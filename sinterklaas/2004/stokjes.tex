\documentclass[12pt]{brief}
\pagestyle{empty}
\usepackage[dutch]{babel}

\date{4 december 2004}

\address{Madrid, Spanje}

\begin{document}

\begin{letter}{Lucie}

\opening{}


\begin{verse}

Laatst zat Sint te denken,\\
wat hij Lucie eens zou schenken.\\
De hele dag ging dit zo door,\\
totdat hij echt zijn geduld verloor.\\[0.5em]

Want wat gebeurde er tien over twee?\\
Hij kreeg plots een geweldig idee:\\
Had hij niet laatst van iemand gehoord,\\
hoe snel Lucie eet, da's echt ongehoord!\\[0.5em]

Gezellig samen aan een rustig diner,\\
dat valt voor Lucie zeker niet mee.\\
`Hap, schrok, weg' is hoe zij eet,\\
is er niemand die daar iets op weet?\\[0.5em]

Dat was dus Sinterklaas, de goede oude man:\\
hij dacht eraan hoe ze dat doen in Japan!\\
Nouja, hij bedoelde dus China,\\
maar het idee was prima.\\[0.5em]

Kijk, dit is wat ze daar doen:\\
zonder een lepel verliest eten fatsoen,\\
maar dat is het gevolg niet alleen,\\
ook `hap, schrok, weg' lukt er daar geneen.\\[0.5em]

Je raadt nu vast wat er komen gaat,\\
als je dit pakje open maakt.\\
Maar dat maakt mij niets uit,\\
dus is hier een olifant met een hele grote snuit.\\

\end{verse}


Sinterklaas


\closing{}

\end{letter}

\end{document}
