\documentclass[12pt]{brief}
\pagestyle{empty}
\usepackage[dutch]{babel}

\date{4 december 2004}

\address{Madrid, Spanje}

\begin{document}

\begin{letter}{Marian Welten}

\opening{}


\begin{verse}

Laatst hoorde de Sint over jou,\\
dat je niet kunt tegen een beetje kou.\\[0.5em]

Zou je niet eens in een ander land gaan wonen?\\
Nederland is niet bij uitstek het land om je als koukleum te vertonen.\\[0.5em]

In de zomer dan valt het nog wel mee,\\
maar 's winters is het ijskoud. Huldigingscomit\'e!\\[0.5em]

Wat een slecht gedicht is dit, zeg!\\
Het springt van de hak op de tak, zeg!\\[2em]

Waar waren we ook 'weer gebleven?\\
Oja, Marian Welten heeft het in de winter nogal moeilijk met haar
leven.\\[0.5em]

Kunnen wij daar niet iets aan doen?\\
misschien krijgt Sinterklaas dan een dikke zoen!\\[0.5em]

Met dat in het vooruitzicht wordt Sint actief,\\
hoe krijgen we haar wat warmer, Marianne Lief?\\[2em]

De oplossing lag eigenlijk voor de hand,\\
ik had namelijk haar verlanglijstje bij de hand.\\[2em]

(Met twee dezelfde woorden is toch niets verkeerd?\\
Als 't maar rijmt is wat ik altijd hebt geleerd.)\\[2em]

Ze wilde dus iets voor over haar handjes,\\
daar hebben we wel iets voor: twee lekker warme wantjes!\\[0.5em]

Helaas, die Marian is nog niet de makkelijkstuh,\\
we moeten iets trendi\"ers dan ouderwetse wantjes verzinnuh.\\[0.5em]

De oplossing gaf ze zelf eigenlijk al,\\
Zwarte handschoentjes in dit geval.\\[2em]

Nou, dit gedicht is echt een grote flop,\\
hopelijk is de inkt van Sint nu heel snel op!\\[0.5em]

Ook al denk je ``veel slechter kan het nooit meer worden nu'',\\
toch stop ik maar, dus ik zeg: ``Ajeto!''\\[2em]


\end{verse}


Sinterklaas


\closing{}

\end{letter}

\end{document}
