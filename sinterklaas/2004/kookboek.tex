\documentclass[12pt]{brief}
\pagestyle{empty}
\usepackage[dutch]{babel}

\date{4 december 2004}

\address{Madrid, Spanje}

\begin{document}

\begin{letter}{Aan Lucie Waals}

\opening{}


\begin{verse}

Vaak zit Sint te denken,\\
wat hij nu weer eens moet schenken.\\
Maar zo niet bij Lucie (ja dat ben jij),\\
was ieder kind maar zo als zij \ldots \\[0.5em]

Nee, die Lucie is niet zo moeilijk,\\
als 't aankomt op cadeau's.\\
Op haar verlanglijst beweert ze vrolijk,\\
``ik vind alles wel grandioos!''\\[0.5em]

Dus Sint trok hierop zijn eigen plan:\\
``Ik geef gewoon 't goedkoopste dat ik vinden kan!''\\[0.5em]

Gelukkig, voor Lucie, was daar ook nog Zwarte Piet.\\
Hij riep: ``Sinterklaas, dat kan toch niet!\\
Een meisje zo lief en bescheiden,\\
kun je toch niet met oude rommel proberen te verleiden?''\\[0.5em]

Sinterklaas gaf Piet daarop zijn zin,\\
``Br\'ad\'a je hebt helemaal gelijk daarin!''\\[0.5em]

Dus zo gezegd, zo gedaan,\\
want Sint wilde het niet hebben een Piet zo kwaad.\\
Vandaar, het cadeautje dat hierin is gegaan,\\
is het lekkerste om te lezen dat er bestaat!\\[2em]

\end{verse}


Sinterklaas en Zwarte Piet


\closing{}

\end{letter}

\end{document}
