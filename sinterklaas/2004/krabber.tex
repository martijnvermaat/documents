\documentclass[12pt]{brief}
\pagestyle{empty}
\usepackage[dutch]{babel}

\date{4 december 2004}

\address{Madrid, Spanje}

\begin{document}

\begin{letter}{Aan een ieder,\\
die 's ochtends wel wat hulp gebruiken kan}

\opening{}


\begin{verse}

Op de bewuste maandag ging het erom\\
wat aan Bartje te geven als ik zaterdag kom?\\[0.5em]

Piet zei: ``Weet U waar ik altijd zoek?\\
in de moderne versie van Uw dikke rode boek.''\\[0.5em]

Hij vroeg mij voor wie ik een cadeautje zocht\\
en ook hoe duur het ongeveer wezen mocht.\\[0.5em]

Vervolgens ging hij naar Sinterklaasje.nl\\
en kwam terug met een groot volgeschreven vel.\\[0.5em]

``Ja, dat daar onderaan, dat ga ik kopen!''\\
riep ik tegen Piet, zo snel was het nog nooit verlopen.\\[2em]


Dus daar ging ik, het was snel gepiept,\\
cadeautjes voor Bartje zijn zo moeilijk nog niet.\\[0.5em]

Veilig thuisgekomen en helemaal tevree,\\
zoek ik zijn verlanglijstje vanachter de pc.\\[0.5em]

Maar wat schetste mijn verbazing toen ik daar op keek?\\
het cadeau was al doorgestreept, dat was waar het op leek!\\[0.5em]

``Hoe heeft dit kunnen gebeuren?'' vroeg ik aan Piet.\\
``Ik wilde U een handje helpen, maar dit was de bedoeling niet!\\[2em]


Dus vandaar, dit pakje is niet de enige in zijn soort,\\
maar laten we er geen probleem van maken, al is het niet zoals het eigenlijk hoort.\\[0.5em]

Mijn oplossing is namelijk een heel simpel plan:\\
er is vast iemand anders die dit ook gebruiken kan.\\[0.5em]

En als het er meer zijn, dat is ook geen probleem,\\
om de beurt laten we jullie raden tussen de tien en de \'e\'en.\\[0.5em]


Dus al was het eigenlijk voor een ander bedoeld,\\
ik hoop dat je je er (vooral 's ochtends) niet te beroerd bij voelt.\\[2em]

\end{verse}


Sinterklaas


\closing{}

\end{letter}

\end{document}
