\documentclass[12pt]{brief}
\pagestyle{empty}
\usepackage[dutch]{babel}

\date{4 december 2004}

\address{Madrid, Spanje}

\begin{document}

\begin{letter}{Ruud}

\opening{}


\begin{verse}

Wat toch te kopen\\
voor Ruud dat kind?\\
De stad door gelopen,\\
maar 't is niets wat ik vind.\\[0.5em]

Ik kan toch niet zonder\\
iets moois binnenkomen?\\
Dat geeft vast gedonder,\\
had'k maar nooit deelgenomen!\\[0.5em]

Dus toch nog eens kijken\\
wat Ruud van mij vraagt.\\
voor ik moet uitwijken,\\
werkelijk naar een andere staat.\\[0.5em]

Ja! Is dit niet iets\\
om hem te geven?\\
een ..., snel op de fiets!\\
we gaan het toch nog beleven!\\[0.5em]

Dus ik kan je wel zeggen,\\
wees zuinig en blij,\\
ik hoef niet meer uit te leggen,\\
't koste veel moeite voor mij!\\[2em]

(\emph{Enige tijd} later\ldots)\\[2em]

De oplettende lezer heeft het al door,\\
``het is niet voor 't eerst dat ik dit gedichtje hoor!''\\
Nee, dat klopt, 't is een gedicht van vorig jaar,\\
toen voor iemand anders, nu is ons Ruudje de sigaar.\\[0.5em]

Het mooie is namelijk was mijn inzicht,\\
dat het geen onderwerp heeft, dit gedicht.\\
Daarom kon hij dit jaar nog een keer,\\
en wellicht zelfs een paar keer meer!\\[1.2em]

\end{verse}


Zwarte Piet


\closing{}

\end{letter}

\end{document}
