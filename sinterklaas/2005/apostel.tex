\documentclass[12pt]{brief}
\pagestyle{empty}
\usepackage[dutch]{babel}

\date{11 december 2005}

\address{Madrid, Spanje}

\begin{document}

\begin{letter}{Ruud}

\opening{}


\begin{verse}

Zaterdagavond is Ruudje in zijn sas,\\
in de Gotcha speelt hij de wildebras.\\
Met vrouwen, bier en Alegrio,\\
Ruud is een man van het goede leven, dat zie je zo.\\[0.5em]

Alles lijkt nog super voor elkaar,\\
in Sinterklaasavond ziet Ruud nog geen gevaar.\\
In gedachten ziet hij vele pakjes,\\
die hij open maken gaat op zijn gemakjes.\\[0.5em]

Ok\'e, een gedichtje hier een daar,\\
even doorbijten en nagenieten maar.\\
Vele kado's heeft Ruud voor ogen,\\
tijdens het zuipen met grote togen.\\[0.5em]

Enkele kroegen verder gebeurt iets onverwacht,\\
dit had Sint toch nooit van Ruud gedacht!\\
Na een bier of 13 pakt Ruud de bijbel er plots bij,\\
met in \emph{Apostelen} vier vers vijf, zo weten wij:\\
\emph{``men moet niet te veel verlangen,\\
maar blij zijn met wat wordt gevangen''}.\\[0.5em]

Dit is waarlijk een hint van Hoger Hand,\\
veel duidelijker hints, ik weet er geen.\\
Ok\'e, een bosje dat vliegt plots in de brand,\\
daar kun je zeker niet omheen,\\
En je weet ook d'r is wat aan de hand,\\
wanneer je als maagd een kindje neemt.\\
Of een beeldje dat moet huilen aan de lopende band,\\
dat vinden we natuurlijk ook wat vreemd.\\[0.5em]

Maar goed, `ons God' was met Ruudje meer subtiel,\\
toch las Ruud niets dat hem niet beviel.\\
Sterker nog, die nacht in bed,\\
heeft Ruud de grootste pret!\\
Het is niet Marjan die daar voor zorgt,\\
maar waar hij aan \emph{denkt}, volkomen onbezorgd.\\
Ruudje is namelijk weer aan het berekenen,\\
waar hij morgen op zou kunnen rekenen.\\
Een cadeautje of negen of tien,\\
voor minder houdt hij het voor gezien.\\
Tenminste, dat is hoe hij er dan nog over denkt,\\
van vorige jaren is hij nauwelijks anders gewend.\\[0.5em]

Inmiddels heeft Ruud zijn verwachtingen bijgesteld,\\
tijdens 't Heerlijk Avondje dat begonnen is\\
zit hij opnieuw te rekenen met dat geld.\\
Vijfentwintig Euro dat is toch niet mis,\\
waarom heb ik nog zo weinig cadeau's geteld?\\[0.5em]

Normaal is Ruud de lulligste niet,\\
maar dit valt hem tegen van Zwarte Piet.\\
Driehondervijfenzestig lange dagen,\\
lijken te eindigen in een groot verdriet.\\
Dit kan Ruud bijna niet langer verdragen,\\
maar nu al opgeven lijkt ook de oplossing niet.\\
Goed, hij besluit nog even niet te klagen,\\
en zich rustig te gedragen.\\[2em]

\end{verse}


Zwarte Piet


\closing{}

\end{letter}

\end{document}
