\documentclass[12pt]{brief}
\pagestyle{empty}
\usepackage[dutch]{babel}

\date{11 december 2005}

\address{Madrid, Spanje}

\begin{document}

\begin{letter}{Sluwe Ruud}

\opening{}


\begin{verse}

Sint zat op een zekere zaterdagmiddag,\\
te denken wat hij moest gaan kopen voor een zekere zondag.\\
De verlanglijst van Ruud was snel gevonden,\\
en Hop sak\'ee, Piet kon heengezonden!\\[0.5em]

Piet kijkt nog eens beter op de lijst en wat ziet-ie?\\
die Ruud lijkt eigenlijk best wel sneeky!\\
Welke truc probeert Ruud namelijk uit te halen?\\
hij wil Sinterklaas wat extra laten betalen!\\
Tenminste, dat is zeker waarop het lijkt,\\
nu Piet nog eens goed naar de verlanglijst kijkt.\\[0.5em]

Wat is er namelijk aan de hand?\\
Ruudje is behoorlijk bijdehand.\\
Vijfentwintig is wat was afgesproken,\\
en dat is waar Ruud zijn kans heeft geroken.\\
Hij dacht ``ik doe het erg slim'',\\
en bedacht en plannetje naar zijn zin.\\[0.5em]

Op zijn lijst plaatste hij vele wensen,\\
met schappelijke prijzen voor alle mensen.\\
Een makkelijke klant die Ruud, is wat je denkt,\\
niets ergs waarvan je hem op het eerste gezicht verdenkt.\\[0.5em]

De prijzen van zijn items gaf hij erbij voor de goede indruk,\\
tussen de vijftien en zeventien Euro was het per stuk.\\
Dat lijken schappelijke prijzen inderdaad,\\
maar bedenk dat het vijfentwintig Euro is waar het om gaat.\\
Ruuudje maakte dus de volgende som,\\
``het zijn vast twee cadeautjes waar ik zo op kom''!\\[0.5em]

Twee van die cadeau's maken iets meer dan dertig,\\
Ruud denkt zeker, die Sinterklaas is vast wat gekkig.\\
Goed ben Ik wel, maar gek gelukkig niet,\\
we laten het dus niet zitten bij deze analyse van Zwarte Piet.\\[0.5em]

Helaas Ruud, je zult het dus niet beleven,\\
dat we het naar boven afronden wat we je gaan geven.\\
Vinden we niets voor een Euro of acht,\\
laten we het lekker zitten bij vijftien is wat ik dacht.\\[0.5em]

Wat je nu nog kunt doen is slechts hopen,\\
dat dit voor jou nog goed is afgelopen.\\
Ik zal vast verklappen, we hebben iets voor je gevonden,\\
maar verwacht niet dat we vijfentwintig naar boven ronden!\\[2em]

\end{verse}


Sinterklaas


\closing{}

\end{letter}

\end{document}
