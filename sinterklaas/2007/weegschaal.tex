\documentclass[12pt]{brief}
\pagestyle{empty}
\usepackage[dutch]{babel}

\date{6 december 2007}

\address{Madrid, Spanje}

\begin{document}

\begin{letter}{Lieve C\'ecile}

\opening{}


\begin{verse}

Lange dagen aan de arbeid,\\
daar vul jij je weken mee.\\
Je schittert in afwezigheid\\
tot een uur of wat voor het diner.\\
Heb jij er wel eens over nagedacht\\
wat je maten daar van vonden?\\
Hoe zij de dagen hebben doorgebracht\\
die ze nauwelijks zonder je konden!\\[1.5em]

Begrijp me alsjeblieft wel goed,\\
het is voor jou soms ook geen pretje.\\
Maar sta even stil bij wat je hen aandoet\\
die bij je wonen in dit vrijgezellenflatje.\\
De uren worden afgeteld,\\
van 's morgens vroeg tot 's middags laat,\\
tot jij je eindelijk weer meldt\\
en voor hen iets lekkers braadt.\\[1.5em]

Want daar is het ze toch om te doen,\\
zo'n lekker maaltje van C\'ecile!\\
Vlees of vegetarisch, rood of groen,\\
kan me niet herinneren dat dat niet beviel!\\
Zo was je laatst weer echt goed bezig,\\
heerlijk in de weer voor het fornuis,\\
eieren, melk en meel mengde je stevig\\
alsof we aten in een pannenkoekenhuis.\\[1.5em]

Over het algemeen ben jij dan ook\\
van alle kookgemakken wel voorzien.\\
Zo breng je water aan de kook\\
met een druk op de knop in een seconde of tien.\\
Voor het snijwerk heb je in de juiste maten\\
speciaal daarvoor bedoelde planken staan.\\
En water wordt vanzelf doorgelaten\\
in iets waarvan Sint niet eens wist dat die bestaan.\\
\emph{(dat ding voor de rijst)}\\[1.5em]

Toch zag ik je eens op een koude avond\\
in de keuken staan te ploeteren.\\
Het recept sprak van een liter, gram, of pond\\
en daarop begon jij te foeteren.\\
Hoe kon jij nu bepalen\\
hoeveel dat dan wel was?\\
Met behulp van ouderwetse schalen\\
of een onbekend groot glas...\\[1.5em]

Dit is waar het ons om gaat\\
en niet die onzin over eenzaamheid\\
(Dat was regels vullen met geblaat\\
zeg ik in alle eerlijkheid).\\
Nu over tot de oplossing\\
van waar jij zo mee zat.\\
Hier komt dan jouw verlossing:\\
het is dat ene wat je nog niet had!\\

\end{verse}


Sintermaat


\closing{}

\end{letter}

\end{document}
