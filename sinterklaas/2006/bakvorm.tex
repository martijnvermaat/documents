\documentclass[12pt]{brief}
\pagestyle{empty}
\usepackage[dutch]{babel}

\date{3 september 2006}

\address{Madrid, Spanje}

\begin{document}

\begin{letter}{Lieve Christa}

\opening{}


\begin{verse}

Het is alweer een tijd geleden\\
dat ik met jouw cadeautjes bezig ging.\\
De doos voor alle aardigheden\\
zit al weken vol bling-bling.\\[1.5em]

Wat was het ook alweer met het cadeau\\
waar ik dit gedicht bij schrijven moet?\\
Oja, ik weet het, het zat zo,\\
inhoud voor een gedicht in overvloed.\\[1.5em]

Het verhaal begint met L. Eliza,\\
een meisje dat je vast wel kent.\\
Ik bedoel hiermee die bitchie troela\\
die term gebruikt ze tenminste zelf frequent.\\[1.5em]

Over eigenschappen zoals bitchie\\
gaat het in dit gedicht toch niet zo zeer.\\
Het is haar university\\
waar we beginnen deze keer.\\[1.5em]

Daarvoor moest ze namelijk deze zomer\\
richting Londen, de grote stad.\\
Daar voelde ze zich soms als nieuwkomer\\
alsof ze het helemaal had gehad.\\[1.5em]

Wat was het nu dat haar zo ver van hier\\
op de been hield in die zware tijden?\\
Het was niet een stevig kratje bier,\\
maar een stukje cake om af te snijden.\\[1.5em]

Het moet bijzonder zijn geweest\\
wat zo'n plakje met je doet.\\
Want dat mistte ze toch het meest\\
nadat ze Nederland weer had begroet.\\[1.5em]

Nu kom ik op het onderwerp van dit gedicht\\
en de persoon aan wie het is gericht.\\
Christa kreeg spontaan ook zin in carrotcake\\
aangestoken door Superhoes zo zag de leek.\\[1.5em]

Helaas is carrotcake een waar\\
die je in Londen kopen moet.\\
Niet bij de hema vind je haar,\\
in Weert geen bakker die het doet.\\[1.5em]

Dit was echter geen probleem\\
voor Christa die van kokkerellen houdt.\\
Dat is tenminste zoals ik het verneem\\
in dat filmpje voor natuurbehoud.\\[1.5em]

Dat was toch echt te mooi voor woorden\\
in vloeiend Engels op de buis.\\
Geen wonder dat we daarmee scoorden,\\
voor eeuwig vastgelegd achter 't fornuis!\\[1.5em]

Dus ik zag dat positief,\\
de carrotcake, die kwam er wel.\\
Christa is zo creatief\\
een taartje bakt ze in een tel.\\[1.5em]

Het bleek toch allemaal niet zo mooi\\
als we stiekem hadden gehoopt.\\
Het internet was werkelijk ene zooi\\
toen ze het voor gerechten hadden afgestroopt.\\[1.5em]

Ze vonden recepten hier en daar\\
de een nog lekkerder dan de ander.\\
Hoe kreeg je zo'n rotcake gaar?\\
de e-chefs hadden ruzie met elkander.\\[1.5em]

De verwarring was allom\\
hoe zo'n cakie nou te bakken?\\
Onze English-speeking-mom\\
liet wat enthousiasme zakken.\\[1.5em]


...




\end{verse}


Sinterklaas


\closing{}

\end{letter}

\end{document}
