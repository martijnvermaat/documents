\documentclass[12pt]{brief}
\pagestyle{empty}
\usepackage[dutch]{babel}

\date{3 september 2006}

\address{Madrid, Spanje}

\begin{document}

\begin{letter}{Lieve Christa}

\opening{}


\begin{verse}

Het is alweer een tijd geleden\\
dat ik met jouw cadeautjes bezig ging.\\
De doos voor alle aardigheden\\
zit al weken vol bling-bling.\\[1.5em]

Wat was het ook alweer met het cadeau\\
waar ik dit gedicht bij schrijven moet?\\
Oja, ik weet het, het zat zo,\\
inhoud voor een gedicht in overvloed.\\[1.5em]

Het verhaal begint met L. Eliza,\\
een meisje dat je vast wel kent.\\
Ik bedoel hiermee die bitchie troela\\
die term gebruikt ze tenminste zelf frequent.\\[1.5em]

Over eigenschappen zoals bitchie\\
gaat het in dit gedicht toch niet zo zeer.\\
Het is haar university\\
waar we beginnen deze keer.\\[1.5em]

Daarvoor moest ze namelijk deze zomer\\
richting Londen, de grote stad.\\
Daar voelde ze zich soms als nieuwkomer\\
alsof ze het helemaal had gehad.\\[3.5em]

Wat was het nu dat haar zo ver van hier\\
op de been hield in die zware tijden?\\
Het was niet een stevig kratje bier,\\
maar een stukje cake om af te snijden.\\[1.5em]

Het moet bijzonder zijn geweest\\
wat zo'n plakje met je doet.\\
Want dat mistte ze toch het meest\\
nadat ze Nederland weer had begroet.\\[1.5em]

Nu kom ik op het onderwerp van dit gedicht\\
en de persoon aan wie het is gericht.\\
Christa kreeg spontaan ook zin in carrotcake\\
aangestoken door Superhoes zo zag de leek.\\[1.5em]

Helaas is carrotcake een waar\\
die je in Londen kopen moet.\\
Niet bij de hema vind je haar,\\
in Weert geen bakker die het doet.\\[1.5em]

Dit was echter geen probleem\\
voor Christa die van kokkerellen houdt.\\
Dat is tenminste zoals ik het verneem\\
in dat filmpje voor natuurbehoud.\\[1.5em]

Dat was toch echt te mooi voor woorden\\
in vloeiend Engels op de buis.\\
Geen wonder dat we daarmee scoorden,\\
voor eeuwig vastgelegd achter 't fornuis!\\[1.5em]

Dus ik zag dat positief,\\
de carrotcake, die kwam er wel.\\
Christa is zo creatief\\
een taartje bakt ze in een tel.\\[2.5em]

Het bleek toch allemaal niet zo mooi\\
als we stiekem hadden gehoopt.\\
Het internet was werkelijk ene zooi\\
toen het voor cakes werd afgestroopt.\\[1.5em]

Ze vonden recepten hier en daar\\
de een nog lekkerder dan de ander.\\
Hoe kreeg je zo'n rotcake gaar?\\
de e-chefs hadden ruzie met elkander.\\[1.5em]

De verwarring was allom\\
hoe zo'n cakie nou te bakken?\\
Onze English-speeking-mom\\
liet wat enthousiasme zakken.\\[1.5em]

Daar zaten we dus lekker mee\\
zonder taartjes in de vuist.\\
Aan recepten geen tekort, oh nee,\\
het waren er te veel, dat was het juist.\\[1.5em]

Dus Christa zat nu te verzinnen\\
hoe ze zich hier uit kon lullen.\\
Vanaf nu zette zij al haar zinnen\\
op een vierkant bakkie om te vullen.\\[1.5em]

Want wat had loes daar in het verre Londen\\
over worteltaart geleerd?\\
Bak hem in het rond dan is dat zonde,\\
het moest in een vierkantje geproduceerd.\\[1.5em]

Vierkant bakken dat ligt echter niet,\\
zoals velen zullen weten,\\
voor een ieder in 't verschiet.\\
Voor je je in't vierkant kunt gaan eten,\\
moet het vierkant zijn waar je 't in giet.\\[1.5em]

Dat was dus de laatste smoes,\\
werden ons de taarten door de neus geboord.\\
Helaas voor Christa ben ik niet voor de poes,\\
reeds in september had ik dit gehoord.\\[1.5em]

En ik zou de Goede Sint niet wezen,\\
als ik dan niks had gedaan.\\
Door de jaren heen geprezen,\\
zal ook dit probleem niet overslaan.\\[1.5em]

Dus je zal er aan geloven\\
dat taartje bakken zul je echt.\\
Het kan nu vierkant in de oven\\
het komt wel goed met dat gerecht.\\[2em]

\end{verse}


Sinterklaas


\closing{}

\end{letter}

\end{document}
