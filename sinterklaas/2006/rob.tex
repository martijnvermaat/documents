\documentclass[12pt]{brief}
\pagestyle{empty}
\usepackage[dutch]{babel}

\date{3 december 2006}

\address{Madrid, Spanje}

\begin{document}

\begin{letter}{Lieve Gerrit}

\opening{}


\begin{verse}

De lange werkweek zit er weer op,\\
ik (Sinterklaas) ben echt doodop.\\
Nog \'e\'en ding waar ik over tob:\\
het gedicht, daar zie ik werkelijk tegenop.\\
Een geniale en rijmende mop,\\
daar wacht iedereen met spanning op.\\[0.5em]

Helaas ben ik in rijmen maar een sufkop,\\
niets meer dan een omhooggevallen bisschop.\\
De voor-de-hand-liggende keuze roept zich op,\\
ik geef dit klusje lekker aan die roetmop!\\
Hij heeft ervaring hierin, volop,\\
dus op naar Piet, in volle galop!\\[1.5em]

Gelukkig zag Piet er geenszins tegenop,\\
een gedicht te schrijven voor een kaaskop.\\
Etalagepop, doodskop, kolenschop,\\
het rijmt er allemaal wel op.\\[0.5em]

Piet verdiende dus een schouderklop,\\
en ik beschreef alleen de enveloppe.\\[1.5em]

Even later, genietend van een taaitaaipop,\\
viel ik in slaap, want ik ben een slaapkop.\\[0.5em]

Totdat Piet mij wakker maakte met een enorme schop,\\
``wakker worden, de tijd is op!''\\
Klaar met de auto stond daar al onze Bob,\\
ik stapte maar snel in met mijn suffe kop.\\[0.5em]

Het gedicht, ging ik vanuit, was vast wel top,\\
zoiets kun je wel over laten aan die zwarte kop.\\
Dus in de auto lees ik het door, niet hardop,\\
wat een schrik, het was een absolute flop!\\
Slechter dan de slechtste belgenmop,\\
dit haalt echt bij lange niet de subtop.\\[0.5em]

De gedachte kwam boven `als ik me nu eens verstop,\\
ik verlaat de auto bij de volgende stop'.\\
Maar al snel begreep ik met mijn witte pruikenkop,\\
hier moest ik het mee doen, met opgeheven hoofd, kom op!\\[0.5em]

Vandaar kom ik vanavond hiermee op de prop,\\
ik begrijp het als hij verdwijnt in een retourenveloppe.\\
Bijna iedere regel rijmt wel ergens op,\\
behalve helaas die allerlaaste onderop.\\
Het is jammer dat je naam niet rijmt op afsluitdop,\\
maar dit was je gedicht, dank voor het lezen Gerrit.\\[2em]

\end{verse}


Sinterklaas


\closing{}

\end{letter}

\end{document}
