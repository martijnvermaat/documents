\documentclass[12pt]{brief}
\pagestyle{empty}
\usepackage[dutch]{babel}

\date{12 augustus 2006}

\address{Madrid, Spanje}

\begin{document}

\begin{letter}{Beste Ruud}

\opening{}


\begin{verse}

Onze Ruud is toch wel zeker\\
een snelle jongen van deze tijd.\\
Dat is wel weer gebleken,\\
wat heeft hij toch een handigheid.\\[1.5em]

Neem nou dat tripje met zijn griet\\
naar dat arme Afrikaanse land.\\
Modern toerisme schuwt hij niet,\\
bakken tussen negerkindjes op het strand.\\[1.5em]

Dat hoort eenmaal bij deze tijd,\\
daar legt Sint zich bij neer.\\
Je zult hem over dit soort narigheid\\
geen kwaad woord horen zeggen meer.\\[1.5em]

Iets anders moet hem daarentegen\\
in dit gedicht wel van het hart.\\
Ruud die is wel van bewegen,\\
oh wat loopt die jongen hard!\\[1.5em]

Dat lijkt iets dat echt hoort\\
bij de moderne jongeman.\\
Maar ene Midas, zegt het voort,\\
heeft daarover een heel ander plan.\\[1.5em]

Want wat denkt nu deze Dekkers,\\
rennen is niet goed voor jou.\\
Raak die kilo's van dat lekkers\\
niet kwijt met dat gesjouw.\\[1.5em]

Toch weet Sint wel, dat gehos\\
is echt iets dat hoort bij deze tijd.\\
Lekker hollen in het bos\\
doet de metrosexueel geheid.\\[1.5em]

Sint zal jullie nog een voorbeeld geven\\
waarom Ruud niet achter loopt.\\
Alles wat hij en die meid beleven\\
wordt naar gelieven bijgephotoshoopt.\\[1.5em]

Boekjes vol met mooie plaatjes\\
flanst hij telkens in elkaar.\\
Daarbij heeft hij mooie praatjes,\\
maar is het allemaal wel waar?\\[1.5em]

Zo had Sint er laatst \'e\'en onder ogen\\
waar toch iets niet helemaal klopt.\\
Mbele had er twee keer op gemogen,\\
door ons Ruudje voor elkaar geshopt.\\[1.5em]

Dus Sint weet niet wat hij denken moet,\\
zijn dit nu wel goede zaken?\\
Het is duidelijk wat die Ruud hier doet,\\
modern de mooiste kiekjes maken.\\[1.5em]

Dit brengt ons bij de clou van het verhaal,\\
onze Ruud is dus een `snelle'.\\
De nieuweste dingen heeft hij telkensmaal,\\
waar zou hij eigenlijk mee bellen?\\[1.5em]

Dat viel Sint toch wel iets tegen\\
met het beeld dat hij van Ruudje heeft.\\
Kan iemand hem misschien bewegen\\
te bellen volgens de tijd waarin hij leeft?\\[1.5em]

Een Nokia drie-en-zestig-tien\\
dat is toch werkelijk om te huilen.\\
Laatst in twee-en-negentig gezien,\\
zonder van die polyphoon geluiden.\\[1.5em]

Goed, dit is dus wat jij wil,\\
zelfs na \'e\'en geval versleten.\\
Jongen, hiermee ga je nooit van bil,\\
maar je moet het zelf weten!\\[1.5em]

Misschien is Sint wat laat met die cadeau,\\
geef hem dan maar aan de Chemokar ofzo.\\[1.5em]

\end{verse}


De groeten van Sinterklaas


\closing{}

\end{letter}

\end{document}
