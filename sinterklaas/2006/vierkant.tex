\documentclass[12pt]{brief}
\pagestyle{empty}
\usepackage[dutch]{babel}

\date{12 augustus 2006}

\address{Madrid, Spanje}

\begin{document}

\begin{letter}{Lieve Christa}

\opening{}


\begin{verse}

Nog net geen jaar geleden zat Sinterklaas te denken,\\
wat hij dit keer eens aan Christa zou gaan schenken.\\
Zoals je weet, Sint is een man die plant vooruit,\\
juni juli of augustus, hij checkt vroeg genoeg die koopgoot uit!\\[1.5em]

Het is wel voorgekomen, dat zijn geen ambtsgeheimen,\\
een duffe hulpsint begint te laat te kopen,\\
of op 't heerlijk avondje zit iemand nog te rijmen.\\
't Is helaas de kwaliteit die het dan vaak moet bekopen,\\
dus de echte Sint zat vanavond niet meer te lijmen\\
en zie je reeds 's zomers in de koopgoot lopen.\\[1.5em]

Oh wat was dat toch verschrikkelijk warm,\\
in een mijter, tabbert en een baard.\\
Tassen vol cadeau's onder de arm,\\
was het maar vast december, dan zat Sint lekker voor de haard.\\[1.5em]

Goed, het was dus lekker ruim gepland,\\
dat moest nu geen problemen geven op het end.\\
Maar het zal je allerminst verbazen,\\
het ging al mis in een zeer vroege fase.\\[1.5em]

Want waar begin je als zijnde een getrainde Sint?\\
op wishdata.net is waar je je cadeautips vindt.\\
Het Sintelijk oog viel op cadeautip nummer tien,\\
Christa noemde dat een `vierkant ding'.\\[1.5em]

Dit vroeg toch om iets meer helderheid,\\
een extra hint wat je hier precies mee had bedoeld.\\
Gelukkig was Christa hier wel toe bereid,\\
misschien had ze wat verwarring gelijk aangevoeld.\\[1.5em]

Christa is in dat soort situaties\\
nooit te beroerd een mens nog wat te helpen.\\
Het voorkomt de Sint een hoop frustraties\\
hem met extra aanwijzingen te overstelpen.\\[1.5em]

Hoe luidde nu die extra zin van Christa\\
waarmee alles duidelijk zou zijn?\\
Na `vierkant ding' kwam dit er achterna\\
niet op het niveau van Sint zijn brein:\\[1.5em]

\emph{``zo'n mooi vierkant, waar je kaarsjes op kunt zetten.''}\\[1.5em]

Nu is de Sint in het kopen van cadeau's\\
toch zeker geen beginneling.\\
Maar dan vraag ik je gewetenloos,\\
wat bedoelt Christa in hemelsnaam met haar `vierkant ding'?\\[1.5em]

Veel gekker moet het toch niet worden\\
met de duidelijkheid van ieders wensen.\\
wat moet er van het feest geworden\\
als Sint geen raad weet met lijstjes van sommige mensen!\\[1.5em]

Dus vandaar die zonnige zomernamiddag\\
dacht Sint ik waag gewoon een gok.\\
Het was het eerste vierkant dat hij zag,\\
waarmee hij uit de koopgoot weer vertrok.\\[8em]

\begin{quote}(En dus lekker op tijd\\
was Sint met dit cadeau\\
Hij hoopt dat het je verblijdt,\\
vierkant is het sowieso.)\\[2em]\end{quote}

\end{verse}


Sinterklaas


\closing{}

\end{letter}

\end{document}
