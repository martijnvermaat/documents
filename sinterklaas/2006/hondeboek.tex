\documentclass[12pt]{brief}
\pagestyle{empty}
\usepackage[dutch]{babel}

\date{12 augustus 2006}

\address{Madrid, Spanje}

\begin{document}

\begin{letter}{Lieve Christa}

\opening{}


\begin{verse}

Ja ik weet nog wel die tijd,\\
cadeautjes aan het zoeken voor die meid.\\
Ik als Sint heb alles voor elkaar,\\
te laat beginnen is bij mij dus geen gevaar.\\[1.5em]

Ik zit dit dus ook niet te schrijven\\
terwijl we stoppen in Den Bosch.\\
Als ik rijm voor oude wijven\\
ben je in augustus reeds de klos.\\[1.5em]

Dit verhaal is daarom voorbereid\\
vele maanden voor je 't leest.\\
Toen wij nog leefden volgens zomertijd\\
was dit gedicht al aan de beurt geweest.\\[1.5em]

Sint die is een man van stiptheid.\\
Ook vanavond zul je zien,\\
toont hij wederom bereidheid\\
te arriveren ruim voor tien.\\[1.5em]

Nee hoe anders zijn die mensen\\
aan de Johan Willem Frisolaan.\\
Zelden zul je mij zien wensen\\
samen met hen uit te gaan!\\[3.5em]

Als ze eenmaal arriveren\\
zijn ze best OK.\\
Maar hoezeer ze ook het tegendeel beweren,\\
Op \emph{tijd} komen, oh nee!\\[1.5em]

Er is er \'e\'entje in dat huis\\
daar geldt dit echter niet voor.\\
Hij ligt gewoonlijk voor de buis\\
maar wat er op is dat heeft hij niet door.\\[1.5em]

Lekker liggen doen er velen,\\
maar ik bedoel hier onze Roef.\\
Oh hij houdt toch zo van spelen,\\
als zijn clubje komt dan zegt hij \emph{woef}!\\[1.5em]

Dit is dan ook wat ik wil zeggen,\\
Roefus die is nooit te laat.\\
Het clubje hoef je hem niet uit te leggen,\\
lang ervoor staat hij paraat.\\[1.5em]

Verder heeft dit in 't geheel\\
niet zozeer met het cadeau te maken.\\
't Is dat ik mij wat verveel\\
bij de werkelijk ingepakte zaken.\\[1.5em]

Maak het cadeau dat voor je ligt\\
daarom maar snel open.\\
Het is nu uit met dit gedicht;\\
de trein is inmiddels Weert binnengelopen...\\[1.5em]

\end{verse}


De groeten van Sinterklaas


\closing{}

\end{letter}

\end{document}
