\documentclass[a4paper,11pt]{article}
\usepackage{amssymb}
\usepackage{fitch}
\usepackage[dutch]{babel}
\usepackage{a4}
\usepackage{color, rotating}
\usepackage{latexsym}
\usepackage[
    colorlinks,
    pdftitle={Uitwerkingen derde huiswerkopdracht},
    pdfsubject={Voortgezette Logica},
    pdfauthor={Martijn Vermaat}
]{hyperref}

% Couldn't find a symbol for bisimulation in standard LaTeX.
% Then I found the following workaround by Rob van Glabbeek.
% http://kilby.stanford.edu/~rvg/

\newcommand{\bis}[1]{ \;% bisimulation
  \raisebox{.3ex}{$\underline{\makebox[.7em]{$\leftrightarrow$}}$}
                  \,_{#1}\,}
\newcommand{\nobis}[1]{\mbox{$\,\not\hspace{-2.5pt}% no bisimulation
    \raisebox{.3ex}{$\underline{\makebox[.7em]{$\leftrightarrow$}}$}
                  \,_{#1}\,$}}

\title{Uitwerkingen derde huiswerkopdracht\\Voortgezette Logica}
\author{
    Martijn Vermaat\\
    mvermaat@cs.vu.nl
}
\date{8 december 2004}

\begin{document}
\maketitle


\section*{Opgave 1 (3.3 6)}

Te bewijzen:

\begin{displaymath}
\forall x (Px \vee Qx) \vDash_{P} \forall x (Px \leftrightarrow \neg Qx) \, \mbox{.}
\end{displaymath}

Daartoe laten we zien dat ieder $P$-minimaal model voor $\forall x (Px \vee
Qx)$ (afgekort $\varphi$), ook $\forall x (Px \leftrightarrow \neg
Qx)$ (afgekort $\psi$) waar maakt.

\paragraph{Bewijs}

We bekijken een willekeurig $P$-minimaal model $\mathcal{M}$ voor
$\varphi$. Nu geldt voor ieder element $e$ in het domein van $\mathcal{M}$ dat
\`ofwel $e \in P^{\mathcal{M}}$, \`ofwel $e \in Q^{\mathcal{M}}$ (maar nooit beide).

Want stel dat er een element $e$ is waarvoor dat niet geldt. Dat in ieder
geval $Pe$ \`of $Qe$ moet gelden is snel duidelijk, anders maakt $\mathcal{M}$
$\varphi$ niet waar. Dat betekent dus dat $Pe$ \`en $Qe$. Maar dan is
$\mathcal{M}$ geen $P$-minimaal model voor $\varphi$. (Neem bijvoorbeeld
$\mathcal{M}'$ gelijk aan $\mathcal{M}$ maar zonder $Pe$. Dan is $\mathcal{M}'
<_{P} \mathcal{M}$ en $\mathcal{M}'$ maakt nog steeds $\varphi$ waar.) Dit is
in tegenspraak met ons gegeven, dus kan dit element $e$ niet bestaan.

Anders gezegd geldt dus voor ieder element $e$:

\begin{displaymath}
Pe \rightarrow \neg Qe
\end{displaymath}

en

\begin{displaymath}
\neg Qe \rightarrow Pe \mbox{.}
\end{displaymath}

Dit is precies wat gezegd wordt met $\psi$, dus maakt ons model $\mathcal{M}$
ook $\psi$ waar. En dat is wat we nodig hadden om te bewijzen dat

\begin{displaymath}
\forall x (Px \vee Qx) \vDash_{P} \forall x (Px \leftrightarrow \neg Qx) \mbox{.}
\end{displaymath}

\hfill\rule{2.1mm}{2.mm}


\section*{Opgave 2 (3.7)}

\begin{enumerate}


\item

\begin{eqnarray*}
\mathcal{M}_{1} & \vDash & \Sigma  \, \mbox{,}\\
\mathcal{M}_{2} & \nvDash & \Sigma \, \mbox{,}\\
\mathcal{M}_{3} & \vDash & \Sigma  \, \mbox{.}
\end{eqnarray*}


\item

\begin{eqnarray*}
\mathcal{M}_{1} & \vDash_{P} & \Sigma  \, \mbox{,}\\
\mathcal{M}_{2} & \nvDash_{P} & \Sigma \, \mbox{,}\\
\mathcal{M}_{3} & \vDash_{P} & \Sigma  \, \mbox{.}
\end{eqnarray*}


\item

\begin{eqnarray*}
\mathcal{M}_{1} & \vDash_{P;B} & \Sigma  \, \mbox{,}\\
\mathcal{M}_{2} & \nvDash_{P;B} & \Sigma \, \mbox{,}\\
\mathcal{M}_{3} & \nvDash_{P;B} & \Sigma  \, \mbox{.}
\end{eqnarray*}


\item

Te bewijzen:

\begin{displaymath}
\Sigma \vDash_{P;B} Pt \, \mbox{.}
\end{displaymath}

Hiertoe laten we zien dat in ieder $<^{P;B}$-minimaal model voor $\Sigma$ ook
$Pt$ waar is.

\paragraph{Bewijs}

Laat $\mathcal{M}$ een willekeurig $<^{P;B}$-minimaal model voor $\Sigma$
zijn. Dan moet $\mathcal{M} \vDash \Sigma$.  Volgens $\Sigma$ is $At$ waar en
volgens $\forall x ((Ax \vee Bx) \rightarrow Px)$  dan ook $Pt$.

Omdat $\mathcal{M}$ willekeurig gekozen was, maakt ieder
$<^{P;B}$-minimaal model voor $\Sigma$ ook $Pt$ waar en dus geldt (volgens
definitie)

\begin{displaymath}
\Sigma \vDash_{P;B} Pt \, \mbox{.}
\end{displaymath}

\hfill\rule{2.1mm}{2.mm}


\item

Te bewijzen:

\begin{displaymath}
\Sigma \nvDash_{P;B} Pu \, \mbox{.}
\end{displaymath}

We geven een tegenvoorbeeld voor het geval dit niet zo was.

\paragraph{Bewijs}

Bekijk het gegeven model $\mathcal{M}_{1}$. Dit model is een
$<^{P;B}$-minimaal model voor $\Sigma$, maar maakt niet $Pu$ waar. Dus $Pu$ is
niet waar in alle $<^{P;B}$-minimale modellen voor $\Sigma$ en dus

\begin{displaymath}
\Sigma \nvDash_{P;B} Pu \, \mbox{.}
\end{displaymath}

\hfill\rule{2.1mm}{2.mm}


\item

We laten zien dat

\begin{displaymath}
\Sigma \vDash_{P;B} Ps
\end{displaymath}

geldt.

\paragraph{Bewijs}

We nemen $\mathcal{M}$ als een willekeurig $<^{P;B}$-minimaal model voor
$\Sigma$ aan. Dan moet $\mathcal{M} \vDash \Sigma$.  Volgens $\Sigma$ is $Bs$
waar en volgens $\forall x ((Ax \vee Bx) \rightarrow Px)$  dan ook $Ps$.

Hieruit volgt dat ieder $<^{P;B}$-minimaal model voor $\Sigma$ ook $Ps$ waar
maakt en dus geldt

\begin{displaymath}
\Sigma \vDash_{P;B} Ps \, \mbox{.}
\end{displaymath}

\hfill\rule{2.1mm}{2.mm}

\paragraph{N.B.}

Ook in dit geval kunnen we $P^{\mathcal{M}}$ niet minimaliseren door $s$ eruit
te laten, omdat we dan volgens de eerste formule in $\Sigma$ ook $s$ uit
$B^{\mathcal{M}}$ moeten laten (en dat heeft direct als gevolg dat de derde
formule uit $\Sigma$, $Bs$, niet meer waar is).


\end{enumerate}


\section*{Opgave 3 (4.2 1)}

Te bewijzen:

\begin{displaymath}
\varphi \vDash_{\sqsubset} \psi \rightarrow \chi
\end{displaymath}

gegeven

\begin{equation}\label{eqn:gegeven}
\varphi \wedge \psi \vDash_{\sqsubset} \chi \, \mbox{.}
\end{equation}

\paragraph{Bewijs}

Wat we moeten laten zien is dat, gegeven \ref{eqn:gegeven},  voor ieder
$\sqsubset$-preferent model $\mathcal{M}$ voor $\varphi$ geldt:

\begin{displaymath}
\mathcal{M} \vDash \psi \rightarrow \chi \, \mbox{.}
\end{displaymath}

Hiertoe bekijken we een willekeurig $\sqsubset$-preferent model
$\mathcal{M}$ voor $\varphi$. Nu onderscheiden we voor $\mathcal{M}$ de
volgende twee gevallen:

\begin{enumerate}
\item\label{geval:een} $\mathcal{M} \vDash \psi$
\item\label{geval:twee} $\mathcal{M} \vDash \neg \psi$
\end{enumerate}

In geval \ref{geval:twee} is het duidelijk dat geldt:

\begin{displaymath}
\mathcal{M} \vDash \psi \rightarrow \chi \, \mbox{.}
\end{displaymath}

In het eerste geval moeten we hiertoe laten zien dat $\chi$ waar is. Nu is in
dit geval $\mathcal{M}$ ook een $\sqsubset$-preferent model voor $\varphi
\wedge \psi$. Dan volgt uit ons gegeven \ref{eqn:gegeven} dat $\mathcal{M}$
ook $\chi$ waar maakt.

\paragraph{}

Concluderend hebben we laten zien dat, gegeven \ref{eqn:gegeven}, voor ieder
$\sqsubset$-preferent model $\mathcal{M}$ voor $\varphi$ geldt:

\begin{displaymath}
\mathcal{M} \vDash \psi \rightarrow \chi \, \mbox{.}
\end{displaymath}

En dus geldt, gegeven \ref{eqn:gegeven}, ook

\begin{displaymath}
\varphi \vDash_{\sqsubset} \psi \rightarrow \chi \, \mbox{.}
\end{displaymath}

\hfill\rule{2.1mm}{2.mm}


\subsection*{Geldt de omgekeerde implicatie ook?}

De omgekeerde implicatie

\begin{displaymath}
\varphi \vDash_{\sqsubset} \psi \rightarrow \chi
\, \Longrightarrow \,
\varphi \wedge \psi \vDash_{\sqsubset} \chi
\end{displaymath}

geldt niet. We laten dit zien door een tegenvoorbeeld te geven.

\paragraph{Bewijs}

We kiezen voor $\varphi$, $\psi$ en $\chi$ formules uit de eerste-orde
predikatenlogica:

\begin{eqnarray*}
\varphi & = & Bt \, \mbox{,}\\
\psi    & = & \forall x ((Bx \wedge \neg Px) \rightarrow Fx) \, \mbox{,}\\
\chi    & = & Ft \, \mbox{.}
\end{eqnarray*}

Voor $\sqsubset$ kiezen we de strikte parti\"ele ordening $<^{P}$ zoals deze
is gedefini\"eerd voor predikaatcircumscriptie in definitie 3.1 van de
reader.

We laten nu zien dat

\begin{displaymath}
\varphi \vDash_{<^{P}} \psi \rightarrow \chi
\end{displaymath}

maar niet

\begin{displaymath}
\varphi \wedge \psi \vDash_{<^{P}} \chi \, \mbox{.}
\end{displaymath}

\paragraph{}

Laat $\mathcal{M}$ een $<^{P}$-preferent model zijn voor $\varphi$. Dat
betekent dat $t^{\mathcal{M}} \not \in P^{\mathcal{M}}$. Want stel dat
$t^{\mathcal{M}} \in P^{\mathcal{M}}$. Dan is er een model $\mathcal{M}'$ voor
$\varphi$ gelijk aan $\mathcal{M}$ maar zonder $t^{\mathcal{M}'}$ in
$P^{\mathcal{M}'}$ met $\mathcal{M}' <^{P} \mathcal{M}$. Dat is in tegenspraak
met de gegeven eigenschap van $\mathcal{M}$ en dus geldt $t^{\mathcal{M}} \not
\in P^{\mathcal{M}}$.

Uit de waarheid van $\varphi$ en $\neg Pt$ in $\mathcal{M}$ volgt dat gegeven
de waarheid van $\psi$ ook $\chi$ waar is. En dus geldt $\mathcal{M} \vDash
\psi \rightarrow \chi$. Omdat $\mathcal{M}$ een willekeurig gekozen
$<^{P}$-preferent model zijn voor $\varphi$ is, volgt hier uit dat

\begin{displaymath}
\varphi \vDash_{<^{P}} \psi \rightarrow \chi \, \mbox{.}
\end{displaymath}

\paragraph{}

We bekijken het model $\mathcal{M}$ met domein $\{Tweety\}$, interpretatie
$s^{\mathcal{M}} = Tweety$ en

\begin{eqnarray*}
B^{\mathcal{M}} & = & \{Tweety\} \, \mbox{,} \\
P^{\mathcal{M}} & = & \{Tweety\} \, \mbox{,} \\
F^{\mathcal{M}} & = & \{\} \, \mbox{.}
\end{eqnarray*}

Dit is een $<^{P}$-preferent model voor $\varphi$ en $\psi$, omdat:

\begin{itemize}

\item $\mathcal{M} \vDash \varphi$
\item $\mathcal{M} \vDash \psi$
\item Er is geen model $\mathcal{M}'$ voor $\varphi$ en $\psi$ met
  $\mathcal{M}' <^{P} \mathcal{M}$.

Want stel dat er wel zo'n model $\mathcal{M}'$ zou zijn. De enige manier om
$\mathcal{M}' <^{P} \mathcal{M}$ waar te maken is door $\mathcal{M}'$ gelijk
te nemen aan $\mathcal{M}$, echter met $P^{\mathcal{M}'} = \{\}$. Maar dan is
$\mathcal{M}'$ geen model meer voor $\psi$. Dit is in tegenspraak met de
definitie van $\mathcal{M}'$ en dus bestaat dit model niet.

\end{itemize}

In $\mathcal{M}$ is echter $\chi$ niet waar en dus hebben we

\begin{displaymath}
\varphi \wedge \psi \nvDash_{<^{P}} \chi \, \mbox{.}
\end{displaymath}

\paragraph{}

Uit deze twee resultaten concluderen we dat de implicatie

\begin{displaymath}
\varphi \vDash_{\sqsubset} \psi \rightarrow \chi
\, \Longrightarrow \,
\varphi \wedge \psi \vDash_{\sqsubset} \chi
\end{displaymath}

niet geldt.

\hfill\rule{2.1mm}{2.mm}


\end{document}
