\documentclass[a4paper,11pt]{article}
\usepackage{listings}
\usepackage[dutch]{babel}
\usepackage{a4}
\usepackage{color, rotating}
\usepackage{latexsym}
\usepackage[
    colorlinks,
    pdftitle={Uitwerkingen eerste huiswerkopdracht},
    pdfsubject={Voortgezette Logica},
    pdfauthor={Martijn Vermaat}
]{hyperref}

% Couldn't find a symbol for bisimulation in standard LaTeX.
% Then I found the following workaround by Rob van Glabbeek.
% http://kilby.stanford.edu/~rvg/

\newcommand{\bis}[1]{ \;% bisimulation
  \raisebox{.3ex}{$\underline{\makebox[.7em]{$\leftrightarrow$}}$}
                  \,_{#1}\,}
\newcommand{\nobis}[1]{\mbox{$\,\not\hspace{-2.5pt}% no bisimulation
    \raisebox{.3ex}{$\underline{\makebox[.7em]{$\leftrightarrow$}}$}
                  \,_{#1}\,$}}

\title{Uitwerkingen eerste huiswerkopdracht\\Voortgezette Logica}
\author{
    Martijn Vermaat\\
    mvermaat@cs.vu.nl
}
\date{11 oktober 2004}

\begin{document}
\maketitle


\section*{Opgave 1}

\begin{quote}
Uit: het tentamen van 16 juni 2004
\end{quote}

De volgende modale formule $\varphi$ is equivalent met $\triangle p$:

\begin{eqnarray*}
\varphi & = & p \, \land \, \Diamond \lnot p
\end{eqnarray*}


\section*{Opgave 2}

\begin{quote}
Uit: het tentamen van 16 juni 2004
\end{quote}

\begin{description}

\item[a)]

Plaatjes maken alles makkelijker:\\[1em]

\input{huiswerk-1-opg2.pdftex_t}

De relatie $\mathcal{Z}$ is een bisimulatie tussen de modellen $\mathcal{M}$ en
$\mathcal{M'}$:

\begin{eqnarray*}
\mathcal{Z} & = & \{ \langle s, s' \rangle , \, \langle s, t' \rangle \}
\end{eqnarray*}

\item[b)]

We proberen een tegenspraak af te leiden uit de aanname dat er een modale
basisformule $\psi$ bestaat die equivalent is met $\bowtie p$:

\begin{eqnarray*}
\psi & \leftrightarrow & \bowtie p
\end{eqnarray*}

We bekijken de waarheidswaarde van $\psi$ in wereld $s$. Omdat $\psi$
equivalent is met $\bowtie p$ en $\bowtie p$ volgens de semantiek van
$\bowtie$ niet waar is in $s$ (er bestaat geen wereld $x \not= s$ met $Rsx$)
geldt:

\begin{eqnarray}\label{eqn:first}
\mathcal{M},s & \not\models & \psi
\end{eqnarray}

Vervolgens bekijken we de waarheidswaarde van $\psi$ in wereld $s'$. Omdat
$\bowtie p$ volgens de semantiek van $\bowtie$ waar is in $s'$ (want
$\mathcal{M'},s' \models p$ en er bestaat een wereld $x \not= s'$ met $R's'x$,
namelijk $t'$, zo dat $\mathcal{M'},x \models p$) volgt op gelijke wijze:

\begin{eqnarray}\label{eqn:second}
\mathcal{M'},s' & \models & \psi
\end{eqnarray}

Omdat $s$ bisimuleert met $s'$ volgt uit de preservatie van waarheid onder
bisimulatie en uit \ref{eqn:second} dat:

\begin{eqnarray*}
\mathcal{M},s & \models & \psi
\end{eqnarray*}

Dit is echter in tegenspraak met \ref{eqn:first} waardoor onze aanname niet
juist geweest kan zijn.

\end{description}


\section*{Opgave 3}

\begin{quote}
Uit: \emph{Handout 1: Multimodale Logica}
\end{quote}

\begin{description}

\item[a)]

Het model $\mathcal{M} = (W,<,\pi)$ uit het bewijs van propositie 2.18 uit de
syllabus kan opgevat worden als een $\{F,P\}$-model $\mathcal{N} =
(W,<_{F},<_{P},\pi)$ door voor $<_{F}$ de temporele ordening $<$ te nemen en
voor $<_{P}$ de omkering $>$ van $<$.

Evenzo kunnen we het model $\mathcal{M'} = (W',<',\pi')$ uit hetzelfde bewijs
opvatten als een $\{F,P\}$-model $\mathcal{N'} = (W',<_{F}',<_{P}',\pi')$ door
voor $<_{F}'$ de temporele ordening $<'$ te nemen en voor $<_{P}'$ de omkering
$>'$ van $<'$.

\item[b)]

De relatie $\mathcal{Z}$ met elementen uit $W \times W'$ is een bisimulatie
tussen $\mathcal{N}$ en $\mathcal{N'}$ als aan de drie voorwaarden uit de
definitie van bisimulatie uit de handout is voldaan.

\begin{enumerate}

\item

Voor ieder paar $(w,w')$ uit $\mathcal{Z}$ geldt $\pi(w)(p) = \pi'(w')(p)$ en
$\pi(w)(q) = \pi'(w')(q)$ (bekijk hiervoor de figuur uit de syllabus). Er
worden geen andere propositionele variabelen in $W$ en $W'$ beschouwd en dus
is er aan de eerste voorwaarde voldaan.

\item

We beschouwen de tweede voorwaarde eerste met betrekking tot de
toegankelijkheidsrelaties $>_{F}$ en $>_{F}'$ welke gelijk zijn aan de in de
syllabus met pijlen aangegeven $>$ en $>'$. Om te voldoen aan de voorwaarde
moet er in de figuur voor iedere $(w,w')$ uit $\mathcal{Z}$ gelden:

\begin{quote}

er loopt een pijl van $w$ naar $\psi$

$\quad \Longrightarrow$

er loopt een pijl van $w'$ naar $\psi' \in W'$ en $(\psi,\psi') \in \mathcal{Z}$

\end{quote}

Dit geldt inderdaad. Eigenlijk moeten we er enkele pijlen bij denken, omdat
temporele frames per definitie transitief zijn, maar ook dan blijft
bovenstaande gelden.

Door alle pijlen in de figuur om te draaien krijgen we een voorstelling van de
toegankelijkheidsrelaties $<$ en $<'$ welke gelijk zijn aan $>_{P}$ en
$>_{P}'$. Op dezelfde manier kunnen we inzien dat ook met betrekking tot deze
toegankelijkheidsrelaties voldaan is aan de voorwaarde.

\item

De derde voorwaarde is symmetrisch in $\mathcal{Z}$ met de tweede voorwaarde
en daarom is het nu triviaal om in te zien dat ook aan deze voorwaarde voldaan
is.

\end{enumerate}

\item[c)]

We nemen aan dat er een modale formule $\varphi$ over $\{F,P\}$ bestaat die
equivalent is met $p \mathcal{U} q$:

\begin{eqnarray*}
\psi & \leftrightarrow & \bowtie p
\end{eqnarray*}

We bekijken de waarheidswaarde van $\varphi$ in wereld $s_{0}$. Volgens de
semantiek van $\mathcal{U}$ is $\varphi$ waar in $s_{0}$ dan en slechts dan
als:

\begin{enumerate}

\item

er een wereld $v$ bestaat waarin $q$ waar is; en

\item

er een pijl loopt van $s_{0}$ naar deze wereld $v$; en

\item

in alle werelden waarnaar een pijl loopt vanaf $s_{0}$ en een pijl vanaf loopt
naar $v$ de propositie $p$ waar is.

\end{enumerate}

Kiezen we voor $v$ de wereld $v_{1}$, dan zien we snel dat:

\begin{eqnarray}\label{eqn:third}
\mathcal{M},s_{0} & \models & \varphi
\end{eqnarray}

(Bedenk hierbij weer dat er ook een niet getekende pijl loopt van $s_{0}$ naar
$v_{1}$ wegens de transitiviteit van $<$ en $<_{F}$.)

\paragraph{}

Vervolgens bekijken we de waarheidswaarde van $\varphi$ in wereld
$s'$ op vergelijkbare wijze. Nu kunnen kunnen we echter in bovenstaand verhaal
alleen $v'$ kiezen voor de rol van $v$ (het is immers de enige wereld in
$\mathcal{M'}$ die $q$ waar maakt). Omdat er een pijl van $s'$ naar $t'$ en
een pijl van $t'$ naar $v'$ lopen moet volgens de laatste voorwaarde $p$ waar
zijn in $t'$. Dat is niet het geval, dus geldt:

\begin{eqnarray}\label{eqn:fourth}
\mathcal{M'},s' & \not\models & \varphi
\end{eqnarray}

\paragraph{}

Eerder hebben we aangetoond dat $\mathcal{M},s_{0} \bis{}
\mathcal{M'},s'$. Samen met \ref{eqn:third} volgt daarmee volgens stelling 8
uit de handout:

\begin{eqnarray*}
\mathcal{M'},s' & \models & \varphi
\end{eqnarray*}

Dit is echter in tegenspraak met \ref{eqn:fourth} en dus is de aanname dat
$\varphi$ een modale formule over $\{F,P\}$ is die equivalent is met $p
\mathcal{U} q$ niet juist geweest.

\end{description}


\end{document}
