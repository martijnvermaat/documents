\documentclass[a4paper,11pt]{article}
\usepackage[defs]{ams}
\usepackage[dutch]{babel}
\usepackage{a4}
\usepackage{color, rotating}
\usepackage{latexsym}
\usepackage[
    colorlinks,
    pdftitle={Uitwerkingen tweede huiswerkopdracht},
    pdfsubject={Voortgezette Logica},
    pdfauthor={Martijn Vermaat}
]{hyperref}

% Couldn't find a symbol for bisimulation in standard LaTeX.
% Then I found the following workaround by Rob van Glabbeek.
% http://kilby.stanford.edu/~rvg/

\newcommand{\bis}[1]{ \;% bisimulation
  \raisebox{.3ex}{$\underline{\makebox[.7em]{$\leftrightarrow$}}$}
                  \,_{#1}\,}
\newcommand{\nobis}[1]{\mbox{$\,\not\hspace{-2.5pt}% no bisimulation
    \raisebox{.3ex}{$\underline{\makebox[.7em]{$\leftrightarrow$}}$}
                  \,_{#1}\,$}}

\title{Uitwerkingen eerste huiswerkopdracht\\Voortgezette Logica}
\author{
    Martijn Vermaat\\
    mvermaat@cs.vu.nl
}
\date{21 november 2004}

\begin{document}
\maketitle


\section*{Opgave 1}

Gezocht wordt een model met 3 kenners waar in een bepaalde wereld
$t$ de volgende formules waar zijn:

\begin{eqnarray*}
\mathcal{M},t & \Vdash & K_{1} p        \\
\mathcal{M},t & \Vdash & K_{2} p        \\
\mathcal{M},t & \Vdash & K_{3} K_{1} p  \\
\mathcal{M},t & \nVdash & K_{3} K_{2} p
\end{eqnarray*}

We defini\"eren het frame $\mathcal{F}$ als volgt:

\begin{eqnarray*}
\mathcal{F} & = & (\{t,u,v\}, R_{1}, R_{2}, R_{3}) \\
& & \mbox{waarbij}                                 \\
& & R_{1} = \{(t,t), (u,u), (v,v)\}                \\
& & R_{2} = \{(t,t), (u,u), (v,v), (u,v), (v,u)\}  \\
& & R_{3} = \{(t,t), (u,u), (v,v), (t,u), (u,t)\}
\end{eqnarray*}

Nu voldoet het model $\mathcal{M}$ aan bovengenoemde eisen:

\begin{eqnarray*}
\mathcal{M} & = & (\mathcal{F}, \pi) \\
& & \mbox{met}                       \\
& & \pi(t)(p) = 1                    \\
& & \pi(u)(p) = 1                    \\
& & \pi(v)(p) = 0
\end{eqnarray*}

Dit is een tegenmodel voor de bewering dat wanneer twee kenners iets
weten, ze dit ook van elkaar weten. We kunnen dus zeggen dat deze
bewering niet in het algemeen geldt:

\begin{eqnarray*}
K_{i} p \, \wedge \, K_{j} p & \nvDash & K_{i} K_{j} p
\end{eqnarray*}

waarbij $i$ en $j$ twee kenners zijn.


\end{document}
