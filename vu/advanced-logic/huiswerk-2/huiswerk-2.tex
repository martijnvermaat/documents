\documentclass[a4paper,11pt]{article}
\usepackage{amssymb}
\usepackage{fitch}
\usepackage[dutch]{babel}
\usepackage{a4}
\usepackage{color, rotating}
\usepackage{latexsym}
\usepackage[
    colorlinks,
    pdftitle={Uitwerkingen tweede huiswerkopdracht},
    pdfsubject={Voortgezette Logica},
    pdfauthor={Martijn Vermaat}
]{hyperref}

% Couldn't find a symbol for bisimulation in standard LaTeX.
% Then I found the following workaround by Rob van Glabbeek.
% http://kilby.stanford.edu/~rvg/

\newcommand{\bis}[1]{ \;% bisimulation
  \raisebox{.3ex}{$\underline{\makebox[.7em]{$\leftrightarrow$}}$}
                  \,_{#1}\,}
\newcommand{\nobis}[1]{\mbox{$\,\not\hspace{-2.5pt}% no bisimulation
    \raisebox{.3ex}{$\underline{\makebox[.7em]{$\leftrightarrow$}}$}
                  \,_{#1}\,$}}

\title{Uitwerkingen eerste huiswerkopdracht\\Voortgezette Logica}
\author{
    Martijn Vermaat\\
    mvermaat@cs.vu.nl
}
\date{21 november 2004}

\begin{document}
\maketitle


\section*{Opgave 1}

Gezocht wordt een model met 3 kenners waar in een bepaalde wereld
$t$ de volgende formules waar zijn:

\begin{eqnarray*}
\mathcal{M},t & \Vdash & K_{1} p        \\
\mathcal{M},t & \Vdash & K_{2} p        \\
\mathcal{M},t & \Vdash & K_{3} K_{1} p  \\
\mathcal{M},t & \nVdash & K_{3} K_{2} p
\end{eqnarray*}

We defini\"eren het frame $\mathcal{F}$ als volgt:

\begin{eqnarray*}
\mathcal{F} & = & (\{t,u,v\}, R_{1}, R_{2}, R_{3}) \\
& & \mbox{waarbij}                                 \\
& & R_{1} = \{(t,t), (u,u), (v,v)\}                \\
& & R_{2} = \{(t,t), (u,u), (v,v), (u,v), (v,u)\}  \\
& & R_{3} = \{(t,t), (u,u), (v,v), (t,u), (u,t)\}
\end{eqnarray*}

Nu voldoet het model $\mathcal{M}$ aan bovengenoemde eisen:

\begin{eqnarray*}
\mathcal{M} & = & (\mathcal{F}, \pi) \\
& & \mbox{met}                       \\
& & \pi(t)(p) = 1                    \\
& & \pi(u)(p) = 1                    \\
& & \pi(v)(p) = 0
\end{eqnarray*}

En voor wie $\mathcal{M}$ in een plaatje wil zien:\\[1em]

\input{huiswerk-2-opg1.pdftex_t}

Dit is een tegenmodel voor de bewering dat wanneer twee kenners iets
weten, ze dit ook van elkaar weten. We kunnen dus zeggen dat deze
bewering niet in het algemeen geldt:

\begin{eqnarray*}
K_{i} p \, \wedge \, K_{j} p & \nvDash & K_{i} K_{j} p
\end{eqnarray*}

waarbij $i$ en $j$ twee kenners zijn.


\section*{Opgave 2}

Een afleiding in $S5$ voor $\neg K \neg K p \to K ( p \vee q )$:

\begin{equation*}
\begin{fitch}
\neg K p \to K \neg K p                        & A3        \\
\neg K \neg K p \to K p                        & PROP, 1   \\
K(p \to q) \to (K p \to K q)                   & DB        \\
K(p \to (p \vee q)) \to (K p \to K (p \vee q)) & SUB, 3    \\
p \to (p \vee q)                               & CT        \\
K(p \to (p \vee q))                            & UG, 5     \\
K p \to K (p \vee q)                           & MP, 4,6   \\
\neg K \neg K p \to K (p \vee q)               & \dag, 2,7
\end{fitch}
\end{equation*}


\section*{Opgave 3}

In een willekeurig epistemisch model $\mathcal{M}$ bekijken we de
toegankelijkheidsrelatie die hoort bij de operator $E$ (waarbij $n$ het aantal
kenners is):

\begin{displaymath}
R_{E} = \bigcup_{i=1}^{n} \, R_{i}
\end{displaymath}

Volgens de semantiek van de kennis operator $K$ betekent $(p \to E p)$ nu dat er
voor iedere wereld $s$ die $p$ waar maakt geldt: $p$ is waar in alle werelden
$t$ met $R_{E} st$.

Evenzo betekent $(p \to E E p)$ nu dat er voor iedere wereld $s$ die $p$ waar
maakt geldt: $p$ is waar in alle werelden $u$ met $R_{E} st$ en $R_{E} tu$ en
voor $t$ willekeurige werelden.

\paragraph{}

We zien dat, gegeven $\mathcal{M} \Vdash (p \to E p)$, in deze interpretatie
van $(p \to E E p)$ de werelden $t$ allemaal $p$ waar maken. Eenvoudig is te
zien dat nogmaals toepassen van het gegevene op alle werelden $t$ geeft dat
alle werelden $u$ ook $p$ waar maken. En dat is precies wat nodig was om $(p
\to E E p)$ waar te maken.

Aangezien het model $\mathcal{M}$ willekeurig gekozen was, is onze conclusie

\begin{displaymath}
\mathcal{M} \Vdash (p \to E p) \quad \vDash \quad \mathcal{M} \Vdash(p \to E E p)
\end{displaymath}

voor ieder epistemisch model $\mathcal{M}$.

\paragraph{Vraag:}

Ik ben er niet zeker van dat ik met deze laatste notatie precies aangeef wat
er gevraagd werd. Eerder had ik een alternatieve uitwerking van deze opgave
(deze heb ik hier onder bijgevoegd), maar daarmee kom ik ten eerste tot een
\emph{andere} conclusie (en volgens mij niet gelijk aan die gevraagd werd) en
bovendien tot een \emph{verkeerde} conclusie (maw, ik denk niet dat de
afleiding klopt).

Als alternatief voor de alternatieve uitwerking (snapt u het nog?) bedacht ik
later om $p \to E p$ als extra axioma aan te nemen en van daaruit de conclusie
af te leiden, maar ik denk dat dat ook niet juist is (het gaat namelijk niet
om geldigheid, maar om waarheid). Is dit \"uberhaupt op een aardige manier
mogelijk in een Hilbert systeem (en zou het antwoord geaccepteerd worden)?


\section*{Opgave 3 - alternatieve uitwerking}

In het axiomatisch systeem $(K^{n})^{+}$ hebben we voor alle $i,j,k \in \{1,\ldots,n\}$
de volgende afleiding:

\begin{equation*}
\begin{fitch}
K_{i}(p \to q) \to (K_{i} p \to K_{i} q)                        & DB        \\
K_{i}(p \to K_{j} p) \to (K_{i} p \to K_{i} K_{j} p)            & SUB, 1    \\
(p \to K_{i} p) \to ((p \to K_{j} p) \to K_{i}(p \to K_{j} p))  & PROP      \\
(p \to K_{i} p) \to (p \to K_{j} K_{k} p)                       & PROP
\end{fitch}
\end{equation*}

We kunnen hier volgens $PROP$ ook volstaan met alleen het axioma $DB$ direct
gevolgd door de conclusie middels een enkele $PROP$, maar het is nu al
ingewikkeld genoeg.

We kunnen dit resultaat met het axioma $AE$ omschrijven naar:

\begin{displaymath}
\vdash_{(K^{n})^{+}} \, (p \to E p) \to (p \to E E p)
\end{displaymath}

En gelukkig hebben we volledigheid van $(K^{n})^{+}$ voor de klasse van alle
epistemische frames, dus kunnen we ook zeggen:

\begin{displaymath}
\vDash \, (p \to E p) \to (p \to E E p)
\end{displaymath}

Dit is wat gevraagd werd te bewijzen.

\paragraph{Noot:}

Ik ben mij er van bewust dat deze uitwerking niet juist is.


\section*{Opgave 4}


\end{document}
