\chapter{Introduction}
\setcounter{page}{1}

TODO: write introduction

Infinitary term rewriting, the \Coq proof assistant, etc.

A list of notations can be found on page n.

From now on, we may drop the `infinitary' from infinitary rewriting
and explicitly say so if we mean finitary rewriting.

% Original goal was compression

%\Coq has been used for sizeable projects such as CompCert and a verified proof
%of the Four Color Theorem.

%Examples of formalisations of mathematical theories in \Coq:
%\begin{compactenum}
%\item Logic: A proof of G\"odel's First Incompleteness Theorem (Russel
%  O'Connor).
%\item Analysis: Exact real arithmetic (Russel O'Connor).
%\end{compactenum}

Although this text contains a fair amount of \Coq code, it is not our
intention to completely list a development ready for compiling. Rather, the
included code fragments are thought to be the most interesting ones for the
purpose of discussion of our development. In fact, many of the code listings
are simplified and/or typographically enhanced to a form beyond of what the
\Coq compiler will accept. Furthermore, lemmas are stated without proof. The
reader is invited to study the full source code, with proofs, which is
available at \url{http://martijn.vermaat.name/master-project/}.


\section*{Outline}

In Chapter~\ref{chap:rewriting} we introduce ordinal numbers and the
theory of infinitary term rewriting. This is mostly a recapitulation
of \citet{terese-03}, included for self-containment, and can be seen
as preliminaries for the later chapters.
% TODO: hancock things are not recapitulted from terese

The goal of Chapter~\ref{chap:implementation} is to present our
formalisation of infinitary term rewriting in the \Coq proof
assistant. We first introduce this proof assistant and then review the
main parts of our development.

Our formalisation was used to prove that in infinitary rewriting, weak
orthogonality does not imply uniqueness of normal forms. This
application is discussed in Chapter~\ref{chap:unwo}.

Finally, in Chapter~\ref{chap:discussion} we discuss our results and
draw conclusions.
