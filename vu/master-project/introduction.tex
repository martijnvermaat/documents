\chapter{Introduction}
\setcounter{page}{1}

Infinitary term rewriting, the \Coq proof assistant, etc.

\Coq has been used for sizeable projects such as CompCert and a verified proof
of the Four Color Theorem.

Examples of formalisations of mathematical theories in \Coq:
\begin{compactenum}
\item Logic: A proof of G\"odel's First Incompleteness Theorem (Russel
  O'Connor).
\item Analysis: Exact real arithmetic (Russel O'Connor).
\end{compactenum}

Although this text contains a fair amount of \Coq code, it is not our
intention to completely list a development ready for compiling. Rather, the
included code fragments are thought to be the most interesting ones for the
purpose of discussion of our development. In fact, many of the code listings
are simplified and/or typographically enhanced to a form beyond of what the
\Coq compiler will accept. Furthermore, lemmas are stated without proof. The
reader is invited to study the full source code, with proofs, which is
available at \url{http://martijn.vermaat.name/master-project/}.


\section*{Outline}

TODO: this is no longer correct (no finitary rewriting)

In Chapter~\ref{chap:rewriting} we introduce the theory of infinitary term
rewriting. To this end we first give a quick summary of traditional (finitary)
term rewriting and an introduction to the mathematical concept of ordinal
numbers.

The goal of Chapter~\ref{chap:implementation} is to introduce our formalization
of infinitary term rewriting in the \Coq proof assistant. We first discuss this
proof assistant and then review the main parts of our development.

Our formalization was used to prove that in infinitary rewriting, weak
orthogonality does not imply unique normal forms. This application is
discussed in Chapter~\ref{chap:unwo}.

Finally, in Chapter~\ref{chap:discussion} we discuss our results and draw
conclusions.

% TODO: state that all proofs can be found in the development
