\chapter{Introduction}
\setcounter{page}{1}

This thesis describes a formalisation of the theory of infinitary term
rewriting in the \Coq proof assistant. The foundation of \Coq is a
constructive type theory with inductive types, whereas infinitary term
rewriting, building on the theory of finitary rewriting, uses notions
from topology, set theory and analysis, but not necessarily in a
constructive way.

The central question we aim to answer in this thesis is whether the
traditional notions from infinitary term rewriting can be translated
to \Coq in such a way that the resulting definitions are natural for
the \Coq system. Of course, for such a translation to be satisfactory,
it should preserve the semantics of the original notions.

% functions must be total and terminating for coq to be consistent

In the remainder we may simply write `rewriting' instead of
`infinitary rewriting' and `term' instead of `infinite
term'.

Although this text contains a fair amount of \Coq code, it is not our
intention to completely list a development ready for compiling. Rather, the
included code fragments are thought to be the most interesting ones for the
purpose of discussion of our development. In fact, many of the code listings
are simplified and/or typographically enhanced to a form beyond of what the
\Coq compiler will accept. Furthermore, lemmas are stated without proof. The
reader is invited to study the full source code, with proofs, which is
available at \url{http://martijn.vermaat.name/master-project/}.

A glossary of notations can be found in Appendix~\ref{chap:glossary}.


\section*{Infinitary Rewriting}

% TODO: cite knots
% TODO: or better use braids?
The theory of \emph{(finitary) rewriting} is concerned with the
repeated transformation of objects by discrete steps following a
predefined set of rules. Such a set of rules can be understood as
implementing a programming language if we take programs as the objects
to be transformed. Indeed, \emph{term rewriting} is the foundational
model of functional programming. Other examples of rewriting can be
found in the transformation of braids and the $\lambda$-calculus
with applications as a model of computation or as logical framework.

% TODO:
% as a matter of fact, \lambda-calculus is of prime importance, both
% as a model of computation and as logical framework

\emph{Infinitary rewriting} generalises finitary rewriting by
considering infinitely large objects and series of transfinitely many
transformation steps. One could question the validity of this
generalisation, especially in the context of mechanical formalisation
with which this thesis is concerned. After all, the word `mechanical'
implies finite restrictions on the amounts of space and time we can
use.

% TODO:
% infinitely large objects (terms?)

However, mathematicians (and computer scientists for that matter) have
long had ample reason to include the infinite in their work. In
\emph{The Quadrature of the Parabola}, Archimedes considers the
infinite summation
\begin{align*}
  1 \,+\, \frac{1}{4} \,+\, \frac{1}{16} \,+\, \frac{1}{64} \,+\, \ldots
\end{align*}
in his proof that the area of a parabolic segment is \sfrac{4}{3} that
of a certain inscribed triangle. Of course we cannot carry out the
infinite computation to arrive at the outcome \sfrac{4}{3}, but we can
represent it in finite space and manipulate this representation in
finite time to deduce its outcome.

As another example to motivate the study of infinite objects, consider
the simple \Haskell program
\begin{singlespace}
\begin{coqdoccode}
\coqdocnoindent
\coqdocinductive{f} 0 \coqdocindent{0.2em} \textsf{where}
\coqdocindent{0.2em} \coqdocinductive{f} \coqdocvar{\textsf{n}} =
\coqdocvariable{\textsf{n}} : \coqdocinductive{f}
(\coqdocvar{\textsf{n}} + 1)\coqdoceol
\end{coqdoccode}
\end{singlespace}
that defines the infinite stream of natural numbers. We can inspect
the stream at any position, but by \Haskell's lazy evaluation the
stream is never fully computed. Again, the represented object takes an
infinite amount of space to store and an infinite amount of time to
compute, yet we can perfectly reason about it in finite space and
time.

The theory of infinitary term rewriting is formally introduced in
Chapter~\ref{chap:rewriting}. In that chapter, we define precisely
which infinite objects are allowed and what we understand by
transfinite sequences of rewrite steps.


\section*{Mechanical Formalisation}
\markright{\emph{Mechanical Formalisation}}

The translation of infinitary rewriting to \Coq is an example of the
mechanical formalisation of a mathematical theory. By `mechanical' we
mean that the formalisation must be in a form that can be manipulated
by a machine. Definitions on paper or ideas in one's head do not
qualify as mechanical formalisations. Furthermore, we want the
formalisation to represent the theory's semantics.

The promises of mechanical formalisations include:
\begin{compactdesc}
%  \item[\normalfont{\emph{Confidence}}\hskip .5em]
%  \item[{\makebox[6em][l]{\normalfont{\emph{Confidence}}}}]
  \item[\normalfont{\emph{Confidence}}\hskip .5em]
    Proof-checkers can verify the correctness of proofs in a rigorous
    way. If we trust the implementation of the checker (which is
    usually kept as small as possible) and its execution on a
    computer, we can be confident that a verified proof is correct,
    even if we do not fully comprehend it.
  \item[\normalfont{\emph{Automation}}\hskip .5em]
    Tedious work, whether computationally hard or just boring, is
    often better suited to computers. They are faster than humans and
    precise.
  \item[\normalfont{\emph{Intuition}}\hskip .5em]
    Proofs on paper typically abstract away from seemingly
    uninteresting  details. This is often a good thing, but sometimes
    the level of detail of a mechanical proof gives us that extra bit
    of insight to fully understand its workings. Another way of
    gaining intuition is by building tools on top of the
    formalisation, e.g.\ providing graphical representations of
    definitions.
  \item[\normalfont{\emph{Availability}}\hskip .5em]
    Formalised theories may be searched and browsed semantically using
    a computer instead of just syntactically. Furthermore, the
    internet provides us with excellent methods to share and copy
    these formalisations globally.
\end{compactdesc}

In 1976, the four colour theorem\footnote{`Four colours suffice to
  colour any map such that no two neighbouring countries have the same
  colour.' Posed by Francis Guthrie in 1852. } was proven
\citep{appel-haken-76}. The proof used a computer program to show that
a particular set of 1,936 maps satisfy a certain property by
exhaustive case analysis. This is an
example of automation by translation to a computer. The correctness of
the proof, however, remained debatable because this case analysis by
computer was impossible to perform or verify by hand.

Even though a simpler proof was published by \citet{robertson-97}, it
was not until \citet{gonthier-05} formally verified the entire proof,
including the computer program part, that all remaining doubts were
dispelled. This formalisation thus helped gain confidence in the
validity of the theorem.

Large-scale mechanical formalisation of mathematics goes back to the
late 1960's with de Bruijn's \Automath project
\citep{nederpelt-94}. A complete text book on analysis
\citep{landau-65} was formalised and verified for correctness, but the
system never gained widespread use.

Many formalisation efforts, using many different systems, have since
been undertaken. A list of formally verified proofs for 100
mathematical theorems is maintained by \citet{wiedijk-08}. Ongoing
work is the Flyspeck project \citep{hales-09} with as goal a formal
proof of the Kepler conjecture,\footnote{`The most efficient packing
  of oranges is in a pyramid.' Posed by Johannes Kepler in 1611.}
expected to take up to 20 work-years to complete.


\section*{Outline}

In Chapter~\ref{chap:rewriting} we introduce ordinal numbers and the
theory of infinitary term rewriting. This is mostly a recapitulation
of \citet{terese-03}, included for self-containment, and can be seen
as preliminaries for the later chapters.
% TODO: hancock things are not recapitulted from terese

The purpose of Chapter~\ref{chap:implementation} is to present our
formalisation of infinitary term rewriting in the \Coq proof
assistant. We start with a short introduction to \Coq and then review
the main parts of our development.

Our formalisation was used to prove that in infinitary rewriting, weak
orthogonality does not imply uniqueness of normal forms. This
application is discussed in Chapter~\ref{chap:unwo}.

Finally, in Chapter~\ref{chap:discussion} we discuss our results and
draw conclusions.
