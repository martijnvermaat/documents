% Options are: draft, color
\documentclass[color]{vu}

% Text encoding to take for data input stream (i.e. the text we write in)
\usepackage[latin1]{inputenc}

% Hyphenation patterns for UK English
% TODO: check if UKenglish actually works (otherwise use english)
%\usepackage[british]{babel}
\usepackage[english]{babel}

% Tweak the inner margin to align all text as nice as possible
%\usepackage[inner=3.83cm]{geometry}

% Abbreviations
\usepackage{xspace}

\newcommand{\name}[1]{\textsc{#1}\xspace}
\newcommand{\mathvar}[1]{\mathit{#1}}

%\newcommand{\equpto}[1]{\ensuremath{\overset{#1}{=}}}
\newcommand{\equpto}[1]{\ensuremath{\equiv_{\le \, #1}}}

%\def\bis{\ensuremath{\sim}}
%\newcommand{\sbis}{\text{\raisebox{2pt}{$\underline{\leftrightarrow}$}}}
%\newcommand{\bis}{\mathrel{\sbis}}
\def\bis{\;\raisebox{.3ex}{\underline{\makebox[.8em]{\ensuremath{\leftrightarrow}}}}\;}
\newcommand{\biss}[1]{\;\raisebox{.3ex}{\underline{\makebox[.7em]{\ensuremath{\leftrightarrow}}}}\,_{#1}\,}

\newcommand{\arity}[1]{\ensuremath{\sharp #1}}

%\newcommand{\prefix}[2]{\ensuremath{#1 \! \cdot \! #2}}
\newcommand{\prefix}[2]{\ensuremath{#1 #2}}

\newcommand{\subterm}[2]{\ensuremath{#1 |_{#2}}}

\newcommand{\concat}{\mathrel{\ensuremath{+\!\!+}}}

% TODO: nice symboll for \du?
%\def\du{\raisebox{1pt}{\ensuremath{\Delta}}\hspace{-7.5pt}\raisebox{-0.5pt}{\ensuremath{\Upsilon}}}
%\def\du{\text{\begin{arabtext}d\end{arabtext}}}
%\def\du{\ensuremath{\delta}}
\def\du{\ensuremath{\upsilon}}

% TODO: check out mathrsfs package with \mathscr font
% TODO: maybe \mathcal{O} or \mathcal{TO} for \Ord
% TODO: maybe 3-headed arrow for \rewrites
\def\rewrites{\ensuremath{\twoheadrightarrow}}
\def\nothing{\ensuremath{\varnothing}}
%\def\Ord{\ensuremath{\Omega}}
\def\Ord{\mathcal{T\!O}}
\def\dom{\mathvar{dom}}
\def\X{\ensuremath{\mathcal{X}}}
\def\Var{\mathvar{Var}}
\def\Ter{\mathvar{Ter}_\Sigma}
\def\TerI{\mathvar{Ter}_\Sigma^\infty}
\def\Ctx{\mathvar{Ctx}_\Sigma^\infty}
\def\Pos{\mathcal{P}\!\mathvar{os}}

\def\Coq{\name{Coq}}
\def\CoLoR{\name{CoLoR}}
\def\Coccinelle{\name{Coccinelle}}
\def\CiME{\name{C\textit{i}ME}}
\def\Haskell{\name{Haskell}}
\def\Automath{\name{Automath}}
\def\Isabelle{\name{Isabelle}}
\def\IsabelleHOL{\name{Isabelle}/HOL}
\def\Flyspeck{\name{Flyspeck}}


% Citation aliases
\defcitealias{coq-refman-09}{The \Coq Reference Manual}

% TODO: maybe in vu.cls
\usepackage{tikz}
\usetikzlibrary{arrows,positioning,chains,trees}

% Coq code listings
\usepackage{coqdoc}
\setlength{\coqdocbaseindent}{0.7em}

%\usepackage{hyperref}
\hypersetup{
  final,
  colorlinks=true,
  citecolor=black,
  filecolor=black,
  linkcolor=black,
  urlcolor=black,
  anchorcolor=black,
  pdfauthor={Martijn Vermaat},
  pdftitle={Infinitary Rewriting in Coq},
  pdfsubject={A Mechanic Formalisation of Infinitary Rewriting in the Coq
    Proof Assistant},
  pdfkeywords={tree ordinals, brouwer ordinals, infinitary rewriting,
    formalisation, verification, coq, functional programming}}
%\usepackage[figure]{hypcap}

\title{Infinitary Rewriting in \Coq}

\author{Martijn Vermaat}
\date{August 2010}
%\date{MSc. Thesis (\emph{draft r\svnrev})}
% TODO: good title page
% Thesis submitted for the degree of Master of Science by research at the VU
% University Amsterdam
% by Martijn Vermaat
% August 2010

\begin{document}

\makefrontmatter
\maketitle

\begin{abstract}
  % 250 words
  TODO: this is a temporary abstract

  We formalize the basic notions from the theory of infinitary
  rewriting in the \Coq proof assistant. Infinite terms are defined by
  coinduction and rewrite sequences of ordinal length are based on the
  inductive structure of the tree ordinals. We discuss an
  application of the formalization in a verification of the
  counterexample to unique normal forms in weakly orthogonal iTRSs,
  introduced by \citet{endrullis-10}.
\end{abstract}

\chapter*{Preface}
\thispagestyle{empty}

This thesis is part of my Master Project -- a half-year project that should be
the culmination of one's Master study.


\section*{Acknowledgements}

Joint work with Dimitri Hendriks.

Adam Chlipala for the recursive vector type and other hints on
Coq-club. Bruno Barras for a hint on Ordinal proofs. Matthieu Sozeau for help
with the dependent destruction tactic. Peter Hancock for his notes on Brouwer
ordinals. J\"org Endrullis and Vincent van Oostrom for ideas on pred and
embeddings. Roel de Vrijer. Femke van Raamsdonk.


\tableofcontents

\chapter{Introduction}
\setcounter{page}{1}

This thesis describes a formalisation of the theory of infinitary term
rewriting in the \Coq proof assistant. The foundation of \Coq is a
constructive type theory with inductive types, whereas infinitary term
rewriting, building on the theory of finitary rewriting, uses notions
from topology, set theory and analysis, but not necessarily in a
constructive way.

The central question we aim to answer in this thesis is whether the
traditional notions from infinitary term rewriting can be translated
to \Coq in such a way that the resulting definitions are natural for
the \Coq system. Of course, for such a translation to be satisfactory,
it should preserve the semantics of the original notions.

% functions must be total and terminating for coq to be consistent

In the remainder we may simply write `rewriting' instead of
`infinitary rewriting' and `term' instead of `infinite
term'.

Although this text contains a fair amount of \Coq code, it is not our
intention to completely list a development ready for compiling. Rather, the
included code fragments are thought to be the most interesting ones for the
purpose of discussion of our development. In fact, many of the code listings
are simplified and/or typographically enhanced to a form beyond of what the
\Coq compiler will accept. Furthermore, lemmas are stated without proof. The
reader is invited to study the full source code, with proofs, which is
available at \url{http://martijn.vermaat.name/master-project/}.

A glossary of notations can be found in Appendix~\ref{chap:glossary}.


\section*{Infinitary Rewriting}

The theory of \emph{(finitary) rewriting} is concerned with the
repeated transformation of objects by discrete steps following a
predefined set of rules. Such a set of rules can be understood as
implementing a programming language if we take programs as the objects
to be transformed. Indeed, \emph{term rewriting} is the foundational
model of functional programming. Other examples of rewriting can be
found in the transformation of knots (cite) and the $\lambda$-calculus
with applications as a model of computation or as logical framework.

\emph{Infinitary rewriting} generalizes finitary rewriting by
considering infinitely large objects and series of transfinitely many
transformation steps. One could question the validity of this
generalization, especially in the context of mechanical formalisation
with which this thesis is concerned. After all, the word `mechanical'
implies finite restrictions on the amounts of space and time we can
use.

However, mathematicians (and computer scientists for that matter) have
long had ample reason to include the infinite in their work. In
\emph{The Quadrature of the Parabola}, Archimedes considers the
infinite summation
\begin{align*}
  1 \,+\, \frac{1}{4} \,+\, \frac{1}{16} \,+\, \frac{1}{64} \,+\, \ldots
\end{align*}
in his proof that the area of a parabolic segment is \sfrac{4}{3} that
of a certain inscribed triangle. Of course we cannot carry out the
infinite computation to arrive at the outcome \sfrac{4}{3}, but we can
represent it in finite space and manipulate this representation in
finite time to deduce its outcome.

As another example to motivate the study of infinite objects, consider
the simple \Haskell program
\begin{singlespace}
\begin{coqdoccode}
\coqdocnoindent
\coqdocinductive{f} 0 \coqdocindent{0.2em} \textsf{where}
\coqdocindent{0.2em} \coqdocinductive{f} \coqdocvar{\textsf{n}} =
\coqdocvariable{\textsf{n}} : \coqdocinductive{f}
(\coqdocvar{\textsf{n}} + 1)\coqdoceol
\end{coqdoccode}
\end{singlespace}
that defines the infinite stream of natural numbers. We can inspect
the stream at any position, but by \Haskell's lazy evaluation the
stream is never fully computed. Again, the represented object takes an
infinite amount of space to store and an infinite amount of time to
compute, yet we can perfectly reason about it in finite space and
time.

The theory of infinitary term rewriting is formally introduced in
Chapter~\ref{chap:rewriting}. In that chapter, we define precisely
what infinite objects are allowed and what we understand by
transfinite sequences of rewrite steps.


\section*{Mechanical Formalisation}
\markright{\emph{Mechanical Formalisation}}

The translation of infinitary rewriting to \Coq is an example of the
mechanical formalisation of a mathematical theory. By `mechanical' we
mean that the formalisation must be in a form that can be manipulated
by a machine. Definitions on paper or ideas in one's head do not
qualify as mechanical formalisations. Furthermore, we want the
formalisation to represent the theory's semantics.

% TODO: indent these premises by 2em or so
The premises of such formalisations include:
\begin{compactdesc}
%  \item[\normalfont{\emph{Confidence}}\hskip .5em]
%  \item[{\makebox[6em][l]{\normalfont{\emph{Confidence}}}}]
  \item[\normalfont{\emph{Confidence}}\hskip .5em]
    Proof-checkers can verify the correctness of proofs in a rigorous
    way. If we trust the implementation of the checker (which is
    usually kept as small as possible) and its execution on a
    computer, we can be confident that a verified proof is correct,
    even if we do not fully comprehend it.
  \item[\normalfont{\emph{Automation}}\hskip .5em]
    Tedious work, whether computationally hard or just boring, is
    often better suited to computers. They are faster than humans and
    precise.
  \item[\normalfont{\emph{Intuition}}\hskip .5em]
    Proofs on paper typically abstract away from seemingly
    uninteresting  details. This is often a good thing, but sometimes
    the level of detail of a mechanical proof gives us that extra bit
    of insight to fully understand its workings. Another way of
    gaining intuition is by building tools on top of the
    formalisation, e.g.\ providing graphical representations of
    definitions.
  \item[\normalfont{\emph{Availability}}\hskip .5em]
    Formalised theories may be searched and browsed semantically using
    a computer instead of just syntactically. Furthermore, the
    internet provides us with excellent methods to share and copy
    these formalisations globally.
\end{compactdesc}

In 1976, the four colour theorem\footnote{`Four colours suffice to
  colour any map such that no two neighbouring countries have the same
  colour.' Posed by Francis Guthrie in 1852. } was proven
\citep{appel-haken-76}. The proof used a computer program to show that
a particular set of 1,936 maps satisfy a certain property. This is an
example of automation by translation to a computer. The correctness of
the proof, however, remained debatable because this case analysis by
computer was impossible to perform or verify by hand.

Even though a simpler proof was published by \citet{robertson-97}, it
was not until \citet{gonthier-05} formally verified the entire proof,
including the computer program part, that all remaining doubts were
dispelled. This formalisation thus helped gain confidence in the
validity of the theorem.

Large scale mechanical formalisation of mathematics goes back to the
late 1960's with de Bruijn's \Automath project
\citep{nederpelt-94}. A complete text book on analysis
\citep{landau-65} was formalised and verified for correctness, but the
system never gained widespread use.

Many formalisation efforts, using many different systems, have since
been undertaken. An list of formally verified proofs for 100
mathematical theorems is maintained by \citet{wiedijk-08}. Ongoing
work is the Flyspeck project \citep{hales-09} with as goal a formal
proof of the Kepler conjecture,\footnote{`The most efficient packing
  of oranges is in a pyramid.' Posed by Johannes Kepler in 1611.}
expected to take up to 20 work-years to complete.


\section*{Outline}

In Chapter~\ref{chap:rewriting} we introduce ordinal numbers and the
theory of infinitary term rewriting. This is mostly a recapitulation
of \citet{terese-03}, included for self-containment, and can be seen
as preliminaries for the later chapters.
% TODO: hancock things are not recapitulted from terese

The purpose of Chapter~\ref{chap:implementation} is to present our
formalisation of infinitary term rewriting in the \Coq proof
assistant. We first introduce this proof assistant and then review the
main parts of our development.

Our formalisation was used to prove that in infinitary rewriting, weak
orthogonality does not imply uniqueness of normal forms. This
application is discussed in Chapter~\ref{chap:unwo}.

Finally, in Chapter~\ref{chap:discussion} we discuss our results and
draw conclusions.

\chapter{Infinitary Term Rewriting}\label{chap:rewriting}

Before formally introducing term rewriting, we give an example of an
infinitary TRS taken from \citet{klop-de-vrijer-05} and trust the
reader to pick up the general ideas.

Let $A$, $B$ and $P$ be symbols with arities $0$, $1$ and $2$,
respectively. We consider the rule $A \rightarrow B(A)$. The term $A$
rewrites to $B(B(B(A)))$, written $B^3(A)$, in $3$ steps, or more
generally to $B^n(A)$ in $n$ steps for every $n \in \mathbb{N}$.
%\begin{align*}
%  A \rightarrow B(A) \rightarrow B(B(A)) \rightarrow \cdots \mbox{($n$
%    steps)} \cdots \rightarrow B^n(A)
%\end{align*}
\begin{center}
{\footnotesize\begin{tikzpicture}[node distance=50pt]
\tikzstyle{level}=[level distance=20pt,sibling distance=22pt]
\node (a) {$A$};
\node (b) [right of=a] {$B$} child { node {$A$} };
\node (c) [right of=b] {$B$} child { node {$B$} child { node {$A$} } };
\node (d) [right of=c] {$B$} child { node {$B$} child { node {$B$}
    child { node {$A$} } } };
\path (a) -- (b) node[midway,below=-1pt] {$\rightarrow$};
\path (b) -- (c) node[midway,below=-1pt] {$\rightarrow$};
\path (c) -- (d) node[midway,below=-1pt] {$\rightarrow$};
\end{tikzpicture}}
\end{center}\vspace{-0.8\baselineskip}
In the finitary setting, these are the only rewrite sequences from $A$
and all of them allow an additional step. Hence, $A$ has no normal
form.

The limit of these rewrite sequences, although non-existent in
finitary rewriting, is intuitively well-defined. It is reached in
$\omega$ many rewrite steps and we denote it by $B^\omega$. Infinitary
rewriting allows \begin{inparaenum}[(i)]
  \item infinite terms such as $B^\omega$ and
  \item rewrite sequences of transfinite length such as $A
    \rightarrow^\omega B^\omega$.
\end{inparaenum}\nopagebreak[3]
\begin{center}
{\footnotesize\begin{tikzpicture}[node distance=50pt]
\tikzstyle{level}=[level distance=20pt,sibling distance=22pt]
\node (a) {$A$};
\node (b) [right of=a] {$B$} child { node {$A$} };
\node (c) [right of=b] {$B$} child { node {$B$} child { node {$A$} } };
\node (d) [right of=c,node distance=80pt] {$B$} child { node {$B$} child { node {$B$}
    child { node[below=-6pt] {\scriptsize$\vdots$} } } };
\path (a) -- (b) node[midway,below=-1pt] {$\rightarrow$};
\path (b) -- (c) node[midway,below=-1pt] {$\rightarrow$};
\path (c) -- (d) node[midway,below=-1pt] {$\rightarrow \quad \cdots$};
\end{tikzpicture}}
\end{center}\vspace{-0.8\baselineskip}
The word `transfinite' hints that there are also rewrite sequences of
length $> \omega$. For consider the term $P(A, A)$. It rewrites in
$\omega$ many steps to $P(B^\omega, A)$, which in turn rewrites in
$\omega$ many steps to $P(B^\omega, B^\omega)$. Therefore we have
$P(A, A) \rightarrow^{\omega \times 2} P(B^\omega, B^\omega)$.
\begin{center}
{\footnotesize\begin{tikzpicture}[node distance=41pt]
\tikzstyle{level}=[level distance=17pt,sibling distance=16pt]
\node (a) {$P$} child { node {$A$} } child { node {$A$} };
\node (b) [right of=a] {$P$} child { node {$B$} child { node {$A$} } } child { node {$A$} };
\node (c) [right of=b] {$P$} child { node {$B$} child { node {$B$}
    child { node {$A$} } } } child { node {$A$} };
\node (d) [right of=c,node distance=60pt] {$P$} child { node {$B$} child { node {$B$}
    child { node {$B$} child { node[below=-6pt] {\scriptsize$\vdots$} } } } } child { node {$A$} };
\node (e) [right of=d] {$P$} child { node {$B$} child { node {$B$}
    child { node {$B$} child { node[below=-6pt] {\scriptsize$\vdots$} } } } }
child { node {$B$} child { node {$A$} } };
\node (f) [right of=e] {$P$} child { node {$B$} child { node {$B$}
    child { node {$B$} child { node[below=-6pt] {\scriptsize$\vdots$} } } } }
child { node {$B$} child { node {$B$} child { node {$A$} } } };
\node (g) [right of=f,node distance=60pt] {$P$} child { node {$B$} child { node {$B$}
    child { node {$B$} child { node[below=-6pt] {\scriptsize$\vdots$} } } } }
child { node {$B$} child { node {$B$} child { node {$B$} child {
        node[below=-6pt] {\scriptsize$\vdots$} } } } };
\path (a) -- (b) node[midway,below=-1pt] {$\rightarrow$};
\path (b) -- (c) node[midway,below=-1pt] {$\rightarrow$};
\path (c) -- (d) node[midway,below=-1pt] {$\rightarrow \, \: \cdots$};
\path (d) -- (e) node[midway,below=-1pt] {$\rightarrow$};
\path (e) -- (f) node[midway,below=-1pt] {$\rightarrow$};
\path (f) -- (g) node[midway,below=-1pt] {$\rightarrow \, \: \cdots$};
\end{tikzpicture}}
\end{center}\vspace{-0.8\baselineskip}
However, $P(B^\omega, B^\omega)$ can also be reached from $P(A, A)$ in
$\omega$ many steps, alternating the steps from the two $\omega$-step
rewrite sequences.
\begin{center}
{\footnotesize\begin{tikzpicture}[node distance=50pt]
\tikzstyle{level}=[level distance=20pt,sibling distance=22pt]
\node (a) {$P$} child { node {$A$} } child { node {$A$} };
\node (b) [right of=a] {$P$} child { node {$B$} child { node {$A$} } } child { node {$A$} };
\node (c) [right of=b] {$P$} child { node {$B$} child { node {$A$} } }
child { node {$B$} child { node {$A$} } };
\node (d) [right of=c] {$P$} child { node {$B$} child { node {$B$}
    child { node {$A$} } } }
child { node {$B$} child { node {$A$} } };
\node (e) [right of=d] {$P$} child { node {$B$} child { node {$B$}
    child { node {$A$} } } }
child { node {$B$} child { node {$B$} child { node {$A$} } } };
\node (f) [right of=e,node distance=80pt] {$P$} child { node {$B$} child { node {$B$}
    child { node {$B$} child { node[below=-6pt] {\scriptsize$\vdots$} } } } }
child { node {$B$} child { node {$B$} child { node {$B$} child {
        node[below=-6pt] {\scriptsize$\vdots$} } } } };
\path (a) -- (b) node[midway,below=-1pt] {$\rightarrow$};
\path (b) -- (c) node[midway,below=-1pt] {$\rightarrow$};
\path (c) -- (d) node[midway,below=-1pt] {$\rightarrow$};
\path (d) -- (e) node[midway,below=-1pt] {$\rightarrow$};
\path (e) -- (f) node[midway,below=-1pt] {$\rightarrow \quad \cdots$};
\end{tikzpicture}}
\end{center}\vspace{-0.8\baselineskip}
This observation is generalised in the Compression Lemma
(page~\pageref{lem:compression}).

In the following sections, we introduce the ordinal numbers and a
representation for them known as tree ordinals, and we present the
basic notions from the theory of term rewriting as required in the
following chapters. None of this is original material.

%This chapter does not contain original material. The construction of the
%order relation on tree ordinals in Section~\ref{sub:tree} is by
%\citet{hancock-08}.

% finitary rewriting: TRS as a programming language, or in theorem
% provers

% nonterminating processes, stream-based programming languages

% TODO: Short motivation for term rewriting, summation of its applications and
% aspects of rewriting that are studied.

%It should be noted that an infinite term is to be understood as a term
%with a \emph{possibly} infinite depth, that is, the class of infinite
%terms includes the finite terms.

%Orthogonal TRSs have some nice properties, for example UN$^{\infty}$ and
%compression.

% TODO: note that we are explicit in the wording: sequence vs rewrite sequence


\section{Ordinal Numbers}\label{sec:ordinals}

Ordinal numbers \citep{cantor-15}, or ordinals for short, are an
extension of the natural numbers with transfinite objects. Indeed, the
finite ordinals are just the natural numbers. The smallest infinite
ordinal is called $\omega$ and following $\omega$ we have $\omega +
1$, $\omega + 2$, \ldots, $\omega \times 2$. Then there are the
ordinals $\omega \times 2 + 1$, $\omega \times 2 + 2$, \ldots, $\omega
\times 3$. Some other (still relatively small) ordinals are:
\begin{displaymath}
  \omega^2 \qquad
  \omega^\omega \qquad
  \omega^{\omega^2} \qquad
  \omega^{\omega^\omega} \qquad
  \omega^{\omega^{\omega^{\iddots}}} = \epsilon_0
\end{displaymath}
Note that this is all merely notation, we have not yet defined a
representation for ordinals or what $+$ and $\times$ are.
% TODO: and exponentiation

In set theory, ordinals are usually represented by hereditarily transitive
sets. Zero corresponds to the empty set $\nothing$, one to the
singleton $\{ \nothing \}$ and so on, and $\omega$ is represented by
$\{ \nothing, \{ \nothing \}, \{ \nothing, \{ \nothing \} \} , \ldots
\}$. Now $\in$ constitutes a well-founded total order on the
ordinals.

We abbreviate $\alpha \cup \{ \alpha \}$ by $\alpha^+$ and say that
$\alpha$ is a \emph{successor ordinal} if $\alpha = \beta^+$ for some
ordinal $\beta$. If $\alpha$ is not a successor ordinal and $\alpha
\neq \nothing$, it is called a \emph{limit ordinal}. Hence, an ordinal
can be either zero, a successor ordinal, or a limit ordinal.

From now on, we make no distinction between an ordinal and its set-theoretic
representation (e.g.\ between $0$ and $\nothing$). Examples of successor
ordinals are $4$, $\omega + 7$  and $\omega^{\omega \times 2} + 1$. Examples
of limit ordinals are $\omega$ and $\omega \times 3$. Henceforth we
use $\alpha, \beta, \gamma, \lambda$ to denote ordinals where
$\lambda$ always denotes a limit ordinal.

% TODO: we also make no distinction between the ordinal and its tree
% ordinal

One can do arithmetics on ordinals much like we do arithmetics on natural
numbers. For example, addition can be defined by recursion on the right
argument:
\begin{align*}
  \alpha + 0       &= \alpha\\
  \alpha + \beta^+ &= (\alpha + \beta)^+\\
  \alpha + \lambda &= \bigcup \{ \alpha + \gamma \; | \; \gamma \in \lambda \}
\end{align*}


\subsection{Tree Ordinals}\label{sub:tree}

% TODO: Possible reference for Brouwer ordinals:
% Constructivism in Mathematics: An Introduction, A.S. Troelstra en Dirk
% van Dalen, North-Holland Publishing, Amsterdam, 1988.

% Other possible reference:
% Combinators, Lambda-Terms and Proof Theory, S. Stenlund, 1972

% Reference for tree ordinals:
% Subrecursive hierarchies via direct limits, E.C. Dennis-Jones and
% S.S. Wainer
%
% See also Hydra Games and Tree Ordinals (Ariya Isihara)

The tree ordinals \citep{dennis-jones-wainer-84} are a representation
of the countable ordinals as countably branching well-founded
trees. Their inductive definition uses constructors $0$ (zero), $^+$
(successor) and $\sqcup$ (limit).

% Be alerted that all ordinals cannot form a set (only a class), but we are
% defining a subset here
\begin{definition}\label{def:ordinals}%[Ordinals]
The set of \emph{tree ordinals} $\Ord$ is defined by induction:
\begin{compactenum}
  \item
    $0 \in \Ord$.
  \item
    If $\alpha \in \Ord$, then $\alpha^+ \in \Ord$.
  \item
    If $\alpha_i \in \Ord$ for all $i \in \mathbb{N}$, then $\sqcup_i
    \alpha_i \in \Ord$.
\end{compactenum}
\end{definition}
The $\sqcup$ constructor has type $(\mathbb{N} \rightarrow \Ord) \rightarrow
\Ord$, but for our convenience we write $\sqcup_i \cdots i \cdots$ instead
of $\sqcup (\lambda i . \cdots i \cdots)$. Sometimes we explicitly enumerate
the argument, writing for example $\sqcup \{ \alpha_1, \alpha_2,
\alpha_3, \ldots \}$.

% TODO: make this a nice picture
\begin{figure}
\begin{center}
\begin{tikzpicture}
\tikzstyle{level 1}=[level distance=0cm, sibling distance=3cm]
\tikzstyle{level 2}=[level distance=0.8cm, sibling distance=3cm]
\tikzstyle{level 3}=[level distance=0.8cm, sibling distance=3cm]
\tikzstyle{level 4}=[level distance=1.5cm, sibling distance=3cm]
\tikzstyle{level 5}=[level distance=0.8cm, sibling distance=0.7cm]
\tikzstyle{level 6}=[level distance=0.8cm, sibling distance=0.7cm]
\tikzstyle{level 7}=[level distance=0.8cm, sibling distance=0.7cm]
\tikzstyle{level 8}=[level distance=0.8cm, sibling distance=0.7cm]

\coordinate
child {
  [fill] circle (1.5pt) edge from parent
child {
  [fill] circle (1.5pt) edge from parent
child {
 edge from parent
child {
  child {
    circle (1.5pt) edge from parent
  }
  child {
    [fill] circle (1.5pt)
    child {
      circle (1.5pt) edge from parent
    }
    edge from parent
  }
  child {
    [fill] circle (1.5pt)
    child {
      [fill] circle (1.5pt)
      child {
        circle (1.5pt) edge from parent
      }
      edge from parent
    }
    edge from parent
  }
  child {
    [fill] circle (1.5pt)
    child {
      [fill] circle (1.5pt)
       child {
        [fill] circle (1.5pt)
         child {
           circle (1.5pt) edge from parent
        }
        edge from parent
      }
      edge from parent
    }
    edge from parent node[at end] {$\qquad \quad \ldots$}
  }
}
child {
  [fill] circle (1.5pt)
  child {
    child {
      circle (1.5pt)
      edge from parent
    }
    child {
      [fill] circle (1.5pt)
      child {
        circle (1.5pt) edge from parent
      }
      edge from parent
    }
    child {
      [fill] circle (1.5pt)
      child {
        [fill] circle (1.5pt)
        child {
          circle (1.5pt) edge from parent
        }
        edge from parent
      }
      edge from parent
    }
    child {
      [fill] circle (1.5pt)
      child {
        [fill] circle (1.5pt)
        child {
          [fill] circle (1.5pt)
          child {
            circle (1.5pt) edge from parent
          }
          edge from parent
        }
        edge from parent
      }
      edge from parent node[at end] {$\qquad \quad \ldots$}
    }
    edge from parent
  }
}
child {
  [fill] circle (1.5pt)
  child {
    [fill] circle (1.5pt)
    child {
      child {
        circle (1.5pt)
        edge from parent
      }
      child {
        [fill] circle (1.5pt)
        child {
          circle (1.5pt) edge from parent
        }
        edge from parent
      }
      child {
        [fill] circle (1.5pt)
        child {
          [fill] circle (1.5pt)
          child {
            circle (1.5pt) edge from parent
          }
          edge from parent
        }
        edge from parent
      }
      child {
        [fill] circle (1.5pt)
        child {
          [fill] circle (1.5pt)
          child {
            [fill] circle (1.5pt)
            child {
              circle (1.5pt) edge from parent
            }
            edge from parent
          }
          edge from parent
        }
        edge from parent node[at end] {$\qquad \quad \ldots$}
      }
      edge from parent
    }
  }
}
child {
  [fill] circle (1.5pt)
  child {
    [fill] circle (1.5pt)
    child {
      [fill] circle (1.5pt)
      child {
        child {
          circle (1.5pt)
          edge from parent
        }
        child {
          [fill] circle (1.5pt)
          child {
            circle (1.5pt) edge from parent
          }
          edge from parent
        }
        child {
          [fill] circle (1.5pt)
          child {
            [fill] circle (1.5pt)
            child {
              circle (1.5pt) edge from parent
            }
            edge from parent
          }
          edge from parent
        }
        child {
          [fill] circle (1.5pt)
          child {
            [fill] circle (1.5pt)
            child {
              [fill] circle (1.5pt)
              child {
                circle (1.5pt) edge from parent
              }
              edge from parent
            }
            edge from parent
          }
          edge from parent node[at end] {$\qquad \quad \ldots$}
        }
        edge from parent
      }
    }
  }
  edge from parent node[at end] {$\qquad \quad \quad \ldots$}
}
}
}
};

\end{tikzpicture}
\end{center}
\caption{A representation of $\omega \times 2 + 2$.}\label{fig:tree}
\end{figure}

Now zero is represented by $0$, a successor ordinal $\alpha +1$ is represented
by $\alpha^+$ and a limit ordinal $\lambda$ is represented by $\sqcup_i
\alpha_i$ if $\lambda$ is the least upper bound of the sequence $\alpha_1,
\alpha_2, \alpha_3, \ldots$. As an example of a tree ordinal representation,
Figure~\ref{fig:tree} visualises $\omega \times 2 + 2$ as a tree
ordinal. Again, we identify ordinals and their representation as tree
ordinal.

% TODO: some ordinals have no representation as tree ordinal at all
Some ordinals have no unique representation as tree ordinal. Consider for
example the limit ordinals $\sqcup_i i + 3$ and $\sqcup_i i \times 2$. Both
are representations of $\omega$ and a meaningful order relation would
have to position them at the same rank.

A more intricate issue is what to make of ordinals such as $\sqcup \{
3, 3, 3, \ldots \}$. In spirit of the intuition given above it
represents $3$, that being the non-strict least upper bound of $3, 3,
3, \ldots$.\footnote{Or, if we take the \emph{strict} least upper bound, $\sqcup \{
3, 3, 3, \ldots \}$ represents $4$.}
% TODO: this makes it undecidable to compare an ordinal to 0 or 3
We might like to exclude such representations and require that
$\sqcup_i \alpha_i$ always represents a limit ordinal. This can be
done by imposing a strict monotonicity property on the limit
sequences. Some order relation on tree ordinals is needed for that.

The following construction (i.e.\ Definitions~\ref{def:indices}
through \ref{def:order}) is due to \citet{hancock-08}. In preparation
for an extensional order relation on $\Ord$, we define a structural
strict order relation.
% TODO: , is too close on $\Omega$

\begin{definition}\label{def:indices}%[Predecessor indices]
The set-valued function $\Phi$ defines the \emph{predecessor indices}
$\Phi(\alpha)$ \emph{for} $\alpha$ by recursion on $\alpha$:
\begin{align*}
  \Phi(0)                 &= \nothing \\
  \Phi(\alpha^+)          &= \Phi(\alpha)^? \\
  \Phi(\sqcup_i \alpha_i) &= (\Sigma n \in \mathbb{N}) \; \Phi(\alpha_n)
\end{align*}
\end{definition}

By $A^?$ we mean the option type over $A$, or equivalently the
disjoint sum $1 + A$ of the unit type $1$ and $A$. We use
\textsc{left} and \textsc{right $a$} (for $a \in A$) as constructors
of $A^?$. Note that the set $\Phi(0)$ of predecessor indices for $0$
has no inhabitants.
% TODO: just use disjoint sum, no option type?

The predecessor indices of an ordinal $\alpha$ are essentially the
paths on its tree structure starting from the root that cross at least
one $^+$ constructor.

\begin{definition}%[Predecessor]
The function $\_[\_] : (\prod \alpha : \Ord) \; \Phi(\alpha)
\rightarrow \Ord$ defines the \emph{predecessor} $\alpha[\iota]$
\emph{of} $\alpha$ \emph{indexed by} $\iota$ recursively on $\alpha$:
\begin{align*}
  \alpha^+[\textsc{left}]                     &= \alpha \\
  \alpha^+[\textsc{right $\iota$}]            &= \alpha[\iota] \\
  \sqcup_i \alpha_i[\langle n, \iota \rangle] &= \alpha_n[\iota]
\end{align*}
\end{definition}

% TODO: explain what I and _[_] mean

This structural predecessor function can be seen as defining a `subtree'
partial order on $\Ord$. With it we are ready to define an extensional
non-strict order relation on $\Ord$ that classifies ordinals by rank.

% TODO: note <= infix notation
% TODO: or use the set-theoretic definitions from hancock?
% TODO: infix notations are used all the time, maybe state this at front
% TODO: does \sqcup_i start at 0 or 1? we have pred_type saying all of
% the natural numbers, but this definition starting at \alpha_1...
\begin{definition}\label{def:order}%[Order]
We define the \emph{order} $\preceq$ as a binary relation on $\Ord$ by
induction:
% TODO: maybe make a not of infix notation in introduction
% (and write $\alpha \preceq \beta$ for $\langle \alpha, \beta \rangle
% \in \; \preceq$)
\begin{compactenum}
  \item
    $0 \preceq \beta$ for every ordinal $\beta \in \Ord$.
  \item\label{def:order:succ}
    For all $\alpha, \beta \in \Ord$ and $\iota \in \Phi(\beta)$, if
    $\alpha \preceq \beta[\iota]$ then $\alpha^+ \preceq \beta$.
  \item
    For all $\alpha_1, \alpha_2, \alpha_3, \ldots, \beta \in \Ord$, if
    $\alpha_n \preceq \beta$ for all $n \in \mathbb{N}$, then $\sqcup_i
    \alpha_i \preceq \beta$.
\end{compactenum}
\end{definition}

Using this order, we can introduce two other useful binary relations
on $\Ord$. First, the extensional \emph{equality} $\alpha \simeq
\beta$ is defined by the conjunction of $\alpha \preceq \beta$ and
$\beta \preceq \alpha$. Second, the extensional \emph{strict order}
$\alpha \prec \beta$ holds if $\alpha \preceq \beta[\iota]$ holds for
some $\iota \in \Phi(\beta)$.

%\begin{proposition}\label{prop:ord}
%$\alpha \le \beta \Leftrightarrow \alpha \preceq \beta$ for all finite ordinals $\alpha$ and $\beta$.
%\end{proposition}
%\begin{proof}
%By induction on $\beta$.
%\end{proof}


\section{Term Rewriting}\label{sec:rewriting}

We give a short introduction to the basic notions of infinitary term
rewriting as required in the following chapters. For a more in-depth
treatment of the theory of term rewriting, consult
\citet{terese-03}. Discussion of infinitary rewriting specifically,
can be found in \citet[Chapter 12]{terese-03} and
\citet{klop-de-vrijer-05}. In this section, we try to conform to
definitions and notations from \citet{terese-03}, but sometimes choose
to follow more closely our \Coq formalisation discussed in
Chapter~\ref{chap:implementation} when the two diverge.
% TODO: wording is not so nice
% TODO: hyphenation of de vrijer

\subsection{Definition of a TRS}\label{sub:trs}

\begin{definition}%[Signature]
A \emph{signature} $\Sigma$ is a non-empty set of \emph{function symbols} $f,
g, \ldots$. Each function symbol $f$ has a fixed natural number
$\arity{f}$, which we call its \emph{arity}. A function symbol with
arity $0$ is also called a \emph{constant}.
\end{definition}

\begin{definition}%[Term]
The set of \emph{terms} $\TerI(X)$ over a signature $\Sigma$ and a
set of variables $\X = \{x, y, \ldots\}$ is defined by coinduction:
\begin{compactenum}
  \item
    $x \in \TerI(\X)$ for every variable $x \in \X$.
  \item
    For every $f \in \Sigma$, if $t_1, \ldots, t_{\arity{f}} \in
    \TerI(\X)$, then $f(t_1, \ldots, t_{\arity{f}}) \in \TerI(\X)$.
%  \item
%    If $f$ is a function symbol with arity $n$ and $t_1, \ldots, t_n \in
%    \TerI(\X)$, then $f(t_1, \ldots, t_n) \in \TerI(\X)$.
\end{compactenum}
\end{definition}
% TODO: maybe use Ter(\Sigma, X) instead of Ter_\Sigma(X)

The symbol $f$ is called the \emph{root} of $f(t_1, \ldots, t_n)$ and
the terms $t_i$ are called the \emph{arguments} of $f$. By $\Var(t)$
we denote the set of variables occurring in $t$, and $t$ is
\emph{closed} if $\Var(t) = \nothing$. If no variable occurs more than
once in $t$, we say $t$ is \emph{linear}. Often, the set of variables
$\X$ is left implicit and $\TerI(\X)$ is denoted simply by $\TerI$. By
the set of \emph{finite terms} $\Ter$ we mean the subset of
well-founded terms of $\TerI$.

Preparing for the mechanised setting of Section~\ref{chap:implementation} with
its constrains of finite memory and computing time, we want to be precise
about the notions of equality on infinite objects we employ. We consider terms
to be equal if they are
\begin{inparaenum}[(i)]
  \item bisimilar or
  \item pointwise equivalent up to every depth.
\end{inparaenum}
According to Proposition~\ref{prop:equalities} it does not matter
which equality we use.

\begin{definition}\label{def:bisimilarity}%[Bisimilarity]
We define the \emph{bisimilarity relation} $\bis$ on $\TerI$ by
coinduction:
\begin{compactenum}
  \item
    $x \bis x$ for every variable $x \in \X$.
  \item
    For every $f \in \Sigma$, if $s_i \bis t_i$ for all $1 \leq i \leq
    \arity{f}$, then $f(s_1, \ldots s_{\arity{f}}) \bis f(t_1, \ldots,
    t_{\arity{f}})$.
\end{compactenum}
%$\bis$ is the greatest bisimulation on $\TerI$ and
We say that $s$ and $t$ are \emph{bisimilar} if $s \bis t$.
\end{definition}

\begin{definition}\label{def:equiv}%[Pointwise equivalence]
\emph{(Pointwise) equivalence} of terms $s$ and $t$ \emph{up to depth} $d$,
written $s \equpto{d} t$, is defined by induction:
\begin{compactenum}
  \item $s \equpto{0} t$ for every $s, t \in \TerI$.
  \item $x \equpto{d} x$ for every $d \in \mathbb{N}$ and $x \in \X$.
  \item For every $f \in \Sigma$, if $s_i \equpto{d} t_i$ for all $1 \leq i
    \leq \arity{f}$, then $f(s_1, \ldots s_{\arity{f}}) \equpto{d+1}
    f(t_1, \ldots, t_{\arity{f}})$.
\end{compactenum}
The \emph{(pointwise) equivalence} $s \equiv t$ holds if $s \equpto{d}
t$ for every depth $d$.
\end{definition}

\begin{proposition}\label{prop:equalities}
$s \bis t \; \Leftrightarrow \; s \equiv t$.
\end{proposition}
\begin{proof}
By induction on the depth of pointwise equivalence ($\Rightarrow$) and
by coinduction on $s$ ($\Leftarrow$).
%\footnote{This proposition is proved in our \Coq development.}
\end{proof}

\begin{definition}%[Rewrite rule]
  A \emph{rewrite rule} $\rho$ on a signature $\Sigma$ is a pair
  $\langle l, r \rangle$ of finite terms in $\Ter$ (written $\rho : l
  \rightarrow r$). We restrict ourselves to rewrite rules where $l$ is
  not a variable and $\Var(r) \subseteq \Var(l)$.
\end{definition}

The two restrictions on rewrite rules are standard and prevent our
theory from misbehaving in some particular ways. We say a rewrite rule
is \emph{left-linear} if its left-hand side is linear.

\begin{definition}%[TRS]
A \emph{term rewriting system} (TRS) $\mathcal{R}$ is a pair $\langle \Sigma,
R \rangle$ of a signature $\Sigma$ and a finite set of rewrite rules
$R$ on $\Sigma$.
\end{definition}


\subsection{Rewriting}

Positions are sequences of natural numbers. The empty sequence is
denoted by $\epsilon$ and $\prefix{i}{p}$ is the prefixing of a
sequence $p$ with the number $i$.

\begin{definition}%[Subterm positions]
  The set of \emph{positions} $\Pos(t)$ \emph{of a term} $t$ is
  inductively defined:
  \begin{compactenum}
    \item $\epsilon \in \Pos(t)$ for every $t \in \TerI$.
    \item For every $f \in \Sigma$ and $1 \le i \le \arity{f}$, if $p
      \in \Pos(t_i)$ then $\prefix{i}{p} \in \Pos(f(t_1, \ldots,
      t_{\arity{f}}))$.
  \end{compactenum}
  The \emph{subterm} of term $t$ at position $p$, written
  $\subterm{t}{p}$, is inductively defined by
  \begin{inparaenum}[(i)]
    \item $\subterm{t}{\epsilon} = t$ and
    \item $\subterm{f(t_1, \ldots, t_n)}{\prefix{i}{p}} =
      \subterm{t_i}{p}$.
  \end{inparaenum}
  Similarly, \emph{updating} a term $t$ at position $p$ with term $s$,
  written $t[s]_p$, is defined by replacing the subterm
  $\subterm{t}{p}$ at position $p$ in $t$ with $s$.
\end{definition}

In contrast to \cite{terese-03}, we do not define contexts as terms over an
extended signature. Instead, a direct inductive definition is given since this
is how we defined the notion of context in our \Coq development (the main
reason being that we choose not to consider multi-hole contexts).
% TODO: maybe this needs more explaining

\begin{definition}%[Context]
The set of (one-hole) \emph{contexts} $\Ctx$ over a signature
$\Sigma$ is defined by induction:
\begin{compactenum}
  \item
    $\Box \in \Ctx$.
%  \item
%    For every $f \in \Sigma$ and $1 \le n \le \arity{f}$, if $t_1,
%    \ldots, t_{\arity{f} - 1} \in \TerI$ and $C \in \Ctx$, then
%    $f(t_1, \ldots, t_{n - 1}, C, t_{n + 1}, \ldots, t_{\arity{f} -
%      1}) \in \Ctx$.
  \item
    For every $f \in \Sigma$ and $1 \le i \le \arity{f}$, if $t_1,
    \ldots, t_{i - 1}, t_{i + 1}, \ldots, t_{\arity{f}} \in \TerI$ and
    $C \in \Ctx$, then $f(t_1, \ldots, t_{i - 1}, C, t_{i + 1},
    \ldots, t_{\arity{f}}) \in \Ctx$.
\end{compactenum}
\end{definition}

Thus every context $C$ has exactly one occurrence of the symbol $\Box$, called
its \emph{hole}. By the term $C[t]$ we mean the result of replacing the hole
of $C$ by $t$. We allow a slight abuse of notation by writing
$t[\Box]_p$ for the context $C$ with $C[\subterm{t}{p}] \equiv t$. We
also assume obvious extensions to contexts for notions on terms
(e.g.\ $\Var(C)$ and $\Pos(C)$ for $C \in \Ctx$).
The \emph{depth} of a context $C$ is defined by the length of the
(unique) position $p$ with $\subterm{C}{p} = \Box$.

%The \emph{hole depth} of a context $C$ is defined by the number
%of `$($' symbols minus the number of `$)$' symbols preceding the $\Box$
%symbol in $C$.
%TODO: this seemed to me the shortest way to define the hole depth?

\begin{definition}%[Substitution]
% TODO: now we only generalise to finite terms
Given a signature $\Sigma$ and a set of variables $\X$, a
\emph{substitution} $\sigma$ is a mapping from $\X$ to $\TerI(\X)$. It
can be generalised to a mapping $\bar{\sigma} : \TerI(\X) \rightarrow
\TerI(\X)$ corecursively:
\begin{align*}
  \bar{\sigma}(x) &= \sigma(x)\\
  \bar{\sigma}(f(t_1, \ldots, t_n)) &= f(\bar{\sigma}(t_1), \ldots,
  \bar{\sigma}(t_n))
\end{align*}
\end{definition}

Since $\bar{\sigma}$ is completely defined by $\sigma$ we refer to both as
`the' substitution $\sigma$.
%The notation $[x_1, \ldots, x_n := s_1, \ldots, s_n]$ is used for the
%substitution $\sigma$ with $\sigma(x_i) = s_i$ for $1 \leq i \leq n$
%and $\sigma(y) = y$ for all other $y$.
Applying a substitution $\sigma$ to a term $t$ is usually written
$t^\sigma$ and the result is called an \emph{instance} of $t$.
% TODO: at this moment, we don't use the [x := y] notation (get rid of it?)

If we view a rewriting rule $\rho : l \rightarrow r$ as a \emph{scheme}, an
\emph{instance} of $\rho$ can be obtained by applying a substitution
$\sigma$. The result is the \emph{atomic} rewrite step $l^\sigma
\rightarrow_\rho r^\sigma$. We call $l^\sigma$ a ($\rho$-) \emph{redex} and
$r^\sigma$ its \emph{contractum}. An atomic rewrite step can be placed in a
context, forming a rewrite step.

\begin{definition}%[Rewrite step]
A \emph{rewrite step} $C[l^\sigma] \rightarrow_\rho C[r^\sigma]$ according to
the rewrite rule $\rho$ consists of rewriting the redex obtained from
$\rho$ and substitution $\sigma$ to its contractum in a context $C$.
\end{definition}

The \emph{depth} of a rewrite step is the depth of its context. We
call $\rightarrow_\rho$ the \emph{one-step rewriting relation}
generated by $\rho$. The one-step rewriting relation $\rightarrow$ of
a TRS $\mathcal{R}$ with rewrite rules $R$ is defined as the union of
$\{ \rightarrow_\rho | \; \rho \in R \}$.

\begin{definition}\label{def:stepeq}%[Equality of steps]
%Rewrite steps $s_1 \rightarrow s_2$ and $t_1 \rightarrow t_2$ are
%defined to be \emph{equal} if
Two rewrite steps are defined to be \emph{equal} if
\begin{compactenum}
  \item they use the same rewrite rule $\rho$,
  \item their contexts are bisimilar and
  \item their substitutions agree on all variables in $\rho$.
\end{compactenum}
\end{definition}

\begin{definition}%[Rewrite sequence]
A \emph{rewrite sequence} of ordinal length $\alpha$ is a sequence of rewrite
steps $(t_\beta \rightarrow t_{\beta^+})_{\beta \prec \alpha}$.
\end{definition}

This definition only makes sense if we somehow require that for every limit
ordinal $\lambda \prec \alpha$, the terms $(t_\beta)_{\beta \prec
  \lambda}$ approach $t_\lambda$ in the limit and sometimes even
further restrictions are desirable. We define the notion of Cauchy
convergence and, using that, four conditions on rewrite sequences.

\begin{definition}\label{def:cauchy}%[Cauchy convergence]
  Let $\lambda$ be a limit ordinal. A sequence of terms
  $(t_\beta)_{\beta \prec \lambda}$ \emph{(Cauchy-) converges to the
    term} $t$ if for every depth $d$ there exists $\alpha \prec
  \lambda$ such that for all $\alpha \preceq \beta \prec \lambda$ we
  have $t_\beta \equpto{d} t$.
\end{definition}

\begin{definition}%[Continuity and convergence]
A rewrite sequence $(t_\beta \rightarrow t_{\beta^+})_{\beta \prec
  \alpha}$ of length $\alpha$ is
\begin{compactenum}
  \item
    \emph{weakly continuous} if for every limit ordinal $\lambda \prec
    \alpha$, the sequence $(t_\beta)_{\beta \prec \lambda}$ converges
    to the term $t_\lambda$,
  \item
    \emph{strongly continuous} if it is weakly continuous and for every limit
    ordinal $\lambda \prec \alpha$, the depth of the rewrite steps $(t_\beta
    \rightarrow t_{\beta^+})_{\beta \prec \lambda}$ tends to infinity,
  \item
    \emph{weakly convergent} if for every limit ordinal $\lambda
    \preceq \alpha$, the sequence $(t_\beta)_{\beta \prec \lambda}$
    converges to the term $t_\lambda$ and
  \item
    \emph{strongly convergent} if it is weakly convergent and for
    every limit ordinal $\lambda \preceq \alpha$, the depth of the
    rewrite steps $(t_\beta \rightarrow t_{\beta^+})_{\beta \prec
      \lambda}$ tends to infinity.
\end{compactenum}
\end{definition}

We write $t_0 \rewrites t_\alpha$ if there exists a strongly
convergent rewrite sequence $(t_\beta \rightarrow t_{\beta^+})_{\beta
  \prec \alpha}$ (or $t_0 \rightarrow^\alpha t_\alpha$ if we want to
stress its length). The \emph{convertibility relation} is defined as
the equivalence closure of $\rewrites$.


\subsection{Normal Forms and Orthogonality}

\begin{definition}\label{def:normalisation}%[Normalisation]
  Let $\mathcal{R}$ be a TRS.
  \begin{compactenum}
    \item
      A term $t$ is a \emph{normal form} if there do not exist a
      rewrite rule $l \rightarrow r$ in $\mathcal{R}$, a substitution
      $\sigma$ and a context $C$ such that $t \equiv C[l^\sigma]$
      (i.e.\ if there is no step from $t$). We say $t$ is a normal
      form \emph{of} $s$ if $s \rewrites t$ and $t$ is a normal form.
    \item
      $\mathcal{R}$ has the (infinitary) \emph{unique normal forms}
      (UN$^\infty$) property if $t \equiv u$ for every two
      convertible normal forms $t$ and $u$.
    \item
      % TODO: make sure \rewrites renders fine in superscript
      $\mathcal{R}$ has the (infinitary) \emph{unique normal forms
      with respect to rewriting} (UN$^\rewrites$) property if for all
      terms $s$, we have $t \equiv u$ for every normal forms $t$ and
      $u$ of $s$.
  \end{compactenum}
\end{definition}
% TODO: klop-de-vrijer-05 uses our UN->> definition for UN^\infty,
% should we distinguish between the two?
% TODO: pictures

% TODO: tegenvoorbeeld voor UN$^\rewrites$ => UN$^\infty$
Obviously we have that UN$^\infty$ implies UN$^\rewrites$. One source
of non-unique normal forms is the interference of two redex occurrences
in a term. Contracting one of them may result in a term where (a
descendant of) the other redex is no longer present, possibly losing
confluence. This phenomenon is made precise in the following
definition.
% TODO: informal definition of descendants and confluence

% TODO: maybe cite joerg's thesis
\begin{definition}\label{def:overlap}%[Overlap and critical pairs]
We say two rewrite rules $\rho_1 : l_1 \rightarrow r_1$ and $\rho_2 :
l_2 \rightarrow r_2$ have \emph{overlap} if there exists a
non-variable position $p$ such that $l_1 |_p$ and $l_2$ have a common
instance. (We exclude the trivial case of overlap between a rewrite
rule and itself at the root position.)
Let $\sigma, \tau$ be substitutions such that $l_1 |_p ^{\,\;\sigma}
\equiv l_2 ^{\,\;\tau}$ is a most general common instance of $l_1 |_p$
and $l_2$, and without loss of generality assume that $\dom(\sigma) =
\Var(l_1 |_p)$ and $\Var(l_1 |_p ^{\,\;\sigma}) \cap \Var(l_1[\Box]_p)
= \nothing$. Then $\langle l_1 ^{\,\;\sigma}[r_2^{\,\;\tau}]_p,
r_1^{\,\;\sigma} \rangle$ is called a \emph{critical pair of} $\rho_1$
\emph{with} $\rho_2$. A critical pair $\langle s, t \rangle$ is called
\emph{trivial} if $s \equiv t$.
\end{definition}

Critical pairs are unique up to renaming of variables. Using these
notions we can define some useful classes of term rewriting systems.

%We say two rewrite rules $\rho_1 : l_1 \rightarrow r_2$ and $\rho_2 : l_2
%\rightarrow r_2$ overlap if there is a term $t$ with overlapping occurrences
%of the pattern of $l_1$ and the pattern of $l_2$.

%Two redex occurrences in a term $t$ overlap if their patterns share at least
%one symbol occurrence. Here we do not count the trivial overlap between a
%redex s and itself, unless s is a redex with respect to two different
%reduction rules.
%We say two rewrite rules $\rho_1, \rho_2$ overlap if there is a term $t$ with
%overlapping occurrences of a $\rho_1$- and $\rho_2$-redex.

\begin{definition}%[Orthogonality]
A TRS is called
\begin{compactenum}
  \item \emph{left-linear} if all its rewrite rules are left-linear,
  \item \emph{orthogonal} if it is left-linear and there are no critical pairs and
  \item \emph{weakly orthogonal} if it is left-linear and all critical pairs
    are trivial.
\end{compactenum}
\end{definition}

A fundamental result in the theory of infinitary rewriting is the
Compression Lemma.
\begin{lemma}[{\normalfont{Compression}}]\label{lem:compression}
  Every strongly convergent rewrite sequence in a left-linear TRS can
  be compressed to length less than or equal to $\omega$.
\end{lemma}
\begin{proof}
  By transfinite induction on the length of the rewrite sequence. See
  for example \citet[Theorem 12.7.1, page 689]{terese-03} or
  \citet{endrullis-10}.
\end{proof}

Orthogonal systems enjoy the UN$^\infty$ property
\citep{kennaway-95,klop-de-vrijer-05}. In Chapter~\ref{chap:unwo} we
formalise the counterexample to UN$^\infty$ for weakly orthogonal TRSs
from \citet{endrullis-10}.

\chapter{A Mechanic Formalisation}\label{chap:implementation}

TODO: introduction.


\section{The \Coq Proof Assistant}

\Coq is based on the formal language \emph{Calculus of Inductive
  Constructions}, which is essentially a typed $\lambda$-calculus with
inductive types. In this language, logical propositions are
represented as types and proofs of such propositions are
$\lambda$-terms, motivated by the Curry-Howard-de Bruijn
correspondence.

Type constructor $\Pi$ (dependent) or $\rightarrow$ (non-dependent).

%Set is the sort of computational nature, Prop is that of logical
%statements.

TODO: text
TODO: very short intro to coq syntax
TODO: some notes about code listings. coq has implicit arguments. we aim for
brevity.
TODO: we will take some notational liberties (syntax precedence levels, infix
notation etc)


\section{Ordinal Numbers}

In the theory of infinitary rewriting, the lengths of rewrite sequences play
a central role. Therefore, any reasonable formalisation of infinitary
rewriting ought to have some notion of ordinal numbers.

We define the ordinal numbers using the representation of Brouwer
ordinals (cf.~Definition~\ref{def:ordinals}) in \Coq by
\coqref{Ordinal.ord}{\coqdocinductive{ord}}.
\begin{singlespace}
\begin{coqdoccode}
\coqdocnoindent
\coqdockw{Inductive} \coqdef{Ordinal.ord}{ord}{\coqdocinductive{ord}} :
\coqdockw{Set} :=\coqdoceol
\coqdocindent{1.00em}
\ensuremath{|} \coqdef{Ordinal.Zero}{Zero}{\coqdocconstructor{Zero}}  :
\coqref{Ordinal.ord}{\coqdocinductive{ord}}\coqdoceol
\coqdocindent{1.00em}
\ensuremath{|} \coqdef{Ordinal.Succ}{Succ}{\coqdocconstructor{Succ}}  :
\coqref{Ordinal.ord}{\coqdocinductive{ord}} \ensuremath{\rightarrow}
\coqref{Ordinal.ord}{\coqdocinductive{ord}}\coqdoceol
\coqdocindent{1.00em}
\ensuremath{|} \coqdef{Ordinal.Limit}{Limit}{\coqdocconstructor{Limit}} :
(\coqexternalref{http://coq.inria.fr/stdlib/Coq.Init.Datatypes}{nat}{\coqdocinductive{nat}}
\ensuremath{\rightarrow} \coqref{Ordinal.ord}{\coqdocinductive{ord}})
\ensuremath{\rightarrow}
\coqref{Ordinal.ord}{\coqdocinductive{ord}}.\coqdoceol
\end{coqdoccode}
\end{singlespace}
In fact, all definitions from Section~\ref{sub:brouwer} translate directly to
\Coq code. We can now prove basic properties of $\preceq$, for example that it
is transitive.
% TODO: maybe add another example
\begin{singlespace}
\begin{coqdoccode}
\coqdocnoindent
\coqdockw{Lemma}
\coqdef{Ordinal.ordletrans}{ord\_le\_trans}{\coqdoclemma{\ensuremath{\preceq_{\text{trans}}}}}
:
\ensuremath{\forall} \coqdocvar{\ensuremath{\alpha}}
\coqdocvar{\ensuremath{\beta}}
\coqdocvar{\ensuremath{\gamma}}, \coqdocvariable{\ensuremath{\alpha}}
\ensuremath{\preceq} \coqdocvariable{\ensuremath{\beta}}
\ensuremath{\rightarrow}
\coqdocvariable{\ensuremath{\beta}} \ensuremath{\preceq}
\coqdocvariable{\ensuremath{\gamma}}
\ensuremath{\rightarrow} \coqdocvariable{\ensuremath{\alpha}}
\ensuremath{\preceq}
\coqdocvariable{\ensuremath{\gamma}}.\coqdoceol
\end{coqdoccode}
\end{singlespace}

Recalling our discussion in Section~\ref{sub:brouwer} of limit ordinals whose
sequences do not actually approximate to a limit ordinal, we consider the
lemma
\coqref{Ordinal.ordlezeroright}{\coqdoclemma{\ensuremath{\preceq_{\text{zero\_right}}}}}
as an example of this issue.
\begin{singlespace}
\begin{coqdoccode}
\coqdocnoindent
\coqdockw{Lemma}
\coqdef{Ordinal.ordlezeroright}{ord\_le\_zero\_right}{\coqdoclemma{\ensuremath{\preceq_{\text{zero\_right}}}}}
:
\ensuremath{\forall} \coqdocvar{\ensuremath{\alpha}} \coqdocvar{\ensuremath{\beta}},
\coqdocvariable{\ensuremath{\alpha}} \ensuremath{\preceq}
\coqref{Ordinal.Zero}{\coqdocconstructor{Zero}}
\ensuremath{\rightarrow}
\coqdocvariable{\ensuremath{\alpha}} \ensuremath{\preceq}
\coqdocvariable{\ensuremath{\beta}}.\coqdoceol
\end{coqdoccode}
\end{singlespace}
We would like to strengthen this, but cannot, since nothing denies
\coqdocvariable{$\alpha$} from being the Brouwer ordinal $\sqcup \{ 0, 0, 0,
\ldots \}$ (which has the same rank as $0$). We therefore turn to a subset of
the Brouwer ordinals where we restrict limit sequences to be strictly
monotonic. This restriction is encoded in the
\coqref{WfOrdinal.wf}{\coqdocdefinition{wf}} (well-formedness)
property. The $\Sigma$-type
\coqref{WfOrdinal.wford}{\coqdocdefinition{ord$^\text{wf}$}} defines
the resulting subset.
\begin{singlespace}
\begin{coqdoccode}
\coqdocnoindent
\coqdockw{Fixpoint} \coqdef{WfOrdinal.wf}{wf}{\coqdocdefinition{wf}}
\coqdocvar{\ensuremath{\alpha}} : \coqdockw{Prop} :=\coqdoceol
\coqdocindent{1.00em}
\coqdockw{match} \coqdocvariable{\ensuremath{\alpha}} \coqdockw{with}\coqdoceol
\coqdocindent{1.00em}
\ensuremath{|} \coqref{Ordinal.Zero}{\coqdocconstructor{Zero}}
\ensuremath{\Rightarrow}
\coqexternalref{http://coq.inria.fr/stdlib/Coq.Init.Logic}{True}{\coqdocinductive{True}}\coqdoceol
\coqdocindent{1.00em}
\ensuremath{|} \coqref{Ordinal.Succ}{\coqdocconstructor{Succ}}
\coqdocvar{\ensuremath{\beta}} \ensuremath{\Rightarrow}
\coqref{WfOrdinal.wf}{\coqdocdefinition{wf}} \coqdocvariable{\ensuremath{\beta}}\coqdoceol
\coqdocindent{1.00em}
\ensuremath{|} \coqref{Ordinal.Limit}{\coqdocconstructor{Limit}} \coqdocvar{f}
\ensuremath{\Rightarrow} \ensuremath{\forall} \coqdocvar{n},
\coqref{WfOrdinal.wf}{\coqdocdefinition{wf}} (\coqdocvariable{f}
\coqdocvariable{n}) \ensuremath{\land} \ensuremath{\forall} \coqdocvar{m},
\coqdocvariable{n} < \coqdocvariable{m} \ensuremath{\rightarrow}
\coqdocvariable{f} \coqdocvariable{n} \ensuremath{\prec}
\coqdocvariable{f} \coqdocvariable{m}\coqdoceol
\coqdocindent{1.00em}
\coqdockw{end}.\coqdoceol
\coqdocemptyline
\coqdocnoindent
\coqdockw{Definition}
\coqdef{WfOrdinal.wford}{wf\_ord}{\coqdocdefinition{ord$^\text{wf}$}} : \coqdockw{Set}
:=
%\coqexternalref{http://coq.inria.fr/stdlib/Coq.Init.Specif}{sig}{\coqdocinductive{sig}}
%\coqref{WfOrdinal.wf}{\coqdocdefinition{wf}}.\coqdoceol
\{ \coqdocvariable{$\alpha$} :
\coqref{Ordinal.ord}{\coqdocinductive{ord}} \ensuremath{|}
\coqref{WfOrdinal.wf}{\coqdocdefinition{wf}} \coqdocvariable{$\alpha$}
\}.\coqdoceol
\end{coqdoccode}
\end{singlespace}
Now we can prove the stronger result we were looking for.\footnote{Although
  \coqdocvariable{\ensuremath{\alpha}} has type
  \coqref{WfOrdinal.wford}{\coqdocdefinition{ord$^\text{wf}$}} and $\preceq$
  has type \coqref{Ordinal.ord}{\coqdocinductive{ord}} $\rightarrow$
  \coqref{Ordinal.ord}{\coqdocinductive{ord}} $\rightarrow$ \coqdockw{Prop},
  we can state the lemma in this concise way by defining a simple coercion
  from \coqref{WfOrdinal.wford}{\coqdocdefinition{ord$^\text{wf}$}} to
  \coqref{Ordinal.ord}{\coqdocinductive{ord}} (first $\Sigma$-type
  projection).}
\begin{singlespace}
\begin{coqdoccode}
\coqdocnoindent
\coqdockw{Lemma}
\coqdef{WfOrdinal.wfordlezeroright}{wf\_ord\_le\_zero\_right}{\coqdoclemma{\ensuremath{\preceq^{\text{wf}}_{\text{zero\_right}}}}}
:
\ensuremath{\forall} \coqdocvar{\ensuremath{\alpha}} :
\coqref{WfOrdinal.wford}{\coqdocdefinition{ord$^\text{wf}$}},
\coqdocvariable{\ensuremath{\alpha}} \ensuremath{\preceq}
\coqref{Ordinal.Zero}{\coqdocconstructor{Zero}}
\ensuremath{\rightarrow}
\coqdocvariable{\ensuremath{\alpha}} =
\coqref{Ordinal.Zero}{\coqdocconstructor{Zero}}.\coqdoceol
\end{coqdoccode}
\end{singlespace}


\section{Coinductive Terms}

In \Coq, coinductive types can be defined using the
\coqdockw{CoInductive} keyword \citep{gimenez-casteran-07}. No
induction principles are defined for these types, because they are not
well-founded.\footnote{\Coq automatically derives induction principles
  for inductive definitions.}
The objects of a coinductive type may contain an infinite number of
constructors, but can only be built in some restricted way to ensure
productivity of the construction.
% TODO: productivity or effectiveness?
We defer discussion of this restriction to
Section~\ref{sub:guardedness}, and define the type
\coqref{Term.term}{\coqdocinductive{term}} of infinite terms with
function symbols in \coqdocvar{F} and variables in \coqdocvar{X}.
\begin{singlespace}
\begin{coqdoccode}
\coqdocnoindent
\coqdockw{CoInductive} \coqdef{Term.term}{term}{\coqdocinductive{term}} :
\coqdockw{Type} :=\coqdoceol
\coqdocindent{1.00em}
\ensuremath{|} \coqdef{Term.Var}{Var}{\coqdocconstructor{Var}} : \coqdocvar{X}
\ensuremath{\rightarrow} \coqref{Term.term}{\coqdocinductive{term}}\coqdoceol
\coqdocindent{1.00em}
\ensuremath{|} \coqdef{Term.Fun}{Fun}{\coqdocconstructor{Fun}} :
\ensuremath{\forall} \coqdocvar{f} : \coqdocvar{F},
\coqref{Vector.vector}{\coqdocdefinition{vector}}
\coqref{Term.term}{\coqdocinductive{term}}
(\coqdocprojection{arity} \coqdocvariable{f})
\ensuremath{\rightarrow} \coqref{Term.term}{\coqdocinductive{term}}.\coqdoceol
\end{coqdoccode}
\end{singlespace}
Here, \coqref{Vector.vector}{\coqdocdefinition{vector}} is assumed to
implement dependently typed lists (the type depending on their length).

The standard Leibniz equality defined in \Coq, written $=$, does not suffice
for establishing that two terms are equal, given that the only way to build
infinite objects is by corecursion. Because the amount of memory available is
finite, we can only unfold the corecursive definition finitely many times, and
then still be left with a corecursive definition. Simply comparing such
definitions will not do, since the corecursive construction of any given
infinite object is not unique. To this end, we define two extensional
equalities on \coqref{Term.term}{\coqdocinductive{term}}, following
Definitions~\ref{def:bisimilarity} and \ref{def:equiv}. The coinductive
predicate \coqref{TermEquality.termbis}{$\bis$} defines bisimilarity and
pointwise equality is defined by \coqref{TermEquality.termeq}{$\equiv$}
inductively.
\begin{singlespace}
\begin{coqdoccode}
\coqdocnoindent
\coqdockw{CoInductive}
\coqdef{TermEquality.termbis}{term\_bis}{$\bis$} :
\coqref{Term.term}{\coqdocinductive{term}} \ensuremath{\rightarrow}
\coqref{Term.term}{\coqdocinductive{term}} \ensuremath{\rightarrow}
\coqdockw{Prop} :=\coqdoceol
\coqdocindent{1.00em}
\ensuremath{|}
\coqdef{TermEquality.Varbis}{Var\_bis}{\coqdocconstructor{$\biss{\text{Var}}$}} :
\ensuremath{\forall} \coqdocvar{x},
\coqref{Term.Var}{\coqdocconstructor{Var}} \coqdocvariable{x}
\coqref{TermEquality.termbis}{$\bis$}
\coqref{Term.Var}{\coqdocconstructor{Var}} \coqdocvariable{x}\coqdoceol
\coqdocindent{1.00em}
\ensuremath{|}
\coqdef{TermEquality.Funbis}{Fun\_bis}{\coqdocconstructor{$\biss{\text{Fun}}$}} :
\ensuremath{\forall} \coqdocvar{f} \coqdocvar{v} \coqdocvar{w},
(\ensuremath{\forall} \coqdocvar{i},
\coqdocvariable{v} \coqdocvariable{i}
\coqref{TermEquality.termbis}{$\bis$}
\coqdocvariable{w} \coqdocvariable{i})
\ensuremath{\rightarrow}
\coqref{Term.Fun}{\coqdocconstructor{Fun}} \coqdocvariable{f}
\coqdocvariable{v}
\coqref{TermEquality.termbis}{$\bis$}
\coqref{Term.Fun}{\coqdocconstructor{Fun}} \coqdocvariable{f}
\coqdocvariable{w}.\coqdoceol
\end{coqdoccode}
\end{singlespace}
Any proof of two infinite terms being bisimilar is an infinite proof, in the
sense that the proof term is built by corecursion.
\begin{singlespace}
\begin{coqdoccode}
\coqdocnoindent
\coqdockw{Inductive}
\coqdef{TermEquality.termequpto}{term\_eq\_up\_to}{\equpto{}}
:
\coqexternalref{http://coq.inria.fr/stdlib/Coq.Init.Datatypes}{nat}{\coqdocinductive{nat}}
\ensuremath{\rightarrow} \coqref{Term.term}{\coqdocinductive{term}}
\ensuremath{\rightarrow} \coqref{Term.term}{\coqdocinductive{term}}
\ensuremath{\rightarrow} \coqdockw{Prop} :=\coqdoceol
\coqdocindent{1.00em}
\ensuremath{|}
\coqdef{TermEquality.teut0}{teut\_0}{\coqdocconstructor{teut$_0$}}   :
\ensuremath{\forall} \coqdocvar{s} \coqdocvar{t},
\coqdocvariable{s} \coqref{TermEquality.termequpto}{\equpto{0}}
\coqdocvariable{t}\coqdoceol
\coqdocindent{1.00em}
\ensuremath{|}
\coqdef{TermEquality.teutvar}{teut\_var}{\coqdocconstructor{teut$_\text{Var}$}} :
\ensuremath{\forall} \coqdocvar{d} \coqdocvar{x},
\coqref{Term.Var}{\coqdocconstructor{Var}} \coqdocvariable{x}
\coqref{TermEquality.termequpto}{\equpto{\coqdocvariable{d}}}
\coqref{Term.Var}{\coqdocconstructor{Var}} \coqdocvariable{x}\coqdoceol
\coqdocindent{1.00em}
\ensuremath{|}
\coqdef{TermEquality.teutfun}{teut\_fun}{\coqdocconstructor{teut$_\text{Fun}$}} :
\ensuremath{\forall} \coqdocvar{d} \coqdocvar{f} \coqdocvar{v}
\coqdocvar{w},
(\ensuremath{\forall} \coqdocvar{i},
\coqdocvariable{v} \coqdocvariable{i}
\coqref{TermEquality.termequpto}{\equpto{\coqdocvariable{d}}}
\coqdocvariable{w} \coqdocvariable{i}) \ensuremath{\rightarrow}
\coqref{Term.Fun}{\coqdocconstructor{Fun}}
\coqdocvariable{f} \coqdocvariable{v}
\coqref{TermEquality.termequpto}{\equpto{\coqexternalref{http://coq.inria.fr/stdlib/Coq.Init.Datatypes}{S}{\coqdocconstructor{S}}
    \, \coqdocvariable{d}}}
\coqref{Term.Fun}{\coqdocconstructor{Fun}} \coqdocvariable{f}
\coqdocvariable{w}.\coqdoceol
\coqdocemptyline
\coqdocnoindent
\coqdockw{Definition}
\coqdocvar{s}
\coqdef{TermEquality.termeq}{term\_eq}{$\equiv$}
\coqdocvar{t} :=
\ensuremath{\forall} \coqdocvar{d},
\coqdocvariable{s}
\coqref{TermEquality.termequpto}{\equpto{\coqdocvariable{d}}}
\coqdocvariable{t}.\coqdoceol
\end{coqdoccode}
\end{singlespace}

We can prove that \coqref{TermEquality.termbis}{$\bis$} and
\coqref{TermEquality.termeq}{$\equiv$} are the same
relation, and that indeed it is an equivalence.
\begin{singlespace}
\begin{coqdoccode}
\coqdocnoindent
\coqdockw{Lemma}
\coqdef{TermEquality.termbistermeq}{term\_bis\_term\_eq}{\coqdoclemma{term\_bis\_term\_eq}}
: \ensuremath{\forall} \coqdocvar{s} \coqdocvar{t},
\coqdocvariable{s}
\coqref{TermEquality.termbis}{$\bis$}
\coqdocvariable{t} \ensuremath{\leftrightarrow}
\coqdocvariable{s}
\coqref{TermEquality.termeq}{$\equiv$}
\coqdocvariable{t}.\coqdoceol
\coqdocemptyline
\coqdocnoindent
\coqdockw{Lemma}
\coqdef{TermEquality.termbisrefl}{term\_bis\_refl}{\coqdoclemma{$\biss{\text{refl}}$}}
: \ensuremath{\forall} \coqdocvar{t},
\coqdocvariable{t}
\coqref{TermEquality.termbis}{$\bis$}
\coqdocvariable{t}.\coqdoceol
\coqdocemptyline
\coqdocnoindent
\coqdockw{Lemma}
\coqdef{TermEquality.termbissymm}{term\_bis\_symm}{\coqdoclemma{$\biss{\text{symm}}$}}
: \ensuremath{\forall} \coqdocvar{s} \coqdocvar{t},
\coqdocvariable{s}
\coqref{TermEquality.termbis}{$\bis$}
\coqdocvariable{t} $\rightarrow$
\coqdocvariable{t}
\coqref{TermEquality.termbis}{$\bis$}
\coqdocvariable{s}.\coqdoceol
\coqdocemptyline
\coqdocnoindent
\coqdockw{Lemma}
\coqdef{TermEquality.termbistrans}{term\_bis\_trans}{\coqdoclemma{$\biss{\text{trans}}$}}
: \ensuremath{\forall} \coqdocvar{s} \coqdocvar{t} \coqdocvar{u},
\coqdocvariable{s}
\coqref{TermEquality.termbis}{$\bis$}
\coqdocvariable{t} $\rightarrow$
\coqdocvariable{t}
\coqref{TermEquality.termbis}{$\bis$}
\coqdocvariable{u} $\rightarrow$
\coqdocvariable{s}
\coqref{TermEquality.termbis}{$\bis$}
\coqdocvariable{u}.\coqdoceol
\end{coqdoccode}
\end{singlespace}

In Section~\ref{sec:seq} we need some notion of convergence for functions of
type
\coqexternalref{http://coq.inria.fr/stdlib/Coq.Init.Datatypes}{nat}{\coqdocinductive{nat}}
$\rightarrow$
\coqref{Term.term}{\coqdocinductive{term}}. We implement
Definition~\ref{def:cauchy} in \Coq for sequences of length $\omega$.
\begin{singlespace}
\begin{coqdoccode}
\coqdocnoindent
\coqdockw{Definition}
\coqdef{Rewriting.converges}{converges}{\coqdocdefinition{converges}}
(\coqdocvar{f} :
\coqexternalref{http://coq.inria.fr/stdlib/Coq.Init.Datatypes}{nat}{\coqdocinductive{nat}}
\ensuremath{\rightarrow} \coqref{Term.term}{\coqdocinductive{term}})
(\coqdocvar{t} : \coqref{Term.term}{\coqdocinductive{term}}) :
\coqdockw{Prop} :=\coqdoceol
\coqdocindent{1.00em}
\ensuremath{\forall} \coqdocvar{d}, \ensuremath{\exists} \coqdocvar{n},
\ensuremath{\forall} \coqdocvar{m},
\coqdocvariable{n} \ensuremath{\le} \coqdocvariable{m}
\ensuremath{\rightarrow}
\coqdocvariable{f} \coqdocvariable{m}
\coqref{TermEquality.termequpto}{\equpto{\coqdocvariable{d}}}
\coqdocvariable{t}.\coqdoceol
\end{coqdoccode}
\end{singlespace}
The definitions of finite term, rewrite rule, TRS and left-linearity
from Section~\ref{sub:trs} translate to \Coq directly. We define
\coqdef{Rewriting.lhs}{lhs}{\coqdocprojection{lhs}} and
\coqdef{Rewriting.rhs}{rhs}{\coqdocprojection{rhs}} to be first and
second projection on rewrite rules.

The type of contexts is inductively defined, where the hole always
occurs at a finite depth.
\begin{singlespace}
\begin{coqdoccode}
\coqdocnoindent
\coqdockw{Inductive}
\coqdef{Context.context}{context}{\coqdocinductive{context}} :
\coqdockw{Type} :=\coqdoceol
\coqdocindent{1.00em}
\ensuremath{|} $\Box$ :
\coqref{Context.context}{\coqdocinductive{context}}\coqdoceol
\coqdocindent{1.00em}
\ensuremath{|} \coqdef{Context.CFun}{CFun}{\coqdocconstructor{CFun}} :
\ensuremath{\forall} (\coqdocvar{f} : \coqdocvar{F}) (\coqdocvar{i}
\coqdocvar{j} :
\coqexternalref{http://coq.inria.fr/stdlib/Coq.Init.Datatypes}{nat}{\coqdocinductive{nat}}),
\coqdocvariable{i} +
\coqexternalref{http://coq.inria.fr/stdlib/Coq.Init.Datatypes}{S}{\coqdocconstructor{S}}
\coqdocvariable{j} =
\coqdocprojection{arity} \coqdocvariable{f}
\ensuremath{\rightarrow}\coqdoceol
\coqdocindent{5.50em}
\coqref{Vector.vector}{\coqdocdefinition{vector}}
\coqref{Term.term}{\coqdocinductive{term}} \coqdocvariable{i}
\ensuremath{\rightarrow}
\coqref{Context.context}{\coqdocinductive{context}}
\ensuremath{\rightarrow}
\coqref{Vector.vector}{\coqdocdefinition{vector}}
\coqref{Term.term}{\coqdocinductive{term}} \coqdocvariable{j}
\ensuremath{\rightarrow}
\coqref{Context.context}{\coqdocinductive{context}}.\coqdoceol
\end{coqdoccode}
\end{singlespace}
Applying a substitution \coqdocvariable{$\sigma$} to a term
\coqdocvariable{t} is defined by corecursion over
\coqdocvariable{t}. We also use the
notation \begin{coqdoccode}\coqdocvariable{t}$^\coqdocvariable{$\sigma$}$\end{coqdoccode}
for \begin{coqdoccode}\coqref{Substitution.substitute}{\coqdocdefinition{substitute}}
  \coqdocvariable{$\sigma$} \coqdocvariable{t}\end{coqdoccode}.
\begin{singlespace}
\begin{coqdoccode}
\coqdocnoindent
\coqdockw{Definition}
\coqdef{Substitution.substitution}{substitution}{\coqdocdefinition{substitution}}
:= \coqdocvar{X} \ensuremath{\rightarrow}
\coqref{Term.term}{\coqdocinductive{term}}.\coqdoceol
\coqdocemptyline
\coqdocnoindent
\coqdockw{CoFixpoint}
\coqdef{Substitution.substitute}{substitute}{\coqdocdefinition{substitute}}
(\coqdocvar{$\sigma$} :
\coqref{Substitution.substitution}{\coqdocdefinition{substitution}})
(\coqdocvar{t} :
\coqref{Term.term}{\coqdocinductive{term}}) :
\coqref{Term.term}{\coqdocinductive{term}} :=\coqdoceol
\coqdocindent{1.00em}
\coqdockw{match} \coqdocvariable{t} \coqdockw{with}\coqdoceol
\coqdocindent{1.00em}
\ensuremath{|} \coqref{Term.Var}{\coqdocconstructor{Var}}
\coqdocvar{x}      \ensuremath{\Rightarrow} \coqdocvariable{$\sigma$}
\coqdocvariable{x}\coqdoceol
\coqdocindent{1.00em}
\ensuremath{|} \coqref{Term.Fun}{\coqdocconstructor{Fun}}
\coqdocvar{f} \coqdocvar{args} \ensuremath{\Rightarrow}
\coqref{Term.Fun}{\coqdocconstructor{Fun}} \coqdocvariable{f}
(\coqref{Vector.vmap}{\coqdocdefinition{vmap}}
(\coqref{Substitution.substitute}{\coqdocdefinition{substitute}}
\coqdocvariable{$\sigma$}) \coqdocvariable{args})\coqdoceol
\coqdocindent{1.00em}
\coqdockw{end}.\coqdoceol
\end{coqdoccode}
\end{singlespace}
We apply the recursive function
\coqdef{Context.fill}{fill}{\coqdocdefinition{fill}} (not shown here)
to a context \coqdocvariable{C} and a term \coqdocvariable{t}
(written \begin{coqdoccode}\coqdocvariable{C}[\coqdocvariable{t}]\end{coqdoccode})
to replace the hole in \coqdocvariable{C} with \coqdocvariable{t}.

Positions are represented by simple lists of natural numbers. This
means the subterm at some position in some term may not actually
exist. For this reason we employ
\coqexternalref{http://coq.inria.fr/stdlib/Coq.Init.Datatypes}{option}{\coqdocinductive{option}}
types in functions that do a lookup by position (functions in \Coq are
always \emph{total}).
\begin{singlespace}
\begin{coqdoccode}
\coqdocnoindent
\coqdockw{Fixpoint}
\coqdef{Term.subterm}{subterm}{\coqdocdefinition{subterm}}
(\coqdocvar{t} : \coqref{Term.term}{\coqdocabbreviation{term}})
(\coqdocvar{p} :
\coqdocabbreviation{position})
\{\coqdockw{struct} \coqdocvar{p}\} :
\coqexternalref{http://coq.inria.fr/stdlib/Coq.Init.Datatypes}{option}{\coqdocinductive{option}}
\coqref{Term.term}{\coqdocabbreviation{term}} :=\coqdoceol
\coqdocindent{1.00em}
\coqdockw{match} \coqdocvariable{p} \coqdockw{with}\coqdoceol
\coqdocindent{1.00em}
\ensuremath{|}
\coqexternalref{http://coq.inria.fr/stdlib/Coq.Init.Datatypes}{nil}{\coqdocconstructor{nil}}
\ensuremath{\Rightarrow}
\coqexternalref{http://coq.inria.fr/stdlib/Coq.Init.Datatypes}{Some}{\coqdocconstructor{Some}}
\coqdocvariable{t}\coqdoceol
\coqdocindent{1.00em}
\ensuremath{|} \coqdocvar{n} :: \coqdocvar{p} \ensuremath{\Rightarrow}
\coqdockw{match} \coqdocvariable{t} \coqdockw{with}\coqdoceol
\coqdocindent{7.00em}
\ensuremath{|} \coqref{Term.Var}{\coqdocconstructor{Var}}
\coqdocvar{\_}      \ensuremath{\Rightarrow}
\coqexternalref{http://coq.inria.fr/stdlib/Coq.Init.Datatypes}{None}{\coqdocconstructor{None}}\coqdoceol
\coqdocindent{7.00em}
\ensuremath{|} \coqref{Term.Fun}{\coqdocconstructor{Fun}}
\coqdocvar{f} \coqdocvar{args} \ensuremath{\Rightarrow}
\coqdockw{match}
\coqexternalref{http://coq.inria.fr/stdlib/Coq.Arith.Bool\_nat}{ltgedec}{\coqdocdefinition{lt\_ge\_dec}}
\coqdocvariable{n} (\coqdocprojection{arity}
\coqdocvariable{f}) \coqdockw{with}\coqdoceol
\coqdocindent{15.00em}
\ensuremath{|}
\coqexternalref{http://coq.inria.fr/stdlib/Coq.Init.Specif}{left}{\coqdocconstructor{left}}
\coqdocvar{h}  \ensuremath{\Rightarrow}
\coqref{Term.subterm}{\coqdocdefinition{subterm}}
(\coqdocdefinition{vnth} \coqdocvariable{h}
\coqdocvariable{args}) \coqdocvariable{p}\coqdoceol
\coqdocindent{15.00em}
\ensuremath{|}
\coqexternalref{http://coq.inria.fr/stdlib/Coq.Init.Specif}{right}{\coqdocconstructor{right}}
\coqdocvar{\_} \ensuremath{\Rightarrow}
\coqexternalref{http://coq.inria.fr/stdlib/Coq.Init.Datatypes}{None}{\coqdocconstructor{None}}\coqdoceol
\coqdocindent{15.00em}
\coqdockw{end}\coqdoceol
\coqdocindent{7.00em}
\coqdockw{end}\coqdoceol
\coqdocindent{1.00em}
\coqdockw{end}.\coqdoceol
\coqdocemptyline
\coqdocnoindent
\coqdockw{Fixpoint} \coqdef{Context.dig}{dig}{\coqdocdefinition{dig}}
(\coqdocvar{t} : \coqref{Term.term}{\coqdocabbreviation{term}})
(\coqdocvar{p} :
\coqdocabbreviation{position})
\{\coqdockw{struct} \coqdocvar{p}\} :
\coqexternalref{http://coq.inria.fr/stdlib/Coq.Init.Datatypes}{option}{\coqdocinductive{option}}
\coqref{Context.context}{\coqdocinductive{context}} :=\coqdoceol
\coqdocindent{1.00em}
\coqdockw{match} \coqdocvariable{p} \coqdockw{with}\coqdoceol
\coqdocindent{1.00em}
\ensuremath{|}
\coqexternalref{http://coq.inria.fr/stdlib/Coq.Init.Datatypes}{nil}{\coqdocconstructor{nil}}
\ensuremath{\Rightarrow}
\coqexternalref{http://coq.inria.fr/stdlib/Coq.Init.Datatypes}{Some}{\coqdocconstructor{Some}}
$\Box$\coqdoceol
\coqdocindent{1.00em}
\ensuremath{|} \coqdocvar{n} :: \coqdocvar{p} \ensuremath{\Rightarrow}
\coqdockw{match} \coqdocvariable{t} \coqdockw{with}\coqdoceol
\coqdocindent{3.00em}
\ensuremath{|} \coqref{Term.Var}{\coqdocconstructor{Var}}
\coqdocvar{\_}      \ensuremath{\Rightarrow}
\coqexternalref{http://coq.inria.fr/stdlib/Coq.Init.Datatypes}{None}{\coqdocconstructor{None}}\coqdoceol
\coqdocindent{3.00em}
\ensuremath{|} \coqref{Term.Fun}{\coqdocconstructor{Fun}}
\coqdocvar{f} \coqdocvar{args} \ensuremath{\Rightarrow}
\coqdockw{match}
\coqexternalref{http://coq.inria.fr/stdlib/Coq.Arith.Bool\_nat}{ltgedec}{\coqdocdefinition{lt\_ge\_dec}}
\coqdocvariable{n} (\coqdocprojection{arity}
\coqdocvariable{f}) \coqdockw{with}\coqdoceol
\coqdocindent{5.00em}
\ensuremath{|}
\coqexternalref{http://coq.inria.fr/stdlib/Coq.Init.Specif}{left}{\coqdocconstructor{left}}
\coqdocvar{h}  \ensuremath{\Rightarrow} \coqdockw{match}
\coqref{Context.dig}{\coqdocdefinition{dig}}
(\coqdocdefinition{vnth} \coqdocvariable{h}
\coqdocvariable{args}) \coqdocvariable{p} \coqdockw{with}\coqdoceol
\coqdocindent{7.00em}
\ensuremath{|}
\coqexternalref{http://coq.inria.fr/stdlib/Coq.Init.Datatypes}{None}{\coqdocconstructor{None}}
\ensuremath{\Rightarrow}
\coqexternalref{http://coq.inria.fr/stdlib/Coq.Init.Datatypes}{None}{\coqdocconstructor{None}}\coqdoceol
\coqdocindent{7.00em}
\ensuremath{|}
\coqexternalref{http://coq.inria.fr/stdlib/Coq.Init.Datatypes}{Some}{\coqdocconstructor{Some}}
\coqdocvar{C} \ensuremath{\Rightarrow}
\coqexternalref{http://coq.inria.fr/stdlib/Coq.Init.Datatypes}{Some}{\coqdocconstructor{Some}}
(\coqref{Context.CFun}{\coqdocconstructor{CFun}} \coqdocvariable{f}
(\coqdoclemma{lt\_plus\_minus\_r}
\coqdocvariable{h})\coqdoceol
\coqdocindent{14.00em}
(\coqdocdefinition{vtake}
(\coqexternalref{http://coq.inria.fr/stdlib/Coq.Arith.Lt}{ltleweak}{\coqdoclemma{lt\_le\_weak}}
\coqdocvariable{n} (\coqdocprojection{arity}
\coqdocvariable{f}) \coqdocvariable{h})
\coqdocvariable{args})\coqdoceol
\coqdocindent{14.00em}
\coqdocvariable{C}\coqdoceol
\coqdocindent{14.00em}
(\coqdocdefinition{vdrop} \coqdocvariable{h}
\coqdocvariable{args}))\coqdoceol
\coqdocindent{7.00em}
\coqdockw{end}\coqdoceol
\coqdocindent{5.00em}
\ensuremath{|}
\coqexternalref{http://coq.inria.fr/stdlib/Coq.Init.Specif}{right}{\coqdocconstructor{right}}
\coqdocvar{\_} \ensuremath{\Rightarrow}
\coqexternalref{http://coq.inria.fr/stdlib/Coq.Init.Datatypes}{None}{\coqdocconstructor{None}}\coqdoceol
\coqdocindent{5.00em}
\coqdockw{end}\coqdoceol
\coqdocindent{3.00em}
\coqdockw{end}\coqdoceol
\coqdocindent{1.00em}
\coqdockw{end}.\coqdoceol
\end{coqdoccode}
\end{singlespace}
Now \begin{coqdoccode}\coqref{Term.subterm}{\coqdocdefinition{subterm}}
  \coqdocvariable{t} \coqdocvariable{p}\end{coqdoccode} gives the
subterm of \coqdocvariable{t} (if it exists)
and \begin{coqdoccode}\coqref{Context.dig}{\coqdocdefinition{dig}}
  \coqdocvariable{t} \coqdocvariable{p}\end{coqdoccode} gives the
context \coqdocvariable{C} (if it exists) that is \coqdocvariable{t}
with \begin{coqdoccode}\coqref{Term.subterm}{\coqdocdefinition{subterm}}
  \coqdocvariable{t} \coqdocvariable{p}\end{coqdoccode} replaced by
$\Box$ at position \coqdocvariable{p}.


\section{Transfinite Rewrite Sequences}\label{sec:seq}

TODO: this section is the crux of our development and is original work.

Throughout this section, we let $\mathcal{R}$ be a fixed TRS. We
define the type of steps using rewrite rules in $\mathcal{R}$,
parameterised by their source and target terms.
\begin{singlespace}
\begin{coqdoccode}
\coqdocnoindent
\coqdockw{Inductive} \coqdef{Rewriting.step}{step}{$\rightarrow_\mathcal{R}$} :
\coqref{Term.term}{\coqdocinductive{term}} \ensuremath{\rightarrow}
\coqref{Term.term}{\coqdocinductive{term}} \ensuremath{\rightarrow}
\coqdockw{Type} :=\coqdoceol
\coqdocindent{1.00em}
\ensuremath{|} \coqdef{Rewriting.Step}{Step}{\coqdocconstructor{Step}} :
\ensuremath{\forall} (\coqdocvar{s} \coqdocvar{t} :
\coqref{Term.term}{\coqdocinductive{term}}) (\coqdocvar{$\rho$} :
\coqdocrecord{rule}) (\coqdocvar{C} :
\coqref{Context.context}{\coqdocinductive{context}}) (\coqdocvar{$\sigma$} :
\coqref{Substitution.substitution}{\coqdocdefinition{substitution}}),\coqdoceol
\coqdocindent{6.50em} \coqdocvariable{$\rho$}
\coqexternalref{http://coq.inria.fr/stdlib/Coq.Lists.List}{In}{\coqdocdefinition{$\in$}}
\coqdocvar{$\mathcal{R}$} \ensuremath{\rightarrow}\coqdoceol
\coqdocindent{6.50em}
\coqdocvariable{C}[(\coqref{Rewriting.lhs}{\coqdocprojection{lhs}}
\coqdocvariable{$\rho$})\coqdocvariable{$^\sigma$}] \coqref{TermEquality.termbis}{$\bis$} \coqdocvariable{s}
\ensuremath{\rightarrow}\coqdoceol
\coqdocindent{6.50em}
\coqdocvariable{C}[(\coqref{Rewriting.rhs}{\coqdocprojection{rhs}}
\coqdocvariable{$\rho$})\coqdocvariable{$^\sigma$}] \coqref{TermEquality.termbis}{$\bis$} \coqdocvariable{t}
\ensuremath{\rightarrow}\coqdoceol
\coqdocindent{6.50em}
\coqdocvariable{s} \coqref{Rewriting.step}{$\rightarrow_\mathcal{R}$}
\coqdocvariable{t}.\coqdoceol
\end{coqdoccode}
\end{singlespace}
For the translation of Definition~\ref{def:stepeq} (equality of steps)
to \Coq, we assume the lifting of bisimilarity to contexts and that
\coqdef{Substitution.substitutioneq}{substitution\_eq}{\coqdocdefinition{substitution\_eq}}
defines agreement of substitutions on a given list of variables.
\begin{singlespace}
\begin{coqdoccode}
\coqdocnoindent
\coqdockw{Definition}
\coqdef{Rewriting.stepeq}{step\_eq}{$\approx$}
(\coqdocvar{s} \coqdocvar{t} :
\coqref{Term.term}{\coqdocinductive{term}}) (\coqdocvar{$\pi$} :
\coqdocvar{s} \coqref{Rewriting.step}{$\rightarrow_\mathcal{R}$} \coqdocvar{t}) (\coqdocvar{u} \coqdocvar{v} :
\coqref{Term.term}{\coqdocinductive{term}}) (\coqdocvar{$o$} :
\coqdocvar{u} \coqref{Rewriting.step}{$\rightarrow_\mathcal{R}$}
\coqdocvar{v}) : \coqdockw{Prop} :=\coqdoceol
\coqdocindent{1.00em}
\coqdockw{match} \coqdocvariable{$\pi$}, \coqdocvariable{$o$}
\coqdockw{with}\coqdoceol
\coqdocindent{1.00em}
\ensuremath{|} \coqref{Rewriting.Step}{\coqdocconstructor{Step}}
\coqdocvar{\_} \coqdocvar{\_} \coqdocvar{$\rho$} \coqdocvar{C}
\coqdocvar{$\sigma$} \coqdocvar{\_} \coqdocvar{\_} \coqdocvar{\_},
\coqref{Rewriting.Step}{\coqdocconstructor{Step}} \coqdocvar{\_}
\coqdocvar{\_} \coqdocvar{$\rho'$} \coqdocvar{C$'$} \coqdocvar{$\sigma'$}
\coqdocvar{\_} \coqdocvar{\_} \coqdocvar{\_}
\ensuremath{\Rightarrow}\coqdoceol
\coqdocindent{2.00em}
\coqdocvariable{C} $\bis$ \coqdocvariable{C$'$} \ensuremath{\land}
\coqdocvariable{$\rho$} = \coqdocvariable{$\rho'$} \ensuremath{\land}
\coqref{Substitution.substitutioneq}{\coqdocdefinition{substitution\_eq}}
(\coqdocdefinition{vars}
(\coqref{Rewriting.lhs}{\coqdocprojection{lhs}} \coqdocvariable{$\rho$}))
\coqdocvariable{$\sigma$} \coqdocvariable{$\sigma'$}\coqdoceol
\coqdocindent{1.00em}
\coqdockw{end}.\coqdoceol
\end{coqdoccode}
\end{singlespace}
% TODO: naming of variables is not so nice here

We describe a way to define rewrite sequences as an inductive type. A rewrite
sequence of length $\alpha$ can be represented by the Brouwer ordinal $\alpha$
where we label every occurrence of the $^+$ constructor with a rewrite
step. To ensure that successive steps have the same target and source terms,
respectively, we include the source and target terms of the rewrite sequence
in its type and label accordingly.

At this point, it is not immediately clear what the type of the limit
constructor should be. Following the Brouwer ordinals, we think of a rewrite
sequence as a countably branching tree with every branching node representing
the least upper bound of its branches.
\begin{singlespace}
\begin{coqdoccode}
\coqdocnoindent
\coqdockw{Inductive}
\coqdef{Rewriting.sequence}{sequence}{$\rewrites_\mathcal{R}$} :
\coqref{Term.term}{\coqdocinductive{term}} \ensuremath{\rightarrow}
\coqref{Term.term}{\coqdocinductive{term}} \ensuremath{\rightarrow}
\coqdockw{Type} :=\coqdoceol
\coqdocindent{1.00em}
\ensuremath{|} \coqdef{Rewriting.Nil}{Nil}{\coqdocconstructor{Nil}} :
\ensuremath{\forall} \coqdocvar{t}, \coqdocvariable{t}
\coqref{Rewriting.sequence}{$\rewrites_\mathcal{R}$}
\coqdocvariable{t}\coqdoceol \coqdocindent{1.00em}
\ensuremath{|} \coqdef{Rewriting.Cons}{Cons}{\coqdocconstructor{Cons}} :
\ensuremath{\forall} \coqdocvar{s} \coqdocvar{t} \coqdocvar{u}, \coqdocvar{s}
\coqref{Rewriting.sequence}{$\rewrites_\mathcal{R}$} \coqdocvar{t}
$\rightarrow$
\coqdocvariable{t} \coqref{Rewriting.step}{$\rightarrow_\mathcal{R}$}
\coqdocvar{u} $\rightarrow$ \coqdocvariable{s}
\coqref{Rewriting.sequence}{$\rewrites_\mathcal{R}$}
\coqdocvariable{u}\coqdoceol \coqdocindent{1.00em}
\ensuremath{|} \coqdocconstructor{Lim}   :
\ensuremath{\forall} \coqdocvar{s} \coqdocvar{t},
(\coqexternalref{http://coq.inria.fr/stdlib/Coq.Init.Datatypes}{nat}{\coqdocinductive{nat}}
\ensuremath{\rightarrow} \coqdocvariable{s}
\coqref{Rewriting.sequence}{$\rewrites_\mathcal{R}$}
\coqdef{Rewriting.LimPlaceholder}{LimPlaceholder}{\textbf{?}}) $\rightarrow$
\coqdocvariable{s}
\coqref{Rewriting.sequence}{$\rewrites_\mathcal{R}$}
\coqdocvariable{t}.\coqdoceol
\end{coqdoccode}
\end{singlespace}

TODO: (in prelims) $\rewrites_\mathcal{R}$ binds stronger than
$\rightarrow$

This is not yet a satisfying definition, because we cannot fix a value for
\coqref{Rewriting.LimPlaceholder}{\textbf{?}}. We complete the type for
\coqref{Rewriting.Lim}{\coqdocconstructor{Lim}} as follows. First, we
parameterise it with the target terms of the branches. Second, we add the
condition that these terms must converge to the target term
\coqdocvariable{t}.
\begin{singlespace}
\begin{coqdoccode}
\coqdocindent{1.00em}
\ensuremath{|} \coqdef{Rewriting.Lim}{Lim}{\coqdocconstructor{Lim}} :
\ensuremath{\forall} \coqdocvar{s} \coqdocvar{t}
(\coqdocvar{ts} :
\coqexternalref{http://coq.inria.fr/stdlib/Coq.Init.Datatypes}{nat}{\coqdocinductive{nat}}
\ensuremath{\rightarrow} \coqref{Term.term}{\coqdocinductive{term}}),
(\ensuremath{\forall} \coqdocvar{n}, \coqdocvar{s}
\coqref{Rewriting.sequence}{$\rewrites_\mathcal{R}$}
\coqdocvar{ts} \coqdocvariable{n}) $\rightarrow$
\coqref{Rewriting.converges}{\coqdocdefinition{converges}} \coqdocvariable{ts}
\coqdocvariable{t} $\rightarrow$
\coqdocvariable{s}
\coqref{Rewriting.sequence}{$\rewrites_\mathcal{R}$}
\coqdocvariable{t}\coqdoceol
\end{coqdoccode}
\end{singlespace}
% TODO: maybe first convergence, then branches

Of course, the branches of a \coqref{Rewriting.Lim}{\coqdocconstructor{Lim}}
constructor may still not actually approximate to a rewrite sequence (of
length a limit ordinal). The intuition is that each branch should extend on
its preceeding ones. This would correspond to the
\coqref{WfOrdinal.wf}{\coqdocdefinition{wf}} property we defined on
\coqref{Ordinal.ord}{\coqdocinductive{ord}}, where we lift $\prec$ to a
strict prefix relation on
\coqref{Rewriting.sequence}{$\rewrites_\mathcal{R}$}.
We return to this issue in Section~\ref{sub:wf}, but
first consider the definition of an embedding relation on
\coqref{Rewriting.sequence}{$\rewrites_\mathcal{R}$}.


\subsection{Embeddings of Rewrite Sequences}

We lift the notions of predecessor and predecessor indices to the domain of
rewrite sequences. The set of predecessor indices is easily defined as
\coqref{Rewriting.predtype}{\coqdocdefinition{pred\_type}}.
%\footnote{We employ some notational overloading by reusing $I(\_)$ and
%  $[\_]\_$ for the corresponding definitions on rewrite sequences.}
\begin{singlespace}
\begin{coqdoccode}
\coqdocnoindent
\coqdockw{Fixpoint}
\coqdef{Rewriting.predtype}{pred\_type}{\coqdocdefinition{pred\_type}}
\coqdocvar{s} \coqdocvar{t}
(\coqdocvar{$\varphi$} : \coqdocvar{s}
\coqref{Rewriting.sequence}{$\rewrites_\mathcal{R}$} \coqdocvar{t}) :
\coqdockw{Type} :=\coqdoceol
\coqdocindent{1.00em}
\coqdockw{match} \coqdocvariable{$\varphi$} \coqdockw{with}\coqdoceol
\coqdocindent{1.00em}
\ensuremath{|} \coqref{Rewriting.Nil}{\coqdocconstructor{Nil}} \coqdocvar{\_}
\ensuremath{\Rightarrow}
\coqexternalref{http://coq.inria.fr/stdlib/Coq.Init.Logic}{False}{\coqdocinductive{False}}\coqdoceol
\coqdocindent{1.00em}
\ensuremath{|} \coqref{Rewriting.Cons}{\coqdocconstructor{Cons}}
\coqdocvar{\_} \coqdocvar{\_} \coqdocvar{\_} \coqdocvar{$\psi$} \coqdocvar{\_}
\ensuremath{\Rightarrow}
\coqexternalref{http://coq.inria.fr/stdlib/Coq.Init.Datatypes}{unit}{\coqdocinductive{unit}}
+ \coqref{Rewriting.predtype}{\coqdocdefinition{pred\_type}}
\coqdocvariable{$\psi$}\coqdoceol
\coqdocindent{1.00em}
\ensuremath{|} \coqref{Rewriting.Lim}{\coqdocconstructor{Lim}} \coqdocvar{\_}
\coqdocvar{\_} \coqdocvar{\_} \coqdocvar{f} \coqdocvar{\_}
\ensuremath{\Rightarrow} \{ \coqdocvar{n} :
\coqexternalref{http://coq.inria.fr/stdlib/Coq.Init.Datatypes}{nat}{\coqdocinductive{nat}}
\& \coqref{Rewriting.predtype}{\coqdocdefinition{pred\_type}}
(\coqdocvariable{f} \coqdocvariable{n}) \}\coqdoceol
\coqdocindent{1.00em}
\coqdockw{end}.\coqdoceol
\end{coqdoccode}
\end{singlespace}

The predecessor indices defined by
\coqref{Rewriting.predtype}{\coqdocdefinition{pred\_type}} point to a specific
occurrence of the \coqref{Rewriting.Cons}{\coqdocconstructor{Cons}}
constructor in a rewrite sequence. This constructor does not only contain a
rewrite sequence (analoguous to an ordinal in the
\coqref{Ordinal.ord}{\coqdocinductive{ord}} case), but also a rewrite
step. The \coqref{Rewriting.pred}{\coqdocdefinition{pred}} function gives us
both the rewrite sequence and the step. For the type checker to accept the
definition, we use a $\Sigma$-type that contains this pair, parameterised by
the source and target terms of the rewrite step.
\begin{singlespace}
\begin{coqdoccode}
\coqdocnoindent
\coqdockw{Fixpoint} \coqdef{Rewriting.pred}{pred}{\coqdocdefinition{pred}}
\coqdocvar{s} \coqdocvar{t} (\coqdocvar{$\varphi$} : \coqdocvar{s}
\coqref{Rewriting.sequence}{$\rewrites_\mathcal{R}$} \coqdocvar{t})
(\coqdocvar{$\iota$} : \coqref{Rewriting.predtype}{\coqdocdefinition{pred\_type}}
\coqdocvariable{$\varphi$})
:\coqdoceol \coqdocindent{2.00em}
\{ \coqdocvar{ts} :
\coqref{Term.term}{\coqdocabbreviation{term}} \ensuremath{\times}
\coqref{Term.term}{\coqdocabbreviation{term}} \&
(\coqdocvariable{s} \coqref{Rewriting.sequence}{$\rewrites_\mathcal{R}$}
\coqexternalref{http://coq.inria.fr/stdlib/Coq.Init.Datatypes}{fst}{\coqdocdefinition{fst}}
\coqdocvariable{ts}) \ensuremath{\times}
(\coqexternalref{http://coq.inria.fr/stdlib/Coq.Init.Datatypes}{fst}{\coqdocdefinition{fst}}
\coqdocvariable{ts} \coqref{Rewriting.step}{$\rightarrow_\mathcal{R}$}
\coqexternalref{http://coq.inria.fr/stdlib/Coq.Init.Datatypes}{snd}{\coqdocdefinition{snd}}
\coqdocvariable{ts}) \} :=\coqdoceol
\coqdocindent{1.00em}
\coqdockw{match} \coqdocvariable{$\varphi$} \coqdockw{with}\coqdoceol
\coqdocindent{1.00em}
\ensuremath{|} \coqref{Rewriting.Nil}{\coqdocconstructor{Nil}} \coqdocvar{\_}
\ensuremath{\Rightarrow}
(\coqexternalref{http://coq.inria.fr/stdlib/Coq.Init.Logic}{Falserect}{\coqdocdefinition{False\_rect}}
\coqdocvar{\_}) \coqdocvariable{$\iota$} \coqdoceol
\coqdocindent{1.00em}
\ensuremath{|} \coqref{Rewriting.Cons}{\coqdocconstructor{Cons}} \coqdocvar{\_}
\coqdocvar{u} \coqdocvar{t} \coqdocvar{$\psi$} \coqdocvar{$\pi$}
\ensuremath{\Rightarrow}
\coqdockw{match} \coqdocvariable{$\iota$} \coqdockw{with}\coqdoceol
\coqdocindent{10.00em}
\ensuremath{|}
\coqexternalref{http://coq.inria.fr/stdlib/Coq.Init.Datatypes}{inl}{\coqdocconstructor{inl}}
\coqexternalref{http://coq.inria.fr/stdlib/Coq.Init.Datatypes}{tt}{\coqdocconstructor{tt}}
\ensuremath{\Rightarrow}
\coqexternalref{http://coq.inria.fr/stdlib/Coq.Init.Specif}{existT}{\coqdocconstructor{existT}}
\coqdocvar{\_} (\coqdocvariable{u}, \coqdocvariable{t})
(\coqdocvariable{$\psi$}, \coqdocvariable{$\pi$})\coqdoceol
\coqdocindent{10.00em}
\ensuremath{|}
\coqexternalref{http://coq.inria.fr/stdlib/Coq.Init.Datatypes}{inr}{\coqdocconstructor{inr}}
\coqdocvar{$\kappa$}  \ensuremath{\Rightarrow}
\coqref{Rewriting.pred}{\coqdocdefinition{pred}} \coqdocvariable{$\psi$}
\coqdocvariable{$\kappa$}\coqdoceol
\coqdocindent{10.00em}
\coqdockw{end}\coqdoceol
\coqdocindent{1.00em}
\ensuremath{|} \coqref{Rewriting.Lim}{\coqdocconstructor{Lim}} \coqdocvar{\_}
\coqdocvar{\_} \coqdocvar{\_} \coqdocvar{f} \coqdocvar{\_}
\ensuremath{\Rightarrow}
\coqdockw{match} \coqdocvariable{$\iota$} \coqdockw{with}\coqdoceol
\coqdocindent{10.00em}
\ensuremath{|}
\coqexternalref{http://coq.inria.fr/stdlib/Coq.Init.Specif}{existT}{\coqdocconstructor{existT}}
\coqdocvar{n} \coqdocvar{$\kappa$} \ensuremath{\Rightarrow}
\coqref{Rewriting.pred}{\coqdocdefinition{pred}} (\coqdocvariable{f}
\coqdocvariable{n}) \coqdocvariable{$\kappa$}\coqdoceol
\coqdocindent{10.00em}
\coqdockw{end}\coqdoceol
\coqdocindent{1.00em}
\coqdockw{end}.\coqdoceol
\end{coqdoccode}
\end{singlespace}
% TODO: we simplified this (mainly type inference helpers)

In an effort to prevent getting lost in a syntactical labyrinth, we define
the following notational shortcuts:

% TODO: keep an eye on placement of the table
% TODO: maybe add real Coq code in a third column
{\renewcommand{\arraystretch}{1.1}
\renewcommand{\tabcolsep}{10pt}
\begin{tabular}{ll}
$\varphi[\iota]$ & location of $\varphi$ indexed by $\iota$\\
$\varphi[\iota]^\textsc{seq}$ & predecessor rewrite sequence of $\varphi$ indexed by $\iota$\\
$\varphi[\iota]^\textsc{stp}$ & step of $\varphi$ indexed by $\iota$\\
$\varphi[\iota]^\textsc{l}$ & source term of $\varphi[\iota]^\textsc{stp}$ (also target term
  of $\varphi[\iota]^\textsc{seq}$)\\
$\varphi[\iota]^\textsc{r}$ & target term of $\varphi[\iota]^\textsc{stp}$
\end{tabular}}

As an example of predecessor indexing, consider the graphical
representation of a rewrite sequence $\varphi$ of length $\omega + 2$
and its predecessor index $\iota = \coqdocconstructor{inr} \;
(\coqdocconstructor{inr} \; \langle 4,
\coqdocconstructor{inl}\rangle)$ in Figure~\ref{fig:pred}. The initial
part of length $\omega$ is represented by a series of finite rewrite
sequences, each one extending on the previous one by one step. The
sequence of terms $\{ t_1,t_2, t_3, \ldots \}$ converges to the term
$t_\omega$. Here, $\varphi[\iota]^\textsc{seq}$ is a rewrite sequence
from $t_1$ to $t_3$ and $\varphi[\iota]^\textsc{stp}$ is a step from
$t_3$ to $t_4$.

\begin{figure}
\begin{center}
\begin{tikzpicture}
  \node at (1, 0) {$t_{1}$};
  \foreach \i in {2, ..., 5}{%
    \begin{scope}[start chain=\i,every join/.style=->,node
        distance=0.5]
      \node [on chain=\i, join] at (1, -\i + 1) {$t_{1}$};
      \foreach \j in {2, ..., \i}{%
        \node (node\i\j) [on chain=\i, join] {$t_{\j}$};
      }
    \end{scope}
  }
  \node at (1, -5) {$\vdots$};
  \node at (5, -5) {$\ddots$};
  \begin{scope}[start chain=i,every join/.style=->,node distance=0.5]
    \node [on chain=i, join] at (1, -6) {$t_{1}$};
    \foreach \j in {2, ..., 5}{%
      \node [on chain=i, join] {$t_{\j}$};
    }
    \node [on chain=i, join] {$\cdots$};
    \node [on chain=i, join] {$t_{i}$};
  \end{scope}
  \node at (1, -7) {$\vdots$};
  \node at (7, -7) {$\ddots$};
  \begin{scope}[start chain=o,every join/.style=->,node distance=0.5]
    \node (nodeO) [on chain=o, join] at (10, -3.5) {$t_\omega$};
    \node (nodeO1) [on chain=o, join] {$t_{\omega+1}$};
    \node (nodeO2) [on chain=o, join] {$t_{\omega+2}$};
  \end{scope}
  \begin{scope}[auto]
    \draw [->, thick] (nodeO2) to [out=110,in=70] node [swap]
          {\coqdocconstructor{inr}}
          (nodeO1);
    \draw [->, thick] (nodeO1) to [out=110,in=70] node [swap]
          {\coqdocconstructor{inr}}
          (nodeO);
    \draw [->, thick] (nodeO) to [out=100,in=50] node [swap] {4}
    (node44);
    \draw [->, thick] (node44) to [out=110,in=70] node [swap]
          {\coqdocconstructor{inl}}
          (node43);
  \end{scope}

\end{tikzpicture}
\end{center}
\caption{Example of a rewrite sequence and predecessor
  index.}\label{fig:pred}
\end{figure}

% TODO: better wording for 'canceled out'?
% TODO: picture of this embedding behaviour?
Having a closer look at the order $\preceq$ on the Brouwer ordinals, we can
see that it really defines embeddings of their tree structures. This is due to
clause (\ref{def:order:succ}) of Definition~\ref{def:order}. In this clause,
two occurrences of the $^+$ constructor (one in both ordinals) are cancelled
out against each other, but the positions of these occurences in their
respective ordinals do not necessarily correspond. Since occurrences
of $^+$ carry no additional information, this has no effect on the resulting
relation.

% TODO: should we use the word 'iff'?
What this means for a translation of $\preceq$ to the domain of our
inductively defined rewrite sequences is that, indeed, we get an embedding
relation. We only have to make sure that in the
\coqref{Rewriting.Cons}{\coqdocconstructor{Cons}} case, we cancel out two
equal steps against each other.
We say that $\varphi$ is embedded in $\psi$ (written $\varphi
\sqsubseteq \psi$) if $\psi$ can be obtained from $\varphi$ by inserting
any number of steps in $\varphi$. We distinguish between inserting a step
\begin{inparaenum}[(i)]
  \item before the first step,
  \item after the last step and
  \item in between steps
\end{inparaenum}
in a rewrite sequence. Note that any steps inserted consecutively in between
steps necessarily form a cycle, because of the typing constraints in
the definition of rewrite sequence.
% TODO: is the wording good here (the three cases)?
% TODO: define cycle
% TODO: explaining image
\begin{singlespace}
\begin{coqdoccode}
\coqdocnoindent
\coqdockw{Inductive} \coqdef{Rewriting.embed}{embed}{$\sqsubseteq$}
: \ensuremath{\forall} \coqdocvar{s} \coqdocvar{t} \coqdocvar{u}
\coqdocvar{v}, \coqdocvariable{s}
\coqref{Rewriting.sequence}{$\rewrites_\mathcal{R}$} \coqdocvariable{t}
$\rightarrow$ \coqdocvariable{u}
\coqref{Rewriting.sequence}{$\rewrites_\mathcal{R}$} \coqdocvariable{v}
$\rightarrow$ \coqdockw{Prop} :=\coqdoceol
\coqdocindent{1.00em}
\ensuremath{|}
\coqdef{Rewriting.EmbedNil}{Embed\_Nil}{\coqdocconstructor{$\sqsubseteq_\text{Nil}$}}  :
\ensuremath{\forall} \coqdocvar{s} \coqdocvar{u} \coqdocvar{v} (\coqdocvar{$\psi$}
: \coqdocvariable{u} \coqref{Rewriting.sequence}{$\rewrites_\mathcal{R}$}
\coqdocvariable{v}),\coqdoceol
\coqdocindent{9.50em}
\coqref{Rewriting.Nil}{\coqdocconstructor{Nil}} \coqdocvariable{s}
\coqref{Rewriting.embed}{$\sqsubseteq$} \coqdocvariable{$\psi$}\coqdoceol
\coqdocindent{1.00em}
\ensuremath{|}
\coqdef{Rewriting.EmbedCons}{Embed\_Cons}{\coqdocconstructor{$\sqsubseteq_\text{Cons}$}} :
\ensuremath{\forall} \coqdocvar{s} \coqdocvar{t} \coqdocvar{u} \coqdocvar{v}
(\coqdocvar{$\psi$}
: \coqdocvariable{u} \coqref{Rewriting.sequence}{$\rewrites_\mathcal{R}$}
\coqdocvariable{v}) (\coqdocvar{$\iota$} :
\coqref{Rewriting.predtype}{\coqdocdefinition{pred\_type}}
\coqdocvariable{$\psi$})
(\coqdocvar{$\varphi$} : \coqdocvariable{s}
\coqref{Rewriting.sequence}{$\rewrites_\mathcal{R}$}
\coqdocvariable{$\psi$}[\coqdocvariable{$\iota$}]$^\textsc{l}$)\coqdoceol
\coqdocindent{5.00em}
(\coqdocvar{$\pi$} :
\coqdocvariable{$\psi$}[\coqdocvariable{$\iota$}]$^\textsc{l}$
\coqref{Rewriting.step}{$\rightarrow_\mathcal{R}$}
\coqdocvariable{t}),\coqdoceol
\coqdocindent{9.50em}
\coqdocvariable{$\varphi$} \coqref{Rewriting.embed}{$\sqsubseteq$}
\coqdocvariable{$\psi$}[\coqdocvariable{$\iota$}]$^\textsc{seq}$
\ensuremath{\rightarrow}\coqdoceol
\coqdocindent{9.50em}
\coqdocvariable{$\pi$} \coqref{Rewriting.stepeq}{$\approx$}
\coqdocvariable{$\psi$}[\coqdocvariable{$\iota$}]$^\textsc{stp}$
\ensuremath{\rightarrow}\coqdoceol
\coqdocindent{9.50em}
\coqref{Rewriting.Cons}{\coqdocconstructor{Cons}}
\coqdocvariable{$\varphi$} \coqdocvariable{$\pi$}
\coqref{Rewriting.embed}{$\sqsubseteq$}
\coqdocvariable{$\psi$}\coqdoceol
\coqdocindent{1.00em}
\ensuremath{|}
\coqdef{Rewriting.EmbedLim}{Embed\_Lim}{\coqdocconstructor{$\sqsubseteq_\text{Lim}$}}  :
\ensuremath{\forall} \coqdocvar{s} \coqdocvar{t} \coqdocvar{u} \coqdocvar{v}
(\coqdocvar{ts} :
\coqexternalref{http://coq.inria.fr/stdlib/Coq.Init.Datatypes}{nat}{\coqdocinductive{nat}}
\ensuremath{\rightarrow} \coqref{Term.term}{\coqdocinductive{term}})
(\coqdocvar{f} : \ensuremath{\forall} \coqdocvar{n},
\coqdocvariable{s}
\coqref{Rewriting.sequence}{$\rewrites_\mathcal{R}$}
\coqdocvariable{ts} \coqdocvariable{n})\coqdoceol
\coqdocindent{5.00em}
(\coqdocvar{c} :
\coqref{Rewriting.converges}{\coqdocdefinition{converges}} \coqdocvariable{ts}
\coqdocvar{t}) (\coqdocvar{$\psi$} : \coqdocvar{u}
\coqref{Rewriting.sequence}{$\rewrites_\mathcal{R}$}
\coqdocvar{v}),\coqdoceol
\coqdocindent{9.50em}
(\ensuremath{\forall} \coqdocvar{n}, (\coqdocvariable{f} \coqdocvariable{n})
\coqref{Rewriting.embed}{$\sqsubseteq$} \coqdocvariable{$\psi$})
\ensuremath{\rightarrow}\coqdoceol
\coqdocindent{9.50em}
\coqref{Rewriting.Lim}{\coqdocconstructor{Lim}} \coqdocvariable{f}
\coqdocvariable{c} \coqref{Rewriting.embed}{$\sqsubseteq$}
\coqdocvariable{$\psi$}.\coqdoceol
\end{coqdoccode}
\end{singlespace}
% TODO: (f n) without ()
% TODO: say what implicit/explicit arguments for cons and lim are

TODO: Strict embedding relation. It really expands after the last step!
\begin{singlespace}
\begin{coqdoccode}
\coqdocnoindent
\coqdockw{Definition}
\coqdef{Rewriting.embedstrict}{embed\_strict}{$\sqsubset$}
\coqdocvar{s} \coqdocvar{t} \coqdocvar{u} \coqdocvar{v}
(\coqdocvar{$\varphi$} : \coqdocvariable{s}
\coqref{Rewriting.sequence}{$\rewrites_\mathcal{R}$}
\coqdocvariable{t},
\coqdocvar{$\psi$} :
\coqdocvariable{u}
\coqref{Rewriting.sequence}{$\rewrites_\mathcal{R}$}
\coqdocvariable{v}) := \ensuremath{\exists} \coqdocvar{$\iota$},
\coqdocvariable{$\varphi$} $\sqsubseteq$
\coqdocvariable{$\psi$}[\coqdocvariable{$\iota$}]$^\textsc{seq}$.\coqdoceol
\end{coqdoccode}
\end{singlespace}


\subsection{Well-formed Rewrite Sequences and Convergence}\label{sub:wf}

On \coqref{Ordinal.ord}{\coqdocinductive{ord}} we defined the
\coqref{WfOrdinal.wf}{\coqdocdefinition{wf}} property to rule out a certain
class of ordinal representations. This issue translates directly to our
inductive representation of rewrite sequences.

$\sqsubset$ turns out to be satisfying for our present purposes.

TODO: think harder about what it means to have an embedding relation instead
of a prefix relation. Vincent said this about it:
\begin{quote}
Overigens bedachten dat het in principe niet heel erg is $\sqsubseteq$ te
definieren voor reducties zoals voor ordinalen. alleen zie je dan het meer een
notie van embedding/deelreductie ipv een notie van prefix geeft, maar ook daar
kun je denkelijk goed mee werken.

\ldots

als je een stuk invoegt in het midden in sigma moet dat noodzakelijkerwijs,
vanwege de constraints op begin-en eindpunten, een reductie cykel zijn. ik zou
verwachten dat dat uiteindelijk een goede notie van ordening (goede reducties)
oplevert (``de cykels doen er niet toe voor convergente rijen, en kun je
weglaten bij compressie''), die natuurlijk niet overeenkomt, ook niet in het
eindige geval met de prefix notie; het is deelwoord, of (op reductie rijtjes
als bomen) deelboom (en niet boom-factor).
\end{quote}

\begin{singlespace}
\begin{coqdoccode}
\coqdocnoindent
\coqdockw{Fixpoint} \coqdef{Rewriting.wf}{wf}{\coqdocdefinition{wf}}
\coqdocvar{s} \coqdocvar{t}
(\coqdocvar{$\varphi$} : \coqdocvariable{s}
\coqref{Rewriting.sequence}{$\rewrites_\mathcal{R}$}
\coqdocvariable{t}) : \coqdockw{Prop}
:=\coqdoceol
\coqdocindent{1.00em}
\coqdockw{match} \coqdocvariable{$\varphi$} \coqdockw{with}\coqdoceol
\coqdocindent{1.00em}
\ensuremath{|} \coqref{Rewriting.Nil}{\coqdocconstructor{Nil}}
\coqdocvar{\_}          \ensuremath{\Rightarrow}
\coqexternalref{http://coq.inria.fr/stdlib/Coq.Init.Logic}{True}{\coqdocinductive{True}}\coqdoceol
\coqdocindent{1.00em}
\ensuremath{|} \coqref{Rewriting.Cons}{\coqdocconstructor{Cons}}
\coqdocvar{\_} \coqdocvar{\_} \coqdocvar{$\psi$} \coqdocvar{\_}
\coqdocvar{\_} \ensuremath{\Rightarrow}
\coqref{Rewriting.wf}{\coqdocdefinition{wf}}
\coqdocvariable{\coqdocvariable{$\psi$}}\coqdoceol
\coqdocindent{1.00em}
\ensuremath{|} \coqref{Rewriting.Lim}{\coqdocconstructor{Lim}}
\coqdocvar{\_} \coqdocvar{\_} \coqdocvar{f} \coqdocvar{\_}
\coqdocvar{\_}  \ensuremath{\Rightarrow}
(\ensuremath{\forall} \coqdocvar{n},
\coqref{Rewriting.wf}{\coqdocdefinition{wf}} (\coqdocvariable{f}
\coqdocvariable{n})) \ensuremath{\land}
\ensuremath{\forall} \coqdocvar{n} \coqdocvar{m}, \coqdocvariable{n}
< \coqdocvariable{m} \ensuremath{\rightarrow} \coqdocvariable{f}
\coqdocvariable{n}
\coqref{Rewriting.embedstrict}{$\sqsubset$}
\coqdocvariable{f} \coqdocvariable{m}\coqdoceol
\coqdocindent{1.00em}
\coqdockw{end}.\coqdoceol
\end{coqdoccode}
\end{singlespace}


\section{Properties of Terms and TRSs}

We define some predicates on terms and TRSs. Again, we let
$\mathcal{R}$ be a fixed TRS throughout this section.

We work with a somewhat relaxed definition of critical pairs. First,
we do not require the common instance to be a most general
one. Second, the substitution $\sigma$ might not be minimal and might
not introduce only fresh variables
(cf.\ Definition~\ref{def:overlap}). The effect of this relaxation is
that for every critical pair, we have a series of critical pairs by
this \Coq definition. This is precise enough for our present
purposes, however, since it has no effect on questions such as
\emph{`are there critical pairs?'} or \emph{`are all critical pairs
  trivial?'}.
\begin{singlespace}
\begin{coqdoccode}
\coqdocnoindent
\coqdockw{Definition}
\coqdef{Rewriting.criticalpair}{critical\_pair}{\coqdocdefinition{critical\_pair}}
(\coqdocvar{$\mathcal{R}$} : \coqdocdefinition{trs})
(\coqdocvar{t$_1$} \coqdocvar{t$_2$} :
\coqref{Term.term}{\coqdocinductive{term}}) : \coqdockw{Prop}
:=\coqdoceol
\coqdocindent{1.00em}
\ensuremath{\exists} \coqdocvar{$\rho_1$} :
\coqdocrecord{rule}, \ensuremath{\exists}
\coqdocvar{$\rho_2$} :
\coqdocrecord{rule},
\ensuremath{\exists} \coqdocvar{p} :
\coqdocabbreviation{position},
\ensuremath{\exists} \coqdocvar{$\sigma$},
\ensuremath{\exists} \coqdocvar{$\tau$},\coqdoceol
\coqdocindent{3.00em}
\coqdocvariable{$\rho_1$} \coqdocdefinition{$\in$}
\coqdocvariable{$\mathcal{R}$}
\ensuremath{\land}
\coqdocvariable{$\rho_2$} \coqdocdefinition{$\in$}
\coqdocvar{$\mathcal{R}$} \ensuremath{\land}
(\coqdocvariable{$\rho_1$} = \coqdocvariable{$\rho_2$}
\ensuremath{\rightarrow}
\coqdocvariable{p} \ensuremath{\not=}
\coqexternalref{http://coq.inria.fr/stdlib/Coq.Init.Datatypes}{nil}{\coqdocconstructor{nil}})
\ensuremath{\land}\coqdoceol
\coqdocindent{3.00em}
\coqdockw{match}
\coqref{Term.subterm}{\coqdocdefinition{subterm}} (\coqref{Rewriting.lhs}{\coqdocprojection{lhs}}
\coqdocvariable{$\rho_1$}) \coqdocvariable{p},
\coqref{Context.dig}{\coqdocdefinition{dig}} (\coqref{Rewriting.lhs}{\coqdocprojection{lhs}}
\coqdocvariable{$\rho_1$})$^{\coqdocvariable{$\sigma$}}$ \coqdocvariable{p}
\coqdockw{with}\coqdoceol
\coqdocindent{3.00em}
\ensuremath{|}
\coqexternalref{http://coq.inria.fr/stdlib/Coq.Init.Datatypes}{Some}{\coqdocconstructor{Some}}
\coqdocvar{s},
\coqexternalref{http://coq.inria.fr/stdlib/Coq.Init.Datatypes}{Some}{\coqdocconstructor{Some}}
\coqdocvar{C} \ensuremath{\Rightarrow}
\coqdocdefinition{is\_var} \coqdocvariable{s} =
\coqexternalref{http://coq.inria.fr/stdlib/Coq.Init.Datatypes}{false}{\coqdocconstructor{false}}
\ensuremath{\land}
\coqdocvariable{s}$^{\coqdocvariable{$\sigma$}}$ \coqref{TermEquality.termbis}{$\bis$}
(\coqref{Rewriting.lhs}{\coqdocprojection{lhs}}
\coqdocvariable{$\rho_2$})$^{\coqdocvariable{$\tau$}}$
\ensuremath{\land}\coqdoceol
\coqdocindent{12.00em}
\coqdocvariable{t$_1$} \coqref{TermEquality.termbis}{$\bis$}
\coqdocvariable{C}[(\coqref{Rewriting.rhs}{\coqdocprojection{rhs}}
\coqdocvariable{$\rho_2$})$^{\coqdocvariable{$\tau$}}$]
\ensuremath{\land}
\coqdocvariable{t$_2$} \coqref{TermEquality.termbis}{$\bis$}
(\coqref{Rewriting.rhs}{\coqdocprojection{rhs}}
\coqdocvariable{$\rho_1$})$^{\coqdocvariable{$\sigma$}}$\coqdoceol
\coqdocindent{3.00em}
\ensuremath{|} \coqdocvar{\_}, \coqdocvar{\_}
\ensuremath{\Rightarrow}
\coqexternalref{http://coq.inria.fr/stdlib/Coq.Init.Logic}{False}{\coqdocinductive{False}}\coqdoceol
\coqdocindent{3.00em}
\coqdockw{end}.\coqdoceol
\end{coqdoccode}
\end{singlespace}
Now we can in a straightforward manner define the properties of
orthogonality and weak orhogonality.
\begin{singlespace}
\begin{coqdoccode}
\coqdocnoindent
\coqdockw{Definition}
\coqdef{Rewriting.orthogonal}{orthogonal}{\coqdocdefinition{orthogonal}} (\coqdocvar{$\mathcal{R}$} :
\coqdocdefinition{trs})
: \coqdockw{Prop} :=\coqdoceol
\coqdocindent{1.00em}
\coqdocdefinition{trs\_left\_linear} % TODO: maybe define trs_left_linear and link to it
\coqdocvariable{$\mathcal{R}$} \ensuremath{\land}
\ensuremath{\forall} \coqdocvar{t$_1$} \coqdocvar{t$_2$},
\ensuremath{\lnot}
\coqref{Rewriting.criticalpair}{\coqdocdefinition{critical\_pair}}
\coqdocvariable{t$_1$} \coqdocvariable{t$_2$}.\coqdoceol
\coqdocemptyline
\coqdocnoindent
\coqdockw{Definition}
\coqdef{Rewriting.weaklyorthogonal}{weakly\_orthogonal}{\coqdocdefinition{weakly\_orthogonal}} (\coqdocvar{$\mathcal{R}$} :
\coqdocdefinition{trs})
: \coqdockw{Prop} :=\coqdoceol
\coqdocindent{1.00em}
\coqdocdefinition{trs\_left\_linear}
\coqdocvariable{$\mathcal{R}$} \ensuremath{\land}
\ensuremath{\forall} \coqdocvar{t$_1$} \coqdocvar{t$_2$},
\coqref{Rewriting.criticalpair}{\coqdocdefinition{critical\_pair}}
\coqdocvariable{t$_1$} \coqdocvariable{t$_2$} \ensuremath{\rightarrow}
\coqdocvariable{t$_1$} \coqref{TermEquality.termbis}{$\bis$} \coqdocvariable{t$_2$}.\coqdoceol
\end{coqdoccode}
\end{singlespace}

Next we define when a term is a normal form and when we have unique
normal forms.
\begin{singlespace}
\begin{coqdoccode}
\coqdocnoindent
\coqdockw{Definition}
\coqdef{Rewriting.normalform}{normal\_form}{\coqdocdefinition{normal\_form}}
\coqdocvar{t} : \coqdockw{Prop} :=\coqdoceol
\coqdocindent{1.00em}
\ensuremath{\lnot} \ensuremath{\exists} \coqdocvar{C} : \coqref{Context.context}{\coqdocinductive{context}},
\ensuremath{\exists} \coqdocvar{$\rho$} : \coqdocrecord{rule},
\ensuremath{\exists} \coqdocvar{$\sigma$} :
\coqref{Substitution.substitution}{\coqdocdefinition{substitution}},
\coqdocvariable{$\rho$}
\coqexternalref{http://coq.inria.fr/stdlib/Coq.Lists.List}{In}{\coqdocdefinition{$\in$}}
\coqdocvar{$\mathcal{R}$} \ensuremath{\land}
\coqdocvariable{C}[(\coqref{Rewriting.lhs}{\coqdocprojection{lhs}}
\coqdocvariable{r})\coqdocvariable{$^\sigma$}] \coqref{TermEquality.termbis}{$\bis$}
\coqdocvariable{t}.\coqdoceol
\coqdocemptyline
\coqdocnoindent
\coqdockw{Definition}
\coqdef{Rewriting.uniquenormalforms}{unique\_normal\_forms}{\coqdocdefinition{unique\_normal\_forms}}
: \coqdockw{Prop} :=\coqdoceol
\coqdocindent{1.00em}
\ensuremath{\forall} \coqdocvar{s} \coqdocvar{t} \coqdocvar{u}
(\coqdocvar{$\varphi$} : \coqdocvariable{s} \coqref{Rewriting.sequence}{$\rewrites_\mathcal{R}$} \coqdocvariable{t})
(\coqdocvar{$\psi$} : \coqdocvariable{s} \coqref{Rewriting.sequence}{$\rewrites_\mathcal{R}$} \coqdocvariable{u}),\coqdoceol
\coqdocindent{2.00em}
\coqref{Rewriting.wf}{\coqdocdefinition{wf}} \coqdocvariable{$\varphi$}
\ensuremath{\rightarrow}
\coqref{Rewriting.wf}{\coqdocdefinition{wf}} \coqdocvariable{$\psi$}
\ensuremath{\rightarrow}
\coqref{Rewriting.normalform}{\coqdocdefinition{normal\_form}}
\coqdocvariable{t} \ensuremath{\rightarrow}
\coqref{Rewriting.normalform}{\coqdocdefinition{normal\_form}}
\coqdocvariable{u} \ensuremath{\rightarrow}
\coqdocvariable{t} \coqref{TermEquality.termbis}{$\bis$} \coqdocvariable{u}.\coqdoceol
\end{coqdoccode}
\end{singlespace}
Note that the
\coqref{Rewriting.uniquenormalforms}{\coqdocdefinition{unique\_normal\_forms}}
definition is only a translation of the $UN^\rewrites$ property, not
of the more general $UN^\infty$ property (see also
Definition~\ref{def:normalisation}).

% TODO: note in chapter unwo that we proved ~UN^\infty with ~UN^->

\chapter[\texorpdfstring{UN$^\infty$ in Weakly Orthogonal Systems}{UN
  in Weakly Orthogonal Systems}]{\texorpdfstring{Unique Normal Forms
    in\\Weakly Orthogonal Systems}{Unique Normal Forms in Weakly
    Orthogonal Systems}}\label{chap:unwo}

TODO: this intro

Every orthogonal TRS exhibits the infinitary unique normal forms
(UN$^\infty$) property. We might expect this property to generalize to
weakly orthogonal systems. However, this does not turn out to be the
case.


\section{A Counterexample}\label{sec:counterexample}

% TODO: is the naming D,U really a good idea?

We describe a simple counterexample showing that weak orthogonality
does not imply the UN$^\infty$ property \citep{endrullis-10}.

We work in a signature with unary function symbols $D$ and
$U$.\footnote{We can think of $D$ and $U$ as `down' and `up'. The
  original formulation of this TRS uses $P$ and $S$ (`predecessor' and
  `successor'), but to avoid notational conflicts with the
  \coqexternalref{http://coq.inria.fr/stdlib/Coq.Init.Datatypes}{S}{\coqdocconstructor{S}}
  constructor for
  \coqexternalref{http://coq.inria.fr/stdlib/Coq.Init.Datatypes}{nat}{\coqdocinductive{nat}}
  in \Coq, we proceed with this modification.}
In the notation of terms, we omit the brackets around arguments and
assume right-associativity of function symbol application,
e.g.\ writing $DU$ for $D(U(x))$. A notation for finite repetitions of
a function symbol $f$ terminated by a term $t$ is defined by
\begin{inparaenum}[(i)]
\item $f^0 t = t$ and
\item $f^{n+1} = ff^nt$.
\end{inparaenum}
The infinite nesting $fff \ldots$ of $f$ is written $f^\omega$.
% TODO: nesting vs repetition

Consider the TRS consisting of the two left-linear rewrite rules
$\rho_1$ and $\rho_2$:
\begin{align*}
  \rho_1 \, : \, DUx \to x \qquad \qquad \qquad
  \rho_2 \, : \, UDx \to x
\end{align*}
This system has two critical pairs $\langle Dx, Dx \rangle$ and
$\langle Ux, Ux \rangle$, both of which are trivial, establishing
weak orthogonality. The infinite term $\psi = D^1 U^2 D^3 U^4 \ldots$
has two normal forms. It rewrites to $U^\omega$ in $\omega$ many
$\rho_1$-steps and to $D^\omega$ in $\omega$ many $\rho_2$-steps.

Other interesting properties of this TRS (e.g.\ weak normalisation is
not preserved under rewriting) and a translation to the infinitary
$\lambda \beta \eta$-calculus are discussed by \citet{endrullis-10}.


\subsection{\texorpdfstring{Rewriting $\psi$ to $U^\omega$}{Rewriting
    DUUDDD... to UUU...}}\label{sub:counterexample}

We show briefly what rewriting $\psi$ to $U^\omega$ amounts
to. Rewriting $\psi$ to $D^\omega$ is done in a similar way.
An obvious way to define $\psi$ by corecursion is via auxiliary terms
$\psi'_n$ parameterized by $n$ as follows:
\begin{align*}
  \psi'_n = U^n D^{n + 1} \psi'_{n + 2} \qquad \qquad \qquad
  \psi = \psi'_0
\end{align*}
But a more useful definition for our present purposes, and the one we
stick with, is the slight reformulation:
\begin{align*}
  \psi'_n = D^{2 n + 1} U^{2 n + 2} \psi'_{n + 1} \qquad
  \qquad \qquad
  \psi = \psi'_0
\end{align*}
For any term $t$ and natural numbers $n$ and $m$ we have $U^n D^{m+1}
U^{m+1} t \rightarrow_{\rho_1} U^n D^m U^m t$ and thus $U^n D^m U^m t
\twoheadrightarrow U^n t$ by iterating $m$ such steps. Instantiating
$m$ with $2 n + 1$ and $t$ with $U \psi'_{n + 1}$, we obtain
%$S^n P^{2 n + 1} S^{2 n + 1} S \psi'_{n + 1} \equiv S^n \psi'_n$
$U^n \psi'_n \twoheadrightarrow U^{n+1} \psi'_{n + 1}$ for any $n$.
Concatenating these sequences, iterating $n$ from $0$ onwards, we
arrive at $\psi \twoheadrightarrow U^\omega$.
% TODO: wording 'iterating n from 0 onwards'


\section{The Counterexample in \Coq}

We implement the counterexample from Section~\ref{sec:counterexample}
using the \Coq development described in
Chapter~\ref{chap:implementation}.

The rewrite rules $\rho_1$ and $\rho_2$ are straightforwardly defined
and shown left-linear. By a simple proof we obtain that all critical
pairs are trivial and hence that the TRS is weakly orthogonal.
\begin{singlespace}
\begin{coqdoccode}
%\coqdocnoindent
%\coqdockw{Definition}
%\coqdef{ExampleUNWO.UNWOtrs}{UNWO\_trs}{\coqdocdefinition{$\mathcal{R}$}}
%:=
%\coqdocdefinition{$\rho_1$} ::
%\coqdocdefinition{$\rho_2$} ::
%\coqexternalref{http://coq.inria.fr/stdlib/Coq.Init.Datatypes}{nil}{\coqdocconstructor{nil}}.\coqdoceol
%\coqdocemptyline
\coqdocnoindent
\coqdockw{Lemma}
\coqdoclemma{wo$_\mathcal{R}$} :
\coqdocdefinition{weakly\_orthogonal} % TODO: link to definition
%\coqref{ExampleUNWO.UNWOtrs}{\coqdocdefinition{$\mathcal{R}$}}.\coqdoceol
\coqdocdefinition{$\mathcal{R}$}.\coqdoceol
\end{coqdoccode}
\end{singlespace}
We introduce the notation \begin{coqdoccode}\coqdocvariable{f} @
  \coqdocvariable{t}\end{coqdoccode} to mean
\begin{coqdoccode}\coqref{Term.Fun}{\coqdocconstructor{Fun}}
  \coqdocvariable{f} (\coqdocdefinition{vcons} \coqdocvariable{t}
  (\coqdocdefinition{vnil}
  \coqref{Term.term}{\coqdocinductive{term}}))\end{coqdoccode}. For
brevity, mirrored constructions for both function symbols are only
discussed for one of them. The infinite term $U^\omega$ is defined by
corecursion and finite repetitions $U^n t$ are defined by recursion
(and are assumed to generalize to contexts with the same notation).
% TODO: wording of the generalization to contexts
\begin{singlespace}
\begin{coqdoccode}
\coqdocnoindent
\coqdockw{CoFixpoint}
\coqdef{ExampleUNWO.repeatU}{repeat\_U}{\coqdocdefinition{U$^\omega$}}
: \coqref{Term.term}{\coqdocinductive{term}} :=
\coqdocconstructor{U} @
\coqref{ExampleUNWO.repeatU}{\coqdocdefinition{U$^\omega$}}.\coqdoceol
\coqdocemptyline
\coqdocnoindent
\coqdockw{Fixpoint}
\coqdef{ExampleUNWO.Unt}{Unt}{\coqdocdefinition{U}}$^\coqdocvar{n}$
\coqdocvar{t} :=\coqdoceol
\coqdocindent{1.00em}
\coqdockw{match} \coqdocvariable{n} \coqdockw{with}\coqdoceol
\coqdocindent{1.00em}
\ensuremath{|}
\coqexternalref{http://coq.inria.fr/stdlib/Coq.Init.Datatypes}{O}{\coqdocconstructor{O}}
\ensuremath{\Rightarrow} \coqdocvariable{t}\coqdoceol
\coqdocindent{1.00em}
\ensuremath{|}
\coqexternalref{http://coq.inria.fr/stdlib/Coq.Init.Datatypes}{S}{\coqdocconstructor{S}}
\coqdocvar{n} \ensuremath{\Rightarrow}
\coqdocconstructor{U} @
(\coqref{ExampleUNWO.Unt}{\coqdocdefinition{U}}$^\coqdocvariable{n}$
\coqdocvariable{t})\coqdoceol
\coqdocindent{1.00em}
\coqdockw{end}.\coqdoceol
\end{coqdoccode}
\end{singlespace}
Unfortunately, $\psi$ is not as easily defined. Although clearly
productive, direct translations of the corecursions in
Section~\ref{sub:counterexample} do not satisfy \Coq's guardedness
condition (see also Section~\ref{sub:guardedness}). The conclusion of
a \emph{trial and error} approach is that we must use anonymous cofix
constructions. The definition we proceed with is the following.
% TODO: wording 'trial and error approach'
\begin{singlespace}
\begin{coqdoccode}
\coqdocnoindent
\coqdockw{CoFixpoint}
\coqdef{ExampleUNWO.psi'}{psi'}{\coqdocdefinition{$\psi'$}} \coqdocvar{n}
: \coqref{Term.term}{\coqdocinductive{term}} :=\coqdoceol
\coqdocindent{1.00em}
(\coqdocvar{cofix} \coqdocvar{Ds} (\coqdocvar{d} :
\coqexternalref{http://coq.inria.fr/stdlib/Coq.Init.Datatypes}{nat}{\coqdocinductive{nat}})
:=\coqdoceol
\coqdocindent{2.00em}
\coqdockw{match} \coqdocvariable{d} \coqdockw{with}\coqdoceol
\coqdocindent{2.00em}
\ensuremath{|}
\coqexternalref{http://coq.inria.fr/stdlib/Coq.Init.Datatypes}{O}{\coqdocconstructor{O}}
\ensuremath{\Rightarrow} \coqdocconstructor{D}
@ (\coqdocvar{cofix} \coqdocvar{Us} (\coqdocvar{u} :
\coqexternalref{http://coq.inria.fr/stdlib/Coq.Init.Datatypes}{nat}{\coqdocinductive{nat}})
:=\coqdoceol
\coqdocindent{7.50em}
\coqdockw{match} \coqdocvariable{u} \coqdockw{with}\coqdoceol
\coqdocindent{7.50em}
\ensuremath{|}
\coqexternalref{http://coq.inria.fr/stdlib/Coq.Init.Datatypes}{O}{\coqdocconstructor{O}}
\ensuremath{\Rightarrow}
\coqref{ExampleUNWO.psi'}{\coqdocdefinition{$\psi'$}}
(\coqexternalref{http://coq.inria.fr/stdlib/Coq.Init.Datatypes}{S}{\coqdocconstructor{S}}
\coqdocvariable{n})\coqdoceol
\coqdocindent{7.50em}
\ensuremath{|}
\coqexternalref{http://coq.inria.fr/stdlib/Coq.Init.Datatypes}{S}{\coqdocconstructor{S}}
\coqdocvar{u} \ensuremath{\Rightarrow}
\coqdocconstructor{U} @
\coqdocconstructor{U} @ (\coqdocvariable{Us}
\coqdocvariable{u})\coqdoceol
\coqdocindent{7.50em}
\coqdockw{end})
(\coqexternalref{http://coq.inria.fr/stdlib/Coq.Init.Datatypes}{S}{\coqdocconstructor{S}}
\coqdocvariable{n})\coqdoceol
\coqdocindent{2.00em}
\ensuremath{|}
\coqexternalref{http://coq.inria.fr/stdlib/Coq.Init.Datatypes}{S}{\coqdocconstructor{S}}
\coqdocvar{d} \ensuremath{\Rightarrow}
\coqdocconstructor{D} @
\coqdocconstructor{D} @
(\coqdocvariable{Ds} \coqdocvariable{d})\coqdoceol
\coqdocindent{2.00em}
\coqdockw{end}) \coqdocvariable{n}.\coqdoceol
\coqdocemptyline
\coqdocnoindent
\coqdockw{Definition}
\coqdef{ExampleUNWO.psi}{psi}{\coqdocdefinition{$\psi$}} :=
\coqref{ExampleUNWO.psi'}{\coqdocdefinition{$\psi'$}} 0.\coqdoceol
\end{coqdoccode}
\end{singlespace}

We now prove that $U^\omega$ and $D^\omega$ are (distinct) normal
forms. This is essentially done by exhaustive case analysis of
the position of redex occurrences in the terms, yielding that there
can not be such an occurence.
\begin{singlespace}
\begin{coqdoccode}
\coqdocnoindent
\coqdockw{Lemma} \coqdoclemma{nf$_{\text{U}^\omega}$} :
\coqdocdefinition{normal\_form} % TODO: link to definition
(\coqdocvar{system} :=
\coqdocdefinition{$\mathcal{R}$})
\coqref{ExampleUNWO.repeatU}{\coqdocdefinition{U$^\omega$}}.\coqdoceol
\coqdocemptyline
\coqdocnoindent
\coqdockw{Lemma} \coqdoclemma{nf$_{\text{D}^\omega}$} :
\coqdocdefinition{normal\_form}
(\coqdocvar{system} :=
\coqdocdefinition{$\mathcal{R}$})
\coqdocdefinition{D$^\omega$}.\coqdoceol
\coqdocemptyline
\coqdocnoindent
\coqdockw{Lemma}
\coqdoclemma{neq$^{\text{U}^\omega}_{\text{D}^\omega}$} :
\ensuremath{\lnot}
\coqref{ExampleUNWO.repeatU}{\coqdocdefinition{U$^\omega$}}
\coqref{TermEquality.termbis}{$\sim$}
\coqdocdefinition{D$^\omega$}.\coqdoceol
\end{coqdoccode}
\end{singlespace}

Constructing a rewrite sequence from $\psi$ to $U^\omega$ is done in
much the same way as described in
Section~\ref{sub:counterexample}. First, we define the parameterized
step that is used in the rewrite sequence. It eliminates one pair of $D,
U$ constructors in a term of the form $U^n D^{m+1} U^{m+1} t$. The
omitted argument of the \coqref{Rewriting.Step}{\coqdocconstructor{Step}}
constructor (denoted by \coqdoclemma{\_}) is a proof of $\rho_1 \in
\mathcal{R}$.
\begin{singlespace}
\begin{coqdoccode}
\coqdocnoindent
\coqdockw{Definition}
\coqdef{ExampleUNWO.sigma}{sigma}{\coqdocdefinition{$\sigma$}}
\coqdocvar{t} (\coqdocvar{y} :
\coqdocdefinition{X}) :
\coqref{Term.term}{\coqdocinductive{term}} :=\coqdoceol
\coqdocindent{1.00em}
\coqdockw{match} \coqdocdefinition{beq\_var} \coqdocvariable{y} \coqdocvariable{x} \coqdockw{with}\coqdoceol
\coqdocindent{1.00em}
\ensuremath{|} \coqexternalref{http://coq.inria.fr/stdlib/Coq.Init.Datatypes}{true}{\coqdocconstructor{true}} \ensuremath{\Rightarrow}
\coqdocvariable{t}\coqdoceol
\coqdocindent{1.00em}
\ensuremath{|} \coqexternalref{http://coq.inria.fr/stdlib/Coq.Init.Datatypes}{false}{\coqdocconstructor{false}} \ensuremath{\Rightarrow}
\coqref{Term.Var}{\coqdocconstructor{Var}}
\coqdocvariable{y}\coqdoceol
\coqdocindent{1.00em}
\coqdockw{end}.\coqdoceol
\coqdocemptyline
\coqdocnoindent
\coqdockw{Lemma}
\coqdef{ExampleUNWO.facttermbisUmDSnUSnt}{fact\_term\_bis\_UmDSnUSnt}{\coqdoclemma{fact$_\pi^\text{source}$}}
:
\ensuremath{\forall} (\coqdocvar{n} \coqdocvar{m} :
\coqexternalref{http://coq.inria.fr/stdlib/Coq.Init.Datatypes}{nat}{\coqdocinductive{nat}})
(\coqdocvar{t} :
\coqref{Term.term}{\coqdocinductive{term}}),\coqdoceol
\coqdocindent{1.00em}
(\coqref{ExampleUNWO.Unt}{\coqdocdefinition{U}}$^\coqdocvariable{n}$
\coqdocdefinition{D}$^\coqdocvariable{m}$
$\Box$)[\coqref{Substitution.substitute}{\coqdocdefinition{substitute}}
  (\coqref{ExampleUNWO.sigma}{\coqdocdefinition{$\sigma$}}
(\coqref{ExampleUNWO.Unt}{\coqdocdefinition{U}}$^\coqdocvariable{m}$
    \coqdocvariable{t})) (\coqref{Rewriting.lhs}{\coqdocprojection{lhs}}
\coqdocdefinition{$\rho_1$})] \coqref{TermEquality.termbis}{$\sim$}
\coqref{ExampleUNWO.Unt}{\coqdocdefinition{U}}$^\coqdocvariable{n}$
\coqdocdefinition{D}$^{\coqexternalref{http://coq.inria.fr/stdlib/Coq.Init.Datatypes}{S}{\coqdocconstructor{S}} \coqdocvariable{m}}$
\coqref{ExampleUNWO.Unt}{\coqdocdefinition{U}}$^{\coqexternalref{http://coq.inria.fr/stdlib/Coq.Init.Datatypes}{S}{\coqdocconstructor{S}}
\coqdocvariable{m}}$
\coqdocvariable{t}.\coqdoceol
\coqdocemptyline
\coqdocnoindent
\coqdockw{Lemma}
\coqdef{ExampleUNWO.facttermbisUmDnUnt}{fact\_term\_bis\_UmDnUnt}{\coqdoclemma{fact$_\pi^\text{target}$}}
:
\ensuremath{\forall} (\coqdocvar{n} \coqdocvar{m} :
\coqexternalref{http://coq.inria.fr/stdlib/Coq.Init.Datatypes}{nat}{\coqdocinductive{nat}})
(\coqdocvar{t} :
\coqref{Term.term}{\coqdocinductive{term}}),\coqdoceol
\coqdocindent{1.00em}
(\coqref{ExampleUNWO.Unt}{\coqdocdefinition{U}}$^\coqdocvariable{n}$
\coqdocdefinition{D}$^\coqdocvariable{m}$
$\Box$)[\coqref{Substitution.substitute}{\coqdocdefinition{substitute}}
  (\coqref{ExampleUNWO.sigma}{\coqdocdefinition{$\sigma$}}
(\coqref{ExampleUNWO.Unt}{\coqdocdefinition{U}}$^\coqdocvariable{m}$
    \coqdocvariable{t})) (\coqref{Rewriting.rhs}{\coqdocprojection{rhs}}
\coqdocdefinition{$\rho_1$})] \coqref{TermEquality.termbis}{$\sim$}
\coqref{ExampleUNWO.Unt}{\coqdocdefinition{U}}$^\coqdocvariable{n}$
\coqdocdefinition{D}$^\coqdocvariable{m}$
\coqref{ExampleUNWO.Unt}{\coqdocdefinition{U}}$^\coqdocvariable{m}$
\coqdocvariable{t}.\coqdoceol
\coqdocemptyline
\coqdocnoindent
\coqdockw{Definition}
\coqdef{ExampleUNWO.pi}{pi}{\coqdocdefinition{$\pi$}}
\coqdocvar{n} \coqdocvar{m} \coqdocvar{t} :
\coqref{ExampleUNWO.Unt}{\coqdocdefinition{U}}$^\coqdocvariable{n}$
\coqdocdefinition{D}$^{\coqexternalref{http://coq.inria.fr/stdlib/Coq.Init.Datatypes}{S}{\coqdocconstructor{S}} \coqdocvariable{m}}$
\coqref{ExampleUNWO.Unt}{\coqdocdefinition{U}}$^{\coqexternalref{http://coq.inria.fr/stdlib/Coq.Init.Datatypes}{S}{\coqdocconstructor{S}}
\coqdocvariable{m}}$
\coqdocvariable{t} \coqref{Rewriting.step}{$\rightarrow_\mathcal{R}$}
\coqref{ExampleUNWO.Unt}{\coqdocdefinition{U}}$^\coqdocvariable{n}$
\coqdocdefinition{D}$^\coqdocvariable{m}$
\coqref{ExampleUNWO.Unt}{\coqdocdefinition{U}}$^\coqdocvariable{m}$
\coqdocvariable{t} :=\coqdoceol
\coqdocindent{1.00em}
\coqref{Rewriting.Step}{\coqdocconstructor{Step}}
\coqdocdefinition{$\rho_1$}
(\coqref{ExampleUNWO.Unt}{\coqdocdefinition{U}}$^\coqdocvariable{n}$
\coqdocdefinition{D}$^\coqdocvariable{m}$ $\Box$)
(\coqref{ExampleUNWO.sigma}{\coqdocdefinition{$\sigma$}}
(\coqref{ExampleUNWO.Unt}{\coqdocdefinition{U}}$^\coqdocvariable{m}$ \coqdocvariable{t}))
\coqdoclemma{\_}
(\coqref{ExampleUNWO.facttermbisUmDSnUSnt}{\coqdoclemma{fact$_\pi^\text{source}$}}
\coqdocvariable{n} \coqdocvariable{m} \coqdocvariable{t})
(\coqref{ExampleUNWO.facttermbisUmDnUnt}{\coqdoclemma{fact$_\pi^\text{target}$}}
\coqdocvariable{n} \coqdocvariable{m} \coqdocvariable{t}).\coqdoceol
\end{coqdoccode}
\end{singlespace}
Generalizing these rewrite steps
\coqref{ExampleUNWO.pi}{\coqdocdefinition{$\pi$}}, we construct
the rewrite sequences
\coqref{ExampleUNWO.phia}{\coqdocdefinition{$\varphi_a$}}. In their
recursive definition, the \coqdocdefinition{snoc}
function\footnote{With the
  \coqref{Rewriting.Cons}{\coqdocconstructor{Cons}} constructor, we
  can extend a rewrite sequence with one step at the end. In contrast,
  \coqdocdefinition{snoc} extends a rewrite sequence
  with one step at the start. It is the dual of $1 + \alpha$ on
  ordinals. The type of \coqdocdefinition{snoc}
  is \begin{coqdoccode}\ldots\end{coqdoccode}. TODO: define snoc and
    append in a separate subsection (3.4.2)} is used to
prepend \begin{coqdoccode}(\coqref{ExampleUNWO.pi}{\coqdocdefinition{$\pi$}}
\coqdocvariable{n} \coqdocvariable{m}
\coqdocvariable{t})\end{coqdoccode} to
\begin{coqdoccode}(\coqref{ExampleUNWO.phia}{\coqdocdefinition{$\varphi_a$}}
\coqdocvariable{n} \coqdocvariable{m}
\coqdocvariable{t})\end{coqdoccode}. Doing some arithmetic, we obtain
that these rewrite sequences can be used to define rewrite sequences
\coqref{ExampleUNWO.phib}{\coqdocdefinition{$\varphi_b$}} of a more
useful type.\footnote{TODO: note about \coqdockw{Program}
  construction.}
\begin{singlespace}
\begin{coqdoccode}
\coqdocnoindent
\coqdockw{Fixpoint}
\coqdef{ExampleUNWO.phia}{phia}{\coqdocdefinition{$\varphi_a$}}
\coqdocvar{n} \coqdocvar{m} \coqdocvar{t} :
\coqref{ExampleUNWO.Unt}{\coqdocdefinition{U}}$^\coqdocvariable{n}$
\coqdocdefinition{D}$^\coqdocvariable{m}$
\coqref{ExampleUNWO.Unt}{\coqdocdefinition{U}}$^\coqdocvariable{m}$
\coqdocvariable{t}
\coqref{Rewriting.sequence}{$\twoheadrightarrow_\mathcal{R}$}
\coqref{ExampleUNWO.Unt}{\coqdocdefinition{U}}$^\coqdocvariable{n}$
\coqdocvariable{t} :=\coqdoceol
\coqdocindent{1.00em}
\coqdockw{match} \coqdocvariable{m} \coqdockw{with}\coqdoceol
\coqdocindent{1.00em}
\ensuremath{|}
\coqdocconstructor{O}
\ensuremath{\Rightarrow}
\coqref{Rewriting.Nil}{\coqdocconstructor{Nil}}
(\coqref{ExampleUNWO.Unt}{\coqdocdefinition{U}}$^\coqdocvariable{n}$
\coqdocvariable{t})\coqdoceol
\coqdocindent{1.00em}
\ensuremath{|}
\coqdocconstructor{S}
\coqdocvar{m} \ensuremath{\Rightarrow}
\coqdocdefinition{snoc} % TODO: define snoc and link to it
(\coqref{ExampleUNWO.pi}{\coqdocdefinition{$\pi$}}
\coqdocvariable{n} \coqdocvariable{m} \coqdocvariable{t})
(\coqref{ExampleUNWO.phia}{\coqdocdefinition{$\varphi_a$}}
\coqdocvariable{n} \coqdocvariable{m} \coqdocvariable{t})\coqdoceol
\coqdocindent{1.00em}
\coqdockw{end}.\coqdoceol
\coqdocemptyline
\coqdocnoindent
\coqdockw{Program Definition} % TODO: here the S(2 n) is not correct (?)
\coqdef{ExampleUNWO.phib}{phib}{\coqdocdefinition{$\varphi_b$}}
\coqdocvar{n} : \coqref{ExampleUNWO.Unt}{\coqdocdefinition{U}}$^\coqdocvariable{n}$
(\coqref{ExampleUNWO.psi'}{\coqdocdefinition{$\psi'$}}
\coqdocvariable{n}) \coqref{Rewriting.sequence}{$\twoheadrightarrow_\mathcal{R}$}
\coqref{ExampleUNWO.Unt}{\coqdocdefinition{U}}$^{\coqexternalref{http://coq.inria.fr/stdlib/Coq.Init.Datatypes}{S}{\coqdocconstructor{S}} \coqdocvariable{n}}$
(\coqref{ExampleUNWO.psi'}{\coqdocdefinition{$\psi'$}}
(\coqexternalref{http://coq.inria.fr/stdlib/Coq.Init.Datatypes}{S}{\coqdocconstructor{S}}
\coqdocvariable{n})) :=\coqdoceol
\coqdocindent{1.00em}
\coqref{ExampleUNWO.phia}{\coqdocdefinition{$\varphi_a$}}
\coqdocvariable{n} (\coqexternalref{http://coq.inria.fr/stdlib/Coq.Init.Datatypes}{S}{\coqdocconstructor{S}}
(2
$\times$ \coqdocvariable{n})) (\coqexternalref{http://coq.inria.fr/stdlib/Coq.Init.Datatypes}{S}{\coqdocconstructor{S}} @
\coqref{ExampleUNWO.psi'}{\coqdocdefinition{$\psi'$}} (\coqexternalref{http://coq.inria.fr/stdlib/Coq.Init.Datatypes}{S}{\coqdocconstructor{S}}
\coqdocvariable{n})).\coqdoceol
\end{coqdoccode}
\end{singlespace}
We concatenate all rewrite sequences
\coqref{ExampleUNWO.phib}{\coqdocdefinition{$\varphi_b$}} to construct
rewrite sequences from $\psi$ to a term that is equal to $U^\omega$ up
to any given depth.
\begin{singlespace}
\begin{coqdoccode}
\coqdocnoindent
\coqdockw{Fixpoint}
\coqdef{ExampleUNWO.phic}{phic}{\coqdocdefinition{$\varphi_c$}}
\coqdocvar{n} : \coqref{ExampleUNWO.psi}{\coqdocdefinition{$\psi$}}
\coqref{Rewriting.sequence}{$\twoheadrightarrow_\mathcal{R}$}
\coqref{ExampleUNWO.Unt}{\coqdocdefinition{U}}$^\coqdocvariable{n}$
(\coqref{ExampleUNWO.psi'}{\coqdocdefinition{$\psi'$}}
\coqdocvariable{n}) :=\coqdoceol
\coqdocindent{1.00em}
\coqdockw{match} \coqdocvariable{n} \coqdockw{with}\coqdoceol
\coqdocindent{1.00em}
\ensuremath{|}
\coqdocconstructor{O}
\ensuremath{\Rightarrow}
\coqref{Rewriting.Nil}{\coqdocconstructor{Nil}}
\coqref{ExampleUNWO.psi}{\coqdocdefinition{$\psi$}}\coqdoceol
\coqdocindent{1.00em}
\ensuremath{|}
\coqdocconstructor{S}
\coqdocvar{n} \ensuremath{\Rightarrow}
\coqdocdefinition{append} % TODO: define append and link to it
(\coqref{ExampleUNWO.phic}{\coqdocdefinition{$\varphi_c$}}
\coqdocvariable{n})
(\coqref{ExampleUNWO.phib}{\coqdocdefinition{$\varphi_b$}}
\coqdocvariable{n})\coqdoceol
\coqdocindent{1.00em}
\coqdockw{end}.\coqdoceol
\end{coqdoccode}
\end{singlespace}

The definition of the final rewrite sequence
\coqref{ExampleUNWO.phi}{\coqdocdefinition{$\varphi$}} is done by combining
\coqref{ExampleUNWO.phic}{\coqdocdefinition{$\varphi_c$}} with a proof
that the target terms converge to $U^\omega$.
\begin{singlespace}
\begin{coqdoccode}
\coqdocnoindent
\coqdockw{Lemma}
\coqdef{ExampleUNWO.conv}{conv}{\coqdoclemma{conv$_{\varphi_c}$}}
: \coqref{Rewriting.converges}{\coqdocdefinition{converges}}
(\coqdockw{fun} \coqdocvar{n} \ensuremath{\Rightarrow}
\coqref{ExampleUNWO.Unt}{\coqdocdefinition{U}}$^\coqdocvariable{n}$
(\coqref{ExampleUNWO.psi'}{\coqdocdefinition{$\psi'$}}
\coqdocvariable{n}))
\coqref{ExampleUNWO.repeatU}{\coqdocdefinition{U$^\omega$}}.\coqdoceol
\coqdocemptyline
\coqdocnoindent
\coqdockw{Definition}
\coqdef{ExampleUNWO.phi}{phi}{\coqdocdefinition{$\varphi$}}
: \coqref{ExampleUNWO.psi}{\coqdocdefinition{$\psi$}}
\coqref{Rewriting.sequence}{$\twoheadrightarrow_\mathcal{R}$}
\coqref{ExampleUNWO.repeatU}{\coqdocdefinition{U$^\omega$}} :=
\coqref{Rewriting.Lim}{\coqdocconstructor{Lim}}
\coqdef{ExampleUNWO.phic}{phic}{\coqdocdefinition{$\varphi_c$}}
\coqdef{ExampleUNWO.conv}{conv}{\coqdoclemma{conv$_{\varphi_c}$}}.\coqdoceol
\coqdocemptyline
\coqdocnoindent
\coqdockw{Lemma}
\coqdoclemma{wf$_\varphi$}
: \coqref{Rewriting.wf}{\coqdocdefinition{wf}}
\coqref{ExampleUNWO.phi}{\coqdocdefinition{$\varphi$}}.\coqdoceol
\end{coqdoccode}
\end{singlespace}

We can prove $\psi \twoheadrightarrow D^\omega$ in a similar way and
conclude by proving two general lemmas.
\begin{singlespace}
\begin{coqdoccode}
\coqdocnoindent
\coqdockw{Lemma}
\coqdoclemma{no\_un$_\mathcal{R}$}
: \ensuremath{\lnot}
\coqref{Rewriting.uniquenormalforms}{\coqdocdefinition{unique\_normal\_forms}}
\coqdocdefinition{$\mathcal{R}$}.\coqdoceol
\coqdocemptyline
\coqdocnoindent
\coqdockw{Lemma} \coqdoclemma{no\_un\_wo}
: \ensuremath{\lnot} \ensuremath{\forall} \coqdocvar{F} \coqdocvar{X}
\coqdocvar{$\mathcal{R}$},\coqdoceol
\coqdocindent{1.00em}
\coqdocdefinition{weakly\_orthogonal}
(\coqdocvar{F} := \coqdocvariable{F}) (\coqdocvar{X} :=
\coqdocvariable{X}) \coqdocvariable{$\mathcal{R}$}
\ensuremath{\rightarrow}
\coqdocdefinition{unique\_normal\_forms}
\coqdocvariable{$\mathcal{R}$}.\coqdoceol
\end{coqdoccode}
\end{singlespace}

\chapter{Discussion}\label{chap:discussion}

We discuss our formalisation. Finally a conclusion.


\section{Convergence of Rewrite Sequences}\label{sec:convergence}

The inductively defined rewrite sequences from Section~\ref{sec:seq}
are not necessarily (weakly) convergent. A rewrite sequence of limit
length satisfies the condition that the target terms of the
\coqref{Rewriting.Lim}{\coqdocconstructor{Lim}} branches converge but
this is obviously too weak to establish convergence of the rewrite
sequence itself.

The depths of the rewrite steps are not considered at all in our
formalisation and therefore it obviously does not implement strong
convergence. Furthermore, the discussion in this section also applies
to the notions of continuity.

We consider an example of a rewrite sequence that satisfies the
inductive definition but is not weakly convergent. Let $A$ be a
constant and $B, C, D$ unary function symbols. We use the following
three rewrite rules:
\begin{align*}
  \rho_1 \, : \, A \to B(A) \qquad \qquad
  \rho_2 \, : \, C(x) \to D(x) \qquad \qquad
  \rho_3 \, : \, D(x) \to C(x)
\end{align*}
The term $C(A)$ rewrites in $\omega$ many $\rho_1$-steps to
$C(B^\omega)$.
\begin{center}
{\footnotesize\begin{tikzpicture}[node distance=50pt]
\tikzstyle{level}=[level distance=20pt,sibling distance=22pt]
\node (a) {$C$} child { node {$A$} };
\node (b) [right of=a] {$C$} child { node {$B$} child { node {$A$} } };
\node (c) [right of=b] {$C$} child { node {$B$} child { node {$B$}
    child { node {$A$} } } };
\node (d) [right of=c,node distance=80pt] {$C$} child { node {$B$}
  child { node {$B$} child { node {$B$} child { node[below=-6pt]
        {\scriptsize$\vdots$} } } } };
\path (a) -- (b) node[midway,below=-1pt] {$\rightarrow_{\rho_1}$};
\path (b) -- (c) node[midway,below=-1pt] {$\rightarrow_{\rho_1}$};
\path (c) -- (d) node[midway,below=-1pt] {$\rightarrow_{\rho_1} \quad \cdots$};
\end{tikzpicture}}
\end{center}\vspace{-0.8\baselineskip}
We modify this rewrite sequence such that in between every two
$\rho_1$-steps, the root symbol $C$ is changed to $D$ and back to
$C$. The resulting rewrite sequence does not have a limit and is not
weakly convergent.
\begin{center}
{\footnotesize\begin{tikzpicture}[node distance=40pt]
\tikzstyle{level}=[level distance=20pt,sibling distance=22pt]
\node (a) {$C$} child { node {$A$} };
\node (a') [right of=a] {$C$} child { node {$B$} child { node {$A$} } };
\node (a'') [right of=a'] {$D$} child { node {$B$} child { node {$A$} } };
\node (b) [right of=a''] {$C$} child { node {$B$} child { node {$A$} } };
\node (b') [right of=b] {$C$} child { node {$B$} child { node {$B$}
    child { node {$A$} } } };
\node (b'') [right of=b'] {$D$} child { node {$B$} child { node {$B$}
    child { node {$A$} } } };
\node (c) [right of=b''] {$C$} child { node {$B$} child { node {$B$}
    child { node {$A$} } } };
\node (d) [right of=c] {};
\path (a) -- (a') node[midway,below=-1pt] {$\rightarrow_{\rho_1}$};
\path (a') -- (a'') node[midway,below=-1pt] {$\rightarrow_{\rho_2}$};
\path (a'') -- (b) node[midway,below=-1pt] {$\rightarrow_{\rho_3}$};
\path (b) -- (b') node[midway,below=-1pt] {$\rightarrow_{\rho_1}$};
\path (b') -- (b'') node[midway,below=-1pt] {$\rightarrow_{\rho_2}$};
\path (b'') -- (c) node[midway,below=-1pt] {$\rightarrow_{\rho_3}$};
\path (c) -- (d) node[pos=.8,below=-1pt] {$\rightarrow_{\rho_1} \quad \cdots$};
\end{tikzpicture}}
\end{center}\vspace{-0.8\baselineskip}
We can define this rewrite sequence as the limit of
$(\varphi_n)_{n \in \mathbb{N}}$, where $\concat$ denotes
concatenation of rewrite sequences:
\begin{align*}
  \varphi_0 \, &: \, \mbox{\emph{empty}}\\ % TODO: or C(A) \to^0 C(A)
  \varphi_{n + 1} \, &: \, \varphi_n \concat C(B^n(A)) \to_{\rho_1}
  C(B^{n + 1}(A)) \to_{\rho_2} D(B^{n + 1}(A)) \to_{\rho_3} C(B^{n +
    1}(A))
\end{align*}
The target terms $C(B^{n + 1}(A))$ converge to $C(B^\omega)$ and this
construction can thus be used with our inductive definition of rewrite
sequences, where we take $\varphi_n$ to be the $n$\textsuperscript{th}
branch of the \coqref{Rewriting.Lim}{\coqdocconstructor{Lim}}
constructor.

% $\varphi_n$ is the prefix of length $3n$.

% 'stuttering convergence'
% convergence with hiccups
% the difference between t_n and the limit t is oscillating

It is not clear to us whether there is some natural translation of the
convergence conditions to our formalisation.
%Without such a translation, we feel our definitions are not
%satisfactory.
For completeness we include a (not so natural) translation of
convergence, but we were not able to use it in our development. Even
proving the simplest convergent rewrite sequences to satisfy these
definitions seems too involved.
\begin{singlespace}
\begin{coqdoccode}
\coqdocnoindent
\coqdockw{Fixpoint}
\coqdef{Rewriting.weaklyconvergent}{weakly\_convergent}{\coqdocdefinition{weakly\_convergent}}
\coqdocvar{s} \coqdocvar{t} (\coqdocvar{$\varphi$} : \coqdocvar{s}
\coqref{Rewriting.sequence}{$\rewrites_\mathcal{R}$} \coqdocvar{t}) :
\coqdockw{Prop}
:=\coqdoceol
\coqdocindent{1.00em}
\coqdockw{match} \coqdocvariable{$\varphi$} \coqdockw{with}\coqdoceol
\coqdocindent{1.00em}
\ensuremath{|} \coqref{Rewriting.Nil}{\coqdocconstructor{Nil}}
\coqdocvar{\_}          \ensuremath{\Rightarrow}
\coqexternalref{http://coq.inria.fr/stdlib/Coq.Init.Logic}{True}{\coqdocinductive{True}}\coqdoceol
\coqdocindent{1.00em}
\ensuremath{|} \coqref{Rewriting.Cons}{\coqdocconstructor{Cons}}
\coqdocvar{\_} \coqdocvar{\_} \coqdocvar{$\psi$} \coqdocvar{\_}
\coqdocvar{\_} \ensuremath{\Rightarrow}
\coqref{Rewriting.weaklyconvergent}{\coqdocdefinition{weakly\_convergent}}
\coqdocvariable{$\psi$}\coqdoceol
\coqdocindent{1.00em}
\ensuremath{|} \coqref{Rewriting.Lim}{\coqdocconstructor{Lim}}
\coqdocvar{\_} \coqdocvar{\_} \coqdocvar{f} \coqdocvar{t}
\coqdocvar{\_}  \ensuremath{\Rightarrow}
(\ensuremath{\forall} \coqdocvar{n},
\coqref{Rewriting.weaklyconvergent}{\coqdocdefinition{weakly\_convergent}}
(\coqdocvariable{f} \coqdocvariable{n})) \ensuremath{\land}\coqdoceol
\coqdocindent{2.00em}
\ensuremath{\forall} \coqdocvar{d}, \ensuremath{\exists}
\coqdocvar{$\iota$}, \ensuremath{\forall} \coqdocvar{$\kappa$},\coqdoceol
\coqdocindent{3.00em}
\coqdocvariable{$\varphi$}[\coqdocvariable{$\iota$}]$^\textsc{seq}$ \coqref{Rewriting.embed}{$\sqsubseteq$}
\coqdocvariable{$\varphi$}[\coqdocvariable{$\kappa$}]$^\textsc{seq}$
\ensuremath{\rightarrow}
\coqdocvariable{$\varphi$}[\coqdocvariable{$\kappa$}]$^\textsc{l}$
\coqref{TermEquality.termequpto}{\equpto{\coqdocvariable{d}}}
\coqdocvariable{t}\coqdoceol
\coqdocindent{1.00em}
\coqdockw{end}.\coqdoceol
\coqdocemptyline
\coqdocnoindent
\coqdockw{Fixpoint}
\coqdef{Rewriting.stronglyconvergent}{strongly\_convergent}{\coqdocdefinition{strongly\_convergent}}
\coqdocvar{s} \coqdocvar{t} (\coqdocvar{$\varphi$} : \coqdocvar{s} \coqref{Rewriting.sequence}{$\rewrites_\mathcal{R}$} \coqdocvar{t}) : \coqdockw{Prop}
:=\coqdoceol
\coqdocindent{1.00em}
\coqdockw{match} \coqdocvariable{$\varphi$} \coqdockw{with}\coqdoceol
\coqdocindent{1.00em}
\ensuremath{|} \coqref{Rewriting.Nil}{\coqdocconstructor{Nil}}
\coqdocvar{\_}          \ensuremath{\Rightarrow}
\coqexternalref{http://coq.inria.fr/stdlib/Coq.Init.Logic}{True}{\coqdocinductive{True}}\coqdoceol
\coqdocindent{1.00em}
\ensuremath{|} \coqref{Rewriting.Cons}{\coqdocconstructor{Cons}}
\coqdocvar{\_} \coqdocvar{\_} \coqdocvar{$\psi$} \coqdocvar{\_}
\coqdocvar{\_} \ensuremath{\Rightarrow}
\coqref{Rewriting.stronglyconvergent}{\coqdocdefinition{strongly\_convergent}}
\coqdocvariable{$\psi$}\coqdoceol
\coqdocindent{1.00em}
\ensuremath{|} \coqref{Rewriting.Lim}{\coqdocconstructor{Lim}}
\coqdocvar{\_} \coqdocvar{\_} \coqdocvar{f} \coqdocvar{t}
\coqdocvar{\_}  \ensuremath{\Rightarrow}
(\ensuremath{\forall} \coqdocvar{n},
\coqref{Rewriting.stronglyconvergent}{\coqdocdefinition{strongly\_convergent}}
(\coqdocvariable{f} \coqdocvariable{n})) \ensuremath{\land}\coqdoceol
\coqdocindent{2.00em}
\ensuremath{\forall} \coqdocvar{d}, \ensuremath{\exists}
\coqdocvar{$\iota$}, \ensuremath{\forall} \coqdocvar{$\kappa$},\coqdoceol
\coqdocindent{3.00em}
\coqdocvariable{$\varphi$}[\coqdocvariable{$\iota$}]$^\textsc{seq}$
\coqref{Rewriting.embed}{$\sqsubseteq$}
\coqdocvariable{$\varphi$}[\coqdocvariable{$\kappa$}]$^\textsc{seq}$
\ensuremath{\rightarrow}
\coqdocvariable{d} $\le$
\coqdocdefinition{depth}
\coqdocvariable{$\varphi$}[\coqdocvariable{$\kappa$}]$^\textsc{stp}$\coqdoceol
\coqdocindent{1.00em}
\coqdockw{end}.\coqdoceol
\end{coqdoccode}
\end{singlespace}
Here it should be possible to prove that
\coqref{Rewriting.stronglyconvergent}{\coqdocdefinition{strongly\_convergent}}
implies
\coqref{Rewriting.weaklyconvergent}{\coqdocdefinition{weakly\_convergent}}.


\section{Representing Rewrite Sequences}

% TODO: leave out ', with the appropriate conditions...' ?
In Definition~\ref{def:seq}, transfinite rewrite sequences are
introduced as partial functions from ordinals to rewrite steps, with
the appropriate conditions on source and target terms of subsequent
rewrite steps. Could we not translate this directly to \Coq?

% TODO: one problem, and other problem?
A problem is that we would need a decidable order on the
ordinals.\footnote{Consider a non-trivial rewrite sequence of length
  $\lambda$. For every ordinal $\alpha < \lambda$, the partial
  function representing this rewrite sequence must decide what step to
  produce.}
The order on our tree ordinals is not decidable. This could be
remedied by using a different representation for the ordinals, for
example axiomatically, as Cantor normal forms, or as sets. We feel
that the inductively defined tree ordinals are a more natural
representation in the constructive type theory of \Coq.

Comparison of ordinals up to some given upper bound may be decidable,
so another remedy for this problem would be to only consider rewrite
sequences of limited length. Motivated by the Compression Lemma, we
could go even further and restrict our representation to rewrite
sequences of length $\le \omega$. This would severely cripple our
formalisation, since much of the theory of infinitary rewriting could
not be developed with this representation (e.g. the Compression Lemma
itself). % and many compressed rewrite sequences are unnatural

Another argument in favour of our representation based on tree
ordinals is that it seems natural for a \Coq formalisation. As a
comparison, lists of finite length are usually defined by induction in
\Coq, not as partial functions from the natural numbers. Our
representation can be seen as a generalisation of inductively defined
finite lists to lists of transfinite length.


\section{Design Choices}

We describe a number of design choices for our formalisation.

%One-hole contexts vs multi-hole contexts (possible using extended signature).

%Casteran's ordinals in Veblen nf vs Mamane's set-theoretic ordinals vs tree
%ordinals.

%Bisimilarity in steps.

%The embedding relation and order on ordinals by Hancock, are there other
%choices?

%Positions are just lists (using option types) versus a safe position
%type parameterised by a term.

%Coinductive terms versus functions from lists of positions.

%Why \Coq?


\section{Conclusion}

Our representation feels natural, but the traditional convergence
definitions do not fit (easily). Maybe there are alternatives to our
embedding relation, or can we index the steps in a different way than
with predecessor indices.

Functions from ordinals to steps have other problems. They could be
investigated, especially using other ordinal representations such as
cantor normal form. But this is very not \Coq-like (partial functions
and algebraic representations). See also recent work by Ketema.

TODO: additional remarks on productivity and guardedness restriction?
There has been a lot of work in termination of recursive definitions.

Representation is original.

Original goal was compression.

%It is not clear to us whether there is some natural translation of the
%convergence conditions to our formalisation.
Without such a translation, we feel our definitions are not
satisfactory.


\appendix

\chapter*{Notation Glossary}

This chapter is used during writing, probably won't be in final thesis. Note
that some notations are used more than once, but with roughly comparable
meaning.

{\renewcommand{\arraystretch}{1.1}
\renewcommand{\tabcolsep}{10pt}
\begin{tabular}{p{150pt} p{175pt}}
Names & Used for\\
\hline
\multicolumn{2}{l}{\bf Ordinals}\\
$\alpha, \beta, \gamma$ & ordinals\\
$\lambda$ & limit ordinals\\
$i, j, k$ & ordinal predecessor indices\\
\multicolumn{2}{l}{\bf Terms}\\
$x, y$ & variables\\
$f, g$ & function symbols\\
$s, t, u, v$ & terms\\
$\rho, \tau$ & rewrite rules\\
$\mathcal{R}$ & TRSs\\
$\sigma$ & substitutions\\
\multicolumn{2}{l}{\bf Rewriting Sequences}\\
$\pi, o$ & rewrite steps\\
$\rho, \tau$ & rewrite sequences\\
\end{tabular}}

{\renewcommand{\arraystretch}{1.1}
\renewcommand{\tabcolsep}{10pt}
\begin{tabular}{l l p{175pt}}
Thesis & \Coq & Meaning\\
\hline
\multicolumn{3}{l}{\bf Ordinals}\\
$I(\alpha)$ & \texttt{pd\_type(alpha)} & predecessor indices of $\alpha$\\
$\alpha[i]$ & \texttt{alpha[i]} & predecessor of $\alpha$ indexed by $i$\\
$\alpha \preceq \beta$ & \texttt{alpha <= beta} & non-strict order\\
$\alpha \prec \beta$ & \texttt{alpha < beta} & strict order\\
$\alpha \simeq \beta$ & \texttt{alpha == beta} & extensional equality\\
\multicolumn{3}{l}{\bf Terms}\\
% TODO: have a look where we use this, i think we could just use =
$s \equiv t$ & \texttt{s = t} & syntactical and \Coq equality \small{(not really the same thing)}\\
$s \sim t$ & \texttt{s [\textasciitilde] t} & bisimilarity\\
$s \equpto{n} t$ & \texttt{term\_eq\_up\_to n s t} & pointwise equality up to depth $n$\\ % \eqcirc_n
$s \doteq t$ & \texttt{s [=] t} & pointwise equality\\ % \eqcirc
\multicolumn{3}{l}{\bf Rewriting Sequences}\\
$s \rightarrow t$ & \texttt{s [>] t} & $s$ rewrites to
$t$ in one step\\
$\pi \approx o$ & \texttt{step\_eq p o} & equality of steps\\
$s \twoheadrightarrow t$ & \texttt{s ->{}> t} & $s$ rewrites to
$t$\\
$I(\rho)$ & \texttt{pred\_type(r)} & predecessor indices of $\rho$\\
$\rho[i]$ & \texttt{r[i]} & predecessor location of $\rho$ indexed by $i$\\
$\rho[i]^\textsc{seq}$ & \texttt{r[seq i]} & predecessor sequence of $\rho$ indexed by $i$\\
$\rho[i]^\textsc{stp}$ & \texttt{r[stp i]} & predecessor step of $\rho$ indexed by $i$\\
$\rho[i]^\textsc{l}$ & \texttt{r[1 i]} & left predecessor term of $\rho$ indexed by $i$\\
$\rho[i]^\textsc{r}$ & \texttt{r[2 i]} & right predecessor term of $\rho$ indexed by $i$\\
$\rho \sqsubseteq \tau$ & \texttt{r <= q} & $\rho$ is embedded in $\tau$\\
$\rho \sqsubset \tau$ & \texttt{r < q} & $\rho$ is strictly embedded in $\tau$\\
\end{tabular}}


\nocite{*}
\bibliography{thesis}

\end{document}
