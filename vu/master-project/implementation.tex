\chapter{A Mechanical Formalisation}\label{chap:implementation}

In this chapter we present a formalisation of some of the notions from
Chapter~\ref{chap:rewriting} in the \Coq proof assistant. Related work
are \CoLoR \citep{blanqui-koprowski-10} and \Coccinelle
\citep{contejean-07}, libraries on finitary rewriting and termination,
and the representation of ordinal numbers up to $\epsilon_0$ and
$\Gamma_0$ in Cantor and Veblen normal form by \citet{casteran-06}.

%\Coq has been used for sizeable projects such as CompCert and a
%verified proof of the Four Color Theorem.

%Examples of formalisations of mathematical theories in \Coq:
%\begin{compactenum}
%\item Logic: A proof of G\"odel's First Incompleteness Theorem (Russel
%  O'Connor).
%\item Analysis: Exact real arithmetic (Russel O'Connor).
%\end{compactenum}

Section~\ref{sec:ordimp} translates the
tree ordinals from Subsection~\ref{sub:tree} to \Coq. Coinductive
terms are defined in Section~\ref{sec:terms}. In Section~\ref{sec:seq}
we present a novel representation of transfinite rewrite sequences
based on the structure of the tree ordinals. This we regard as the main
contribution of this thesis. We start with a short
introduction to \Coq. % in Section~\ref{sec:coq}.

In the \Coq code fragments, we take some notational liberties in favour
of readability. Sometimes we omit (part of) the type information. We
also freely use infix notations without declaration. Furthermore,
variable and definition names are typeset liberally.

Some definitions have implicit arguments, meaning those arguments can
be inferred by the system from the context. As an example, consider
the inductive type
\coqref{CoqIntro.natpos}{\coqdocinductive{nat$^+$}} whose constructor
takes as arguments a natural number \coqdocvar{n} and a proof that
\coqdocvar{n} is greater than $0$.
\begin{singlespace}
\begin{coqdoccode}
\coqdocnoindent
\coqdockw{Inductive}
\coqdef{CoqIntro.natpos}{nat\_pos}{\coqdocinductive{nat$^+$}} :
\coqdockw{Set} :=\coqdoceol
\coqdocindent{1.00em}
\ensuremath{|} \coqdef{CoqIntro.Pos}{Pos}{\coqdocconstructor{Pos}} :
\ensuremath{\forall} \coqdocvar{n} :
\coqexternalref{http://coq.inria.fr/stdlib/Coq.Init.Datatypes}{nat}{\coqdocinductive{nat}},
0 < \coqdocvariable{n} \ensuremath{\rightarrow}
\coqref{CoqIntro.natpos}{\coqdocinductive{nat$^+$}}.\coqdoceol
\end{coqdoccode}
\end{singlespace}
The argument \coqdocvar{n} of
\coqref{CoqIntro.Pos}{\coqdocconstructor{Pos}} can be implicit,
since it can be inferred from (the type of) the other argument. If
\coqdocvar{H} has type \begin{coqdoccode}0 < 3\end{coqdoccode}, we can
write \begin{coqdoccode}\coqref{CoqIntro.Pos}{\coqdocconstructor{Pos}}
  \coqdocvariable{H}\end{coqdoccode} instead
of \begin{coqdoccode}\coqref{CoqIntro.Pos}{\coqdocconstructor{Pos}} 3
  \coqdocvariable{H}\end{coqdoccode}.

% TODO: more?


\section{Ordinal Numbers}\label{sec:ordimp}

% TODO: rewrite this paragraph ('likewise')
In the theory of infinitary rewriting, the lengths of rewrite sequences play
a central role. One might even suspect that any representation of transfinite
rewrite sequences needs a representation of ordinal numbers. But this is not
the case.

Consider as an illustration the example of finite lists. They can be
naturally represented inductively, without the need for a
representation of natural numbers. The usual inductive definition of lists,
using constructors \coqdocconstructor{Nil} and \coqdocconstructor{Cons}, can
be seen as a generalisation of the natural numbers, defined inductively using
constructors \coqdocconstructor{Zero} and \coqdocconstructor{Successor}. The
generalisation consists of labeling the \coqdocconstructor{Cons} constructors
with list members.

Likewise, we now turn to the definition of tree ordinals as a case study in
preparation for the definition of transfinite rewrite sequences in
Section~\ref{sec:seq}.

We define the ordinal numbers using the representation of tree
ordinals (cf.~Definition~\ref{def:ordinals}) in \Coq by
\coqref{Ordinal.ord}{\coqdocinductive{ord}}.
\begin{singlespace}
\begin{coqdoccode}
\coqdocnoindent
\coqdockw{Inductive} \coqdef{Ordinal.ord}{ord}{\coqdocinductive{ord}} :
\coqdockw{Set} :=\coqdoceol
\coqdocindent{1.00em}
\ensuremath{|} \coqdef{Ordinal.Zero}{Zero}{\coqdocconstructor{Zero}}  :
\coqref{Ordinal.ord}{\coqdocinductive{ord}}\coqdoceol
\coqdocindent{1.00em}
\ensuremath{|} \coqdef{Ordinal.Succ}{Succ}{\coqdocconstructor{Succ}}  :
\coqref{Ordinal.ord}{\coqdocinductive{ord}} \ensuremath{\rightarrow}
\coqref{Ordinal.ord}{\coqdocinductive{ord}}\coqdoceol
\coqdocindent{1.00em}
\ensuremath{|} \coqdef{Ordinal.Limit}{Limit}{\coqdocconstructor{Limit}} :
(\coqexternalref{http://coq.inria.fr/stdlib/Coq.Init.Datatypes}{nat}{\coqdocinductive{nat}}
\ensuremath{\rightarrow} \coqref{Ordinal.ord}{\coqdocinductive{ord}})
\ensuremath{\rightarrow}
\coqref{Ordinal.ord}{\coqdocinductive{ord}}.\coqdoceol
\end{coqdoccode}
\end{singlespace}
Arithmetic operations on ordinals, such as addition, are easily
defined.
\begin{singlespace}
\begin{coqdoccode}
\coqdocnoindent
\coqdockw{Fixpoint} \coqdef{Ordinal.add}{add}{$+$}
(\coqdocvar{$\alpha$} \coqdocvar{$\beta$} :
\coqref{Ordinal.ord}{\coqdocinductive{ord}}) :
\coqref{Ordinal.ord}{\coqdocinductive{ord}} :=\coqdoceol
\coqdocindent{1.00em}
\coqdockw{match} \coqdocvariable{$\beta$} \coqdockw{with}\coqdoceol
\coqdocindent{1.00em}
\ensuremath{|} \coqref{Ordinal.Zero}{\coqdocconstructor{Zero}}
\ensuremath{\Rightarrow} \coqdocvariable{$\alpha$}\coqdoceol
\coqdocindent{1.00em}
\ensuremath{|} \coqref{Ordinal.Succ}{\coqdocconstructor{Succ}}
\coqdocvar{$\beta$} \ensuremath{\Rightarrow}
\coqref{Ordinal.Succ}{\coqdocconstructor{Succ}}
(\coqdocvariable{$\alpha$} \coqref{Ordinal.add}{$+$}
\coqdocvariable{$\beta$})\coqdoceol
\coqdocindent{1.00em}
\ensuremath{|} \coqref{Ordinal.Limit}{\coqdocconstructor{Limit}}
\coqdocvar{f}   \ensuremath{\Rightarrow}
\coqref{Ordinal.Limit}{\coqdocconstructor{Limit}} (\coqdockw{fun}
\coqdocvar{n} \ensuremath{\Rightarrow}
\coqdocvariable{$\alpha$} \coqref{Ordinal.add}{$+$}
(\coqdocvariable{f} \coqdocvariable{n}))\coqdoceol
\coqdocindent{1.00em}
\coqdockw{end}.\coqdoceol
\end{coqdoccode}
\end{singlespace}
In fact, all definitions from Subsection~\ref{sub:tree} translate directly to
\Coq code. We can now prove basic properties of $\preceq$, for example
that it is transitive and that, for the finite ordinals, it coincides
with the standard order on the natural numbers.\footnote{Although
  \coqdocvariable{n} and \coqdocvariable{m} have type
  \coqexternalref{http://coq.inria.fr/stdlib/Coq.Init.Datatypes}{nat}{\coqdocinductive{nat}}
  and $\preceq$ has type \coqref{Ordinal.ord}{\coqdocinductive{ord}}
  $\rightarrow$ \coqref{Ordinal.ord}{\coqdocinductive{ord}}
  $\rightarrow$ \coqdockw{Prop}, we can state the lemma in this
  concise way by defining the trivial coercion from
  \coqexternalref{http://coq.inria.fr/stdlib/Coq.Init.Datatypes}{nat}{\coqdocinductive{nat}}
  to \coqref{Ordinal.ord}{\coqdocinductive{ord}}.}
\begin{singlespace}
\begin{coqdoccode}
\coqdocnoindent
\coqdockw{Lemma}
\coqdef{Ordinal.ordletrans}{ord\_le\_trans}{\coqdoclemma{\ensuremath{\preceq_{\text{trans}}}}}
:
\ensuremath{\forall} \coqdocvar{\ensuremath{\alpha}}
\coqdocvar{\ensuremath{\beta}}
\coqdocvar{\ensuremath{\gamma}}, \coqdocvariable{\ensuremath{\alpha}}
\ensuremath{\preceq} \coqdocvariable{\ensuremath{\beta}}
\ensuremath{\rightarrow}
\coqdocvariable{\ensuremath{\beta}} \ensuremath{\preceq}
\coqdocvariable{\ensuremath{\gamma}}
\ensuremath{\rightarrow} \coqdocvariable{\ensuremath{\alpha}}
\ensuremath{\preceq}
\coqdocvariable{\ensuremath{\gamma}}.\coqdoceol
\coqdocemptyline
\coqdocnoindent
\coqdockw{Lemma}
\coqdef{Ordinal.ordlele}{ord\_le\_le}{\coqdoclemma{\ensuremath{\preceq_\text{nat}}}} :
\ensuremath{\forall} \coqdocvar{n} \coqdocvar{m}, \coqdocvariable{n}
\ensuremath{\le} \coqdocvariable{m} \ensuremath{\leftrightarrow}
\coqdocvariable{n} \ensuremath{\preceq} \coqdocvariable{m}.\coqdoceol
\end{coqdoccode}
\end{singlespace}

Recalling our discussion in Subsection~\ref{sub:tree} of limit ordinals whose
sequences do not actually approximate to a limit ordinal, we consider the
lemma
\coqref{Ordinal.ordlezeroright}{\coqdoclemma{\ensuremath{\preceq_{\text{zero\_right}}}}}
as an example of this issue.
\begin{singlespace}
\begin{coqdoccode}
\coqdocnoindent
\coqdockw{Lemma}
\coqdef{Ordinal.ordlezeroright}{ord\_le\_zero\_right}{\coqdoclemma{\ensuremath{\preceq_{\text{zero\_right}}}}}
:
\ensuremath{\forall} \coqdocvar{\ensuremath{\alpha}} \coqdocvar{\ensuremath{\beta}},
\coqdocvariable{\ensuremath{\alpha}} \ensuremath{\preceq}
\coqref{Ordinal.Zero}{\coqdocconstructor{Zero}}
\ensuremath{\rightarrow}
\coqdocvariable{\ensuremath{\alpha}} \ensuremath{\preceq}
\coqdocvariable{\ensuremath{\beta}}.\coqdoceol
\end{coqdoccode}
\end{singlespace}
We would like to strengthen this, but cannot, since nothing denies
\coqdocvariable{$\alpha$} from being the tree ordinal $\sqcup \{ 0, 0, 0,
\ldots \}$ (which has the same rank as $0$). We therefore turn to a subset of
the tree ordinals where we restrict limit sequences to be strictly
monotonic. This restriction is encoded in the
\coqref{WfOrdinal.wf}{\coqdocdefinition{wf}} (well-formedness)
property. The $\Sigma$-type
\coqref{WfOrdinal.wford}{\coqdocdefinition{ord$^\text{wf}$}} defines
the resulting subset.
\begin{singlespace}
\begin{coqdoccode}
\coqdocnoindent
\coqdockw{Fixpoint} \coqdef{WfOrdinal.wf}{wf}{\coqdocdefinition{wf}}
\coqdocvar{\ensuremath{\alpha}} : \coqdockw{Prop} :=\coqdoceol
\coqdocindent{1.00em}
\coqdockw{match} \coqdocvariable{\ensuremath{\alpha}} \coqdockw{with}\coqdoceol
\coqdocindent{1.00em}
\ensuremath{|} \coqref{Ordinal.Zero}{\coqdocconstructor{Zero}}
\ensuremath{\Rightarrow}
\coqexternalref{http://coq.inria.fr/stdlib/Coq.Init.Logic}{True}{\coqdocinductive{True}}\coqdoceol
\coqdocindent{1.00em}
\ensuremath{|} \coqref{Ordinal.Succ}{\coqdocconstructor{Succ}}
\coqdocvar{\ensuremath{\beta}} \ensuremath{\Rightarrow}
\coqref{WfOrdinal.wf}{\coqdocdefinition{wf}} \coqdocvariable{\ensuremath{\beta}}\coqdoceol
\coqdocindent{1.00em}
\ensuremath{|} \coqref{Ordinal.Limit}{\coqdocconstructor{Limit}} \coqdocvar{f}
\ensuremath{\Rightarrow} \ensuremath{\forall} \coqdocvar{n},
\coqref{WfOrdinal.wf}{\coqdocdefinition{wf}} (\coqdocvariable{f}
\coqdocvariable{n}) \ensuremath{\land} \ensuremath{\forall} \coqdocvar{m},
\coqdocvariable{n} < \coqdocvariable{m} \ensuremath{\rightarrow}
\coqdocvariable{f} \coqdocvariable{n} \ensuremath{\prec}
\coqdocvariable{f} \coqdocvariable{m}\coqdoceol
\coqdocindent{1.00em}
\coqdockw{end}.\coqdoceol
\coqdocemptyline
\coqdocnoindent
\coqdockw{Definition}
\coqdef{WfOrdinal.wford}{wf\_ord}{\coqdocdefinition{ord$^\text{wf}$}} : \coqdockw{Set}
:=
%\coqexternalref{http://coq.inria.fr/stdlib/Coq.Init.Specif}{sig}{\coqdocinductive{sig}}
%\coqref{WfOrdinal.wf}{\coqdocdefinition{wf}}.\coqdoceol
\{ \coqdocvariable{$\alpha$} :
\coqref{Ordinal.ord}{\coqdocinductive{ord}} \ensuremath{|}
\coqref{WfOrdinal.wf}{\coqdocdefinition{wf}} \coqdocvariable{$\alpha$}
\}.\coqdoceol
\end{coqdoccode}
\end{singlespace}
Now we can prove the stronger result we were looking
for.\footnote{Again, defining a simple coercion from
  \coqref{WfOrdinal.wford}{\coqdocdefinition{ord$^\text{wf}$}} to
  \coqref{Ordinal.ord}{\coqdocinductive{ord}} (first $\Sigma$-type
  projection) lets us state this lemma concisely.}
\begin{singlespace}
\begin{coqdoccode}
\coqdocnoindent
\coqdockw{Lemma}
\coqdef{WfOrdinal.wfordlezeroright}{wf\_ord\_le\_zero\_right}{\coqdoclemma{\ensuremath{\preceq^{\text{wf}}_{\text{zero\_right}}}}}
:
\ensuremath{\forall} \coqdocvar{\ensuremath{\alpha}} :
\coqref{WfOrdinal.wford}{\coqdocdefinition{ord$^\text{wf}$}},
\coqdocvariable{\ensuremath{\alpha}} \ensuremath{\preceq}
\coqref{Ordinal.Zero}{\coqdocconstructor{Zero}}
\ensuremath{\rightarrow}
\coqdocvariable{\ensuremath{\alpha}} =
\coqref{Ordinal.Zero}{\coqdocconstructor{Zero}}.\coqdoceol
\end{coqdoccode}
\end{singlespace}


\section{Coinductive Terms}\label{sec:terms}

We define the type
\coqref{Term.term}{\coqdocinductive{term}} of infinite terms with
function symbols in \coqdocvar{F} and variables in \coqdocvar{X}.
\begin{singlespace}
\begin{coqdoccode}
\coqdocnoindent
\coqdockw{CoInductive} \coqdef{Term.term}{term}{\coqdocinductive{term}} :
\coqdockw{Type} :=\coqdoceol
\coqdocindent{1.00em}
\ensuremath{|} \coqdef{Term.Var}{Var}{\coqdocconstructor{Var}} : \coqdocvar{X}
\ensuremath{\rightarrow} \coqref{Term.term}{\coqdocinductive{term}}\coqdoceol
\coqdocindent{1.00em}
\ensuremath{|} \coqdef{Term.Fun}{Fun}{\coqdocconstructor{Fun}} :
\ensuremath{\forall} \coqdocvar{f} : \coqdocvar{F},
\coqref{Vector.vector}{\coqdocdefinition{vector}}
\coqref{Term.term}{\coqdocinductive{term}}
(\coqdocprojection{arity} \coqdocvariable{f})
\ensuremath{\rightarrow} \coqref{Term.term}{\coqdocinductive{term}}.\coqdoceol
\end{coqdoccode}
\end{singlespace}
The objects of a coinductive type can only be built in some restricted way to
ensure productivity of the construction. This restriction implies a technical
difficulty in the definition of the
\coqref{Vector.vector}{\coqdocdefinition{vector}} type, which is discussed in
Subsection~\ref{sub:guardedness}.
For now, we assume \coqref{Vector.vector}{\coqdocdefinition{vector}} to
implement dependently typed lists (the type depending on their length).

The standard equality defined in \Coq, equivalent to Leibniz' equality and
written $=$, does not suffice
for establishing that two terms are equal, given that the only way to build
infinite objects is by corecursion. Because the amount of memory available is
finite, we can only unfold the corecursive definition finitely many times, and
then still be left with a non-normal form. Simply comparing such
definitions will not do, since the corecursive construction of any given
infinite object is not unique. To this end, we define two extensional
equalities on \coqref{Term.term}{\coqdocinductive{term}}, following
Definitions~\ref{def:bisimilarity} and \ref{def:equiv}. The coinductive
relation \coqref{TermEquality.termbis}{$\bis$} defines bisimilarity and
pointwise equality is defined by \coqref{TermEquality.termeq}{$\equiv$}
inductively.
\begin{singlespace}
\begin{coqdoccode}
\coqdocnoindent
\coqdockw{CoInductive}
\coqdef{TermEquality.termbis}{term\_bis}{$\bis$} :
\coqref{Term.term}{\coqdocinductive{term}} \ensuremath{\rightarrow}
\coqref{Term.term}{\coqdocinductive{term}} \ensuremath{\rightarrow}
\coqdockw{Prop} :=\coqdoceol
\coqdocindent{1.00em}
\ensuremath{|}
\coqdef{TermEquality.Varbis}{Var\_bis}{\coqdocconstructor{$\biss{\text{Var}}$}} :
\ensuremath{\forall} \coqdocvar{x},
\coqref{Term.Var}{\coqdocconstructor{Var}} \coqdocvariable{x}
\coqref{TermEquality.termbis}{$\bis$}
\coqref{Term.Var}{\coqdocconstructor{Var}} \coqdocvariable{x}\coqdoceol
\coqdocindent{1.00em}
\ensuremath{|}
\coqdef{TermEquality.Funbis}{Fun\_bis}{\coqdocconstructor{$\biss{\text{Fun}}$}} :
\ensuremath{\forall} \coqdocvar{f} \coqdocvar{v} \coqdocvar{w},
(\ensuremath{\forall} \coqdocvar{i},
\coqdocvariable{v} \coqdocvariable{i}
\coqref{TermEquality.termbis}{$\bis$}
\coqdocvariable{w} \coqdocvariable{i})
\ensuremath{\rightarrow}
\coqref{Term.Fun}{\coqdocconstructor{Fun}} \coqdocvariable{f}
\coqdocvariable{v}
\coqref{TermEquality.termbis}{$\bis$}
\coqref{Term.Fun}{\coqdocconstructor{Fun}} \coqdocvariable{f}
\coqdocvariable{w}.\coqdoceol
\end{coqdoccode}
\end{singlespace}
Any proof of two infinite terms being bisimilar is an infinite proof, in the
sense that the proof term is built by corecursion.
\begin{singlespace}
\begin{coqdoccode}
\coqdocnoindent
\coqdockw{Inductive}
\coqdef{TermEquality.termequpto}{term\_eq\_up\_to}{\equpto{}}
:
\coqexternalref{http://coq.inria.fr/stdlib/Coq.Init.Datatypes}{nat}{\coqdocinductive{nat}}
\ensuremath{\rightarrow} \coqref{Term.term}{\coqdocinductive{term}}
\ensuremath{\rightarrow} \coqref{Term.term}{\coqdocinductive{term}}
\ensuremath{\rightarrow} \coqdockw{Prop} :=\coqdoceol
\coqdocindent{1.00em}
\ensuremath{|}
\coqdef{TermEquality.teut0}{teut\_0}{\coqdocconstructor{teut$_0$}}   :
\ensuremath{\forall} \coqdocvar{s} \coqdocvar{t},
\coqdocvariable{s} \coqref{TermEquality.termequpto}{\equpto{0}}
\coqdocvariable{t}\coqdoceol
\coqdocindent{1.00em}
\ensuremath{|}
\coqdef{TermEquality.teutvar}{teut\_var}{\coqdocconstructor{teut$_\text{Var}$}} :
\ensuremath{\forall} \coqdocvar{d} \coqdocvar{x},
\coqref{Term.Var}{\coqdocconstructor{Var}} \coqdocvariable{x}
\coqref{TermEquality.termequpto}{\equpto{\coqdocvariable{d}}}
\coqref{Term.Var}{\coqdocconstructor{Var}} \coqdocvariable{x}\coqdoceol
\coqdocindent{1.00em}
\ensuremath{|}
\coqdef{TermEquality.teutfun}{teut\_fun}{\coqdocconstructor{teut$_\text{Fun}$}} :
\ensuremath{\forall} \coqdocvar{d} \coqdocvar{f} \coqdocvar{v}
\coqdocvar{w},
(\ensuremath{\forall} \coqdocvar{i},
\coqdocvariable{v} \coqdocvariable{i}
\coqref{TermEquality.termequpto}{\equpto{\coqdocvariable{d}}}
\coqdocvariable{w} \coqdocvariable{i}) \ensuremath{\rightarrow}
\coqref{Term.Fun}{\coqdocconstructor{Fun}}
\coqdocvariable{f} \coqdocvariable{v}
\coqref{TermEquality.termequpto}{\equpto{\coqexternalref{http://coq.inria.fr/stdlib/Coq.Init.Datatypes}{S}{\coqdocconstructor{S}}
    \, \coqdocvariable{d}}}
\coqref{Term.Fun}{\coqdocconstructor{Fun}} \coqdocvariable{f}
\coqdocvariable{w}.\coqdoceol
\coqdocemptyline
\coqdocnoindent
\coqdockw{Definition}
\coqdocvar{s}
\coqdef{TermEquality.termeq}{term\_eq}{$\equiv$}
\coqdocvar{t} :=
\ensuremath{\forall} \coqdocvar{d},
\coqdocvariable{s}
\coqref{TermEquality.termequpto}{\equpto{\coqdocvariable{d}}}
\coqdocvariable{t}.\coqdoceol
\end{coqdoccode}
\end{singlespace}

We can prove that \coqref{TermEquality.termbis}{$\bis$} and
\coqref{TermEquality.termeq}{$\equiv$} are the same
relation, and that indeed it is an equivalence.
\begin{singlespace}
\begin{coqdoccode}
\coqdocnoindent
\coqdockw{Lemma}
\coqdef{TermEquality.termbistermeq}{term\_bis\_term\_eq}{\coqdoclemma{term\_bis\_term\_eq}}
: \ensuremath{\forall} \coqdocvar{s} \coqdocvar{t},
\coqdocvariable{s}
\coqref{TermEquality.termbis}{$\bis$}
\coqdocvariable{t} \ensuremath{\leftrightarrow}
\coqdocvariable{s}
\coqref{TermEquality.termeq}{$\equiv$}
\coqdocvariable{t}.\coqdoceol
\coqdocemptyline
\coqdocnoindent
\coqdockw{Lemma}
\coqdef{TermEquality.termbisrefl}{term\_bis\_refl}{\coqdoclemma{$\biss{\text{refl}}$}}
: \ensuremath{\forall} \coqdocvar{t},
\coqdocvariable{t}
\coqref{TermEquality.termbis}{$\bis$}
\coqdocvariable{t}.\coqdoceol
\coqdocemptyline
\coqdocnoindent
\coqdockw{Lemma}
\coqdef{TermEquality.termbissymm}{term\_bis\_symm}{\coqdoclemma{$\biss{\text{symm}}$}}
: \ensuremath{\forall} \coqdocvar{s} \coqdocvar{t},
\coqdocvariable{s}
\coqref{TermEquality.termbis}{$\bis$}
\coqdocvariable{t} $\rightarrow$
\coqdocvariable{t}
\coqref{TermEquality.termbis}{$\bis$}
\coqdocvariable{s}.\coqdoceol
\coqdocemptyline
\coqdocnoindent
\coqdockw{Lemma}
\coqdef{TermEquality.termbistrans}{term\_bis\_trans}{\coqdoclemma{$\biss{\text{trans}}$}}
: \ensuremath{\forall} \coqdocvar{s} \coqdocvar{t} \coqdocvar{u},
\coqdocvariable{s}
\coqref{TermEquality.termbis}{$\bis$}
\coqdocvariable{t} $\rightarrow$
\coqdocvariable{t}
\coqref{TermEquality.termbis}{$\bis$}
\coqdocvariable{u} $\rightarrow$
\coqdocvariable{s}
\coqref{TermEquality.termbis}{$\bis$}
\coqdocvariable{u}.\coqdoceol
\end{coqdoccode}
\end{singlespace}

In Section~\ref{sec:seq} we need some notion of convergence for functions of
type
\coqexternalref{http://coq.inria.fr/stdlib/Coq.Init.Datatypes}{nat}{\coqdocinductive{nat}}
$\rightarrow$
\coqref{Term.term}{\coqdocinductive{term}}. We implement
Definition~\ref{def:cauchy} in \Coq for sequences of length $\omega$.
\begin{singlespace}
\begin{coqdoccode}
\coqdocnoindent
\coqdockw{Definition}
\coqdef{Rewriting.converges}{converges}{\coqdocdefinition{converges}}
(\coqdocvar{f} :
\coqexternalref{http://coq.inria.fr/stdlib/Coq.Init.Datatypes}{nat}{\coqdocinductive{nat}}
\ensuremath{\rightarrow} \coqref{Term.term}{\coqdocinductive{term}})
(\coqdocvar{t} : \coqref{Term.term}{\coqdocinductive{term}}) :
\coqdockw{Prop} :=\coqdoceol
\coqdocindent{1.00em}
\ensuremath{\forall} \coqdocvar{d}, \ensuremath{\exists} \coqdocvar{n},
\ensuremath{\forall} \coqdocvar{m},
\coqdocvariable{n} \ensuremath{\le} \coqdocvariable{m}
\ensuremath{\rightarrow}
\coqdocvariable{f} \coqdocvariable{m}
\coqref{TermEquality.termequpto}{\equpto{\coqdocvariable{d}}}
\coqdocvariable{t}.\coqdoceol
\end{coqdoccode}
\end{singlespace}
The definitions of finite term, rewrite rule, TRS and left-linearity
from Subsection~\ref{sub:trs} translate to \Coq directly. We define
\coqdef{Rewriting.lhs}{lhs}{\coqdocprojection{lhs}} and
\coqdef{Rewriting.rhs}{rhs}{\coqdocprojection{rhs}} to be first and
second projection on rewrite rules.

The type of contexts is inductively defined, where the hole always
occurs at a finite depth.
\begin{singlespace}
\begin{coqdoccode}
\coqdocnoindent
\coqdockw{Inductive}
\coqdef{Context.context}{context}{\coqdocinductive{context}} :
\coqdockw{Type} :=\coqdoceol
\coqdocindent{1.00em}
\ensuremath{|} $\Box$ :
\coqref{Context.context}{\coqdocinductive{context}}\coqdoceol
\coqdocindent{1.00em}
\ensuremath{|} \coqdef{Context.CFun}{CFun}{\coqdocconstructor{CFun}} :
\ensuremath{\forall} (\coqdocvar{f} : \coqdocvar{F}) (\coqdocvar{i}
\coqdocvar{j} :
\coqexternalref{http://coq.inria.fr/stdlib/Coq.Init.Datatypes}{nat}{\coqdocinductive{nat}}),
\coqdocvariable{i} +
\coqexternalref{http://coq.inria.fr/stdlib/Coq.Init.Datatypes}{S}{\coqdocconstructor{S}}
\coqdocvariable{j} =
\coqdocprojection{arity} \coqdocvariable{f}
\ensuremath{\rightarrow}\coqdoceol
\coqdocindent{5.50em}
\coqref{Vector.vector}{\coqdocdefinition{vector}}
\coqref{Term.term}{\coqdocinductive{term}} \coqdocvariable{i}
\ensuremath{\rightarrow}
\coqref{Context.context}{\coqdocinductive{context}}
\ensuremath{\rightarrow}
\coqref{Vector.vector}{\coqdocdefinition{vector}}
\coqref{Term.term}{\coqdocinductive{term}} \coqdocvariable{j}
\ensuremath{\rightarrow}
\coqref{Context.context}{\coqdocinductive{context}}.\coqdoceol
\end{coqdoccode}
\end{singlespace}
Applying a substitution \coqdocvariable{$\sigma$} to a term
\coqdocvariable{t} is defined by corecursion over
\coqdocvariable{t}. We also use the
notation \begin{coqdoccode}\coqdocvariable{t}$^\coqdocvariable{$\sigma$}$\end{coqdoccode}
for \begin{coqdoccode}\coqref{Substitution.substitute}{\coqdocdefinition{substitute}}
  \coqdocvariable{$\sigma$} \coqdocvariable{t}\end{coqdoccode}.
\begin{singlespace}
\begin{coqdoccode}
\coqdocnoindent
\coqdockw{Definition}
\coqdef{Substitution.substitution}{substitution}{\coqdocdefinition{substitution}}
:= \coqdocvar{X} \ensuremath{\rightarrow}
\coqref{Term.term}{\coqdocinductive{term}}.\coqdoceol
\coqdocemptyline
\coqdocnoindent
\coqdockw{CoFixpoint}
\coqdef{Substitution.substitute}{substitute}{\coqdocdefinition{substitute}}
(\coqdocvar{$\sigma$} :
\coqref{Substitution.substitution}{\coqdocdefinition{substitution}})
(\coqdocvar{t} :
\coqref{Term.term}{\coqdocinductive{term}}) :
\coqref{Term.term}{\coqdocinductive{term}} :=\coqdoceol
\coqdocindent{1.00em}
\coqdockw{match} \coqdocvariable{t} \coqdockw{with}\coqdoceol
\coqdocindent{1.00em}
\ensuremath{|} \coqref{Term.Var}{\coqdocconstructor{Var}}
\coqdocvar{x}      \ensuremath{\Rightarrow} \coqdocvariable{$\sigma$}
\coqdocvariable{x}\coqdoceol
\coqdocindent{1.00em}
\ensuremath{|} \coqref{Term.Fun}{\coqdocconstructor{Fun}}
\coqdocvar{f} \coqdocvar{args} \ensuremath{\Rightarrow}
\coqref{Term.Fun}{\coqdocconstructor{Fun}} \coqdocvariable{f}
(\coqref{Vector.vmap}{\coqdocdefinition{vmap}}
(\coqref{Substitution.substitute}{\coqdocdefinition{substitute}}
\coqdocvariable{$\sigma$}) \coqdocvariable{args})\coqdoceol
\coqdocindent{1.00em}
\coqdockw{end}.\coqdoceol
\end{coqdoccode}
\end{singlespace}
We apply the recursive function
\coqdef{Context.fill}{fill}{\coqdocdefinition{fill}} (not shown here)
to a context \coqdocvariable{C} and a term \coqdocvariable{t}
(written \begin{coqdoccode}\coqdocvariable{C}[\coqdocvariable{t}]\end{coqdoccode})
to replace the hole in \coqdocvariable{C} with \coqdocvariable{t}.

Positions are represented by simple lists of natural numbers. This
means the subterm at some position in some term may not actually
exist. For this reason we employ
\coqexternalref{http://coq.inria.fr/stdlib/Coq.Init.Datatypes}{option}{\coqdocinductive{option}}
types in functions that do a lookup by position (functions in \Coq are
always \emph{total}). For further discussion of positions, see
Section~\ref{sec:design}.
\begin{singlespace}
\begin{coqdoccode}
\coqdocnoindent
\coqdockw{Fixpoint}
\coqdef{Term.subterm}{subterm}{\coqdocdefinition{subterm}}
(\coqdocvar{t} : \coqref{Term.term}{\coqdocinductive{term}})
(\coqdocvar{p} :
\coqdocabbreviation{position})
\{\coqdockw{struct} \coqdocvar{p}\} :
\coqexternalref{http://coq.inria.fr/stdlib/Coq.Init.Datatypes}{option}{\coqdocinductive{option}}
\coqref{Term.term}{\coqdocinductive{term}} :=\coqdoceol
\coqdocindent{1.00em}
\coqdockw{match} \coqdocvariable{p} \coqdockw{with}\coqdoceol
\coqdocindent{1.00em}
\ensuremath{|}
\coqexternalref{http://coq.inria.fr/stdlib/Coq.Init.Datatypes}{nil}{\coqdocconstructor{nil}}
\ensuremath{\Rightarrow}
\coqexternalref{http://coq.inria.fr/stdlib/Coq.Init.Datatypes}{Some}{\coqdocconstructor{Some}}
\coqdocvariable{t}\coqdoceol
\coqdocindent{1.00em}
\ensuremath{|} \coqdocvar{n} :: \coqdocvar{p} \ensuremath{\Rightarrow}
\coqdockw{match} \coqdocvariable{t} \coqdockw{with}\coqdoceol
\coqdocindent{7.00em}
\ensuremath{|} \coqref{Term.Var}{\coqdocconstructor{Var}}
\coqdocvar{\_}      \ensuremath{\Rightarrow}
\coqexternalref{http://coq.inria.fr/stdlib/Coq.Init.Datatypes}{None}{\coqdocconstructor{None}}\coqdoceol
\coqdocindent{7.00em}
\ensuremath{|} \coqref{Term.Fun}{\coqdocconstructor{Fun}}
\coqdocvar{f} \coqdocvar{args} \ensuremath{\Rightarrow}
\coqdockw{match}
\coqexternalref{http://coq.inria.fr/stdlib/Coq.Arith.Bool\_nat}{ltgedec}{\coqdocdefinition{lt\_ge\_dec}}
\coqdocvariable{n} (\coqdocprojection{arity}
\coqdocvariable{f}) \coqdockw{with}\coqdoceol
\coqdocindent{15.00em}
\ensuremath{|}
\coqexternalref{http://coq.inria.fr/stdlib/Coq.Init.Specif}{left}{\coqdocconstructor{left}}
\coqdocvar{h}  \ensuremath{\Rightarrow}
\coqref{Term.subterm}{\coqdocdefinition{subterm}}
(\coqdocdefinition{vnth} \coqdocvariable{h}
\coqdocvariable{args}) \coqdocvariable{p}\coqdoceol
\coqdocindent{15.00em}
\ensuremath{|}
\coqexternalref{http://coq.inria.fr/stdlib/Coq.Init.Specif}{right}{\coqdocconstructor{right}}
\coqdocvar{\_} \ensuremath{\Rightarrow}
\coqexternalref{http://coq.inria.fr/stdlib/Coq.Init.Datatypes}{None}{\coqdocconstructor{None}}\coqdoceol
\coqdocindent{15.00em}
\coqdockw{end}\coqdoceol
\coqdocindent{7.00em}
\coqdockw{end}\coqdoceol
\coqdocindent{1.00em}
\coqdockw{end}.\coqdoceol
\coqdocemptyline
\coqdocnoindent
\coqdockw{Fixpoint} \coqdef{Context.dig}{dig}{\coqdocdefinition{dig}}
(\coqdocvar{t} : \coqref{Term.term}{\coqdocinductive{term}})
(\coqdocvar{p} :
\coqdocabbreviation{position})
\{\coqdockw{struct} \coqdocvar{p}\} :
\coqexternalref{http://coq.inria.fr/stdlib/Coq.Init.Datatypes}{option}{\coqdocinductive{option}}
\coqref{Context.context}{\coqdocinductive{context}} :=\coqdoceol
\coqdocindent{1.00em}
\coqdockw{match} \coqdocvariable{p} \coqdockw{with}\coqdoceol
\coqdocindent{1.00em}
\ensuremath{|}
\coqexternalref{http://coq.inria.fr/stdlib/Coq.Init.Datatypes}{nil}{\coqdocconstructor{nil}}
\ensuremath{\Rightarrow}
\coqexternalref{http://coq.inria.fr/stdlib/Coq.Init.Datatypes}{Some}{\coqdocconstructor{Some}}
$\Box$\coqdoceol
\coqdocindent{1.00em}
\ensuremath{|} \coqdocvar{n} :: \coqdocvar{p} \ensuremath{\Rightarrow}
\coqdockw{match} \coqdocvariable{t} \coqdockw{with}\coqdoceol
\coqdocindent{3.00em}
\ensuremath{|} \coqref{Term.Var}{\coqdocconstructor{Var}}
\coqdocvar{\_}      \ensuremath{\Rightarrow}
\coqexternalref{http://coq.inria.fr/stdlib/Coq.Init.Datatypes}{None}{\coqdocconstructor{None}}\coqdoceol
\coqdocindent{3.00em}
\ensuremath{|} \coqref{Term.Fun}{\coqdocconstructor{Fun}}
\coqdocvar{f} \coqdocvar{args} \ensuremath{\Rightarrow}
\coqdockw{match}
\coqexternalref{http://coq.inria.fr/stdlib/Coq.Arith.Bool\_nat}{ltgedec}{\coqdocdefinition{lt\_ge\_dec}}
\coqdocvariable{n} (\coqdocprojection{arity}
\coqdocvariable{f}) \coqdockw{with}\coqdoceol
\coqdocindent{5.00em}
\ensuremath{|}
\coqexternalref{http://coq.inria.fr/stdlib/Coq.Init.Specif}{left}{\coqdocconstructor{left}}
\coqdocvar{h}  \ensuremath{\Rightarrow} \coqdockw{match}
\coqref{Context.dig}{\coqdocdefinition{dig}}
(\coqdocdefinition{vnth} \coqdocvariable{h}
\coqdocvariable{args}) \coqdocvariable{p} \coqdockw{with}\coqdoceol
\coqdocindent{7.00em}
\ensuremath{|}
\coqexternalref{http://coq.inria.fr/stdlib/Coq.Init.Datatypes}{None}{\coqdocconstructor{None}}
\ensuremath{\Rightarrow}
\coqexternalref{http://coq.inria.fr/stdlib/Coq.Init.Datatypes}{None}{\coqdocconstructor{None}}\coqdoceol
\coqdocindent{7.00em}
\ensuremath{|}
\coqexternalref{http://coq.inria.fr/stdlib/Coq.Init.Datatypes}{Some}{\coqdocconstructor{Some}}
\coqdocvar{C} \ensuremath{\Rightarrow}
\coqexternalref{http://coq.inria.fr/stdlib/Coq.Init.Datatypes}{Some}{\coqdocconstructor{Some}}
(\coqref{Context.CFun}{\coqdocconstructor{CFun}} \coqdocvariable{f}
(\coqdoclemma{lt\_plus\_minus\_r}
\coqdocvariable{h})\coqdoceol
\coqdocindent{14.00em}
(\coqdocdefinition{vtake}
(\coqexternalref{http://coq.inria.fr/stdlib/Coq.Arith.Lt}{ltleweak}{\coqdoclemma{lt\_le\_weak}}
\coqdocvariable{n} (\coqdocprojection{arity}
\coqdocvariable{f}) \coqdocvariable{h})
\coqdocvariable{args})\coqdoceol
\coqdocindent{14.00em}
\coqdocvariable{C}\coqdoceol
\coqdocindent{14.00em}
(\coqdocdefinition{vdrop} \coqdocvariable{h}
\coqdocvariable{args}))\coqdoceol
\coqdocindent{7.00em}
\coqdockw{end}\coqdoceol
\coqdocindent{5.00em}
\ensuremath{|}
\coqexternalref{http://coq.inria.fr/stdlib/Coq.Init.Specif}{right}{\coqdocconstructor{right}}
\coqdocvar{\_} \ensuremath{\Rightarrow}
\coqexternalref{http://coq.inria.fr/stdlib/Coq.Init.Datatypes}{None}{\coqdocconstructor{None}}\coqdoceol
\coqdocindent{5.00em}
\coqdockw{end}\coqdoceol
\coqdocindent{3.00em}
\coqdockw{end}\coqdoceol
\coqdocindent{1.00em}
\coqdockw{end}.\coqdoceol
\end{coqdoccode}
\end{singlespace}
Now \begin{coqdoccode}\coqref{Term.subterm}{\coqdocdefinition{subterm}}
  \coqdocvariable{t} \coqdocvariable{p}\end{coqdoccode} gives the
subterm of \coqdocvariable{t} (if it exists)
and \begin{coqdoccode}\coqref{Context.dig}{\coqdocdefinition{dig}}
  \coqdocvariable{t} \coqdocvariable{p}\end{coqdoccode} gives the
context \coqdocvariable{C} (if it exists) that is \coqdocvariable{t}
with \begin{coqdoccode}\coqref{Term.subterm}{\coqdocdefinition{subterm}}
  \coqdocvariable{t} \coqdocvariable{p}\end{coqdoccode} replaced by
$\Box$ at position \coqdocvariable{p}.


\section{Transfinite Rewrite Sequences}\label{sec:seq}

In this section we present the essence of our development. Rewrite
sequences of length $\alpha$ are represented using the tree structure
of the tree ordinal $\alpha$. Much of the definitions on ordinals
are lifted to rewrite sequences and we once more come to a notion of
well-formedness. The resulting representation is discussed in relation
to the traditional theory of rewriting in
Section~\ref{sec:convergence}.

Throughout this section, we let $\mathcal{R}$ be a fixed TRS. We
define the type of steps using rewrite rules in $\mathcal{R}$,
parameterised by their source and target terms. These terms are generalised up
to bisimilarity, motivated in Subsection~\ref{sub:bissteps}.
\begin{singlespace}
\begin{coqdoccode}
\coqdocnoindent
\coqdockw{Inductive} \coqdef{Rewriting.step}{step}{$\rightarrow_\mathcal{R}$} :
\coqref{Term.term}{\coqdocinductive{term}} \ensuremath{\rightarrow}
\coqref{Term.term}{\coqdocinductive{term}} \ensuremath{\rightarrow}
\coqdockw{Type} :=\coqdoceol
\coqdocindent{1.00em}
\ensuremath{|} \coqdef{Rewriting.Step}{Step}{\coqdocconstructor{Step}} :
\ensuremath{\forall} (\coqdocvar{s} \coqdocvar{t} :
\coqref{Term.term}{\coqdocinductive{term}}) (\coqdocvar{$\rho$} :
\coqdocrecord{rule}) (\coqdocvar{C} :
\coqref{Context.context}{\coqdocinductive{context}}) (\coqdocvar{$\sigma$} :
\coqref{Substitution.substitution}{\coqdocdefinition{substitution}}),\coqdoceol
\coqdocindent{6.50em} \coqdocvariable{$\rho$}
\coqexternalref{http://coq.inria.fr/stdlib/Coq.Lists.List}{In}{\coqdocdefinition{$\in$}}
\coqdocvar{$\mathcal{R}$} \ensuremath{\rightarrow}\coqdoceol
\coqdocindent{6.50em}
\coqdocvariable{C}[(\coqref{Rewriting.lhs}{\coqdocprojection{lhs}}
\coqdocvariable{$\rho$})\coqdocvariable{$^\sigma$}] \coqref{TermEquality.termbis}{$\bis$} \coqdocvariable{s}
\ensuremath{\rightarrow}\coqdoceol
\coqdocindent{6.50em}
\coqdocvariable{C}[(\coqref{Rewriting.rhs}{\coqdocprojection{rhs}}
\coqdocvariable{$\rho$})\coqdocvariable{$^\sigma$}] \coqref{TermEquality.termbis}{$\bis$} \coqdocvariable{t}
\ensuremath{\rightarrow}\coqdoceol
\coqdocindent{6.50em}
\coqdocvariable{s} \coqref{Rewriting.step}{$\rightarrow_\mathcal{R}$}
\coqdocvariable{t}.\coqdoceol
\end{coqdoccode}
\end{singlespace}
For the translation of Definition~\ref{def:stepeq} (equality of steps)
to \Coq, we assume the lifting of bisimilarity to contexts and that
\coqdef{Substitution.substitutioneq}{substitution\_eq}{\coqdocdefinition{substitution\_eq}}
defines agreement of substitutions on a given list of variables.
\begin{singlespace}
\begin{coqdoccode}
\coqdocnoindent
\coqdockw{Definition}
\coqdef{Rewriting.stepeq}{step\_eq}{$\approx$}
(\coqdocvar{s} \coqdocvar{t} :
\coqref{Term.term}{\coqdocinductive{term}}) (\coqdocvar{$\pi$} :
\coqdocvar{s} \coqref{Rewriting.step}{$\rightarrow_\mathcal{R}$} \coqdocvar{t}) (\coqdocvar{u} \coqdocvar{v} :
\coqref{Term.term}{\coqdocinductive{term}}) (\coqdocvar{$o$} :
\coqdocvar{u} \coqref{Rewriting.step}{$\rightarrow_\mathcal{R}$}
\coqdocvar{v}) : \coqdockw{Prop} :=\coqdoceol
\coqdocindent{1.00em}
\coqdockw{match} \coqdocvariable{$\pi$}, \coqdocvariable{$o$}
\coqdockw{with}\coqdoceol
\coqdocindent{1.00em}
\ensuremath{|} \coqref{Rewriting.Step}{\coqdocconstructor{Step}}
\coqdocvar{\_} \coqdocvar{\_} \coqdocvar{$\rho$} \coqdocvar{C}
\coqdocvar{$\sigma$} \coqdocvar{\_} \coqdocvar{\_} \coqdocvar{\_},
\coqref{Rewriting.Step}{\coqdocconstructor{Step}} \coqdocvar{\_}
\coqdocvar{\_} \coqdocvar{$\rho'$} \coqdocvar{C$'$} \coqdocvar{$\sigma'$}
\coqdocvar{\_} \coqdocvar{\_} \coqdocvar{\_}
\ensuremath{\Rightarrow}\coqdoceol
\coqdocindent{2.00em}
\coqdocvariable{C} $\bis$ \coqdocvariable{C$'$} \ensuremath{\land}
\coqdocvariable{$\rho$} = \coqdocvariable{$\rho'$} \ensuremath{\land}
\coqref{Substitution.substitutioneq}{\coqdocdefinition{substitution\_eq}}
(\coqdocdefinition{vars}
(\coqref{Rewriting.lhs}{\coqdocprojection{lhs}} \coqdocvariable{$\rho$}))
\coqdocvariable{$\sigma$} \coqdocvariable{$\sigma'$}\coqdoceol
\coqdocindent{1.00em}
\coqdockw{end}.\coqdoceol
\end{coqdoccode}
\end{singlespace}
% TODO: naming of variables is not so nice here

We describe a way to define rewrite sequences as an inductive type. A rewrite
sequence of length $\alpha$ can be represented by the tree ordinal $\alpha$
where we label every occurrence of the $^+$ constructor with a rewrite
step. To ensure that successive steps have the same target and source terms,
respectively, we include the source and target terms of the rewrite sequence
in its type and label accordingly.

At this point, it is not immediately clear what the type of the limit
constructor should be. Following the tree ordinals, we think of a rewrite
sequence as a countably branching tree with every branching node representing
the least upper bound of its branches. As a first step towards a type of
rewrite sequences, we write down an incomplete try. Note that the
\coqdef{Rewriting.Cons}{Cons}{\coqdocconstructor{Cons}} constructor appends
(not prepends) a step to a sequence.
\begin{singlespace}
\begin{coqdoccode}
\coqdocnoindent
\coqdockw{Inductive}
\coqdef{Rewriting.sequence}{sequence}{$\rewrites_\mathcal{R}$} :
\coqref{Term.term}{\coqdocinductive{term}} \ensuremath{\rightarrow}
\coqref{Term.term}{\coqdocinductive{term}} \ensuremath{\rightarrow}
\coqdockw{Type} :=\coqdoceol
\coqdocindent{1.00em}
\ensuremath{|} \coqdef{Rewriting.Nil}{Nil}{\coqdocconstructor{Nil}} :
\ensuremath{\forall} \coqdocvar{t}, \coqdocvariable{t}
\coqref{Rewriting.sequence}{$\rewrites_\mathcal{R}$}
\coqdocvariable{t}\coqdoceol \coqdocindent{1.00em}
\ensuremath{|} \coqdef{Rewriting.Cons}{Cons}{\coqdocconstructor{Cons}}:
\ensuremath{\forall} \coqdocvar{s} \coqdocvar{t} \coqdocvar{u}, (\coqdocvar{s}
\coqref{Rewriting.sequence}{$\rewrites_\mathcal{R}$} \coqdocvar{t})
$\rightarrow$
(\coqdocvariable{t} \coqref{Rewriting.step}{$\rightarrow_\mathcal{R}$}
\coqdocvar{u}) $\rightarrow$ (\coqdocvariable{s}
\coqref{Rewriting.sequence}{$\rewrites_\mathcal{R}$}
\coqdocvariable{u})\coqdoceol \coqdocindent{1.00em}
\ensuremath{|} \coqdocconstructor{Lim}   :
\ensuremath{\forall} \coqdocvar{s} \coqdocvar{t},
(\coqexternalref{http://coq.inria.fr/stdlib/Coq.Init.Datatypes}{nat}{\coqdocinductive{nat}}
\ensuremath{\rightarrow} \coqdocvariable{s}
\coqref{Rewriting.sequence}{$\rewrites_\mathcal{R}$}
\coqdef{Rewriting.LimPlaceholder}{LimPlaceholder}{\textbf{?}}) $\rightarrow$
(\coqdocvariable{s}
\coqref{Rewriting.sequence}{$\rewrites_\mathcal{R}$}
\coqdocvariable{t}).\coqdoceol
\end{coqdoccode}
\end{singlespace}
This is not yet satisfying, because we cannot fix a value for
\coqref{Rewriting.LimPlaceholder}{\textbf{?}}. We complete the type for
\coqref{Rewriting.Lim}{\coqdocconstructor{Lim}} as follows. First, we
parameterise it with the target terms of the branches. Second, we add the
condition that these terms must converge to the target term
\coqdocvariable{t}.
\begin{singlespace}
\begin{coqdoccode}
\coqdocindent{1.00em}\label{coq:lim}
\ensuremath{|} \coqdef{Rewriting.Lim}{Lim}{\coqdocconstructor{Lim}} :
\ensuremath{\forall} \coqdocvar{s} \coqdocvar{t}
(\coqdocvar{ts} :
\coqexternalref{http://coq.inria.fr/stdlib/Coq.Init.Datatypes}{nat}{\coqdocinductive{nat}}
\ensuremath{\rightarrow} \coqref{Term.term}{\coqdocinductive{term}}),\coqdoceol
\coqdocindent{5.0em}
(\ensuremath{\forall} \coqdocvar{n}, \coqdocvar{s}
\coqref{Rewriting.sequence}{$\rewrites_\mathcal{R}$}
\coqdocvar{ts} \coqdocvariable{n}) $\rightarrow$
\coqref{Rewriting.converges}{\coqdocdefinition{converges}} \coqdocvariable{ts}
\coqdocvariable{t} $\rightarrow$
(\coqdocvariable{s}
\coqref{Rewriting.sequence}{$\rewrites_\mathcal{R}$}
\coqdocvariable{t})\coqdoceol
\end{coqdoccode}
\end{singlespace}
% TODO: maybe first convergence, then branches

Of course, the branches of a \coqref{Rewriting.Lim}{\coqdocconstructor{Lim}}
constructor may still not actually approximate to a rewrite sequence (of
length a limit ordinal). The intuition is that each branch should extend on
its preceding ones. This would correspond to the
\coqref{WfOrdinal.wf}{\coqdocdefinition{wf}} property we defined on
\coqref{Ordinal.ord}{\coqdocinductive{ord}}, where we lift $\prec$ to a
strict prefix relation on
\coqref{Rewriting.sequence}{$\rewrites_\mathcal{R}$}.
We return to this issue in Subsection~\ref{sub:wf}, but
first consider the definition of an embedding relation on
\coqref{Rewriting.sequence}{$\rewrites_\mathcal{R}$}.


\subsection{Embeddings of Rewrite Sequences}\label{sub:embedding}

We lift the notions of predecessor and predecessor indices to the domain of
rewrite sequences. The set of predecessor indices is easily defined as
\coqref{Rewriting.predtype}{\coqdocdefinition{pred\_type}}.
%\footnote{We employ some notational overloading by reusing $I(\_)$ and
%  $[\_]\_$ for the corresponding definitions on rewrite sequences.}
\begin{singlespace}
\begin{coqdoccode}
\coqdocnoindent
\coqdockw{Fixpoint}
\coqdef{Rewriting.predtype}{pred\_type}{\coqdocdefinition{pred\_type}}
\coqdocvar{s} \coqdocvar{t}
(\coqdocvar{$\varphi$} : \coqdocvar{s}
\coqref{Rewriting.sequence}{$\rewrites_\mathcal{R}$} \coqdocvar{t}) :
\coqdockw{Type} :=\coqdoceol
\coqdocindent{1.00em}
\coqdockw{match} \coqdocvariable{$\varphi$} \coqdockw{with}\coqdoceol
\coqdocindent{1.00em}
\ensuremath{|} \coqref{Rewriting.Nil}{\coqdocconstructor{Nil}} \coqdocvar{\_}
\ensuremath{\Rightarrow}
\coqexternalref{http://coq.inria.fr/stdlib/Coq.Init.Logic}{False}{\coqdocinductive{empty}}\coqdoceol
\coqdocindent{1.00em}
\ensuremath{|} \coqref{Rewriting.Cons}{\coqdocconstructor{Cons}}
\coqdocvar{\_} \coqdocvar{\_} \coqdocvar{\_} \coqdocvar{$\psi$} \coqdocvar{\_}
\ensuremath{\Rightarrow}
\coqexternalref{http://coq.inria.fr/stdlib/Coq.Init.Datatypes}{unit}{\coqdocinductive{unit}}
+ \coqref{Rewriting.predtype}{\coqdocdefinition{pred\_type}}
\coqdocvariable{$\psi$}\coqdoceol
\coqdocindent{1.00em}
\ensuremath{|} \coqref{Rewriting.Lim}{\coqdocconstructor{Lim}} \coqdocvar{\_}
\coqdocvar{\_} \coqdocvar{\_} \coqdocvar{f} \coqdocvar{\_}
\ensuremath{\Rightarrow} \{ \coqdocvar{n} :
\coqexternalref{http://coq.inria.fr/stdlib/Coq.Init.Datatypes}{nat}{\coqdocinductive{nat}}
\& \coqref{Rewriting.predtype}{\coqdocdefinition{pred\_type}}
(\coqdocvariable{f} \coqdocvariable{n}) \}\coqdoceol
\coqdocindent{1.00em}
\coqdockw{end}.\coqdoceol
\end{coqdoccode}
\end{singlespace}

The predecessor indices defined by
\coqref{Rewriting.predtype}{\coqdocdefinition{pred\_type}} point to a specific
occurrence of the \coqref{Rewriting.Cons}{\coqdocconstructor{Cons}}
constructor in a rewrite sequence. This constructor does not only contain a
rewrite sequence (analogous to an ordinal in the
\coqref{Ordinal.ord}{\coqdocinductive{ord}} case), but also a rewrite
step. The \coqref{Rewriting.pred}{\coqdocdefinition{pred}} function gives us
both the rewrite sequence and the step pointed to by a predecessor index. For
the type checker to accept the
definition, we use a $\Sigma$-type that contains this pair, parameterised by
the source and target terms of the rewrite step.
\begin{singlespace}
\begin{coqdoccode}
\coqdocnoindent
\coqdockw{Fixpoint} \coqdef{Rewriting.pred}{pred}{\coqdocdefinition{pred}}
\coqdocvar{s} \coqdocvar{t} (\coqdocvar{$\varphi$} : \coqdocvar{s}
\coqref{Rewriting.sequence}{$\rewrites_\mathcal{R}$} \coqdocvar{t})
(\coqdocvar{$\iota$} : \coqref{Rewriting.predtype}{\coqdocdefinition{pred\_type}}
\coqdocvariable{$\varphi$})
:\coqdoceol \coqdocindent{2.00em}
\{ \coqdocvar{ts} :
\coqref{Term.term}{\coqdocinductive{term}} \ensuremath{\times}
\coqref{Term.term}{\coqdocinductive{term}} \&
(\coqdocvariable{s} \coqref{Rewriting.sequence}{$\rewrites_\mathcal{R}$}
\coqexternalref{http://coq.inria.fr/stdlib/Coq.Init.Datatypes}{fst}{\coqdocdefinition{fst}}
\coqdocvariable{ts}) \ensuremath{\times}
(\coqexternalref{http://coq.inria.fr/stdlib/Coq.Init.Datatypes}{fst}{\coqdocdefinition{fst}}
\coqdocvariable{ts} \coqref{Rewriting.step}{$\rightarrow_\mathcal{R}$}
\coqexternalref{http://coq.inria.fr/stdlib/Coq.Init.Datatypes}{snd}{\coqdocdefinition{snd}}
\coqdocvariable{ts}) \} :=\coqdoceol
\coqdocindent{1.00em}
\coqdockw{match} \coqdocvariable{$\varphi$} \coqdockw{with}\coqdoceol
\coqdocindent{1.00em}
\ensuremath{|} \coqref{Rewriting.Nil}{\coqdocconstructor{Nil}} \coqdocvar{\_}
\ensuremath{\Rightarrow}
(\coqexternalref{http://coq.inria.fr/stdlib/Coq.Init.Logic}{Falserect}{\coqdocdefinition{empty\_rect}}
\coqdocvar{\_}) \coqdocvariable{$\iota$} \coqdoceol
\coqdocindent{1.00em}
\ensuremath{|} \coqref{Rewriting.Cons}{\coqdocconstructor{Cons}} \coqdocvar{\_}
\coqdocvar{u} \coqdocvar{t} \coqdocvar{$\psi$} \coqdocvar{$\pi$}
\ensuremath{\Rightarrow}
\coqdockw{match} \coqdocvariable{$\iota$} \coqdockw{with}\coqdoceol
\coqdocindent{10.00em}
\ensuremath{|}
\coqexternalref{http://coq.inria.fr/stdlib/Coq.Init.Datatypes}{inl}{\coqdocconstructor{inl}}
\coqexternalref{http://coq.inria.fr/stdlib/Coq.Init.Datatypes}{tt}{\coqdocconstructor{tt}}
\ensuremath{\Rightarrow}
\coqexternalref{http://coq.inria.fr/stdlib/Coq.Init.Specif}{existT}{\coqdocconstructor{existT}}
\coqdocvar{\_} (\coqdocvariable{u}, \coqdocvariable{t})
(\coqdocvariable{$\psi$}, \coqdocvariable{$\pi$})\coqdoceol
\coqdocindent{10.00em}
\ensuremath{|}
\coqexternalref{http://coq.inria.fr/stdlib/Coq.Init.Datatypes}{inr}{\coqdocconstructor{inr}}
\coqdocvar{$\kappa$}  \ensuremath{\Rightarrow}
\coqref{Rewriting.pred}{\coqdocdefinition{pred}} \coqdocvariable{$\psi$}
\coqdocvariable{$\kappa$}\coqdoceol
\coqdocindent{10.00em}
\coqdockw{end}\coqdoceol
\coqdocindent{1.00em}
\ensuremath{|} \coqref{Rewriting.Lim}{\coqdocconstructor{Lim}} \coqdocvar{\_}
\coqdocvar{\_} \coqdocvar{\_} \coqdocvar{f} \coqdocvar{\_}
\ensuremath{\Rightarrow}
\coqdockw{match} \coqdocvariable{$\iota$} \coqdockw{with}\coqdoceol
\coqdocindent{10.00em}
\ensuremath{|}
\coqexternalref{http://coq.inria.fr/stdlib/Coq.Init.Specif}{existT}{\coqdocconstructor{existT}}
\coqdocvar{n} \coqdocvar{$\kappa$} \ensuremath{\Rightarrow}
\coqref{Rewriting.pred}{\coqdocdefinition{pred}} (\coqdocvariable{f}
\coqdocvariable{n}) \coqdocvariable{$\kappa$}\coqdoceol
\coqdocindent{10.00em}
\coqdockw{end}\coqdoceol
\coqdocindent{1.00em}
\coqdockw{end}.\coqdoceol
\end{coqdoccode}
\end{singlespace}

% TODO: too dramatic
In an effort to prevent getting lost in a syntactical labyrinth, we define
the following notational shortcuts:
\begin{center}
{\renewcommand{\arraystretch}{1.1}
\renewcommand{\tabcolsep}{8pt}
\begin{tabular}{lll}
& \textsc{description} & \textsc{definition}\\
\hline
\coqdocvar{$\varphi$}[\coqdocvar{$\iota$}] & location in
\coqdocvar{$\varphi$} indexed by \coqdocvar{$\iota$} &
  \begin{coqdoccode}\coqref{Rewriting.pred}{\coqdocdefinition{pred}} \coqdocvar{$\varphi$} \coqdocvar{$\iota$}\end{coqdoccode} \\
\coqdocvar{$\varphi$}[\coqdocvar{$\iota$}]$^\textsc{seq}$ & rewrite sequence in
  \coqdocvar{$\varphi$} indexed by \coqdocvar{$\iota$}
  & \begin{coqdoccode}\coqexternalref{http://coq.inria.fr/stdlib/Coq.Init.Datatypes}{fst}{\coqdocdefinition{fst}}
      (\coqexternalref{http://coq.inria.fr/stdlib/Coq.Init.Specif}{projT2}{\coqdocdefinition{projT2}}
      (\coqref{Rewriting.pred}{\coqdocdefinition{pred}} \coqdocvar{$\varphi$} \coqdocvar{$\iota$}))\end{coqdoccode}
  \\
\coqdocvar{$\varphi$}[\coqdocvar{$\iota$}]$^\textsc{stp}$ & step of \coqdocvar{$\varphi$} indexed by \coqdocvar{$\iota$} &
  \begin{coqdoccode}\coqexternalref{http://coq.inria.fr/stdlib/Coq.Init.Datatypes}{snd}{\coqdocdefinition{snd}}
    (\coqexternalref{http://coq.inria.fr/stdlib/Coq.Init.Specif}{projT2}{\coqdocdefinition{projT2}}
    (\coqref{Rewriting.pred}{\coqdocdefinition{pred}} \coqdocvar{$\varphi$} \coqdocvar{$\iota$}))\end{coqdoccode}
  \\
\coqdocvar{$\varphi$}[\coqdocvar{$\iota$}]$^\textsc{l}$ & source term of \coqdocvar{$\varphi$}[\coqdocvar{$\iota$}]$^\textsc{stp}$
  & \begin{coqdoccode}\coqexternalref{http://coq.inria.fr/stdlib/Coq.Init.Datatypes}{fst}{\coqdocdefinition{fst}}
      (\coqexternalref{http://coq.inria.fr/stdlib/Coq.Init.Specif}{projT1}{\coqdocdefinition{projT1}}
      (\coqref{Rewriting.pred}{\coqdocdefinition{pred}} \coqdocvar{$\varphi$} \coqdocvar{$\iota$}))\end{coqdoccode}
  \\
\coqdocvar{$\varphi$}[\coqdocvar{$\iota$}]$^\textsc{r}$ & target term of
  \coqdocvar{$\varphi$}[\coqdocvar{$\iota$}]$^\textsc{stp}$ & \begin{coqdoccode}\coqexternalref{http://coq.inria.fr/stdlib/Coq.Init.Datatypes}{snd}{\coqdocdefinition{snd}}
    (\coqexternalref{http://coq.inria.fr/stdlib/Coq.Init.Specif}{projT1}{\coqdocdefinition{projT1}}
    (\coqref{Rewriting.pred}{\coqdocdefinition{pred}} \coqdocvar{$\varphi$} \coqdocvar{$\iota$}))\end{coqdoccode}
\end{tabular}}
\end{center}
As an example of predecessor indexing, consider the graphical
representation of a rewrite sequence $\varphi$ of length $\omega + 2$
and its predecessor index $\iota = \coqdocconstructor{inr} \;
(\coqdocconstructor{inr} \; \langle 4,
\coqdocconstructor{inl}\rangle)$ in Figure~\ref{fig:pred}. The initial
part of length $\omega$ is represented by a series of finite rewrite
sequences, each one extending on the previous one by one step. The
sequence of terms $\{ t_1,t_2, t_3, \ldots \}$ converges to the term
$t_\omega$. Here, $\varphi[\iota]^\textsc{seq}$ is a rewrite sequence
from $t_1$ to $t_3$ and $\varphi[\iota]^\textsc{stp}$ is a step from
$t_3$ to $t_4$.

\begin{figure}
\begin{center}
\begin{tikzpicture}[scale=0.85]
  \node at (1, 0) {$t_{1}$};
  \foreach \i in {2, ..., 5}{%
    \begin{scope}[start chain=\i,every join/.style=->,node
        distance=0.5]
      \node [on chain=\i, join] at (1, -\i + 1) {$t_{1}$};
      \foreach \j in {2, ..., \i}{%
        \node (node\i\j) [on chain=\i, join] {$t_{\j}$};
      }
    \end{scope}
  }
  \node at (1, -5) {$\vdots$};
  \node at (5, -5) {$\ddots$};
  \begin{scope}[start chain=i,every join/.style=->,node distance=0.5]
    \node [on chain=i, join] at (1, -6) {$t_{1}$};
    \foreach \j in {2, ..., 5}{%
      \node [on chain=i, join] {$t_{\j}$};
    }
    \node [on chain=i, join] {$\cdots$};
    \node [on chain=i, join] {$t_{i}$};
  \end{scope}
  \node at (1, -7) {$\vdots$};
  \node at (7, -7) {$\ddots$};
  \begin{scope}[start chain=o,every join/.style=->,node distance=0.5]
    \node (nodeO) [on chain=o, join] at (10, -3.5) {$t_\omega$};
    \node (nodeO1) [on chain=o, join] {$t_{\omega+1}$};
    \node (nodeO2) [on chain=o, join] {$t_{\omega+2}$};
  \end{scope}
  \begin{scope}[auto]
    \draw [->, thick] (nodeO2) to [out=110,in=70] node [swap]
          {\coqdocconstructor{inr}}
          (nodeO1);
    \draw [->, thick] (nodeO1) to [out=110,in=70] node [swap]
          {\coqdocconstructor{inr}}
          (nodeO);
    \draw [->, thick] (nodeO) to [out=100,in=50] node [swap] {4}
    (node44);
    \draw [->, thick] (node44) to [out=110,in=70] node [swap]
          {\coqdocconstructor{inl}}
          (node43);
  \end{scope}

\end{tikzpicture}
\end{center}
\caption{Example of a rewrite sequence and predecessor
  index.}\label{fig:pred}
\end{figure}

% TODO: better wording for 'cancelled out'?
% TODO: picture of this embedding behaviour?
Having a closer look at the order $\preceq$ on the tree ordinals, we can
see that it really defines embeddings of their tree structures. This is due to
clause {\sc \ref{def:order:succ}} of Definition~\ref{def:order}. In
this clause, two occurrences of the $^+$ constructor (one in both
ordinals) are cancelled out against each other, but the positions of
these occurrences in their respective ordinals do not necessarily
correspond. Since occurrences of $^+$ carry no additional information,
this has no effect on the resulting relation.

% TODO: should we use the word 'iff'?
What this means for a translation of $\preceq$ to the domain of our
inductively defined rewrite sequences is that, indeed, we get an embedding
relation. We only have to make sure that in the
\coqref{Rewriting.Cons}{\coqdocconstructor{Cons}} case, we cancel out two
equal steps against each other.
We say that $\varphi$ is embedded in $\psi$ (written $\varphi
\sqsubseteq \psi$) if $\psi$ can be obtained from $\varphi$ by inserting
any number of steps in $\varphi$. We distinguish between inserting a step
\begin{inparaenum}[(i)]
  \item before the first step,
  \item after the last step and
  \item in between steps
\end{inparaenum}
in a rewrite sequence. Note that any steps inserted consecutively in between
steps necessarily form a cycle, because of the typing constraints in
the definition of rewrite sequence.
% is this good or bad? (cycle)
\begin{singlespace}
\begin{coqdoccode}
\coqdocnoindent
\coqdockw{Inductive} \coqdef{Rewriting.embed}{embed}{$\sqsubseteq$}
: \ensuremath{\forall} \coqdocvar{s} \coqdocvar{t} \coqdocvar{u}
\coqdocvar{v}, (\coqdocvariable{s}
\coqref{Rewriting.sequence}{$\rewrites_\mathcal{R}$} \coqdocvariable{t})
$\rightarrow$ (\coqdocvariable{u}
\coqref{Rewriting.sequence}{$\rewrites_\mathcal{R}$} \coqdocvariable{v})
$\rightarrow$ \coqdockw{Prop} :=\coqdoceol
\coqdocindent{1.00em}
\ensuremath{|}
\coqdef{Rewriting.EmbedNil}{Embed\_Nil}{\coqdocconstructor{$\sqsubseteq_\text{Nil}$}}  :
\ensuremath{\forall} \coqdocvar{s} \coqdocvar{u} \coqdocvar{v} (\coqdocvar{$\psi$}
: \coqdocvariable{u} \coqref{Rewriting.sequence}{$\rewrites_\mathcal{R}$}
\coqdocvariable{v}),\coqdoceol
\coqdocindent{9.50em}
\coqref{Rewriting.Nil}{\coqdocconstructor{Nil}} \coqdocvariable{s}
\coqref{Rewriting.embed}{$\sqsubseteq$} \coqdocvariable{$\psi$}\coqdoceol
\coqdocindent{1.00em}
\ensuremath{|}
\coqdef{Rewriting.EmbedCons}{Embed\_Cons}{\coqdocconstructor{$\sqsubseteq_\text{Cons}$}} :
\ensuremath{\forall} \coqdocvar{s} \coqdocvar{t} \coqdocvar{u} \coqdocvar{v}
(\coqdocvar{$\psi$}
: \coqdocvariable{u} \coqref{Rewriting.sequence}{$\rewrites_\mathcal{R}$}
\coqdocvariable{v}) (\coqdocvar{$\iota$} :
\coqref{Rewriting.predtype}{\coqdocdefinition{pred\_type}}
\coqdocvariable{$\psi$})
(\coqdocvar{$\varphi$} : \coqdocvariable{s}
\coqref{Rewriting.sequence}{$\rewrites_\mathcal{R}$}
\coqdocvariable{$\psi$}[\coqdocvariable{$\iota$}]$^\textsc{l}$)\coqdoceol
\coqdocindent{5.00em}
(\coqdocvar{$\pi$} :
\coqdocvariable{$\psi$}[\coqdocvariable{$\iota$}]$^\textsc{l}$
\coqref{Rewriting.step}{$\rightarrow_\mathcal{R}$}
\coqdocvariable{t}),\coqdoceol
\coqdocindent{9.50em}
\coqdocvariable{$\varphi$} \coqref{Rewriting.embed}{$\sqsubseteq$}
\coqdocvariable{$\psi$}[\coqdocvariable{$\iota$}]$^\textsc{seq}$
\ensuremath{\rightarrow}\coqdoceol
\coqdocindent{9.50em}
\coqdocvariable{$\pi$} \coqref{Rewriting.stepeq}{$\approx$}
\coqdocvariable{$\psi$}[\coqdocvariable{$\iota$}]$^\textsc{stp}$
\ensuremath{\rightarrow}\coqdoceol
\coqdocindent{9.50em}
\coqref{Rewriting.Cons}{\coqdocconstructor{Cons}}
\coqdocvariable{$\varphi$} \coqdocvariable{$\pi$}
\coqref{Rewriting.embed}{$\sqsubseteq$}
\coqdocvariable{$\psi$}\coqdoceol
\coqdocindent{1.00em}
\ensuremath{|}
\coqdef{Rewriting.EmbedLim}{Embed\_Lim}{\coqdocconstructor{$\sqsubseteq_\text{Lim}$}}  :
\ensuremath{\forall} \coqdocvar{s} \coqdocvar{t} \coqdocvar{u} \coqdocvar{v}
(\coqdocvar{ts} :
\coqexternalref{http://coq.inria.fr/stdlib/Coq.Init.Datatypes}{nat}{\coqdocinductive{nat}}
\ensuremath{\rightarrow} \coqref{Term.term}{\coqdocinductive{term}})
(\coqdocvar{f} : \ensuremath{\forall} \coqdocvar{n},
\coqdocvariable{s}
\coqref{Rewriting.sequence}{$\rewrites_\mathcal{R}$}
\coqdocvariable{ts} \coqdocvariable{n})\coqdoceol
\coqdocindent{5.00em}
(\coqdocvar{c} :
\coqref{Rewriting.converges}{\coqdocdefinition{converges}} \coqdocvariable{ts}
\coqdocvar{t}) (\coqdocvar{$\psi$} : \coqdocvar{u}
\coqref{Rewriting.sequence}{$\rewrites_\mathcal{R}$}
\coqdocvar{v}),\coqdoceol
\coqdocindent{9.50em}
(\ensuremath{\forall} \coqdocvar{n}, \coqdocvariable{f} \coqdocvariable{n}
\coqref{Rewriting.embed}{$\sqsubseteq$} \coqdocvariable{$\psi$})
\ensuremath{\rightarrow}\coqdoceol
\coqdocindent{9.50em}
\coqref{Rewriting.Lim}{\coqdocconstructor{Lim}} \coqdocvariable{f}
\coqdocvariable{c} \coqref{Rewriting.embed}{$\sqsubseteq$}
\coqdocvariable{$\psi$}.\coqdoceol
\end{coqdoccode}
\end{singlespace}

Analogous to the strict order $\prec$ on ordinals, we define a strict
embedding relation $\sqsubset$ on rewrite sequences.
\begin{singlespace}
\begin{coqdoccode}
\coqdocnoindent
\coqdockw{Definition}
\coqdef{Rewriting.embedstrict}{embed\_strict}{$\sqsubset$}
\coqdocvar{s} \coqdocvar{t} \coqdocvar{u} \coqdocvar{v}
(\coqdocvar{$\varphi$} : \coqdocvariable{s}
\coqref{Rewriting.sequence}{$\rewrites_\mathcal{R}$}
\coqdocvariable{t},
\coqdocvar{$\psi$} :
\coqdocvariable{u}
\coqref{Rewriting.sequence}{$\rewrites_\mathcal{R}$}
\coqdocvariable{v}) := \ensuremath{\exists} \coqdocvar{$\iota$},
\coqdocvariable{$\varphi$} \coqref{Rewriting.embed}{$\sqsubseteq$}
\coqdocvariable{$\psi$}[\coqdocvariable{$\iota$}]$^\textsc{seq}$.\coqdoceol
\end{coqdoccode}
\end{singlespace}
Note that, while non-strictly embedded rewrite sequences may differ in
any of the three ways defined above, strictly embedded rewrite
sequences always differ in their last step. Thus, if $\varphi$ is
strictly embedded in $\psi$ then $\psi$ can be obtained from $\varphi$
by inserting any number of steps in $\varphi$, but at least one after
the last step.


\subsection{Well-formed Rewrite Sequences}\label{sub:wf}

The \coqref{WfOrdinal.wf}{\coqdocdefinition{wf}} property on
\coqref{Ordinal.ord}{\coqdocinductive{ord}} is defined in
Section~\ref{sec:ordimp} to rule out a
certain class of ordinal representations. This issue translates
directly to our inductive representation of rewrite sequences. We
define a well-formedness property
\coqref{Rewriting.wf}{\coqdocdefinition{wf}} on rewrite sequences,
using the strict embedding relation $\sqsubset$.
\begin{singlespace}
\begin{coqdoccode}
\coqdocnoindent
\coqdockw{Fixpoint} \coqdef{Rewriting.wf}{wf}{\coqdocdefinition{wf}}
\coqdocvar{s} \coqdocvar{t}
(\coqdocvar{$\varphi$} : \coqdocvariable{s}
\coqref{Rewriting.sequence}{$\rewrites_\mathcal{R}$}
\coqdocvariable{t}) : \coqdockw{Prop}
:=\coqdoceol
\coqdocindent{1.00em}
\coqdockw{match} \coqdocvariable{$\varphi$} \coqdockw{with}\coqdoceol
\coqdocindent{1.00em}
\ensuremath{|} \coqref{Rewriting.Nil}{\coqdocconstructor{Nil}}
\coqdocvar{\_}          \ensuremath{\Rightarrow}
\coqexternalref{http://coq.inria.fr/stdlib/Coq.Init.Logic}{True}{\coqdocinductive{True}}\coqdoceol
\coqdocindent{1.00em}
\ensuremath{|} \coqref{Rewriting.Cons}{\coqdocconstructor{Cons}}
\coqdocvar{\_} \coqdocvar{\_} \coqdocvar{$\psi$} \coqdocvar{\_}
\coqdocvar{\_} \ensuremath{\Rightarrow}
\coqref{Rewriting.wf}{\coqdocdefinition{wf}}
\coqdocvariable{\coqdocvariable{$\psi$}}\coqdoceol
\coqdocindent{1.00em}
\ensuremath{|} \coqref{Rewriting.Lim}{\coqdocconstructor{Lim}}
\coqdocvar{\_} \coqdocvar{\_} \coqdocvar{f} \coqdocvar{\_}
\coqdocvar{\_}  \ensuremath{\Rightarrow}
(\ensuremath{\forall} \coqdocvar{n},
\coqref{Rewriting.wf}{\coqdocdefinition{wf}} (\coqdocvariable{f}
\coqdocvariable{n})) \ensuremath{\land}
\ensuremath{\forall} \coqdocvar{n} \coqdocvar{m}, \coqdocvariable{n}
< \coqdocvariable{m} \ensuremath{\rightarrow} \coqdocvariable{f}
\coqdocvariable{n}
\coqref{Rewriting.embedstrict}{$\sqsubset$}
\coqdocvariable{f} \coqdocvariable{m}\coqdoceol
\coqdocindent{1.00em}
\coqdockw{end}.\coqdoceol
\end{coqdoccode}
\end{singlespace}

% TODO: check if this page ref is correct
On page~\pageref{coq:lim}, we define the
\coqref{Rewriting.Lim}{\coqdocconstructor{Lim}} constructor with the
intuition that each of its branches should extend on the preceding
ones. Naturally, we would implement this condition using a strict prefix
relation on rewrite sequences, but the strict embedding relation
$\sqsubset$ is also satisfying for this purpose.

Consider an instance of
\coqref{Rewriting.Lim}{\coqdocconstructor{Lim}}, satisfying
\coqref{Rewriting.wf}{\coqdocdefinition{wf}}, with branches
\coqdocvar{f}. For every \begin{coqdoccode}\coqdocvariable{n} <
  \coqdocvariable{m}\end{coqdoccode}, we
have \begin{coqdoccode}\coqdocvariable{f} \coqdocvariable{n}
  \coqref{Rewriting.embedstrict}{$\sqsubset$} \coqdocvariable{f}
  \coqdocvariable{m}\end{coqdoccode}. Thus there is a predecessor
sequence % \coqdocvariable{$\varphi$}
of \begin{coqdoccode}\coqdocvariable{f}
  \coqdocvariable{m}\end{coqdoccode} that can be obtained
from \begin{coqdoccode}\coqdocvariable{f}
  \coqdocvariable{n}\end{coqdoccode} by inserting any number of steps
in \begin{coqdoccode}\coqdocvariable{f}
  \coqdocvariable{n}\end{coqdoccode}. Steps inserted before the last
step of \begin{coqdoccode}\coqdocvariable{f}
  \coqdocvariable{n}\end{coqdoccode} must form a cycle (note that all
branches start at the same % TODO: is spacing ok after therefore, ?
term). Therefore, \begin{coqdoccode}\coqdocvariable{f}
  \coqdocvariable{m}\end{coqdoccode} can be obtained
from \begin{coqdoccode}\coqdocvariable{f}
  \coqdocvariable{n}\end{coqdoccode} by adding one or more steps at
the end of \begin{coqdoccode}\coqdocvariable{f}
  \coqdocvariable{n}\end{coqdoccode} and possibly inserting cycles at
other positions of \begin{coqdoccode}\coqdocvariable{f}
  \coqdocvariable{n}\end{coqdoccode}. This shows that, ignoring
cycles, $\sqsubset$ actually defines a strict prefix relation on the
branches of \coqdocvar{f}.

%Vincent said this about it:
%\begin{quote}
%Overigens bedachten dat het in principe niet heel erg is $\sqsubseteq$ te
%definieren voor reducties zoals voor ordinalen. alleen zie je dan het meer een
%notie van embedding/deelreductie ipv een notie van prefix geeft, maar ook daar
%kun je denkelijk goed mee werken.
%
%\ldots
%
%als je een stuk invoegt in het midden in sigma moet dat noodzakelijkerwijs,
%vanwege de constraints op begin-en eindpunten, een reductie cykel zijn. ik zou
%verwachten dat dat uiteindelijk een goede notie van ordening (goede reducties)
%oplevert (``de cykels doen er niet toe voor convergente rijen, en kun je
%weglaten bij compressie''), die natuurlijk niet overeenkomt, ook niet in het
%eindige geval met de prefix notie; het is deelwoord, of (op reductie rijtjes
%als bomen) deelboom (en niet boom-factor).
%\end{quote}

There is still an important omission in our formalisation
though: even rewrite sequences satisfying
\coqref{Rewriting.wf}{\coqdocdefinition{wf}} are not necessarily
convergent. How the convergence conditions from
Definition~\ref{def:convergence} relate to our formalisation is
discussed in Section~\ref{sec:convergence}.


\subsection{Combining Rewrite Sequences}\label{sub:combining}

With the \coqref{Rewriting.Cons}{\coqdocconstructor{Cons}}
constructor, we can extend a rewrite sequence with one step at the
end. Dually, \coqref{Rewriting.snoc}{\coqdocdefinition{snoc}} extends
a rewrite sequence with one step at the start. It is the analogue of
$1 \coqref{Ordinal.add}{+} \alpha$ on ordinals.

\coqref{Rewriting.snoc}{\coqdocdefinition{snoc}} is recursive in its
right argument, but for the \Coq type checker to accept our
definition, we must write it such that it consumes this argument
first.\footnote{The reason for this is rather technical, but the idea
  is that the return type nicely follows the case analysis on the
  rewrite sequence in the \coqdockw{match} construction. We also give
  some hints to the \Coq type checker that are not shown here.}
Hence, we use an auxiliary function
\coqref{Rewriting.snocrec}{\coqdocdefinition{snoc\_rec}}.
\begin{singlespace}
\begin{coqdoccode}
\coqdocnoindent
\coqdockw{Fixpoint}
\coqdef{Rewriting.snocrec}{snoc\_rec}{\coqdocdefinition{snoc\_rec}}
\coqdocvar{s} \coqdocvar{t} \coqdocvar{u} (\coqdocvar{$\varphi$} :
\coqdocvariable{t}
\coqref{Rewriting.sequence}{$\rewrites_\mathcal{R}$}
\coqdocvariable{u}) : (\coqdocvariable{s}
\coqref{Rewriting.step}{$\rightarrow_\mathcal{R}$}
\coqdocvariable{t}) \ensuremath{\rightarrow} (\coqdocvariable{s}
\coqref{Rewriting.sequence}{$\rewrites_\mathcal{R}$}
\coqdocvariable{u}) :=\coqdoceol
\coqdocindent{1.00em}
\coqdockw{match} \coqdocvariable{$\varphi$} \coqdockw{with}\coqdoceol
\coqdocindent{1.00em}
\ensuremath{|} \coqref{Rewriting.Nil}{\coqdocconstructor{Nil}}
\coqdocvar{\_}          \ensuremath{\Rightarrow} \coqdockw{fun}
\coqdocvar{$\pi$} \ensuremath{\Rightarrow}
\coqref{Rewriting.Cons}{\coqdocconstructor{Cons}}
(\coqref{Rewriting.Nil}{\coqdocconstructor{Nil}} \coqdocvariable{s})
\coqdocvariable{$\pi$}\coqdoceol
\coqdocindent{1.00em}
\ensuremath{|} \coqref{Rewriting.Cons}{\coqdocconstructor{Cons}}
\coqdocvar{\_} \coqdocvar{\_} \coqdocvar{$\psi$} \coqdocvar{\_}
\coqdocvar{$o$} \ensuremath{\Rightarrow} \coqdockw{fun} \coqdocvar{$\pi$}
\ensuremath{\Rightarrow}
\coqref{Rewriting.Cons}{\coqdocconstructor{Cons}}
(\coqref{Rewriting.snocrec}{\coqdocdefinition{snoc\_rec}}
\coqdocvariable{$\psi$} \coqdocvariable{$\pi$}) \coqdocvariable{$o$}\coqdoceol
\coqdocindent{1.00em}
\ensuremath{|} \coqref{Rewriting.Lim}{\coqdocconstructor{Lim}}
\coqdocvar{\_} \coqdocvar{\_} \coqdocvar{f} \coqdocvar{u}
\coqdocvar{c}  \ensuremath{\Rightarrow} \coqdockw{fun} \coqdocvar{$\pi$}
\ensuremath{\Rightarrow}
\coqref{Rewriting.Lim}{\coqdocconstructor{Lim}} (\coqdockw{fun}
\coqdocvar{$o$} \ensuremath{\Rightarrow}
\coqref{Rewriting.snocrec}{\coqdocdefinition{snoc\_rec}}
(\coqdocvariable{f} \coqdocvariable{$o$}) \coqdocvariable{$\pi$})
\coqdocvariable{c}\coqdoceol
\coqdocindent{1.00em}
\coqdockw{end}.\coqdoceol
\coqdocemptyline
\coqdocnoindent
\coqdockw{Definition}
\coqdef{Rewriting.snoc}{snoc}{\coqdocdefinition{snoc}} \coqdocvar{s}
\coqdocvar{t} \coqdocvar{u} (\coqdocvar{$\pi$}
: \coqdocvar{s} \coqref{Rewriting.step}{$\rightarrow_\mathcal{R}$}
\coqdocvar{t}) (\coqdocvar{$\varphi$} : \coqdocvariable{t}
\coqref{Rewriting.sequence}{$\rewrites_\mathcal{R}$} \coqdocvar{u}) :
\coqdocvariable{s}
\coqref{Rewriting.sequence}{$\rewrites_\mathcal{R}$}
\coqdocvariable{u} :=
\coqref{Rewriting.snocrec}{\coqdocdefinition{snoc\_rec}}
\coqdocvariable{$\varphi$} \coqdocvariable{$\pi$}.\coqdoceol
\end{coqdoccode}
\end{singlespace}
A related operation is concatenation of rewrite sequences, the
analogue of addition on ordinals. It is defined in the same way as
\coqref{Rewriting.snoc}{\coqdocdefinition{snoc}}.
\begin{singlespace}
\begin{coqdoccode}
\coqdocnoindent
\coqdockw{Fixpoint}
\coqdef{Rewriting.appendrec}{append\_rec}{\coqdocdefinition{concat\_rec}}
\coqdocvar{s} \coqdocvar{t} \coqdocvar{u} (\coqdocvar{$\psi$} :
\coqdocvariable{t}
\coqref{Rewriting.sequence}{$\rewrites_\mathcal{R}$}
\coqdocvariable{u}) : (\coqdocvariable{s}
\coqref{Rewriting.sequence}{$\rewrites_\mathcal{R}$}
\coqdocvariable{t}) \ensuremath{\rightarrow} (\coqdocvariable{s}
\coqref{Rewriting.sequence}{$\rewrites_\mathcal{R}$}
\coqdocvariable{u}) :=\coqdoceol
\coqdocindent{1.00em}
\coqdockw{match} \coqdocvariable{$\psi$} \coqdockw{with}\coqdoceol
\coqdocindent{1.00em}
\ensuremath{|} \coqref{Rewriting.Nil}{\coqdocconstructor{Nil}}
\coqdocvar{\_}         \ensuremath{\Rightarrow} \coqdockw{fun}
\coqdocvar{$\varphi$} \ensuremath{\Rightarrow} \coqdocvariable{$\varphi$}\coqdoceol
\coqdocindent{1.00em}
\ensuremath{|} \coqref{Rewriting.Cons}{\coqdocconstructor{Cons}}
\coqdocvar{\_} \coqdocvar{\_} \coqdocvar{$\psi$} \coqdocvar{\_}
\coqdocvar{$\pi$} \ensuremath{\Rightarrow} \coqdockw{fun} \coqdocvar{$\varphi$}
\ensuremath{\Rightarrow}
\coqref{Rewriting.Cons}{\coqdocconstructor{Cons}}
(\coqref{Rewriting.appendrec}{\coqdocdefinition{concat\_rec}}
\coqdocvariable{$\psi$} \coqdocvariable{$\varphi$}) \coqdocvariable{$\pi$}\coqdoceol
\coqdocindent{1.00em}
\ensuremath{|} \coqref{Rewriting.Lim}{\coqdocconstructor{Lim}}
\coqdocvar{\_} \coqdocvar{\_} \coqdocvar{f} \coqdocvar{u}
\coqdocvar{c}  \ensuremath{\Rightarrow} \coqdockw{fun} \coqdocvar{$\varphi$}
\ensuremath{\Rightarrow}
\coqref{Rewriting.Lim}{\coqdocconstructor{Lim}} (\coqdockw{fun}
\coqdocvar{$o$} \ensuremath{\Rightarrow}
\coqref{Rewriting.appendrec}{\coqdocdefinition{concat\_rec}}
(\coqdocvariable{f} \coqdocvariable{$o$}) \coqdocvariable{$\varphi$})
\coqdocvariable{c}\coqdoceol
\coqdocindent{1.00em}
\coqdockw{end}.\coqdoceol
\coqdocemptyline
\coqdocnoindent
\coqdockw{Definition}
\coqdef{Rewriting.append}{append}{\coqdocdefinition{concat}}
\coqdocvar{s} \coqdocvar{t} \coqdocvar{u} (\coqdocvar{$\varphi$} :
\coqdocvar{s}
\coqref{Rewriting.sequence}{$\rewrites_\mathcal{R}$} \coqdocvar{t})
(\coqdocvar{$\psi$} : \coqdocvariable{t}
\coqref{Rewriting.sequence}{$\rewrites_\mathcal{R}$} \coqdocvar{u}) :
\coqdocvariable{s}
\coqref{Rewriting.sequence}{$\rewrites_\mathcal{R}$}
\coqdocvariable{u} :=\coqdoceol
\coqdocindent{1.0em}
\coqref{Rewriting.appendrec}{\coqdocdefinition{concat\_rec}}
\coqdocvariable{$\psi$} \coqdocvariable{$\varphi$}.\coqdoceol
\end{coqdoccode}
\end{singlespace}

Well-formedness is preserved under concatenation.
\begin{singlespace}
\begin{coqdoccode}
\coqdocnoindent
\coqdockw{Lemma}
\coqdef{Rewriting.appendwf}{append\_wf}{\coqdoclemma{concat\_wf}} :
\ensuremath{\forall} \coqdocvar{s} \coqdocvar{t} \coqdocvar{u}
(\coqdocvar{$\varphi$} : \coqdocvar{s}
\coqref{Rewriting.sequence}{$\rewrites_\mathcal{R}$}
\coqdocvar{t}) (\coqdocvar{$\psi$} : \coqdocvariable{t}
\coqref{Rewriting.sequence}{$\rewrites_\mathcal{R}$}
\coqdocvar{u}),\coqdoceol
\coqdocindent{9.00em}
\coqref{Rewriting.wf}{\coqdocdefinition{wf}} \coqdocvariable{$\varphi$}
\ensuremath{\rightarrow}
\coqref{Rewriting.wf}{\coqdocdefinition{wf}} \coqdocvariable{$\psi$}
\ensuremath{\rightarrow}
\coqref{Rewriting.wf}{\coqdocdefinition{wf}}
(\coqref{Rewriting.append}{\coqdocdefinition{concat}}
\coqdocvariable{$\varphi$} \coqdocvariable{$\psi$}).\coqdoceol
\end{coqdoccode}
\end{singlespace}


\section{Properties of Terms and TRSs}

We define some predicates on terms and TRSs. Again, we let
$\mathcal{R}$ be a fixed TRS throughout this section.

We work with a somewhat relaxed definition of critical pairs. First,
we do not require the common instance to be a most general
one. Second, the substitution $\sigma$ might not be minimal and might
not introduce only fresh variables
(cf.\ Definition~\ref{def:overlap}). The effect of this relaxation is
that for every critical pair, we have a series of critical pairs by
this \Coq definition. This is precise enough for our present
purposes, however, since it has no effect on questions such as
\emph{are there critical pairs?} or \emph{are all critical pairs
  trivial?}.
\begin{singlespace}
\begin{coqdoccode}
\coqdocnoindent
\coqdockw{Definition}
\coqdef{Rewriting.criticalpair}{critical\_pair}{\coqdocdefinition{critical\_pair}}
(\coqdocvar{$\mathcal{R}$} : \coqdocdefinition{trs})
(\coqdocvar{t$_1$} \coqdocvar{t$_2$} :
\coqref{Term.term}{\coqdocinductive{term}}) : \coqdockw{Prop}
:=\coqdoceol
\coqdocindent{1.00em}
\ensuremath{\exists} \coqdocvar{$\rho_1$} :
\coqdocrecord{rule}, \ensuremath{\exists}
\coqdocvar{$\rho_2$} :
\coqdocrecord{rule},
\ensuremath{\exists} \coqdocvar{p} :
\coqdocabbreviation{position},
\ensuremath{\exists} \coqdocvar{$\sigma$},
\ensuremath{\exists} \coqdocvar{$\tau$},\coqdoceol
\coqdocindent{3.00em}
\coqdocvariable{$\rho_1$} \coqdocdefinition{$\in$}
\coqdocvariable{$\mathcal{R}$}
\ensuremath{\land}
\coqdocvariable{$\rho_2$} \coqdocdefinition{$\in$}
\coqdocvar{$\mathcal{R}$} \ensuremath{\land}
(\coqdocvariable{$\rho_1$} = \coqdocvariable{$\rho_2$}
\ensuremath{\rightarrow}
\coqdocvariable{p} \ensuremath{\not=}
\coqexternalref{http://coq.inria.fr/stdlib/Coq.Init.Datatypes}{nil}{\coqdocconstructor{nil}})
\ensuremath{\land}\coqdoceol
\coqdocindent{3.00em}
\coqdockw{match}
\coqref{Term.subterm}{\coqdocdefinition{subterm}} (\coqref{Rewriting.lhs}{\coqdocprojection{lhs}}
\coqdocvariable{$\rho_1$}) \coqdocvariable{p},
\coqref{Context.dig}{\coqdocdefinition{dig}} (\coqref{Rewriting.lhs}{\coqdocprojection{lhs}}
\coqdocvariable{$\rho_1$})$^{\coqdocvariable{$\sigma$}}$ \coqdocvariable{p}
\coqdockw{with}\coqdoceol
\coqdocindent{3.00em}
\ensuremath{|}
\coqexternalref{http://coq.inria.fr/stdlib/Coq.Init.Datatypes}{Some}{\coqdocconstructor{Some}}
\coqdocvar{s},
\coqexternalref{http://coq.inria.fr/stdlib/Coq.Init.Datatypes}{Some}{\coqdocconstructor{Some}}
\coqdocvar{C} \ensuremath{\Rightarrow}
\coqdocdefinition{is\_var} \coqdocvariable{s} =
\coqexternalref{http://coq.inria.fr/stdlib/Coq.Init.Datatypes}{false}{\coqdocconstructor{false}}
\ensuremath{\land}
\coqdocvariable{s}$^{\coqdocvariable{$\sigma$}}$ \coqref{TermEquality.termbis}{$\bis$}
(\coqref{Rewriting.lhs}{\coqdocprojection{lhs}}
\coqdocvariable{$\rho_2$})$^{\coqdocvariable{$\tau$}}$
\ensuremath{\land}\coqdoceol
\coqdocindent{12.00em}
\coqdocvariable{t$_1$} \coqref{TermEquality.termbis}{$\bis$}
\coqdocvariable{C}[(\coqref{Rewriting.rhs}{\coqdocprojection{rhs}}
\coqdocvariable{$\rho_2$})$^{\coqdocvariable{$\tau$}}$]
\ensuremath{\land}
\coqdocvariable{t$_2$} \coqref{TermEquality.termbis}{$\bis$}
(\coqref{Rewriting.rhs}{\coqdocprojection{rhs}}
\coqdocvariable{$\rho_1$})$^{\coqdocvariable{$\sigma$}}$\coqdoceol
\coqdocindent{3.00em}
\ensuremath{|} \coqdocvar{\_}, \coqdocvar{\_}
\ensuremath{\Rightarrow}
\coqexternalref{http://coq.inria.fr/stdlib/Coq.Init.Logic}{False}{\coqdocinductive{False}}\coqdoceol
\coqdocindent{3.00em}
\coqdockw{end}.\coqdoceol
\end{coqdoccode}
\end{singlespace}
Now we can in a straightforward manner define the properties of
orthogonality and weak orthogonality.
\begin{singlespace}
\begin{coqdoccode}
\coqdocnoindent
\coqdockw{Definition}
\coqdef{Rewriting.orthogonal}{orthogonal}{\coqdocdefinition{orthogonal}} (\coqdocvar{$\mathcal{R}$} :
\coqdocdefinition{trs})
: \coqdockw{Prop} :=\coqdoceol
\coqdocindent{1.00em}
\coqdocdefinition{trs\_left\_linear} % TODO: maybe define trs_left_linear and link to it
\coqdocvariable{$\mathcal{R}$} \ensuremath{\land}
\ensuremath{\forall} \coqdocvar{t$_1$} \coqdocvar{t$_2$},
\ensuremath{\lnot}
\coqref{Rewriting.criticalpair}{\coqdocdefinition{critical\_pair}}
\coqdocvariable{t$_1$} \coqdocvariable{t$_2$}.\coqdoceol
\coqdocemptyline
\coqdocnoindent
\coqdockw{Definition}
\coqdef{Rewriting.weaklyorthogonal}{weakly\_orthogonal}{\coqdocdefinition{weakly\_orthogonal}} (\coqdocvar{$\mathcal{R}$} :
\coqdocdefinition{trs})
: \coqdockw{Prop} :=\coqdoceol
\coqdocindent{1.00em}
\coqdocdefinition{trs\_left\_linear}
\coqdocvariable{$\mathcal{R}$} \ensuremath{\land}
\ensuremath{\forall} \coqdocvar{t$_1$} \coqdocvar{t$_2$},
\coqref{Rewriting.criticalpair}{\coqdocdefinition{critical\_pair}}
\coqdocvariable{t$_1$} \coqdocvariable{t$_2$} \ensuremath{\rightarrow}
\coqdocvariable{t$_1$} \coqref{TermEquality.termbis}{$\bis$} \coqdocvariable{t$_2$}.\coqdoceol
\end{coqdoccode}
\end{singlespace}

Next we define when a term is a normal form and when we have unique
normal forms.
\begin{singlespace}
\begin{coqdoccode}
\coqdocnoindent
\coqdockw{Definition}
\coqdef{Rewriting.normalform}{normal\_form}{\coqdocdefinition{normal\_form}}
\coqdocvar{t} : \coqdockw{Prop} :=\coqdoceol
\coqdocindent{1.00em}
\ensuremath{\lnot} \ensuremath{\exists} \coqdocvar{C} : \coqref{Context.context}{\coqdocinductive{context}},
\ensuremath{\exists} \coqdocvar{$\rho$} : \coqdocrecord{rule},
\ensuremath{\exists} \coqdocvar{$\sigma$} :
\coqref{Substitution.substitution}{\coqdocdefinition{substitution}},
\coqdocvariable{$\rho$}
\coqexternalref{http://coq.inria.fr/stdlib/Coq.Lists.List}{In}{\coqdocdefinition{$\in$}}
\coqdocvar{$\mathcal{R}$} \ensuremath{\land}
\coqdocvariable{C}[(\coqref{Rewriting.lhs}{\coqdocprojection{lhs}}
\coqdocvariable{r})\coqdocvariable{$^\sigma$}] \coqref{TermEquality.termbis}{$\bis$}
\coqdocvariable{t}.\coqdoceol
\coqdocemptyline
\coqdocnoindent
\coqdockw{Definition}
\coqdef{Rewriting.uniquenormalforms}{unique\_normal\_forms}{\coqdocdefinition{unique\_normal\_forms}}
: \coqdockw{Prop} :=\coqdoceol
\coqdocindent{1.00em}
\ensuremath{\forall} \coqdocvar{s} \coqdocvar{t} \coqdocvar{u}
(\coqdocvar{$\varphi$} : \coqdocvariable{s} \coqref{Rewriting.sequence}{$\rewrites_\mathcal{R}$} \coqdocvariable{t})
(\coqdocvar{$\psi$} : \coqdocvariable{s} \coqref{Rewriting.sequence}{$\rewrites_\mathcal{R}$} \coqdocvariable{u}),\coqdoceol
\coqdocindent{2.00em}
\coqref{Rewriting.wf}{\coqdocdefinition{wf}} \coqdocvariable{$\varphi$}
\ensuremath{\rightarrow}
\coqref{Rewriting.wf}{\coqdocdefinition{wf}} \coqdocvariable{$\psi$}
\ensuremath{\rightarrow}
\coqref{Rewriting.normalform}{\coqdocdefinition{normal\_form}}
\coqdocvariable{t} \ensuremath{\rightarrow}
\coqref{Rewriting.normalform}{\coqdocdefinition{normal\_form}}
\coqdocvariable{u} \ensuremath{\rightarrow}
\coqdocvariable{t} \coqref{TermEquality.termbis}{$\bis$} \coqdocvariable{u}.\coqdoceol
\end{coqdoccode}
\end{singlespace}
Note that the
\coqref{Rewriting.uniquenormalforms}{\coqdocdefinition{unique\_normal\_forms}}
definition is only a translation of the $UN^\rewrites$ property, not
of the more general $UN^\infty$ property (see also
Definition~\ref{def:normalisation}).

% TODO: note in chapter unwo that we proved ~UN^\infty with ~UN^->
