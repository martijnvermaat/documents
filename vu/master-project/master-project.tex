\documentclass[11pt,oneside,a4paper,final]{report}
\usepackage[english]{babel}


% Use some subversion information
\usepackage{svn-multi}
\svnid{$Id$}


% Use scalable, PostScript Type 1 versions of the Computer Modern fonts.
\usepackage{type1cm}

% How to encode fonts.
\usepackage[T1]{fontenc}

% Text encoding to take for data input stream. I.e. the text we write in.
\usepackage[latin1]{inputenc}

\usepackage{amsfonts}
\usepackage{amsmath}
\usepackage{amssymb}
\usepackage{amsthm}

% Times
%\usepackage{txfonts}
%\usepackage{times}

% Times
%\usepackage{mathptmx}

% Palatino
%\usepackage{pxfonts}
%\usepackage{palatino}

% Palatino
%\usepackage[osf,sc]{mathpazo}

% TeX Gyre Pagella (like Palatino)
%\usepackage{qpxmath}
%\usepackage{tgpagella}

% TeX Gyre Termes (like Times)
\usepackage{qtxmath}
\usepackage{tgtermes}

% txtt for tt
\renewcommand{\ttdefault}{txtt}

% Computer Modern for tt
%\renewcommand{\ttdefault}{cmtt}

% Helvetica for sans serif
%\renewcommand{\sfdefault}{qhv}
\usepackage[scaled=0.95]{helvet}


\usepackage{a4}

\raggedbottom

% Dutch style of paragraph formatting, i.e. no indents.
%\setlength{\parskip}{1.3ex plus 0.2ex minus 0.2ex}
\setlength\parskip{\medskipamount}
%\setlength\parskip{\bigskipamount}
\setlength{\parindent}{0pt}

% Linespacing
\usepackage{setspace}
\onehalfspacing
%\doublespacing


% Natbib for the bibliography, always. sort&compress will make i.e. [1-5,7].
%\usepackage[square,numbers,sort&compress]{natbib}
%\usepackage[authoryear,sort]{natbib}
\usepackage[numbers]{natbib}
\setlength{\bibhang}{0ex}


% Must be last in preamble
\usepackage[
  pdftex,
  colorlinks,
  citecolor=black,
  filecolor=black,
  linkcolor=black,
  urlcolor=black,
  pdfauthor={Martijn Vermaat},
  pdftitle={Infinitary Rewriting in Coq},
  pdfsubject={A Mechanically Verifiable Formalisation of Infinitary Rewriting in the Coq Proof Assistant},
  pdfkeywords={rewriting, infinitary rewriting, verification, coq, functional programming},
  draft=true
]{hyperref}
\usepackage[figure]{hypcap}


\title{Infinitary Rewriting in Coq}

\author{Martijn Vermaat}
\date{MSc. Thesis (\emph{draft r\svnrev})}


\begin{document}


\maketitle


\begin{abstract}
  Abstract describing the thesis in a few sentences.
\end{abstract}


\chapter*{Preamble}


\section*{Acknowledgements}


\tableofcontents


\chapter{Introduction}

I have a lot of dummy text here.
That way we can see how the formatting stuff works with these settings.
And you know what, this is just dummy text!

I have a lot of dummy text here.
That way we can see how the formatting stuff works with these settings.
And you know what, \textbf{this is just dummy text}!
That way we can see how the formatting stuff works with these settings.
And you know what, this is just dummy text!

I have a lot of dummy text here.
That way we can see how the \emph{formatting stuff} works with these settings.
And you know what, this is just dummy text!
And you know what, this is just dummy text!


\section*{Outline}

Some Greek letters in math: $\alpha$ $\beta$ $\rho$ $\psi$ $\phi$ $\sigma$ $\lambda$.

This transformation is conceptually easy, but unfortunately yields a lot of
administrative redexes. An administrative redex is a redex introduced by the CPS
transformation, thus not present in the original source term. Consider for example
the transformation of $(\lambda x. \, x) \, y$ and subsequent $\beta$-reduction
of several administrative redexes:
\begin{align*}
  [\![(\lambda x. \, x) \, y]\!] &= \lambda k. \, (\lambda k. \, k \, (\lambda x. \, (\lambda k. \, k \, x))) \, (\lambda m. \, (\lambda k. \, k \, y) \,
 (\lambda n. \, m \, n \, k))\\
                                 &\rightarrow_{\beta} \lambda k. \, (\lambda m. \, (\lambda k. \, k \, y) \, (\lambda n. \, m \, n \, k)) \, (\lambda x.
\, (\lambda k. \, k \, x))\\
                                 &\rightarrow_{\beta} \lambda k. \, (\lambda k. \, k \, y) \, (\lambda n. \, (\lambda x. \, (\lambda k. \, k \, x)) \, n
\, k)\\
                                 &\rightarrow_{\beta} \lambda k. \, (\lambda n. \, (\lambda x. \, (\lambda k. \, k \, x)) \, n \, k) \, y\\
                                 &\rightarrow_{\beta} \lambda k. \, (\lambda x. \, (\lambda k. \, k \, x)) \, y \, k
\end{align*}
Note that the term concluding the above reduction still contains a redex, but
this redex was already present in the source term.
Administrative redexes are to be avoided, because they complicate formal
reasoning about a transformation and add computation steps to a compilation
process.


\chapter{Infinitary Term Rewriting}


\section{Term Rewriting}


\section{Ordinal Numbers}


\section{Infinitary Term Rewriting}


\chapter{Mechanically Verifiable Formalizations}


\section{The Coq Proof Assistent}


\section{Ordinal Numbers}


\section{Coinductive Terms}


\section{Transfinite Rewrite Sequences}


\chapter{Weak Orthogonality and Unique Normal Forms}


\section{Jan Willem's Counterexample}


\section{Description of UNWO.v}


\pagebreak


\nocite{*}
\bibliographystyle{plain}
\bibliography{master-project}


\end{document}
