\chapter{Infinitary Term Rewriting}\label{chap:rewriting}

TODO: write introduction

This chapter does not contain original material.

Before we can study infinitary rewriting, we must introduce ordinal numbers.

Orthogonal TRSs have some nice properties, for example UN$^{\infty}$ and
compression.

% TODO: note that we are explicit in the wording: sequence vs rewrite sequence


\section{Ordinal Numbers}

% ordinals are originally by Cantor

Ordinal numbers (ordinals for short) are an extension of the natural numbers
with transfinite objects. Indeed, the finite ordinals are just the natural
numbers. The smallest infinite ordinal is called $\omega$ and following
$\omega$ we have $\omega + 1$, $\omega + 2$, \ldots, $\omega \times 2$. Then
there are the ordinals $\omega \times 2 + 1$, $\omega \times 2 + 2$, \ldots,
$\omega \times 3$. Some other (still relatively small) ordinals are:
\begin{displaymath}
  \omega^2 \qquad
  \omega^\omega \qquad
  \omega^{\omega^2} \qquad
  \omega^{\omega^\omega} \qquad
  \omega^{\omega^{\omega^{\iddots}}} = \epsilon_0
\end{displaymath}
Note that this is all merely notation, we have not yet defined a
representation for ordinals or what $+$ and $\times$ are.
% TODO: and exponentiation

In set theory, ordinals are usually represented by hereditarily transitive
sets. Zero corresponds to the empty set $\nothing$, one to the
singleton $\{ \nothing \}$ and so on, and $\omega$ is represented by
$\{ \nothing, \{ \nothing \}, \{ \nothing, \{ \nothing \} \} , \ldots
\}$. Now $\in$ constitutes a well-founded total order on the
ordinals.

We abbreviate $\alpha \cup \{ \alpha \}$ by $\alpha^+$ and say that
$\alpha$ is a \emph{successor ordinal} if $\alpha = \beta^+$ for some
ordinal $\beta$. If $\alpha$ is not a successor ordinal and $\alpha
\neq \nothing$, it is called a \emph{limit ordinal}. Hence, an ordinal
can be either zero, a successor ordinal, or a limit ordinal.

From now on, we make no distinction between an ordinal and its set-theoretic
representation (e.g.\ between $0$ and $\nothing$). Examples of successor
ordinals are $4$, $\omega + 7$  and $\omega^{\omega \times 2} + 1$. Examples
of limit ordinals are $\omega$ and $\omega \times 3$.

% TODO: we also make no distinction between the ordinal and its brouwer
% ordinal
% TODO: explain we use \lambda for limit ordinals

One can do arithmetics on ordinals much like we do arithmetics on natural
numbers. For example, addition can be defined by recursion on the right
argument:
\begin{align*}
  \alpha + 0       &= \alpha\\
  \alpha + \beta^+ &= (\alpha + \beta)^+\\
  \alpha + \lambda &= \bigcup \{ \alpha + \gamma \; | \; \gamma \in \lambda \}
\end{align*}


\subsection{Brouwer Ordinals}\label{sub:brouwer}

The Brouwer ordinals are a representation of the countable ordinals as
countably branching well-founded trees. Their inductive definition
uses constructors $0$ (zero), $^+$ (successor) and $\sqcup$ (limit).

The $\sqcup$ constructor has type $(\mathbb{N} \rightarrow \Ord) \rightarrow
\Ord$, but for our convenience we write $\sqcup_i \cdots i \cdots$ instead
of $\sqcup (\lambda i . \cdots i \cdots)$. Sometimes we explicitly enumerate
the argument, writing for example $\sqcup \{ \alpha_1, \alpha_2,
\alpha_3, \ldots \}$.

% Be alerted that all ordinals cannot form a set (only a class), but we are
% defining a subset here
\begin{definition}\label{def:ordinals}%[Ordinals]
The set of \emph{Brouwer ordinals} (ordinals) $\Ord$ is defined by
induction:
\begin{compactenum}
  \item
    $0 \in \Ord$.
  \item
    If $\alpha \in \Ord$, then $\alpha^+ \in \Ord$.
  \item
    If $\alpha_i \in \Ord$ for all $i \in \mathbb{N}$, then $\sqcup_i
    \alpha_i \in \Ord$.
\end{compactenum}
\end{definition}

Now zero is represented by $0$, a successor ordinal $\alpha +1$ is represented
by $\alpha^+$ and a limit ordinal $\lambda$ is represented by $\sqcup_i
\alpha_i$ if $\lambda$ is the least upper bound of the sequence $\alpha_1,
\alpha_2, \alpha_3, \ldots$. % TODO: strict or non-strict upper bound
Again, we identify ordinals and their representation as Brouwer ordinal.

% TODO: some ordinals have no representation as Brouwer ordinal at all
Some ordinals have no unique representation as Brouwer ordinal. Consider for
example the limit ordinals $\sqcup_i i + 3$ and $\sqcup_i i \times 2$. Both
are representations of $\omega$ and a meaningful order relation would
have to position them at the same rank.

A more intricate issue is what to make of ordinals such as $\sqcup \{ 3, 3, 3,
\ldots \}$. In spirit of the intuition given above it represents $3$ ($4$),
that being the non-strict (strict) upper bound of $3, 3, 3, \ldots$.
% TODO: this makes it undecidable to compare an ordinal to 0 or 3
We might like to exclude such representations and require that
$\sqcup_i \alpha_i$ always represents a limit ordinal. This can be
done by imposing a strict monotonicity property on the limit
sequences. Some order relation on the Brouwer ordinals is needed for
that.

% TODO: much of the following is taken from Hancock, but i think we cannot
% cite that paper. however, we must acknowledge this in some way

Before we can define an extensional order relation on $\Ord$, we define a
structural strict order relation as follows.
% TODO: , is too close on $\Omega$

\begin{definition}%[Predecessor indices]
The set-valued function $\Phi$ defines the \emph{predecessor indices} $\Phi(\alpha)$
\emph{for} $\alpha$ by recursion on $\alpha$:
\begin{align*}
  \Phi(0)                 &= \nothing \\
  \Phi(\alpha^+)          &= \Phi(\alpha)^? \\
  \Phi(\sqcup_i \alpha_i) &= (\Sigma n \in \mathbb{N}) \; \Phi(\alpha_n)
\end{align*}
\end{definition}

By $A^?$ we mean the option type over $A$, or equivalently the disjoint sum
$1 + A$ of the unit type $1$ and $A$. We use \textsc{left} and
\textsc{right $a$} (for $a \in A$) as constructors of $A^?$. Note that
the set $\Phi(0)$ of predecessor indices for $0$ has no inhabitants.
% TODO: just use disjoint sum, no option type?

\begin{definition}%[Predecessor]
The function $\_[\_] : (\prod \alpha : \Ord) \; \Phi(\alpha)
\rightarrow \Ord$ defines the \emph{predecessor} $\alpha[\iota]$
\emph{of} $\alpha$ \emph{indexed by} $\iota$ recursively on $\alpha$:
\begin{align*}
  \alpha^+[\textsc{left}]                     &= \alpha \\
  \alpha^+[\textsc{right $\iota$}]            &= \alpha[\iota] \\
  \sqcup_i \alpha_i[\langle n, \iota \rangle] &= \alpha_n[\iota]
\end{align*}
\end{definition}

% TODO: explain what I and _[_] mean

This structural predecessor function can be viewed as defining a `subtree'
partial order on $\Ord$. With it we are ready to define an extensional
non-strict order relation on $\Ord$ that classifies ordinals by rank.

% TODO: note <= infix notation
% TODO: or use the set-theoretic definitions from hancock?
% TODO: infix notations are used all the time, maybe state this at front
% TODO: does \sqcup_i start at 0 or 1? we have pred_type saying all of
% the natural numbers, but this definition starting at \alpha_1...
\begin{definition}\label{def:order}%[Order]
We define the \emph{order} $\preceq$ as a binary relation on $\Ord$ by
induction:
% TODO: maybe make a not of infix notation in introduction
% (and write $\alpha \preceq \beta$ for $\langle \alpha, \beta \rangle
% \in \; \preceq$)
\begin{compactenum}
  \item
    $0 \preceq \beta$ for every ordinal $\beta \in \Ord$.
  \item\label{def:order:succ}
    For all $\alpha, \beta \in \Ord$ and $\iota \in \Phi(\beta)$, if
    $\alpha \preceq \beta[\iota]$ then $\alpha^+ \preceq \beta$.
  \item
    For all $\alpha_1, \alpha_2, \alpha_3, \ldots, \beta \in \Ord$, if
    $\alpha_n \preceq \beta$ for all $n \in \mathbb{N}$, then $\sqcup_i
    \alpha_i \preceq \beta$.
\end{compactenum}
\end{definition}

Using this order, we can introduce two other useful binary relations
on $\Ord$. First, the extensional \emph{equality} $\alpha \simeq
\beta$ is defined by the conjunction of $\alpha \preceq \beta$ and
$\beta \preceq \alpha$. Second, the extensional \emph{strict order}
$\alpha \prec \beta$ holds if $\alpha \preceq \beta[\iota]$ holds for
some $\iota \in \Phi(\beta)$.

TODO: prove that $\prec$ is just $<$ on the finite ordinals
% TODO: maybe define well-formed ordinals here


\section{Term Rewriting}\label{sec:rewriting}

% TODO: Short motivation for term rewriting, summation of its applications and
% aspects of rewriting that are studied.
% TODO: we only give definitions or notions we actually use

We give a short introduction to the basic notions of infinitary term
rewriting. For a more in-depth treatment of the theory of term rewriting,
consult \citet{terese-03}. Discussion of infinitary rewriting
specifically, can be found in \citet[Chap. 12]{terese-03} and
\citet{klop-de-vrijer-05}. In this section, we use definitions and
notations from \citet{terese-03}.

% TODO: rewrite this statement (and perhaps we never mention finitary
% rewriting)
From now on, we drop the `infinitary' from infinitary rewriting and
explicitly say so if we mean finitary rewriting.


\subsection{Definition of a TRS}\label{sub:trs}

\begin{definition}%[Signature]
A \emph{signature} $\Sigma$ is a non-empty set of \emph{function symbols} $f,
g, \ldots$. Each function symbol $f$ has a fixed natural number
$\arity{f}$, which we call its \emph{arity}. A function symbol with
arity $0$ is also called a \emph{constant}.
\end{definition}

\begin{definition}%[Term]
The set of \emph{terms} $\TerI(X)$ over a signature $\Sigma$ and a
set of variables $\X = \{x, y, \ldots\}$ is defined by coinduction:
\begin{compactenum}
  \item
    $x \in \TerI(\X)$ for every variable $x \in \X$.
  \item
    For every $f \in \Sigma$, if $t_1, \ldots, t_{\arity{f}} \in
    \TerI(\X)$, then $f(t_1, \ldots, t_{\arity{f}}) \in \TerI(\X)$.
%  \item
%    If $f$ is a function symbol with arity $n$ and $t_1, \ldots, t_n \in
%    \TerI(\X)$, then $f(t_1, \ldots, t_n) \in \TerI(\X)$.
\end{compactenum}
\end{definition}
% TODO: maybe use Ter(\Sigma, X) instead of Ter_\Sigma(X)

The symbol $f$ is called the \emph{root} of $f(t_1, \ldots, t_n)$ and
the terms $t_i$ are called the \emph{arguments} of $f$. By $\Var(t)$
we denote the set of variables occurring in $t$, and $t$ is
\emph{closed} if $\Var(t) = \nothing$. If no variable occurs more than
once in $t$, we say $t$ is \emph{linear}. Often, the set of variables
$\X$ is left implicit and $\TerI(\X)$ is denoted simply by $\TerI$. By
the set of \emph{finite terms} $\Ter$ we mean the subset of
well-founded terms of $\TerI$.

Preparing for the mechanised setting of Section~\ref{chap:implementation} with
its constrains of finite memory and computing time, we want to be precise
about the notions of equality on infinite objects we employ. We consider terms
to be equal if they are
\begin{inparaenum}[(i)]
  \item bisimilar or
  \item pointwise equivalent up to every depth.
\end{inparaenum}
According to Proposition~\ref{prop:equalities} it does not matter
which equality we use.

\begin{definition}\label{def:bisimilarity}%[Bisimilarity]
We define the \emph{bisimilarity relation} $\bis$ on $\TerI$ by
coinduction:
\begin{compactenum}
  \item
    $x \bis x$ for every variable $x \in \X$.
  \item
    For every $f \in \Sigma$, if $s_i \bis t_i$ for all $1 \leq i \leq
    \arity{f}$, then $f(s_1, \ldots s_{\arity{f}}) \bis f(t_1, \ldots,
    t_{\arity{f}})$.
\end{compactenum}
%$\bis$ is the greatest bisimulation on $\TerI$ and
We say that $s$ and $t$ are \emph{bisimilar} if $s \bis t$.
\end{definition}

\begin{definition}\label{def:equiv}%[Pointwise equivalence]
\emph{(Pointwise) equivalence} of terms $s$ and $t$ \emph{up to depth} $d$,
written $s \equpto{d} t$, is defined by induction:
\begin{compactenum}
  \item $s \equpto{0} t$ for every $s, t \in \TerI$.
  \item $x \equpto{d} x$ for every $d \in \mathbb{N}$ and $x \in \X$.
  \item For every $f \in \Sigma$, if $s_i \equpto{d} t_i$ for all $1 \leq i
    \leq \arity{f}$, then $f(s_1, \ldots s_{\arity{f}}) \equpto{d+1}
    f(t_1, \ldots, t_{\arity{f}})$.
\end{compactenum}
The \emph{(pointwise) equivalence} $s \equiv t$ holds if $s \equpto{d}
t$ for every depth $d$.
\end{definition}

\begin{proposition}\label{prop:equalities}
$s \bis t \; \Leftrightarrow \; s \equiv t$.
\end{proposition}
\begin{proof}
By induction on the depth of pointwise equivalence ($\Rightarrow$) and
by coinduction on $s$ ($\Leftarrow$).
%\footnote{This proposition is proved in our \Coq development.}
\end{proof}

\begin{definition}%[Rewrite rule]
  A \emph{rewrite rule} $\rho$ on a signature $\Sigma$ is a pair
  $\langle l, r \rangle$ of finite terms in $\Ter$ (written $\rho : l
  \rightarrow r$). We restrict ourselves to rewrite rules where $l$ is
  not a variable and $\Var(r) \subseteq \Var(l)$.
\end{definition}

The two restrictions on rewrite rules are standard and prevent our
theory from misbehaving in some particular ways. We say a rewrite rule
is \emph{left-linear} if its left-hand side is linear.

\begin{definition}%[TRS]
A \emph{term rewriting system} (TRS) $\mathcal{R}$ is a pair $\langle \Sigma,
R \rangle$ of a signature $\Sigma$ and a finite set of rewrite rules
$R$ on $\Sigma$.
\end{definition}


\subsection{Rewriting}

Positions are sequences of natural numbers. The empty sequence is
denoted by $\epsilon$ and $\prefix{i}{p}$ is the prefixing of a
sequence $p$ with the number $i$.

\begin{definition}%[Subterm positions]
  The set of \emph{positions} $\Pos(t)$ \emph{of a term} $t$ is
  inductively defined:
  \begin{compactenum}
    \item $\epsilon \in \Pos(t)$ for every $t \in \TerI$.
    \item For every $f \in \Sigma$ and $1 \le i \le \arity{f}$, if $p
      \in \Pos(t_i)$ then $\prefix{i}{p} \in \Pos(f(t_1, \ldots,
      t_{\arity{f}}))$.
  \end{compactenum}
  The \emph{subterm} of term $t$ at position $p$, written
  $\subterm{t}{p}$, is inductively defined by
  \begin{inparaenum}[(i)]
    \item $\subterm{t}{\epsilon} = t$ and
    \item $\subterm{f(t_1, \ldots, t_n)}{\prefix{i}{p}} =
      \subterm{t_i}{p}$.
  \end{inparaenum}
  Similarly, \emph{updating} a term $t$ at position $p$ with term $s$,
  written $t[s]_p$, is defined by replacing the subterm
  $\subterm{t}{p}$ at position $p$ in $t$ with $s$.
\end{definition}

In contrast to \cite{terese-03}, we do not define contexts as terms over an
extended signature. Instead, a direct inductive definition is given since this
is how we defined the notion of context in our \Coq development (the main
reason being that we choose not to consider multi-hole contexts).
% TODO: maybe this needs more explaining

\begin{definition}%[Context]
The set of (one-hole) \emph{contexts} $\Ctx$ over a signature
$\Sigma$ is defined by induction:
\begin{compactenum}
  \item
    $\Box \in \Ctx$.
%  \item
%    For every $f \in \Sigma$ and $1 \le n \le \arity{f}$, if $t_1,
%    \ldots, t_{\arity{f} - 1} \in \TerI$ and $C \in \Ctx$, then
%    $f(t_1, \ldots, t_{n - 1}, C, t_{n + 1}, \ldots, t_{\arity{f} -
%      1}) \in \Ctx$.
  \item
    For every $f \in \Sigma$ and $1 \le i \le \arity{f}$, if $t_1,
    \ldots, t_{i - 1}, t_{i + 1}, \ldots, t_{\arity{f}} \in \TerI$ and
    $C \in \Ctx$, then $f(t_1, \ldots, t_{i - 1}, C, t_{i + 1},
    \ldots, t_{\arity{f}}) \in \Ctx$.
\end{compactenum}
\end{definition}

Thus every context $C$ has exactly one occurrence of the symbol $\Box$, called
its \emph{hole}. By the term $C[t]$ we mean the result of replacing the hole
of $C$ by $t$. We allow a slight abuse of notation by writing
$t[\Box]_p$ for the context $C$ with $C[\subterm{t}{p}] \equiv t$. We
also assume obvious extensions to contexts for notions on terms
(e.g.\ $\Var(C)$ and $\Pos(C)$ for $C \in \Ctx$).
The \emph{depth} of a context $C$ is defined by the length of the
(unique) position $p$ with $\subterm{C}{p} = \Box$.

%The \emph{hole depth} of a context $C$ is defined by the number
%of `$($' symbols minus the number of `$)$' symbols preceding the $\Box$
%symbol in $C$.
%TODO: this seemed to me the shortest way to define the hole depth?

\begin{definition}%[Substitution]
% TODO: now we only generalise to finite terms
Given a signature $\Sigma$ and a set of variables $\X$, a
\emph{substitution} $\sigma$ is a mapping from $\X$ to $\TerI(\X)$. It
can be generalised to a mapping $\bar{\sigma} : \TerI(\X) \rightarrow
\TerI(\X)$ corecursively:
\begin{align*}
  \bar{\sigma}(x) &= \sigma(x)\\
  \bar{\sigma}(f(t_1, \ldots, t_n)) &= f(\bar{\sigma}(t_1), \ldots,
  \bar{\sigma}(t_n))
\end{align*}
\end{definition}

Since $\bar{\sigma}$ is completely defined by $\sigma$ we refer to both as
`the' substitution $\sigma$.
%The notation $[x_1, \ldots, x_n := s_1, \ldots, s_n]$ is used for the
%substitution $\sigma$ with $\sigma(x_i) = s_i$ for $1 \leq i \leq n$
%and $\sigma(y) = y$ for all other $y$.
Applying a substitution $\sigma$ to a term $t$ is usually written
$t^\sigma$ and the result is called an \emph{instance} of $t$.
% TODO: at this moment, we don't use the [x := y] notation (get rid of it?)

If we view a rewriting rule $\rho : l \rightarrow r$ as a \emph{scheme}, an
\emph{instance} of $\rho$ can be obtained by applying a substitution
$\sigma$. The result is the \emph{atomic} rewrite step $l^\sigma
\rightarrow_\rho r^\sigma$. We call $l^\sigma$ a ($\rho$-) \emph{redex} and
$r^\sigma$ its \emph{contractum}. An atomic rewrite step can be placed in a
context, forming a rewrite step.

\begin{definition}%[Rewrite step]
A \emph{rewrite step} $C[l^\sigma] \rightarrow_\rho C[r^\sigma]$ according to
the rewrite rule $\rho$ consists of rewriting the redex obtained from
$\rho$ and substitution $\sigma$ to its contractum in a context $C$.
\end{definition}

The \emph{depth} of a rewrite step is the depth of its context. We
call $\rightarrow_\rho$ the \emph{one-step rewriting relation}
generated by $\rho$. The one-step rewriting relation $\rightarrow$ of
a TRS $\mathcal{R}$ with rewrite rules $R$ is defined as the union of
$\{ \rightarrow_\rho | \; \rho \in R \}$.

\begin{definition}%[Rewrite sequence]
A \emph{rewrite sequence} of ordinal length $\alpha$ is a sequence of rewrite
steps $(t_\beta \rightarrow t_{\beta^+})_{\beta \prec \alpha}$.
\end{definition}

This definition only makes sense if we somehow require that for every limit
ordinal $\lambda \prec \alpha$, the terms $(t_\beta)_{\beta \prec
  \lambda}$ approach $t_\lambda$ in the limit and sometimes even
further restrictions are desirable. We define the notion of Cauchy
convergence and, using that, four conditions on rewrite sequences.

\begin{definition}\label{def:cauchy}%[Cauchy convergence]
  Let $\lambda$ be a limit ordinal. A sequence of terms
  $(t_\beta)_{\beta \prec \lambda}$ \emph{(Cauchy-) converges to the
    term} $t$ if for every depth $d$ there exists $\alpha \prec
  \lambda$ such that for all $\alpha \preceq \beta \prec \lambda$ we
  have $t_\beta \equpto{d} t$.
\end{definition}

\begin{definition}%[Continuity and convergence]
A rewrite sequence $(t_\beta \rightarrow t_{\beta^+})_{\beta \prec
  \alpha}$ of length $\alpha$ is
\begin{compactenum}
  \item
    \emph{weakly continuous} if for every limit ordinal $\lambda \prec
    \alpha$, the sequence $(t_\beta)_{\beta \prec \lambda}$ converges
    to the term $t_\lambda$,
  \item
    \emph{strongly continuous} if it is weakly continuous and for every limit
    ordinal $\lambda \prec \alpha$, the depth of the rewrite steps $(t_\beta
    \rightarrow t_{\beta^+})_{\beta \prec \lambda}$ tends to infinity,
  \item
    \emph{weakly convergent} if for every limit ordinal $\lambda
    \preceq \alpha$, the sequence $(t_\beta)_{\beta \prec \lambda}$
    converges to the term $t_\lambda$ and
  \item
    \emph{strongly convergent} if it is weakly convergent and for
    every limit ordinal $\lambda \preceq \alpha$, the depth of the
    rewrite steps $(t_\beta \rightarrow t_{\beta^+})_{\beta \prec
      \lambda}$ tends to infinity.
\end{compactenum}
\end{definition}

We write $t_0 \rewrites t_\alpha$ if there exists a strongly
convergent rewrite sequence $(t_\beta \rightarrow t_{\beta^+})_{\beta
  \prec \alpha}$. The \emph{convertibility relation} on terms is
defined as the reflexive transitive symmetric closure of $\rewrites$.


\subsection{Normal Forms and Orthogonality}

\begin{definition}\label{def:normalisation}%[Normalisation]
  Let $\mathcal{R}$ be a TRS.
  \begin{compactenum}
    \item
      A term $t$ is a \emph{normal form} if there do not exist rewrite
      rule $\rho$ in $\mathcal{R}$, substitution $\sigma$ and context
      $C$ such that $t \equiv C[l^\sigma]$ (i.e.\ if there is no step
      from $t$). We say $t$ is a normal form \emph{of} $s$ if $s
      \rewrites t$ and $t$ is a normal form.
    \item
      $\mathcal{R}$ has the (infinitary) \emph{unique normal forms}
      (UN$^\infty$) property if $t \equiv u$ for every two
      convertible normal forms $t$ and $u$.
    \item
      % TODO: make sure \rewrites renders fine in superscript
      $\mathcal{R}$ has the (infinitary) \emph{unique normal forms
      with respect to rewriting} (UN$^\rewrites$) property if for all
      terms $s$, we have $t \equiv u$ for every two normal forms $t$
      and $u$ of $s$.
  \end{compactenum}
\end{definition}
% TODO: klop-de-vrijer-05 uses our UN->> definition for UN^\infty,
% should we distinguish between the two?
% TODO: pictures

Obviously we have that UN$^\infty$ implies UN$^\rewrites$. One source
of non-unique normal forms is the interference of two redex occurrences
in a term. Contracting one of them may result in a term where (a
descendant of) the other redex is no longer present, possibly losing
confluence. This phenomenon is made precise in the following
definition.
% TODO: informal definition of descendants and confluence

% TODO: maybe cite joerg's thesis
\begin{definition}\label{def:overlap}%[Overlap and critical pairs]
We say two rewrite rules $\rho_1 : l_1 \rightarrow r_1$ and $\rho_2 :
l_2 \rightarrow r_2$ have \emph{overlap} if there exists a
non-variable position $p$ such that $l_1 |_p$ and $l_2$ have a common
instance. (We exclude the trivial case of overlap between a rewrite
rule and itself at the root position.)
Let $\sigma, \tau$ be substitutions such that $l_1 |_p ^{\,\;\sigma}
\equiv l_2 ^{\,\;\tau}$ is a most general common instance of $l_1 |_p$
and $l_2$, and without loss of generality assume that $\dom(\sigma) =
\Var(l_1 |_p)$ and $\Var(l_1 |_p ^{\,\;\sigma}) \cap \Var(l_1[\Box]_p)
= \nothing$. Then $\langle l_1 ^{\,\;\sigma}[r_2^{\,\;\tau}]_p,
r_1^{\,\;\sigma} \rangle$ is called a \emph{critical pair of} $\rho_1$
\emph{with} $\rho_2$. A critical pair $\langle s, t \rangle$ is called
\emph{trivial} if $s \equiv t$.
\end{definition}

Critical pairs are unique up to renaming. Using these notions we can
define some useful classes of term rewriting systems.

%We say two rewrite rules $\rho_1 : l_1 \rightarrow r_2$ and $\rho_2 : l_2
%\rightarrow r_2$ overlap if there is a term $t$ with overlapping occurrences
%of the pattern of $l_1$ and the pattern of $l_2$.

%Two redex occurrences in a term $t$ overlap if their patterns share at least
%one symbol occurrence. Here we do not count the trivial overlap between a
%redex s and itself, unless s is a redex with respect to two different
%reduction rules.
%We say two rewrite rules $\rho_1, \rho_2$ overlap if there is a term $t$ with
%overlapping occurrences of a $\rho_1$- and $\rho_2$-redex.

\begin{definition}%[Orthogonality]
A TRS is called
\begin{compactenum}
  \item \emph{left-linear} if all its rewrite rules are left-linear,
  \item \emph{orthogonal} if it is left-linear and there are no critical pairs and
  \item \emph{weakly orthogonal} if it is left-linear and all critical pairs
    are trivial.
\end{compactenum}
\end{definition}

Orthogonal systems enjoy the UN$^\infty$ property
\citep{kennaway-95,klop-de-vrijer-05}. In Chapter~\ref{chap:unwo} we
discuss a counterexample to UN$^\infty$ for weakly orthogonal TRSs
\citep{endrullis-10}.
