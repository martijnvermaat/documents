\chapter{The \texorpdfstring{\Coq}{Coq} Proof Assistant}\label{chap:coq}

\Coq \citep{coq-refman-09} is based on the formal
language \emph{Calculus of Inductive Constructions}
\citep{coquand-huet-88,mohring-93}, which is essentially a typed
$\lambda$-calculus with inductive types. In this language, logical
propositions are represented as types and proofs of such propositions
are $\lambda$-terms, motivated by the Curry--Howard--de Bruijn
correspondence \citep{girard-89}. The core of the \Coq system is its type
checking algorithm.

We present a very short introduction to \Coq and refer to
\citet{bertot-casteran-04} and \citet{chlipala-09} for further
reading. Sections~\ref{sec:guardedness} and \ref{sec:positivity} discuss two
technicallities related to the \Coq development described in
Chapter~\ref{chap:implementation}.
%The seasoned \Coq user can skip to Section~\ref{sec:ordimp}.


\section{Types and Propositions}

Every term in \Coq has a type and every type is also a
term. The type of a type is called a \emph{sort} and the sorts
in \Coq are
\begin{compactitem}
\item \coqdockw{Prop}, the sort of logical propositions,
\item \coqdockw{Set}, the sort of program specifications and datatypes,
\item \coqdockw{Type}$_0$, the sort of \coqdockw{Prop} and
  \coqdockw{Set}, and
\item \coqdockw{Type}$_{i+1}$, the sort of
  \coqdockw{Type}$_i$.\footnote{The subscripts $i$ of the sorts
    \coqdockw{Type}$_i$ are invisible to the user and only used
    internally.}
\end{compactitem}
For example,
\coqexternalref{http://coq.inria.fr/stdlib/Coq.Init.Datatypes}{nat}{\coqdocinductive{nat}}
is the datatype of the natural numbers. It lives in \coqdockw{Set} and
is defined inductively.
\begin{singlespace}
\begin{coqdoccode}
\coqdocnoindent
\coqdockw{Inductive} \coqexternalref{http://coq.inria.fr/stdlib/Coq.Init.Datatypes}{nat}{\coqdocinductive{nat}}
:
\coqdockw{Set} :=\coqdoceol
\coqdocindent{1.0em}
\ensuremath{|} \coqexternalref{http://coq.inria.fr/stdlib/Coq.Init.Datatypes}{O}{\coqdocconstructor{O}}
:
\coqexternalref{http://coq.inria.fr/stdlib/Coq.Init.Datatypes}{nat}{\coqdocinductive{nat}}\coqdoceol
\coqdocindent{1.0em}
\ensuremath{|} \coqexternalref{http://coq.inria.fr/stdlib/Coq.Init.Datatypes}{S}{\coqdocconstructor{S}}
:
\coqexternalref{http://coq.inria.fr/stdlib/Coq.Init.Datatypes}{nat}{\coqdocinductive{nat}} $\rightarrow$
\coqexternalref{http://coq.inria.fr/stdlib/Coq.Init.Datatypes}{nat}{\coqdocinductive{nat}}.\coqdoceol
\end{coqdoccode}
\end{singlespace}
The logical proposition that for every natural number $n$, there
exists a natural number $m$ larger than $n$, can be stated as a term
of sort \coqdockw{Prop}.
\begin{singlespace}
\begin{coqdoccode}
\coqdocnoindent
(\ensuremath{\forall} \coqdocvar{n} :
\coqexternalref{http://coq.inria.fr/stdlib/Coq.Init.Datatypes}{nat}{\coqdocinductive{nat}},
\ensuremath{\exists} \coqdocvar{m} :
\coqexternalref{http://coq.inria.fr/stdlib/Coq.Init.Datatypes}{nat}{\coqdocinductive{nat}},
\coqdocvariable{n} < \coqdocvariable{m}) : \coqdockw{Prop}\coqdoceol
\end{coqdoccode}
\end{singlespace}
Using the vocabulary of types and terms, the universal quantifier
$\forall$ is called the \emph{product type constructor}. A product
type \begin{coqdoccode}$\forall$ \coqdocvar{x} : $T$,
  $U$\end{coqdoccode} is called \emph{dependent} if \coqdocvar{x} occurs free
in $U$, otherwise it is written $T \rightarrow U$. The type of the
constructor symbol \coqdocconstructor{S} defined above, for example,
is that of functions from
\coqexternalref{http://coq.inria.fr/stdlib/Coq.Init.Datatypes}{nat}{\coqdocinductive{nat}}
to
\coqexternalref{http://coq.inria.fr/stdlib/Coq.Init.Datatypes}{nat}{\coqdocinductive{nat}}
and is not dependent. The non-dependent function space notation $T \rightarrow
U$ is also used for logical implication, justified by the Curry--Howard--de
Bruijn correspondence.


\section{Terms and Proofs}

Proofs of logical propositions can be defined in two ways. First, we
can write a proof term directly. The only requirement is that this
term has as type the logical proposition that we want to
prove (again, justified by the Curry--Howard--de Bruijn
correspondence). Second, we can use \emph{tactics} to construct a proof term
interactively, in a way mimicking natural deduction.

As an example of the use of tactics, we prove the proposition from the
previous section. This is done by stating the proposition, after which
the system enters a goal-directed proof mode. In this mode, we are
presented with a goal, consisting of
\begin{inparaenum}[(i)]
\item a context of local variables that are available
\item a proposition denoting what remains to be proven.
\end{inparaenum}
Tactics can now be applied to progressively transform the goal into a
simpler goal. When the goal is simple enough to be solved directly by
applying a tactic, we are done proving the proposition.
\begin{singlespace}
\begin{coqdoccode}
\coqdocnoindent
\coqdockw{Lemma}
\coqdef{CoqIntro.ltserial}{lt\_serial}{\coqdoclemma{lt\_serial}} :
\ensuremath{\forall} \coqdocvar{n} :
\coqexternalref{http://coq.inria.fr/stdlib/Coq.Init.Datatypes}{nat}{\coqdocinductive{nat}},
\ensuremath{\exists} \coqdocvar{m} :
\coqexternalref{http://coq.inria.fr/stdlib/Coq.Init.Datatypes}{nat}{\coqdocinductive{nat}},
\coqdocvariable{n} < \coqdocvariable{m}.\coqdoceol
\coqdocnoindent
\coqdockw{Proof}.\coqdoceol
\coqdocindent{1.0em}
\coqdoctac{intro} \coqdocvar{n}. $\,$
\coqdoctac{exists}
(\coqexternalref{http://coq.inria.fr/stdlib/Coq.Init.Datatypes}{S}{\coqdocconstructor{S}}
\coqdocvar{n}). $\,$
\coqdoctac{auto}.\coqdoceol
\coqdocnoindent
\coqdockw{Qed}.\coqdoceol
\end{coqdoccode}
\end{singlespace}
In this example, we use the tactic \coqdoctac{intro} to introduce the
variable \coqdocvariable{n} to the context. With \coqdoctac{exists},
we supply a witness for the existential quantification. At this point,
the goal is simple enough to be solved directly by the
\coqdoctac{auto} tactic.

Recursive functions are defined using the \coqdockw{Fixpoint}
keyword. The function must have an argument of an inductive type that is
structurally decreasing with each recursive call. Consider for example the
definition of
the factorial function, which also shows how a case analysis on values
of inductive types can be done with the \coqdockw{match} keyword.
\begin{singlespace}
\begin{coqdoccode}
\coqdocnoindent
\coqdockw{Fixpoint}
\coqdef{CoqIntro.factorial}{factorial}{\coqdocdefinition{factorial}}
(\coqdocvar{n} :
\coqexternalref{http://coq.inria.fr/stdlib/Coq.Init.Datatypes}{nat}{\coqdocinductive{nat}})
:
\coqexternalref{http://coq.inria.fr/stdlib/Coq.Init.Datatypes}{nat}{\coqdocinductive{nat}}
:=\coqdoceol
\coqdocindent{1.00em}
\coqdockw{match} \coqdocvariable{n} \coqdockw{with}\coqdoceol
\coqdocindent{1.00em}
\ensuremath{|}
\coqexternalref{http://coq.inria.fr/stdlib/Coq.Init.Datatypes}{O}{\coqdocconstructor{O}}
\ensuremath{\Rightarrow}
\coqexternalref{http://coq.inria.fr/stdlib/Coq.Init.Datatypes}{S}{\coqdocconstructor{S}}
\coqexternalref{http://coq.inria.fr/stdlib/Coq.Init.Datatypes}{O}{\coqdocconstructor{O}}\coqdoceol
\coqdocindent{1.00em}
\ensuremath{|}
\coqexternalref{http://coq.inria.fr/stdlib/Coq.Init.Datatypes}{S}{\coqdocconstructor{S}}
\coqdocvar{n} \ensuremath{\Rightarrow}
\coqexternalref{http://coq.inria.fr/stdlib/Coq.Init.Datatypes}{S}{\coqdocconstructor{S}}
\coqdocvariable{n} \ensuremath{\times}
(\coqref{CoqIntro.factorial}{\coqdocdefinition{factorial}}
\coqdocvariable{n})\coqdoceol
\coqdocindent{1.00em}
\coqdockw{end}.\coqdoceol
\end{coqdoccode}
\end{singlespace}


%\chapter{Two Syntactical Criteria in \texorpdfstring{\Coq}{Coq}}\label{chap:appendix}
%
%We include a short discussion of two \Coq technicalities.


\section{The Positivity Condition}\label{sec:positivity}

\Coq restricts inductive definitions to those that satisfy the
\emph{positivity condition}. The reason for this is that definitions
that fail this (syntactic) criterion may lead to an inconsistent
  system. For a precise definition of positivity, consult
  \citetalias[Section 4.5.3]{coq-refman-09}.

Consider again the definition of rewrite sequences from
Section~\ref{sec:seq}. A more natural way to define the type of the
\coqref{Rewriting.Lim}{\coqdocconstructor{Lim}} constructor might be
by using a $\Sigma$-type instead of a separate function for the target
terms of the branches.
\begin{singlespace}
\begin{coqdoccode}
\coqdocindent{1.00em}
\ensuremath{|} \coqdocconstructor{Lim} :
\ensuremath{\forall} \coqdocvar{s} \coqdocvar{t}
(\coqdocvar{f} :
\coqexternalref{http://coq.inria.fr/stdlib/Coq.Init.Datatypes}{nat}{\coqdocinductive{nat}}
\ensuremath{\rightarrow} \{ \coqdocvar{t'} : \coqref{Term.term}{\coqdocinductive{term}}
\& \coqdocvar{s}
\coqref{Rewriting.sequence}{$\rewrites_\mathcal{R}$}
\coqdocvariable{t'} \}),\coqdoceol
\coqdocindent{5.00em}
\coqref{Rewriting.converges}{\coqdocdefinition{converges}}
(\coqdockw{fun} \coqdocvar{n} \ensuremath{\Rightarrow}
\coqexternalref{http://coq.inria.fr/stdlib/Coq.Init.Specif}{projT1}{\coqdocdefinition{projT1}}
(\coqdocvariable{f} \coqdocvariable{n})) \coqdocvariable{t}
$\rightarrow$ (\coqdocvariable{s}
\coqref{Rewriting.sequence}{$\rewrites_\mathcal{R}$}
\coqdocvariable{t})\coqdoceol
\end{coqdoccode}
\end{singlespace}
However, this type definition does not satisfy the positivity
condition and therefore we cannot use it. We feel that the definition
from Section~\ref{sec:seq}, which does satisfy the condition, models
our intentions adequately.


\section{Guardedness in Corecursive Definitions}\label{sec:guardedness}

In \Coq, coinductive types can be defined using the
\coqdockw{CoInductive} keyword \citep{gimenez-casteran-07}. No
induction principles are defined for these types, because they are not
necessarily well-founded.\footnote{\Coq automatically derives induction
  principles for inductive definitions.}
Objects in a coinductive type may be infinite (i.e.\ contain an infinite
amount of constructors). However, in order to guarantee productivity,
definitions of such objects are required by \Coq to be in \emph{guarded}
form \citep{gimenez-94}. A corecursive definition in guarded form
satisfies two (syntactical) conditions. First, every corecursive call
must occur inside at least one constructor (of the same coinductive
type). Second, every corecursive call may only occur inside
abstractions or constructors (of the same coinductive
type).\footnote{To be more precise, the corecursive call is also
  allowed to occur inside \coqdockw{match} constructs and other
  corecursive definitions.}

In the \coqref{Term.term}{\coqdocinductive{term}} definition, we use a vector
type, parameterised by the type of its element and its size. Naturally, one
would implement a vector type in \Coq inductively, as for example has been
done in the standard library.
\begin{singlespace}
\begin{coqdoccode}
\coqdocnoindent
\coqdockw{Inductive} \coqdef{Bvector.vector}{vector}{\coqdocinductive{vector}}
(\coqdocvar{A} : \coqdockw{Type}) :
\coqexternalref{http://coq.inria.fr/stdlib/Coq.Init.Datatypes}{nat}{\coqdocinductive{nat}}
\ensuremath{\rightarrow} \coqdockw{Type} :=\coqdoceol
\coqdocindent{1.00em}
\ensuremath{|} \coqdef{Bvector.Vnil}{Vnil}{\coqdocconstructor{Vnil}}  :
\coqref{Bvector.vector}{\coqdocinductive{vector}} \coqdocvariable{A} 0\coqdoceol
\coqdocindent{1.00em}
\ensuremath{|} \coqdef{Bvector.Vcons}{Vcons}{\coqdocconstructor{Vcons}} :
\coqdocvariable{A} \ensuremath{\rightarrow} \ensuremath{\forall} \coqdocvar{n},
\coqref{Bvector.vector}{\coqdocinductive{vector}} \coqdocvariable{A}
\coqdocvariable{n} \ensuremath{\rightarrow}
\coqref{Bvector.vector}{\coqdocinductive{vector}} \coqdocvariable{A}
(\coqexternalref{http://coq.inria.fr/stdlib/Coq.Init.Datatypes}{S}{\coqdocconstructor{S}}
\coqdocvariable{n}).\coqdoceol
\end{coqdoccode}
\end{singlespace}

Now consider the following trivial example of a basic operation on terms by
corecursive traversal.
\begin{singlespace}
\begin{coqdoccode}
\coqdocnoindent
\coqdockw{CoFixpoint} \coqdef{Term.id}{id}{\coqdocdefinition{id}}
(\coqdocvar{t} : \coqref{Term.term}{\coqdocinductive{term}}) :
\coqref{Term.term}{\coqdocinductive{term}} :=\coqdoceol
\coqdocindent{1.00em}
\coqdockw{match} \coqdocvariable{t} \coqdockw{with}\coqdoceol
\coqdocindent{1.00em}
\ensuremath{|} \coqref{Term.Var}{\coqdocconstructor{Var}} \coqdocvar{x}
\ensuremath{\Rightarrow} \coqref{Term.Var}{\coqdocconstructor{Var}}
\coqdocvariable{x}\coqdoceol
\coqdocindent{1.00em}
\ensuremath{|} \coqref{Term.Fun}{\coqdocconstructor{Fun}} \coqdocvar{f}
\coqdocvar{args} \ensuremath{\Rightarrow}
\coqref{Term.Fun}{\coqdocconstructor{Fun}} \coqdocvariable{f}
(\coqdocdefinition{vmap} \coqref{Term.id}{\coqdocdefinition{id}}
\coqdocvariable{args})\coqdoceol
\coqdocindent{1.00em}
\coqdockw{end}.\coqdoceol
\end{coqdoccode}
\end{singlespace}
This definition is ill-formed, since the corecursive call to
\coqref{Term.id}{\coqdocdefinition{id}} is not guarded.\footnote{The call to
  \coqref{Term.id}{\coqdocdefinition{id}} is hidden inside
  \coqdocdefinition{vmap}, which is defined by recursion on the vector
  \coqdocvariable{args}.}
We define a recursive type of vectors as an alternative to the
inductive type.
\begin{singlespace}
\begin{coqdoccode}
\coqdocnoindent
\coqdockw{Inductive} \coqdef{Vector.Fin}{Fin}{\coqdocinductive{Fin}} :
\coqexternalref{http://coq.inria.fr/stdlib/Coq.Init.Datatypes}{nat}{\coqdocinductive{nat}}
\ensuremath{\rightarrow} \coqdockw{Type} :=\coqdoceol
\coqdocindent{1.00em}
\ensuremath{|} \coqdef{Vector.First}{First}{\coqdocconstructor{First}} :
\ensuremath{\forall} \coqdocvar{n}, \coqref{Vector.Fin}{\coqdocinductive{Fin}}
(\coqexternalref{http://coq.inria.fr/stdlib/Coq.Init.Datatypes}{S}{\coqdocconstructor{S}}
\coqdocvariable{n})\coqdoceol
\coqdocindent{1.00em}
\ensuremath{|} \coqdef{Vector.Next}{Next}{\coqdocconstructor{Next}}  :
\ensuremath{\forall} \coqdocvar{n}, \coqref{Vector.Fin}{\coqdocinductive{Fin}}
\coqdocvariable{n} \ensuremath{\rightarrow}
\coqref{Vector.Fin}{\coqdocinductive{Fin}}
(\coqexternalref{http://coq.inria.fr/stdlib/Coq.Init.Datatypes}{S}{\coqdocconstructor{S}}
\coqdocvariable{n}).\coqdoceol
\coqdocemptyline
\coqdocnoindent
\coqdockw{Definition}
\coqdef{Vector.vector}{vector}{\coqdocdefinition{vector}} (\coqdocvar{A} :
\coqdockw{Type}) (\coqdocvar{n} :
\coqexternalref{http://coq.inria.fr/stdlib/Coq.Init.Datatypes}{nat}{\coqdocinductive{nat}})
:= \coqref{Vector.Fin}{\coqdocinductive{Fin}} \coqdocvariable{n}
\ensuremath{\rightarrow} \coqdocvariable{A}.\coqdoceol
\end{coqdoccode}
\end{singlespace}
% further explain this vector type?
This makes for a definition of \coqref{Vector.vmap}{\coqdocdefinition{vmap}}
that is just an abstraction, and therefore solves the guardedness problem in
\coqref{Term.id}{\coqdocdefinition{id}}.
\begin{singlespace}
\begin{coqdoccode}
\coqdocnoindent
\coqdockw{Definition} \coqdef{Vector.vmap}{vmap}{\coqdocdefinition{vmap}}
\coqdocvar{A} \coqdocvar{B} (\coqdocvar{f} :
\coqdocvariable{A} \ensuremath{\rightarrow} \coqdocvariable{B}) \coqdocvar{n}
: \coqref{Vector.vector}{\coqdocdefinition{vector}} \coqdocvariable{A}
\coqdocvariable{n} \ensuremath{\rightarrow}
\coqref{Vector.vector}{\coqdocdefinition{vector}} \coqdocvariable{B}
\coqdocvariable{n} :=\coqdoceol
\coqdocindent{1.00em}
\coqdockw{fun} \coqdocvar{v} \coqdocvar{i} \ensuremath{\Rightarrow}
\coqdocvariable{f} (\coqdocvariable{v} \coqdocvariable{i}).\coqdoceol
\end{coqdoccode}
\end{singlespace}
