\documentclass[notheorems]{beamer}

\usepackage[english]{babel}

\usepackage[latin1]{inputenc}

\usepackage{xspace}

\newcommand{\name}[1]{\textsc{#1}\xspace}

\def\Coq{\name{Coq}}


\usepackage[T1]{fontenc}

\usepackage{amsmath}
\usepackage{amsfonts}
\usepackage{amssymb}
\usepackage{amsthm}

\theoremstyle{definition}
\newtheorem{definition}{Definition}

%\usepackage{tgtermes}
\usepackage{tgheros}
%\usepackage{cmbright}
\usepackage{qtxmath}
%\usepackage{kpfonts}
\renewcommand{\ttdefault}{txtt}

%\usepackage[amsmath,thmmarks]{ntheorem}

\usepackage[color]{coqdoc}
\setlength{\coqdocbaseindent}{0.7em}

\usepackage{beamerthemesplit}

\setbeamertemplate{background canvas}[vertical shading][bottom=red!10,top=blue!10]
\setbeamertemplate{navigation symbols}{}
\setbeamertemplate{headline}{}
\usetheme{Warsaw}
\useinnertheme{rectangles}

\colorlet{darkred}{red!80!black}
\colorlet{darkblue}{blue!80!black}
\colorlet{darkgreen}{green!80!black}

\title{Infinitary Rewriting in \Coq}

\author{Martijn Vermaat}
\institute{VU University Amsterdam}
\date{August 2010}


\begin{document}


\frame{\titlepage}


\frame{

  \frametitle{Infinitary rewriting in \Coq}

  \tableofcontents

}


\section{Introduction}


\frame{

  \frametitle{Why?}

  Why formalize a mathematical theory mechanically?

}


\frame{

  \frametitle{\Coq}

  \Coq is a \emph{proof assistant}

  \pause
  \begin{itemize}[<+->]
    \item (interactive) theorem prover
    \item proof checker
    \item programming language
    \item certified program extractor
  \end{itemize}

}


\frame{

  \frametitle{Related work}

  \begin{block}{Rewriting in \Coq}
    \begin{description}
    \item[\CoLoR] focussed on termination
    \item[\Coccinelle] base of \CiME termination tool
    \end{description}
  \end{block}

  \begin{block}{Ordinals in \Coq}
    \begin{itemize}
    \item Veblen Normal Form (Casteran)
    \item Set-theoretic ordinals (Mamane)
    \end{itemize}
  \end{block}

}


\section{Infinite terms}


\frame{

  \frametitle{Infinitary rewriting in \Coq}

  \tableofcontents[currentsection]

}


\frame{

  \frametitle{Coinductive terms}

  $\TerI(\X)$ defined in \Coq:\\[2em]

\begin{coqdoccode}
\coqdocnoindent
\coqdockw{CoInductive}
\coqdocinductive{term} :
\coqdockw{Type} :=\coqdoceol
\coqdocindent{1.00em}
\ensuremath{|} \coqdocconstructor{Var} :
\coqdocvar{X}
\ensuremath{\rightarrow}
\coqdocinductive{term}\coqdoceol
\coqdocindent{1.00em}
\ensuremath{|} \coqdocconstructor{Fun} :
\ensuremath{\forall} \coqdocvar{f} : \coqdocvar{F},
\coqdocdefinition{vector}
\coqdocinductive{term}
(\coqdocprojection{arity} \coqdocvariable{f})
\ensuremath{\rightarrow}
\coqdocinductive{term}.\coqdoceol
\end{coqdoccode}

% TODO: guardedness problem?

}


\frame{

  \frametitle{Equalities on infinite terms}

\begin{coqdoccode}
\coqdocnoindent
\coqdockw{CoInductive}
$\sim$ :
\coqdocinductive{term} \ensuremath{\rightarrow}
\coqdocinductive{term} \ensuremath{\rightarrow}
\coqdockw{Prop} :=\coqdoceol
\coqdocindent{1.00em}
\ensuremath{|}
\coqdocconstructor{$\sim_\text{Var}$}
:
\ensuremath{\forall} \coqdocvar{x},
\coqdocconstructor{Var} \coqdocvariable{x}
$\sim$
\coqdocconstructor{Var}
\coqdocvariable{x}\coqdoceol
\coqdocindent{1.00em}
\ensuremath{|}
\coqdocconstructor{$\sim_\text{Fun}$}
:
\ensuremath{\forall} \coqdocvar{f} \coqdocvar{v} \coqdocvar{w},
(\ensuremath{\forall} \coqdocvar{i},
\coqdocvariable{v} \coqdocvariable{i}
$\sim$
\coqdocvariable{w} \coqdocvariable{i})
\ensuremath{\rightarrow}
\coqdocconstructor{Fun} \coqdocvariable{f}
\coqdocvariable{v}
$\sim$
\coqdocconstructor{Fun} \coqdocvariable{f}
\coqdocvariable{w}.\coqdoceol
\coqdocemptyline
\coqdocemptyline
\coqdocemptyline
\coqdocnoindent
\coqdockw{Inductive}
\equpto{}
:
\coqdocinductive{nat}
\ensuremath{\rightarrow} \coqdocinductive{term}
\ensuremath{\rightarrow} \coqdocinductive{term}
\ensuremath{\rightarrow} \coqdockw{Prop} :=\coqdoceol
\coqdocindent{1.00em}
\ensuremath{|}
\coqdocconstructor{teut$_0$}   :
\ensuremath{\forall} \coqdocvar{s} \coqdocvar{t},
\coqdocvariable{s} \equpto{0}
\coqdocvariable{t}\coqdoceol
\coqdocindent{1.00em}
\ensuremath{|}
\coqdocconstructor{teut$_\text{Var}$}
:
\ensuremath{\forall} \coqdocvar{d} \coqdocvar{x},
\coqdocconstructor{Var} \coqdocvariable{x}
\equpto{\coqdocvariable{d}}
\coqdocconstructor{Var}
\coqdocvariable{x}\coqdoceol
\coqdocindent{1.00em}
\ensuremath{|}
\coqdocconstructor{teut$_\text{Fun}$}
:
\ensuremath{\forall} \coqdocvar{d} \coqdocvar{f} \coqdocvar{v}
\coqdocvar{w},\coqdoceol
\coqdocindent{4em}
(\ensuremath{\forall} \coqdocvar{i},
\coqdocvariable{v} \coqdocvariable{i}
\equpto{\coqdocvariable{d}}
\coqdocvariable{w} \coqdocvariable{i}) \ensuremath{\rightarrow}
\coqdocconstructor{Fun}
\coqdocvariable{f} \coqdocvariable{v}
\equpto{\coqdocconstructor{S} \, \coqdocvariable{d}}
\coqdocconstructor{Fun} \coqdocvariable{f}
\coqdocvariable{w}.\coqdoceol
\coqdocemptyline
\coqdocnoindent
\coqdockw{Definition}
\coqdocvar{s}
$\equiv$
\coqdocvar{t} :=
\ensuremath{\forall} \coqdocvar{d},
\coqdocvariable{s}
\equpto{\coqdocvariable{d}}
\coqdocvariable{t}.\coqdoceol
\coqdocemptyline
\coqdocemptyline
\coqdocemptyline
\coqdocnoindent
\coqdockw{Lemma}
\coqdoclemma{term\_bis\_term\_eq}
: \ensuremath{\forall} \coqdocvar{s} \coqdocvar{t},
\coqdocvariable{s}
$\sim$
\coqdocvariable{t} \ensuremath{\leftrightarrow}
\coqdocvariable{s}
$\equiv$
\coqdocvariable{t}.\coqdoceol
\end{coqdoccode}

}


\section{Ordinal Numbers}


\frame{

  \frametitle{Infinitary rewriting in \Coq}

  \tableofcontents[currentsection]

}


\frame{

  \frametitle{Brouwer ordinals}

\begin{definition}
The set of \emph{Brouwer ordinals} $\Ord$ is defined by induction:
\begin{itemize}
  \item
    $0 \in \Ord$
  \item
    If $\alpha \in \Ord$, then $\alpha^+ \in \Ord$
  \item
    If $\alpha_i \in \Ord$ for all $i \in \mathbb{N}$, then $\sqcup_i
    \alpha_i \in \Ord$
\end{itemize}
\end{definition}

\begin{itemize}
  \item We need an order on $\Ord$
  \item It should position $\sqcup_i i + 3$ and $\sqcup_i i \times 2$
    at the same rank
  \item And what to make of $\sqcup \{ 3, 3, 3, \ldots \}$?
\end{itemize}

}


\frame{

  \frametitle{Structural partial order on $\Ord$}

\begin{definition}
The set-valued function $\Phi$ defines the \emph{predecessor indices}
$\Phi(\alpha)$
\emph{for} $\alpha$ by recursion on $\alpha$:
\begin{align*}
  \Phi(0)                 &= \emptyset \\
  \Phi(\alpha^+)          &= \Phi(\alpha)^? \\
  \Phi(\sqcup_i \alpha_i) &= (\Sigma n \in \mathbb{N}) \;
  \Phi(\alpha_n)
\end{align*}
\end{definition}

\begin{definition}
The function $\_[\_] : (\prod \alpha : \Ord) \; \Phi(\alpha)
\rightarrow
\Ord$ defines the \emph{predecessor} $\alpha[\iota]$ \emph{of}
$\alpha$ \emph{indexed by} $\iota$ recursively on $\alpha$:
\begin{align*}
  \alpha^+[\textsc{none}]                     &= \alpha \\
  \alpha^+[\textsc{some $\iota$}]             &= \alpha[\iota] \\
  \sqcup_i \alpha_i[\langle n, \iota \rangle] &= \alpha_n[\iota]
\end{align*}
\end{definition}

}


\frame{

  \frametitle{Extensional order on $\Ord$}

\begin{definition}
We define the \emph{order} $\preceq$ as a binary relation on $\Ord$ by
induction:
\begin{itemize}
  \item
    $0 \preceq \beta$ for every ordinal $\beta \in \Ord$
  \item
    For all $\alpha, \beta \in \Ord$ and $\iota \in \Phi(\beta)$, if
    $\alpha \preceq \beta[\iota]$ then $\alpha^+ \preceq \beta$
  \item
    For all $\alpha_0, \alpha_1, \alpha_2, \ldots, \beta \in \Ord$, if
    $\alpha_n \preceq \beta$ for all $n \in \mathbb{N}$, then
    $\sqcup_i
    \alpha_i \preceq \beta$
\end{itemize}
\end{definition}

\begin{itemize}
  \item $\alpha \simeq \beta$ if $\alpha \preceq \beta$ and $\beta
    \preceq \alpha$
  \item $\alpha \prec \beta$ if $\alpha \preceq \beta[\iota]$ for some
    $\iota \in \Phi(\beta)$
\end{itemize}

}


\frame{

  \frametitle{Brouwer ordinals in \Coq}

\begin{coqdoccode}
\coqdocnoindent
\coqdockw{Inductive} \coqdocinductive{ord}
:
\coqdockw{Set} :=\coqdoceol
\coqdocindent{1.00em}
\ensuremath{|} \coqdocconstructor{Zero}
:
\coqdocinductive{ord}\coqdoceol
\coqdocindent{1.00em}
\ensuremath{|} \coqdocconstructor{Succ}
:
\coqdocinductive{ord} \ensuremath{\rightarrow}
\coqdocinductive{ord}\coqdoceol
\coqdocindent{1.00em}
\ensuremath{|}
\coqdocconstructor{Limit} :
(\coqdocinductive{nat}
\ensuremath{\rightarrow} \coqdocinductive{ord})
\ensuremath{\rightarrow}
\coqdocinductive{ord}.\coqdoceol
\end{coqdoccode}

}


\section{Rewrite Sequences of Ordinal Length}


\frame{

  \frametitle{Infinitary rewriting in \Coq}

  \tableofcontents[currentsection]

}


\frame{

  \frametitle{Rewrite steps}

\begin{coqdoccode}
\coqdocnoindent
\coqdockw{Inductive}
$\rightarrow_\mathcal{R}$ :
\coqdocinductive{term} \ensuremath{\rightarrow}
\coqdocinductive{term} \ensuremath{\rightarrow}
\coqdockw{Type} :=\coqdoceol
\coqdocindent{1.00em}
\ensuremath{|} \coqdocconstructor{Step}
:
\ensuremath{\forall} (\coqdocvar{s} \coqdocvar{t} :
\coqdocinductive{term}) (\coqdocvar{$\rho$} :
\coqdocrecord{rule}) (\coqdocvar{c} :
\coqdocinductive{context}) (\coqdocvar{$\sigma$} :
\coqdocdefinition{substitution}),\coqdoceol
\coqdocindent{6.50em} \coqdocvariable{$\rho$}
\coqdocdefinition{$\in$}
\coqdocvar{$\mathcal{R}$} \ensuremath{\rightarrow}\coqdoceol
\coqdocindent{6.50em}
\coqdocvariable{c}
[(\coqdocprojection{lhs}
\coqdocvariable{$\rho$})\coqdocvariable{$^\sigma$}] $\sim$
\coqdocvariable{s}
\ensuremath{\rightarrow}\coqdoceol
\coqdocindent{6.50em}
\coqdocvariable{c}
[(\coqdocprojection{rhs}
\coqdocvariable{$\rho$})\coqdocvariable{$^\sigma$}] $\sim$
\coqdocvariable{t}
\ensuremath{\rightarrow}\coqdoceol
\coqdocindent{6.50em}
\coqdocvariable{s} $\rightarrow_\mathcal{R}$
\coqdocvariable{t}.\coqdoceol
\end{coqdoccode}

}


\section{Conclusions}


\frame{

  \frametitle{Infinitary rewriting in \Coq}

  \tableofcontents[currentsection]

}


\end{document}
