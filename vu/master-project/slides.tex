\documentclass[notheorems]{beamer}

\usepackage[english]{babel}

\usepackage[latin1]{inputenc}

\usepackage{xspace}

\newcommand{\name}[1]{\textsc{#1}\xspace}

\def\Coq{\name{Coq}}


\usepackage[T1]{fontenc}

\usepackage{amsmath}
\usepackage{amsfonts}
\usepackage{amssymb}

%\usepackage{tgtermes}
\usepackage{tgheros}
%\usepackage{cmbright}
\usepackage{qtxmath}
%\usepackage{kpfonts}
\renewcommand{\ttdefault}{txtt}

%\usepackage[amsmath,thmmarks]{ntheorem}

\usepackage[color]{coqdoc}
\setlength{\coqdocbaseindent}{0.7em}

\usepackage{beamerthemesplit}

\setbeamertemplate{background canvas}[vertical shading][bottom=red!10,top=blue!10]
\setbeamertemplate{navigation symbols}{}
\setbeamertemplate{headline}{}
\usetheme{Warsaw}
\useinnertheme{rectangles}

\colorlet{darkred}{red!80!black}
\colorlet{darkblue}{blue!80!black}
\colorlet{darkgreen}{green!80!black}

\title{Infinitary Rewriting in \Coq}

\author{Martijn Vermaat}
\institute{VU University Amsterdam}
\date{August 2010}


\begin{document}


\frame{\titlepage}


\frame{

  \frametitle{Infinitary Rewriting in \Coq}

  \tableofcontents

}


\section{Introduction}


\frame{

  \frametitle{Why?}

  Why formalize a mathematical theory mechanically?

}


\frame{

  \frametitle{\Coq}

  \Coq is a \emph{proof assistant}

  \pause
  \begin{itemize}[<+->]
    \item (interactive) theorem prover
    \item proof checker
    \item programming language
    \item certified program extractor
  \end{itemize}

}


\frame{

  \frametitle{Related Work}

  Rewriting in \Coq

  Ordinals in \Coq

}


\section{Infinite Terms}

\section{Ordinal Numbers}

\section{Rewrite Sequences of Ordinal Length}

\section{Conclusions}


\end{document}
