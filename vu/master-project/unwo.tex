\chapter[\texorpdfstring{UN$^\infty$ in Weakly Orthogonal Systems}{UN
  in Weakly Orthogonal Systems}]{\texorpdfstring{Unique Normal Forms
    in\\Weakly Orthogonal Systems}{Unique Normal Forms in Weakly
    Orthogonal Systems}}\label{chap:unwo}

TODO: this intro

Every orthogonal TRS exhibits the infinitary unique normal forms
(UN$^\infty$) property. We might expect this property to generalize to
weakly orthogonal systems. However, this does not turn out to be the
case.


\section{A Counterexample}

We describe a simple counterexample showing that weak orthogonality
does not imply the UN$^\infty$ property \citep{endrullis-10}.

We work in a signature with unary function symbols $P$ and $S$. In the
notation of terms, we omit the brackets around arguments and assume
right-associativity of function symbol application, e.g.\ writing
$PSx$ for $P(S(x))$. A notation for finite repetitions of a function
symbol $f$ terminated by a term $t$ is defined by
\begin{inparaenum}[(i)]
\item $f^0 t = t$ and
\item $f^{n+1} = ff^nt$.
\end{inparaenum}
The infinite nesting $fff \ldots$ of $f$ is written $f^\omega$.
% TODO: nesting vs repetition

Consider the TRS consisting of the two left-linear rewrite rules
$\rho_1$ and $\rho_2$:
\begin{align*}
  \rho_1 \, : \, PSx \to x \qquad \qquad \qquad
  \rho_2 \, : \, SPx \to x
\end{align*}
This system has two critical pairs $\langle Px, Px \rangle$ and
$\langle Sx, Sx \rangle$, both of which are trivial, establishing
weak orthogonality. The infinite term $\psi = P^1 S^2 P^3 S^4 \ldots$
has two normal forms. It rewrites to $S^\omega$ in $\omega$ many
$\rho_1$-steps and to $P^\omega$ in $\omega$ many $\rho_2$-steps.

Other interesting properties of this TRS (e.g.\ weak normalisation is
not preserved under rewriting) and a translation to the infinitary
$\lambda \beta \eta$-calculus can be found in \citet{endrullis-10}.


\subsection{\texorpdfstring{Rewriting $\psi$ to $S^\omega$}{Rewriting
    PSSPPP... to SSS...}}

We show briefly what rewriting $\psi$ to $S^\omega$ amounts
to. Rewriting $\psi$ to $P^\omega$ is done in a similar way.
An obvious way to define $\psi$ by corecursion is via auxiliary terms
$\psi'_n$ parameterized by $n$ as follows:
\begin{align*}
  \psi'_n = S^n P^{n + 1} \psi'_{n + 2} \qquad \qquad \qquad
  \psi = \psi'_0
\end{align*}
But a more useful definition for our present purposes, and the one we
stick with, is the slight reformulation:
\begin{align*}
  \psi'_n = P^{2 n + 1} S^{2 n + 2} \psi'_{n + 1} \qquad
  \qquad \qquad
  \psi = \psi'_0
\end{align*}
For any term $t$ and natural numbers $n$ and $m$ we have $S^n P^{m+1}
S^{m+1} t \rightarrow_{\rho_1} S^n P^m S^m t$ and thus $S^n P^m S^m t
\twoheadrightarrow S^n t$ by iterating $m$ such steps. Instantiating
$m$ with $2 n + 1$ and $t$ with $S \psi'_{n + 1}$, we obtain
%$S^n P^{2 n + 1} S^{2 n + 1} S \psi'_{n + 1} \equiv S^n \psi'_n$
$S^n \psi'_n \twoheadrightarrow S^{n+1} \psi'_{n + 1}$ for any $n$.
Concatenating these sequences, iterating $n$ from $0$ onwards, we
arrive at $\psi \twoheadrightarrow S^\omega$.
% TODO: wording 'iterating n from 0 onwards'


\section{A Mechanic Formalization}

Implementation in \Coq.
