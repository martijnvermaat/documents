\chapter[\texorpdfstring{UN$^\infty$ in Weakly Orthogonal Systems}{UN
  in Weakly Orthogonal Systems}]{\texorpdfstring{Unique Normal Forms
    in\\Weakly Orthogonal Systems}{Unique Normal Forms in Weakly
    Orthogonal Systems}}\label{chap:unwo}

TODO: this intro

Every orthogonal TRS exhibits the infinitary unique normal forms
(UN$^\infty$) property. We might expect this property to generalise to
weakly orthogonal systems. However, this does not turn out to be the
case.


\section{A Counterexample}\label{sec:counterexample}

% TODO: is the naming D,U really a good idea?

We describe a simple counterexample showing that weak orthogonality
does not imply the UN$^\infty$ property \citep{endrullis-10}.

We work in a signature with unary function symbols $D$ and
$U$.\footnote{We can think of $D$ and $U$ as `down' and `up'. The
  original formulation of this TRS uses $P$ and $S$ (`predecessor' and
  `successor'), but to avoid notational conflicts with the
  \coqexternalref{http://coq.inria.fr/stdlib/Coq.Init.Datatypes}{S}{\coqdocconstructor{S}}
  constructor for
  \coqexternalref{http://coq.inria.fr/stdlib/Coq.Init.Datatypes}{nat}{\coqdocinductive{nat}}
  in \Coq, we proceed with this modification.}
In the notation of terms, we omit the brackets around arguments and
assume right-associativity of function symbol application,
e.g.\ writing $DU$ for $D(U(x))$. A notation for finite repetitions of
a function symbol $f$ terminated by a term $t$ is defined by
\begin{inparaenum}[(i)]
\item $f^0 t = t$ and
\item $f^{n+1} = ff^nt$.
\end{inparaenum}
The infinite nesting $fff \ldots$ of $f$ is written $f^\omega$.
% TODO: nesting vs repetition

Consider the TRS consisting of the two left-linear rewrite rules
$\rho_1$ and $\rho_2$:
\begin{align*}
  \rho_1 \, : \, DUx \to x \qquad \qquad \qquad
  \rho_2 \, : \, UDx \to x
\end{align*}
This system has two critical pairs $\langle Dx, Dx \rangle$ and
$\langle Ux, Ux \rangle$, both of which are trivial, establishing
weak orthogonality. The infinite term $\du = D^1 U^2 D^3 U^4 \ldots$
has two normal forms. It rewrites to $U^\omega$ in $\omega$ many
$\rho_1$-steps and to $D^\omega$ in $\omega$ many $\rho_2$-steps.

Other interesting properties of this TRS (e.g.\ weak normalisation is
not preserved under rewriting) and a translation to the infinitary
$\lambda \beta \eta$-calculus are discussed by \citet{endrullis-10}.


\subsection{\texorpdfstring{Rewriting $\du$ to
    $U^\omega$}{Rewriting DUUDDD... to
    UUU...}}\label{sub:counterexample}

We show briefly what rewriting $\du$ to $U^\omega$ amounts
to. Rewriting $\du$ to $D^\omega$ is done in a similar way.
An obvious way to define $\du$ by corecursion is via auxiliary terms
$\du'_n$ parameterised by $n$ as follows:
\begin{align*}
  \du'_n = U^n D^{n + 1} \du'_{n + 2} \qquad \qquad \qquad
  \du = \du'_0
\end{align*}
But a more useful definition for our present purposes, and the one we
stick with, is the slight reformulation:
\begin{align*}
  \du'_n = D^{2 n + 1} U^{2 n + 2} \du'_{n + 1} \qquad
  \qquad \qquad
  \du = \du'_0
\end{align*}
For any term $t$ and natural numbers $n$ and $m$ we have $U^n D^{m+1}
U^{m+1} t \rightarrow_{\rho_1} U^n D^m U^m t$ and thus $U^n D^m U^m t
\rewrites U^n t$ by iterating $m$ such steps. Instantiating
$m$ with $2 n + 1$ and $t$ with $U \du'_{n + 1}$, we obtain
%$S^n P^{2 n + 1} S^{2 n + 1} S \du'_{n + 1} \equiv S^n \du'_n$
$U^n \du'_n \rewrites U^{n+1} \du'_{n + 1}$ for any $n$.
Concatenating these sequences, iterating $n$ from $0$ onwards, we
arrive at $\du \rewrites U^\omega$.
% TODO: wording 'iterating n from 0 onwards'


\section{The Counterexample in \Coq}

We implement the counterexample from Section~\ref{sec:counterexample}
using the \Coq development described in
Chapter~\ref{chap:implementation}.

The rewrite rules $\rho_1$ and $\rho_2$ are straightforwardly defined
and shown left-linear. By a simple proof we obtain that all critical
pairs are trivial and hence that the TRS is weakly orthogonal.
\begin{singlespace}
\begin{coqdoccode}
%\coqdocnoindent
%\coqdockw{Definition}
%\coqdef{ExampleUNWO.UNWOtrs}{UNWO\_trs}{\coqdocdefinition{$\mathcal{R}$}}
%:=
%\coqdocdefinition{$\rho_1$} ::
%\coqdocdefinition{$\rho_2$} ::
%\coqexternalref{http://coq.inria.fr/stdlib/Coq.Init.Datatypes}{nil}{\coqdocconstructor{nil}}.\coqdoceol
%\coqdocemptyline
\coqdocnoindent
\coqdockw{Lemma}
\coqdoclemma{wo$_\mathcal{R}$} :
\coqdocdefinition{weakly\_orthogonal} % TODO: link to definition
%\coqref{ExampleUNWO.UNWOtrs}{\coqdocdefinition{$\mathcal{R}$}}.\coqdoceol
\coqdocdefinition{$\mathcal{R}$}.\coqdoceol
\end{coqdoccode}
\end{singlespace}
We introduce the notation \begin{coqdoccode}\coqdocvariable{f} @
  \coqdocvariable{t}\end{coqdoccode} to mean
\begin{coqdoccode}\coqref{Term.Fun}{\coqdocconstructor{Fun}}
  \coqdocvariable{f} (\coqdocdefinition{vcons} \coqdocvariable{t}
  (\coqdocdefinition{vnil}
  \coqref{Term.term}{\coqdocinductive{term}}))\end{coqdoccode}. For
brevity, mirrored constructions for both function symbols are only
discussed for one of them. The infinite term $U^\omega$ is defined by
corecursion and finite repetitions $U^n t$ are defined by recursion
(and are assumed to generalise to contexts with the same notation).
% TODO: wording of the generalisation to contexts
\begin{singlespace}
\begin{coqdoccode}
\coqdocnoindent
\coqdockw{CoFixpoint}
\coqdef{ExampleUNWO.repeatU}{repeat\_U}{\coqdocdefinition{U$^\omega$}}
: \coqref{Term.term}{\coqdocinductive{term}} :=
\coqdocconstructor{U} @
\coqref{ExampleUNWO.repeatU}{\coqdocdefinition{U$^\omega$}}.\coqdoceol
\coqdocemptyline
\coqdocnoindent
\coqdockw{Fixpoint}
\coqdef{ExampleUNWO.Unt}{Unt}{\coqdocdefinition{U}}$^\coqdocvar{n}$
\coqdocvar{t} :=\coqdoceol
\coqdocindent{1.00em}
\coqdockw{match} \coqdocvariable{n} \coqdockw{with}\coqdoceol
\coqdocindent{1.00em}
\ensuremath{|}
\coqexternalref{http://coq.inria.fr/stdlib/Coq.Init.Datatypes}{O}{\coqdocconstructor{O}}
\ensuremath{\Rightarrow} \coqdocvariable{t}\coqdoceol
\coqdocindent{1.00em}
\ensuremath{|}
\coqexternalref{http://coq.inria.fr/stdlib/Coq.Init.Datatypes}{S}{\coqdocconstructor{S}}
\coqdocvar{n} \ensuremath{\Rightarrow}
\coqdocconstructor{U} @
(\coqref{ExampleUNWO.Unt}{\coqdocdefinition{U}}$^\coqdocvariable{n}$
\coqdocvariable{t})\coqdoceol
\coqdocindent{1.00em}
\coqdockw{end}.\coqdoceol
\end{coqdoccode}
\end{singlespace}
Unfortunately, $\du$ is not as easily defined. Although clearly
productive, direct translations of the corecursions in
Section~\ref{sub:counterexample} do not satisfy \Coq's guardedness
condition (see also Section~\ref{sub:guardedness}). The conclusion of
a \emph{trial and error} approach is that we must use anonymous cofix
constructions. The definition we proceed with is the following.
% TODO: wording 'trial and error approach'
\begin{singlespace}
\begin{coqdoccode}
\coqdocnoindent
\coqdockw{CoFixpoint}
\coqdef{ExampleUNWO.psi'}{psi'}{\coqdocdefinition{$\du'$}} \coqdocvar{n}
: \coqref{Term.term}{\coqdocinductive{term}} :=\coqdoceol
\coqdocindent{1.00em}
(\coqdocvar{cofix} \coqdocvar{Ds} (\coqdocvar{d} :
\coqexternalref{http://coq.inria.fr/stdlib/Coq.Init.Datatypes}{nat}{\coqdocinductive{nat}})
:=\coqdoceol
\coqdocindent{2.00em}
\coqdockw{match} \coqdocvariable{d} \coqdockw{with}\coqdoceol
\coqdocindent{2.00em}
\ensuremath{|}
\coqexternalref{http://coq.inria.fr/stdlib/Coq.Init.Datatypes}{O}{\coqdocconstructor{O}}
\ensuremath{\Rightarrow} \coqdocconstructor{D}
@ (\coqdocvar{cofix} \coqdocvar{Us} (\coqdocvar{u} :
\coqexternalref{http://coq.inria.fr/stdlib/Coq.Init.Datatypes}{nat}{\coqdocinductive{nat}})
:=\coqdoceol
\coqdocindent{7.50em}
\coqdockw{match} \coqdocvariable{u} \coqdockw{with}\coqdoceol
\coqdocindent{7.50em}
\ensuremath{|}
\coqexternalref{http://coq.inria.fr/stdlib/Coq.Init.Datatypes}{O}{\coqdocconstructor{O}}
\ensuremath{\Rightarrow}
\coqref{ExampleUNWO.psi'}{\coqdocdefinition{$\du'$}}
(\coqexternalref{http://coq.inria.fr/stdlib/Coq.Init.Datatypes}{S}{\coqdocconstructor{S}}
\coqdocvariable{n})\coqdoceol
\coqdocindent{7.50em}
\ensuremath{|}
\coqexternalref{http://coq.inria.fr/stdlib/Coq.Init.Datatypes}{S}{\coqdocconstructor{S}}
\coqdocvar{u} \ensuremath{\Rightarrow}
\coqdocconstructor{U} @
\coqdocconstructor{U} @ (\coqdocvariable{Us}
\coqdocvariable{u})\coqdoceol
\coqdocindent{7.50em}
\coqdockw{end})
(\coqexternalref{http://coq.inria.fr/stdlib/Coq.Init.Datatypes}{S}{\coqdocconstructor{S}}
\coqdocvariable{n})\coqdoceol
\coqdocindent{2.00em}
\ensuremath{|}
\coqexternalref{http://coq.inria.fr/stdlib/Coq.Init.Datatypes}{S}{\coqdocconstructor{S}}
\coqdocvar{d} \ensuremath{\Rightarrow}
\coqdocconstructor{D} @
\coqdocconstructor{D} @
(\coqdocvariable{Ds} \coqdocvariable{d})\coqdoceol
\coqdocindent{2.00em}
\coqdockw{end}) \coqdocvariable{n}.\coqdoceol
\coqdocemptyline
\coqdocnoindent
\coqdockw{Definition}
\coqdef{ExampleUNWO.psi}{psi}{\coqdocdefinition{$\du$}} :=
\coqref{ExampleUNWO.psi'}{\coqdocdefinition{$\du'$}} 0.\coqdoceol
\end{coqdoccode}
\end{singlespace}

We now prove that $U^\omega$ and $D^\omega$ are (distinct) normal
forms. This is essentially done by exhaustive case analysis of
the position of redex occurrences in the terms, yielding that there
can not be such an occurence.
\begin{singlespace}
\begin{coqdoccode}
\coqdocnoindent
\coqdockw{Lemma} \coqdoclemma{nf$_{\text{U}^\omega}$} :
\coqdocdefinition{normal\_form} % TODO: link to definition
(\coqdocvar{system} :=
\coqdocdefinition{$\mathcal{R}$})
\coqref{ExampleUNWO.repeatU}{\coqdocdefinition{U$^\omega$}}.\coqdoceol
\coqdocemptyline
\coqdocnoindent
\coqdockw{Lemma} \coqdoclemma{nf$_{\text{D}^\omega}$} :
\coqdocdefinition{normal\_form}
(\coqdocvar{system} :=
\coqdocdefinition{$\mathcal{R}$})
\coqdocdefinition{D$^\omega$}.\coqdoceol
\coqdocemptyline
\coqdocnoindent
\coqdockw{Lemma}
\coqdoclemma{neq$^{\text{U}^\omega}_{\text{D}^\omega}$} :
\ensuremath{\lnot}
\coqref{ExampleUNWO.repeatU}{\coqdocdefinition{U$^\omega$}}
\coqref{TermEquality.termbis}{$\bis$}
\coqdocdefinition{D$^\omega$}.\coqdoceol
\end{coqdoccode}
\end{singlespace}
% TODO: for general equality statements, use \bis or \equiv?

Constructing a rewrite sequence from $\du$ to $U^\omega$ is done in
much the same way as described in
Section~\ref{sub:counterexample}. First, we define the parameterised
step that is used in the rewrite sequence. It eliminates one pair of $D,
U$ constructors in a term of the form $U^n D^{m+1} U^{m+1} t$. The
omitted argument of the \coqref{Rewriting.Step}{\coqdocconstructor{Step}}
constructor (denoted by \coqdoclemma{\_}) is a proof of $\rho_1 \in
\mathcal{R}$.
\begin{singlespace}
\begin{coqdoccode}
\coqdocnoindent
\coqdockw{Definition}
\coqdef{ExampleUNWO.sigma}{sigma}{\coqdocdefinition{$\sigma$}}
\coqdocvar{t} (\coqdocvar{y} :
\coqdocdefinition{X}) :
\coqref{Term.term}{\coqdocinductive{term}} :=\coqdoceol
\coqdocindent{1.00em}
\coqdockw{match} \coqdocdefinition{beq\_var} \coqdocvariable{y} \coqdocvariable{x} \coqdockw{with}\coqdoceol
\coqdocindent{1.00em}
\ensuremath{|} \coqexternalref{http://coq.inria.fr/stdlib/Coq.Init.Datatypes}{true}{\coqdocconstructor{true}} \ensuremath{\Rightarrow}
\coqdocvariable{t}\coqdoceol
\coqdocindent{1.00em}
\ensuremath{|} \coqexternalref{http://coq.inria.fr/stdlib/Coq.Init.Datatypes}{false}{\coqdocconstructor{false}} \ensuremath{\Rightarrow}
\coqref{Term.Var}{\coqdocconstructor{Var}}
\coqdocvariable{y}\coqdoceol
\coqdocindent{1.00em}
\coqdockw{end}.\coqdoceol
\coqdocemptyline
\coqdocnoindent
\coqdockw{Lemma}
\coqdef{ExampleUNWO.facttermbisUmDSnUSnt}{fact\_term\_bis\_UmDSnUSnt}{\coqdoclemma{fact$_\pi^\text{source}$}}
:
\ensuremath{\forall} (\coqdocvar{n} \coqdocvar{m} :
\coqexternalref{http://coq.inria.fr/stdlib/Coq.Init.Datatypes}{nat}{\coqdocinductive{nat}})
(\coqdocvar{t} :
\coqref{Term.term}{\coqdocinductive{term}}),\coqdoceol
\coqdocindent{1.00em}
(\coqref{ExampleUNWO.Unt}{\coqdocdefinition{U}}$^\coqdocvariable{n}$
\coqdocdefinition{D}$^\coqdocvariable{m}$
$\Box$)[\coqref{Substitution.substitute}{\coqdocdefinition{substitute}}
  (\coqref{ExampleUNWO.sigma}{\coqdocdefinition{$\sigma$}}
(\coqref{ExampleUNWO.Unt}{\coqdocdefinition{U}}$^\coqdocvariable{m}$
    \coqdocvariable{t})) (\coqref{Rewriting.lhs}{\coqdocprojection{lhs}}
\coqdocdefinition{$\rho_1$})] \coqref{TermEquality.termbis}{$\bis$}
\coqref{ExampleUNWO.Unt}{\coqdocdefinition{U}}$^\coqdocvariable{n}$
\coqdocdefinition{D}$^{\coqexternalref{http://coq.inria.fr/stdlib/Coq.Init.Datatypes}{S}{\coqdocconstructor{S}} \coqdocvariable{m}}$
\coqref{ExampleUNWO.Unt}{\coqdocdefinition{U}}$^{\coqexternalref{http://coq.inria.fr/stdlib/Coq.Init.Datatypes}{S}{\coqdocconstructor{S}}
\coqdocvariable{m}}$
\coqdocvariable{t}.\coqdoceol
\coqdocemptyline
\coqdocnoindent
\coqdockw{Lemma}
\coqdef{ExampleUNWO.facttermbisUmDnUnt}{fact\_term\_bis\_UmDnUnt}{\coqdoclemma{fact$_\pi^\text{target}$}}
:
\ensuremath{\forall} (\coqdocvar{n} \coqdocvar{m} :
\coqexternalref{http://coq.inria.fr/stdlib/Coq.Init.Datatypes}{nat}{\coqdocinductive{nat}})
(\coqdocvar{t} :
\coqref{Term.term}{\coqdocinductive{term}}),\coqdoceol
\coqdocindent{1.00em}
(\coqref{ExampleUNWO.Unt}{\coqdocdefinition{U}}$^\coqdocvariable{n}$
\coqdocdefinition{D}$^\coqdocvariable{m}$
$\Box$)[\coqref{Substitution.substitute}{\coqdocdefinition{substitute}}
  (\coqref{ExampleUNWO.sigma}{\coqdocdefinition{$\sigma$}}
(\coqref{ExampleUNWO.Unt}{\coqdocdefinition{U}}$^\coqdocvariable{m}$
    \coqdocvariable{t})) (\coqref{Rewriting.rhs}{\coqdocprojection{rhs}}
\coqdocdefinition{$\rho_1$})] \coqref{TermEquality.termbis}{$\bis$}
\coqref{ExampleUNWO.Unt}{\coqdocdefinition{U}}$^\coqdocvariable{n}$
\coqdocdefinition{D}$^\coqdocvariable{m}$
\coqref{ExampleUNWO.Unt}{\coqdocdefinition{U}}$^\coqdocvariable{m}$
\coqdocvariable{t}.\coqdoceol
\coqdocemptyline
\coqdocnoindent
\coqdockw{Definition}
\coqdef{ExampleUNWO.pi}{pi}{\coqdocdefinition{$\pi$}}
\coqdocvar{n} \coqdocvar{m} \coqdocvar{t} :
\coqref{ExampleUNWO.Unt}{\coqdocdefinition{U}}$^\coqdocvariable{n}$
\coqdocdefinition{D}$^{\coqexternalref{http://coq.inria.fr/stdlib/Coq.Init.Datatypes}{S}{\coqdocconstructor{S}} \coqdocvariable{m}}$
\coqref{ExampleUNWO.Unt}{\coqdocdefinition{U}}$^{\coqexternalref{http://coq.inria.fr/stdlib/Coq.Init.Datatypes}{S}{\coqdocconstructor{S}}
\coqdocvariable{m}}$
\coqdocvariable{t} \coqref{Rewriting.step}{$\rightarrow_\mathcal{R}$}
\coqref{ExampleUNWO.Unt}{\coqdocdefinition{U}}$^\coqdocvariable{n}$
\coqdocdefinition{D}$^\coqdocvariable{m}$
\coqref{ExampleUNWO.Unt}{\coqdocdefinition{U}}$^\coqdocvariable{m}$
\coqdocvariable{t} :=\coqdoceol
\coqdocindent{1.00em}
\coqref{Rewriting.Step}{\coqdocconstructor{Step}}
\coqdocdefinition{$\rho_1$}
(\coqref{ExampleUNWO.Unt}{\coqdocdefinition{U}}$^\coqdocvariable{n}$
\coqdocdefinition{D}$^\coqdocvariable{m}$ $\Box$)
(\coqref{ExampleUNWO.sigma}{\coqdocdefinition{$\sigma$}}
(\coqref{ExampleUNWO.Unt}{\coqdocdefinition{U}}$^\coqdocvariable{m}$ \coqdocvariable{t}))
\coqdoclemma{\_}
(\coqref{ExampleUNWO.facttermbisUmDSnUSnt}{\coqdoclemma{fact$_\pi^\text{source}$}}
\coqdocvariable{n} \coqdocvariable{m} \coqdocvariable{t})
(\coqref{ExampleUNWO.facttermbisUmDnUnt}{\coqdoclemma{fact$_\pi^\text{target}$}}
\coqdocvariable{n} \coqdocvariable{m} \coqdocvariable{t}).\coqdoceol
\end{coqdoccode}
\end{singlespace}
Generalising these rewrite steps
\coqref{ExampleUNWO.pi}{\coqdocdefinition{$\pi$}}, we construct
the rewrite sequences
\coqref{ExampleUNWO.phia}{\coqdocdefinition{$\varphi_a$}}. In their
recursive definition, the \coqdocdefinition{snoc}
function\footnote{With the
  \coqref{Rewriting.Cons}{\coqdocconstructor{Cons}} constructor, we
  can extend a rewrite sequence with one step at the end. In contrast,
  \coqdocdefinition{snoc} extends a rewrite sequence
  with one step at the start. It is the dual of $1 + \alpha$ on
  ordinals. The type of \coqdocdefinition{snoc}
  is \begin{coqdoccode}\ldots\end{coqdoccode}. TODO: define snoc and
    append in a separate subsection (3.4.2)} is used to
prepend \begin{coqdoccode}(\coqref{ExampleUNWO.pi}{\coqdocdefinition{$\pi$}}
\coqdocvariable{n} \coqdocvariable{m}
\coqdocvariable{t})\end{coqdoccode} to
\begin{coqdoccode}(\coqref{ExampleUNWO.phia}{\coqdocdefinition{$\varphi_a$}}
\coqdocvariable{n} \coqdocvariable{m}
\coqdocvariable{t})\end{coqdoccode}. Doing some arithmetic, we obtain
that these rewrite sequences can be used to define rewrite sequences
\coqref{ExampleUNWO.phib}{\coqdocdefinition{$\varphi_b$}} of a more
useful type.\footnote{TODO: note about \coqdockw{Program}
  construction.}
\begin{singlespace}
\begin{coqdoccode}
\coqdocnoindent
\coqdockw{Fixpoint}
\coqdef{ExampleUNWO.phia}{phia}{\coqdocdefinition{$\varphi_a$}}
\coqdocvar{n} \coqdocvar{m} \coqdocvar{t} :
\coqref{ExampleUNWO.Unt}{\coqdocdefinition{U}}$^\coqdocvariable{n}$
\coqdocdefinition{D}$^\coqdocvariable{m}$
\coqref{ExampleUNWO.Unt}{\coqdocdefinition{U}}$^\coqdocvariable{m}$
\coqdocvariable{t}
\coqref{Rewriting.sequence}{$\rewrites_\mathcal{R}$}
\coqref{ExampleUNWO.Unt}{\coqdocdefinition{U}}$^\coqdocvariable{n}$
\coqdocvariable{t} :=\coqdoceol
\coqdocindent{1.00em}
\coqdockw{match} \coqdocvariable{m} \coqdockw{with}\coqdoceol
\coqdocindent{1.00em}
\ensuremath{|}
\coqdocconstructor{O}
\ensuremath{\Rightarrow}
\coqref{Rewriting.Nil}{\coqdocconstructor{Nil}}
(\coqref{ExampleUNWO.Unt}{\coqdocdefinition{U}}$^\coqdocvariable{n}$
\coqdocvariable{t})\coqdoceol
\coqdocindent{1.00em}
\ensuremath{|}
\coqdocconstructor{S}
\coqdocvar{m} \ensuremath{\Rightarrow}
\coqdocdefinition{snoc} % TODO: define snoc and link to it
(\coqref{ExampleUNWO.pi}{\coqdocdefinition{$\pi$}}
\coqdocvariable{n} \coqdocvariable{m} \coqdocvariable{t})
(\coqref{ExampleUNWO.phia}{\coqdocdefinition{$\varphi_a$}}
\coqdocvariable{n} \coqdocvariable{m} \coqdocvariable{t})\coqdoceol
\coqdocindent{1.00em}
\coqdockw{end}.\coqdoceol
\coqdocemptyline
\coqdocnoindent
\coqdockw{Program Definition} % TODO: here the S(2 n) is not correct (?)
\coqdef{ExampleUNWO.phib}{phib}{\coqdocdefinition{$\varphi_b$}}
\coqdocvar{n} : \coqref{ExampleUNWO.Unt}{\coqdocdefinition{U}}$^\coqdocvariable{n}$
(\coqref{ExampleUNWO.psi'}{\coqdocdefinition{$\du'$}}
\coqdocvariable{n}) \coqref{Rewriting.sequence}{$\rewrites_\mathcal{R}$}
\coqref{ExampleUNWO.Unt}{\coqdocdefinition{U}}$^{\coqexternalref{http://coq.inria.fr/stdlib/Coq.Init.Datatypes}{S}{\coqdocconstructor{S}} \coqdocvariable{n}}$
(\coqref{ExampleUNWO.psi'}{\coqdocdefinition{$\du'$}}
(\coqexternalref{http://coq.inria.fr/stdlib/Coq.Init.Datatypes}{S}{\coqdocconstructor{S}}
\coqdocvariable{n})) :=\coqdoceol
\coqdocindent{1.00em}
\coqref{ExampleUNWO.phia}{\coqdocdefinition{$\varphi_a$}}
\coqdocvariable{n} (\coqexternalref{http://coq.inria.fr/stdlib/Coq.Init.Datatypes}{S}{\coqdocconstructor{S}}
(2
$\times$ \coqdocvariable{n})) (\coqexternalref{http://coq.inria.fr/stdlib/Coq.Init.Datatypes}{S}{\coqdocconstructor{S}} @
\coqref{ExampleUNWO.psi'}{\coqdocdefinition{$\du'$}} (\coqexternalref{http://coq.inria.fr/stdlib/Coq.Init.Datatypes}{S}{\coqdocconstructor{S}}
\coqdocvariable{n})).\coqdoceol
\end{coqdoccode}
\end{singlespace}
We concatenate all rewrite sequences
\coqref{ExampleUNWO.phib}{\coqdocdefinition{$\varphi_b$}} to construct
rewrite sequences from $\du$ to a term that is equal to $U^\omega$ up
to any given depth.
\begin{singlespace}
\begin{coqdoccode}
\coqdocnoindent
\coqdockw{Fixpoint}
\coqdef{ExampleUNWO.phic}{phic}{\coqdocdefinition{$\varphi_c$}}
\coqdocvar{n} : \coqref{ExampleUNWO.psi}{\coqdocdefinition{$\du$}}
\coqref{Rewriting.sequence}{$\rewrites_\mathcal{R}$}
\coqref{ExampleUNWO.Unt}{\coqdocdefinition{U}}$^\coqdocvariable{n}$
(\coqref{ExampleUNWO.psi'}{\coqdocdefinition{$\du'$}}
\coqdocvariable{n}) :=\coqdoceol
\coqdocindent{1.00em}
\coqdockw{match} \coqdocvariable{n} \coqdockw{with}\coqdoceol
\coqdocindent{1.00em}
\ensuremath{|}
\coqdocconstructor{O}
\ensuremath{\Rightarrow}
\coqref{Rewriting.Nil}{\coqdocconstructor{Nil}}
\coqref{ExampleUNWO.psi}{\coqdocdefinition{$\du$}}\coqdoceol
\coqdocindent{1.00em}
\ensuremath{|}
\coqdocconstructor{S}
\coqdocvar{n} \ensuremath{\Rightarrow}
\coqdocdefinition{append} % TODO: define append and link to it
(\coqref{ExampleUNWO.phic}{\coqdocdefinition{$\varphi_c$}}
\coqdocvariable{n})
(\coqref{ExampleUNWO.phib}{\coqdocdefinition{$\varphi_b$}}
\coqdocvariable{n})\coqdoceol
\coqdocindent{1.00em}
\coqdockw{end}.\coqdoceol
\end{coqdoccode}
\end{singlespace}

The definition of the final rewrite sequence
\coqref{ExampleUNWO.phi}{\coqdocdefinition{$\varphi$}} is done by combining
\coqref{ExampleUNWO.phic}{\coqdocdefinition{$\varphi_c$}} with a proof
that the target terms converge to $U^\omega$.
\begin{singlespace}
\begin{coqdoccode}
\coqdocnoindent
\coqdockw{Lemma}
\coqdef{ExampleUNWO.conv}{conv}{\coqdoclemma{conv$_{\varphi_c}$}}
: \coqref{Rewriting.converges}{\coqdocdefinition{converges}}
(\coqdockw{fun} \coqdocvar{n} \ensuremath{\Rightarrow}
\coqref{ExampleUNWO.Unt}{\coqdocdefinition{U}}$^\coqdocvariable{n}$
(\coqref{ExampleUNWO.psi'}{\coqdocdefinition{$\du'$}}
\coqdocvariable{n}))
\coqref{ExampleUNWO.repeatU}{\coqdocdefinition{U$^\omega$}}.\coqdoceol
\coqdocemptyline
\coqdocnoindent
\coqdockw{Definition}
\coqdef{ExampleUNWO.phi}{phi}{\coqdocdefinition{$\varphi$}}
: \coqref{ExampleUNWO.psi}{\coqdocdefinition{$\du$}}
\coqref{Rewriting.sequence}{$\rewrites_\mathcal{R}$}
\coqref{ExampleUNWO.repeatU}{\coqdocdefinition{U$^\omega$}} :=
\coqref{Rewriting.Lim}{\coqdocconstructor{Lim}}
\coqdef{ExampleUNWO.phic}{phic}{\coqdocdefinition{$\varphi_c$}}
\coqdef{ExampleUNWO.conv}{conv}{\coqdoclemma{conv$_{\varphi_c}$}}.\coqdoceol
\coqdocemptyline
\coqdocnoindent
\coqdockw{Lemma}
\coqdoclemma{wf$_\varphi$}
: \coqref{Rewriting.wf}{\coqdocdefinition{wf}}
\coqref{ExampleUNWO.phi}{\coqdocdefinition{$\varphi$}}.\coqdoceol
\end{coqdoccode}
\end{singlespace}

We can prove $\du \rewrites D^\omega$ in a similar way and
conclude by proving two general lemmas.
\begin{singlespace}
\begin{coqdoccode}
\coqdocnoindent
\coqdockw{Lemma}
\coqdoclemma{no\_un$_\mathcal{R}$}
: \ensuremath{\lnot}
\coqref{Rewriting.uniquenormalforms}{\coqdocdefinition{unique\_normal\_forms}}
\coqdocdefinition{$\mathcal{R}$}.\coqdoceol
\coqdocemptyline
\coqdocnoindent
\coqdockw{Lemma} \coqdoclemma{no\_un\_wo}
: \ensuremath{\lnot} \ensuremath{\forall} \coqdocvar{F} \coqdocvar{X}
\coqdocvar{$\mathcal{R}$},\coqdoceol
\coqdocindent{1.00em}
\coqdocdefinition{weakly\_orthogonal}
(\coqdocvar{F} := \coqdocvariable{F}) (\coqdocvar{X} :=
\coqdocvariable{X}) \coqdocvariable{$\mathcal{R}$}
\ensuremath{\rightarrow}
\coqdocdefinition{unique\_normal\_forms}
\coqdocvariable{$\mathcal{R}$}.\coqdoceol
\end{coqdoccode}
\end{singlespace}
