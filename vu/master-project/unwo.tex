\chapter[\texorpdfstring{UN$^\infty$ in Weakly Orthogonal Systems}{UN
  in Weakly Orthogonal Systems}]{\texorpdfstring{Unique Normal Forms
    in\\Weakly Orthogonal Systems}{Unique Normal Forms in Weakly
    Orthogonal Systems}}\label{chap:unwo}

TODO: this intro

Every orthogonal TRS exhibits the infinitary unique normal forms
(UN$^\infty$) property. We might expect this property to generalize to
weakly orthogonal systems. However, this does not turn out to be the
case.


\section{A Counterexample}

We describe a simple counterexample showing that weak orthogonality
does not imply the UN$^\infty$ property.

We work in a signature with unary function symbols $P$ and $S$. In the
notation of terms, we omit the brackets around arguments and assume
right-associativity of function symbol application, e.g.\ writing
$PSx$ for $P(S(x))$. A notation for finite repetitions of a function
symbol $f$ terminated by a term $t$ is defined by
\begin{inparaenum}[(i)]
\item $f^0 t = t$ and
\item $f^{n+1} = ff^nt$.
\end{inparaenum}
The infinite nesting $fff \ldots$ of $f$ is written $f^\omega$.
% TODO: nesting vs repetition

Consider the TRS consisting of the two left-linear rewrite rules
$\rho_1$ and $\rho_2$:
\begin{align*}
  \rho_1 \, : \, PSx \to x \qquad \qquad \qquad
  \rho_2 \, : \, SPx \to x
\end{align*}
This system has two critical pairs $\langle Px, Px \rangle$ and
$\langle Sx, Sx \rangle$, both of which are trivial, establishing
weak orthogonality. The infinite term $\psi = P^1 S^2 P^3 S^4 \ldots$
has two normal forms. It rewrites to $S^\omega$ in $\omega$ many
$\rho_1$-steps and to $P^\omega$ in $\omega$ may $\rho_2$-steps.


\section{A Mechanic Formalization}

Implementation in \Coq.
