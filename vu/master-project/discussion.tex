\chapter{Discussion and Conclusion}\label{chap:discussion}


\section{Design Choices}

One-hole contexts vs multi-hole contexts (possible using extended signature).

Casteran's ordinals in Veblen nf vs Mamane's set-theoretic ordinals vs Brouwer
ordinals.

Rewriting sequences as functions from ordinals to steps vs inductive
definition.

Bisimilarity in steps

The embedding relation and order on ordinals by Hancock, are there other
choices?


\subsection{The Positivity Condition}\label{sub:positivity}

TODO: Sigma type in Lim constructor vs two separate functions

Using a $\Sigma$-type
\begin{singlespace}
\begin{coqdoccode}
\coqdocindent{1.00em}
\ensuremath{|} \coqdocconstructor{Lim}   :
\ensuremath{\forall} \coqdocvar{s} \coqdocvar{t},
(\coqexternalref{http://coq.inria.fr/stdlib/Coq.Init.Datatypes}{nat}{\coqdocinductive{nat}}
\ensuremath{\rightarrow} \{ \coqdocvar{t'} \& \coqdocvariable{s}
$\twoheadrightarrow_\mathcal{R}$
\coqdocvariable{t'}\}) $\rightarrow$
\coqdocvariable{s} $\twoheadrightarrow_\mathcal{R}$
\coqdocvariable{t}\coqdoceol
\end{coqdoccode}
\end{singlespace}


\subsection{Guardedness}\label{sub:guardedness}

Objects in a coinductive type may be infinite (i.e.\ contain an infinite
amount of constructors). However, in order to guarantee productivity,
definitions of such objects are required by \Coq to be in \emph{guarded}
form. A corecursive definition in guarded form satisfies two (syntactical)
conditions. First, every corecursive call must occur inside at least one
constructor (of the same coinductive type). Second, every corecursive call may
only occur inside abstractions or constructors (of the same coinductive
type).\footnote{To be more precise, the corecursive call is also allowed to
  occur inside \coqdockw{match} constructs and other corecursive definitions.}

TODO: additional remarks on productivity and guardedness restriction?

In the \coqref{Term.term}{\coqdocinductive{term}} definition, we used a vector
type, parameterized by the type of its element and its size. Naturally, one
would implement a vector type in \Coq inductively, as for example has been
done in the standard library.
\begin{singlespace}
\begin{coqdoccode}
\coqdocnoindent
\coqdockw{Inductive} \coqdef{Bvector.vector}{vector}{\coqdocinductive{vector}}
(\coqdocvar{A} : \coqdockw{Type}) :
\coqexternalref{http://coq.inria.fr/stdlib/Coq.Init.Datatypes}{nat}{\coqdocinductive{nat}}
\ensuremath{\rightarrow} \coqdockw{Type} :=\coqdoceol
\coqdocindent{1.00em}
\ensuremath{|} \coqdef{Bvector.Vnil}{Vnil}{\coqdocconstructor{Vnil}}  :
\coqref{Bvector.vector}{\coqdocinductive{vector}} \coqdocvariable{A} 0\coqdoceol
\coqdocindent{1.00em}
\ensuremath{|} \coqdef{Bvector.Vcons}{Vcons}{\coqdocconstructor{Vcons}} :
\coqdocvariable{A} \ensuremath{\rightarrow} \ensuremath{\forall} \coqdocvar{n},
\coqref{Bvector.vector}{\coqdocinductive{vector}} \coqdocvariable{A}
\coqdocvariable{n} \ensuremath{\rightarrow}
\coqref{Bvector.vector}{\coqdocinductive{vector}} \coqdocvariable{A}
(\coqexternalref{http://coq.inria.fr/stdlib/Coq.Init.Datatypes}{S}{\coqdocconstructor{S}}
\coqdocvariable{n}).\coqdoceol
\end{coqdoccode}
\end{singlespace}

Now consider the following trivial example of a basic operation on terms by
corecursive traversal.
\begin{singlespace}
\begin{coqdoccode}
\coqdocnoindent
\coqdockw{CoFixpoint} \coqdef{Term.id}{id}{\coqdocdefinition{id}}
(\coqdocvar{t} : \coqref{Term.term}{\coqdocinductive{term}}) :
\coqref{Term.term}{\coqdocinductive{term}} :=\coqdoceol
\coqdocindent{1.00em}
\coqdockw{match} \coqdocvariable{t} \coqdockw{with}\coqdoceol
\coqdocindent{1.00em}
\ensuremath{|} \coqref{Term.Var}{\coqdocconstructor{Var}} \coqdocvar{x}
\ensuremath{\Rightarrow} \coqref{Term.Var}{\coqdocconstructor{Var}}
\coqdocvariable{x}\coqdoceol
\coqdocindent{1.00em}
\ensuremath{|} \coqref{Term.Fun}{\coqdocconstructor{Fun}} \coqdocvar{f}
\coqdocvar{args} \ensuremath{\Rightarrow}
\coqref{Term.Fun}{\coqdocconstructor{Fun}} \coqdocvariable{f}
(\coqdocdefinition{vmap} \coqref{Term.id}{\coqdocdefinition{id}}
\coqdocvariable{args})\coqdoceol
\coqdocindent{1.00em}
\coqdockw{end}.\coqdoceol
\end{coqdoccode}
\end{singlespace}
This definition is ill-formed, since the corecursive call to
\coqref{Term.id}{\coqdocdefinition{id}} is not guarded.\footnote{The call to
  \coqref{Term.id}{\coqdocdefinition{id}} is hidden inside
  \coqdocdefinition{vmap}, which is defined by recursion on the vector
  \coqdocvariable{args}.}
We define a recursive type of vectors as an alternative to the inductive type:
\begin{singlespace}
\begin{coqdoccode}
\coqdocnoindent
\coqdockw{Inductive} \coqdef{Vector.Fin}{Fin}{\coqdocinductive{Fin}} :
\coqexternalref{http://coq.inria.fr/stdlib/Coq.Init.Datatypes}{nat}{\coqdocinductive{nat}}
\ensuremath{\rightarrow} \coqdockw{Type} :=\coqdoceol
\coqdocindent{1.00em}
\ensuremath{|} \coqdef{Vector.First}{First}{\coqdocconstructor{First}} :
\ensuremath{\forall} \coqdocvar{n}, \coqref{Vector.Fin}{\coqdocinductive{Fin}}
(\coqexternalref{http://coq.inria.fr/stdlib/Coq.Init.Datatypes}{S}{\coqdocconstructor{S}}
\coqdocvariable{n})\coqdoceol
\coqdocindent{1.00em}
\ensuremath{|} \coqdef{Vector.Next}{Next}{\coqdocconstructor{Next}}  :
\ensuremath{\forall} \coqdocvar{n}, \coqref{Vector.Fin}{\coqdocinductive{Fin}}
\coqdocvariable{n} \ensuremath{\rightarrow}
\coqref{Vector.Fin}{\coqdocinductive{Fin}}
(\coqexternalref{http://coq.inria.fr/stdlib/Coq.Init.Datatypes}{S}{\coqdocconstructor{S}}
\coqdocvariable{n}).\coqdoceol
\coqdocemptyline
\coqdocnoindent
\coqdockw{Definition}
\coqdef{Vector.vector}{vector}{\coqdocdefinition{vector}} (\coqdocvar{A} :
\coqdockw{Type}) (\coqdocvar{n} :
\coqexternalref{http://coq.inria.fr/stdlib/Coq.Init.Datatypes}{nat}{\coqdocinductive{nat}})
:= \coqref{Vector.Fin}{\coqdocinductive{Fin}} \coqdocvariable{n}
\ensuremath{\rightarrow} \coqdocvariable{A}.\coqdoceol
\end{coqdoccode}
\end{singlespace}
% TODO: explain this vector type
This makes for a definition of \coqref{Vector.vmap}{\coqdocdefinition{vmap}}
that is just an abstraction, and therefore solves the guardedness problem in
\coqref{Term.id}{\coqdocdefinition{id}}.
\begin{singlespace}
\begin{coqdoccode}
\coqdocnoindent
\coqdockw{Definition} \coqdef{Vector.vmap}{vmap}{\coqdocdefinition{vmap}}
\coqdocvar{A} \coqdocvar{B} (\coqdocvar{f} :
\coqdocvariable{A} \ensuremath{\rightarrow} \coqdocvariable{B}) \coqdocvar{n}
: \coqref{Vector.vector}{\coqdocdefinition{vector}} \coqdocvariable{A}
\coqdocvariable{n} \ensuremath{\rightarrow}
\coqref{Vector.vector}{\coqdocdefinition{vector}} \coqdocvariable{B}
\coqdocvariable{n} :=\coqdoceol
\coqdocindent{1.00em}
\coqdockw{fun} \coqdocvar{v} \coqdocvar{i} \ensuremath{\Rightarrow}
\coqdocvariable{f} (\coqdocvariable{v} \coqdocvariable{i}).\coqdoceol
\end{coqdoccode}
\end{singlespace}


\section{Discussion}

Bvector vs Vector: we did not really hit the guardedness restriction, so we
could have used inductive Bvector. But it spells trouble later.


\section{Conclusions}
