\documentclass[11pt]{article}
\usepackage[dutch]{babel}
\usepackage{a4}
\usepackage{latexsym}
\usepackage[
    colorlinks,
    pdftitle={Filosofie en Etiek van de Techniek college 3},
    pdfsubject={Techniques}
    pdfauthor={Martijn Vermaat}
]{hyperref}

\title{Commentaar voor FeEvdT college 3}
\author{
    Martijn Vermaat, groep 11
}
\date{20 januari 2004}

\begin{document}
\maketitle

\begin{quote}
Commentaar op ``Techniques'' (Jacques Ellul) uit ``The Technological Society'' door Martijn Vermaat voor het college ``Techniek als kenmerk van de moderne samenleving''.
\end{quote}

Waar het Jacques Ellul in ``Techniques'' voornamelijk om gaat is het vinden van een definitie voor het woord `techniek' (deze vertaling van zijn `technique' houdt goed stand). Hij begint echter met het rechtzetten van veelgemaakte misvattingen over dit woord.

De verwarring van `techniek' met `machines' wordt aangekaart en het enorme verschil tussen beiden duidelijk gemaakt. Ook is `techniek' volgens hem niet een applicatie van `science'\footnote{`Science' is in dit verband voor de niet Amerikaanse of Engelse lezer een erg vervelend woord, wellicht is de vertaler van het boek hier schuldig aan. De betekenis van `science' is in het Engels vanuit de Verlichting niet direct te vertalen naar het Nederlandse `wetenschap', maar slaat vooral op de b\`eta wetenschappen.}, het raakvlak tussen technische wetenschap en werkelijkheid, zoals velen het zien. Verder betoogt Ellul dat waar sommigen het woord `organisatie' gebruiken en het vervolgens tegenover `techniek' zetten, ze in werkelijkheid ook `techniek' bedoelen en dus ook gewoon dat woord ervoor zouden moeten gebruiken.

Al deze punten lijken misschien wat gericht op kleine verschillen van meningen op het gebied van taal, maar het illustreert juist goed wat Ellul precies bedoelt met `techniek'--ik kan hem er slechts groot gelijk in geven. De definitie waar hij uiteindelijk naartoe wil ziet `techiek' in de breedst mogelijke zin; een manier van handelen.

Het is vervelend dat Ellul na deze abstractie technieken alsnog in enkele zinnen wil opdelen in vijf strak begrensde hokjes. Is het niet een beetje na\"ief (een punt waarop hij juist anderen graag becritiseert) om te denken dat daar alle verschillende soorten `techniek' in gevangen zitten? Na het zorgvuldig zoeken naar en formuleren van een abstracte definitie zou ik zeggen dat het zonde is deze abstractie direct weer op een snelle (onzorgvuldige?) manier weer weg te nemen.

\paragraph{}

Na deze inhoudelijke beschouwing tot slot een opmerking over de vorm van deze tekst. Misschien ligt het aan de vertaler (Ellul is een Fransman, ik las de tekst in het Engels), misschien aan de leeftijd van Ellul, of misschien aan iets geheel anders. In ieder geval lijkt Ellul een stuk uitgeprokener dan in ``The Search for Ethics in a Technicist Society'' (welke zo'n 25 jaar later geschreven is). De wat minder gerelativeerde manier van schrijven maakt het voor de lezer in ieder geval een stuk gemakkelijker te volgen waar hij naar toe wil.

\end{document}
