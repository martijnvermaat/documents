\documentclass[11pt]{article}
\usepackage[dutch]{babel}
\usepackage{a4}
\usepackage{latexsym}
\usepackage[
    colorlinks,
    pdftitle={Filosofie en Etiek van de Techniek college 1},
    pdfsubject={Engineering Ethics and Political Imagination},
    pdfauthor={Martijn Vermaat}
]{hyperref}

\title{Commentaar voor FeEvdT college 1}
\author{
    Martijn Vermaat, groep 11
}
\date{12 januari 2004}

\begin{document}
\maketitle

\begin{quote}
Commentaar op ``Engineering Ethics and Political Imagination'' (Winner, L.)
door Martijn Vermaat voor het college ``Techniek, ethiek en politiek''.
\end{quote}

In ``Engineering Ethics and Political Imagination'', heeft Winner sterke
kritiek op bestaande manieren van lesgeven in ethiek aan technische studenten.
Het centrale punt waar hij deze kritiek op baseert is het beperkte zicht op de
technologische cultuur (en de positie van de mens daarin) in de traditionele
manier van lesgeven.

Natuurlijk heeft Winner gelijk wanneer hij betoogt dat studenten nu en vooral
later keuzes hebben, meer keuzes dan het bedrijfsleven en de maatschappij hen
doen geloven. En natuurlijk moet er in de techniek stilgestaan worden bij het
nut van het inslaan van nieuwe wegen alvorens de precieze route over deze
wegen te gaan bepalen.

In het kort valt Winner kortzichtigheid in de ethiek van de techniek aan en
daarin kunnen we hem alleen maar gelijk geven---als deze kortzichtigheid ook
daadwerkelijk aanwezig is! Het is aardig om slechte zaken aan de kaak te
stellen, maar om daarbij nauwelijks in te gaan op de vraag of deze zaken in de
praktijk uberhaupt veel voorkomen is op zich een vorm van kortzichtigheid.
Want is het ook daadwerkelijk zo dat afgestudeerden hun ziel als blinden aan
de technische maatschappij verkopen? Voor het beantwoorden van deze vraag is
het niet voldoende slechts te kijken naar lessen ethiek op technische
opleidigen, maar moet er ook stilgestaan worden bij het vermogen na te denken
van de studenten zelf en bij invloeden van buiten de studie, zoals sociale
contacten en de media. Misschien is het allemaal niet zo erg als Winner
impliciet stelt?

Bovendien valt Winner enkele zeer concrete voorbeelden van lesgeven aan, zoals
in de vierde alinea, met als toevoeging ``I have exaggerated, but not by very
much''. Misschien overdreef hij daar inderdaad niet sterk, maar is de door hem
geleverde kritiek op dit voorbeeld sowieso terecht? Dergelijke case-studies
zijn geisoleerd niet voldoende in een degelijke les ethiek, maar men moet
ergens beginnen. Wat is er mis mee wanneer ze binnen de les aangevuld worden
met voldoende bredere inzichten?

\end{document}
