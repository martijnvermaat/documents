\documentclass[11pt]{article}
\usepackage[dutch]{babel}
\usepackage{a4}
\usepackage{latexsym}
\usepackage[
    colorlinks,
    pdftitle={Filosofie en Etiek van de Techniek college 2},
    pdfsubject={The Religion of Technology}
    pdfauthor={Martijn Vermaat}
]{hyperref}

\title{Commentaar voor FeEvdT college 2}
\author{
    Martijn Vermaat, groep 11
}
\date{12 januari 2004}

\begin{document}
\maketitle

\begin{quote}
Commentaar op ``The Religion of Technology'' (David F. Noble) door Martijn Vermaat voor het college ``Techniek en religie''.
\end{quote}

Centraal in Noble's betoog ``The Religion of Technology'' staat de verbintenis
tussen technische wetenschappen en religie. In het tweede deel, ``Paradise
Restored'', schetst hij een uitgebreid beeld van techniek in de zestiende
eeuw, waarin vooruitgang direct gerelateerd was aan religie en zelfs
behoorlijk concrete passages uit de Bijbel. Dit ter ondersteuning van het
eerste deel, ``Technology and Religion'', waarin Noble deze relatie doortrekt
naar de moderne tijd en daarmee wil zeggen dat religie altijd de drijfveer van
technologische vooruitgang was, is en zal blijven.

Wat Noble echter nalaat, is een degelijke behandeling van religie en techniek
zoals deze op dit moment met elkaar in verhouding staan. Alle vormen van
technologische vooruitgang en de relatie hiervan tot religie in tijden voor
het jaar 1000 worden bovendien compleet genegeerd.

Storend is ook het gegeven dat Noble individuen uit de zestiende eeuw, totaal
begeesterd door religie, als voorbeeld neemt. Deze voorbeelden zijn in de
geschiedenis slechts incidenten en alleen in die eeuwen gewoon. Vervolgens
projecteert hij de drijfveren van deze individuen op de complete maatschappij
van de huidige eeuw en gaat daarbij voorbij aan het feit dat groepen mensen
anders bewegen dan individuen. Individuen zoals door hem genoemd van enkele
eeuwen geleden zijn er tegenwoordig niet meer, of zijn slechts dat: een
individu. Probeer eens een wetenschapper te vinden die zijn werk niet net zo
benaderd als een willekeurige arbeider dat doet, maar als de grote verlossing
van de mensheid---ze bestaan niet.

Als Noble dan ook voor het gemak alle stromingen van `zijn' geloof (het
Christendom) op een hoop gooit is het duidelijk: zijn idee\"en zijn slechts
gestoeld op exemplarische observaties van een beperkte wereld in een beperkt
tijdvak.

\end{document}
