\documentclass[11pt]{article}
\usepackage[dutch]{babel}
\usepackage{a4}
\usepackage{latexsym}
\usepackage[
    colorlinks,
    pdftitle={Filosofie en Etiek van de Techniek college 5},
    pdfsubject={Techniek, modernisering en globalisering}
    pdfauthor={Martijn Vermaat}
]{hyperref}

\title{Commentaar voor FeEvdT college 5}
\author{
    Martijn Vermaat, groep 11
}
\date{1 februari 2004}

\begin{document}
\maketitle

\begin{quote}
Commentaar op ``The Archaeology of the Development Idea'' (Sachs, W.) uit ``Planet Dialectics: Explorations in Environment and Development'' en ``The Pressure to Modernize and Globalize'' (Norberg-Hodge, H.) uit ``The Case Against the Global Economy: And for a Turn Toward the Local'' door Martijn Vermaat voor het college ``Techniek, modernisering en globalisering''.
\end{quote}

Als het gaat om globaliseringsvraagstukken, staan Sachs en Norberg-Hodge aan dezelfde kant. Beiden zijn ze van mening dat de economische maatschappij kunstmatig en geforceerd doorgevoerd is in Derde Wereld-landen. Waar mensen eeuwenlang zonder de moderne Westerse normen en waarden hebben kunnen overleven sloeg de modernisering in als een bom. Men was er niet op voorbereid en ze hebben er vooral zelf niet om gevraagd.

Norberg-Hodge gaat wat dit betreft iets verder--waarschijnlijk ziet zij traditionele culturen het liefst tot in de oneindigheid afgezonderd van het moderne Westen. Sachs heeft vooral kritiek op de manier waarop traditionele culturen kennis hebben moeten maken met het Westen. Als een ware archeoloog betoogt hij nauwgezet hoe Derde Wereld-landen vanaf het colonialisme in de Amerikaanse val zijn gelopen, hoe er door het Westen met traditionele culturen gespeeld is.

Waar ze het beiden over eens zijn is dat de schuld bij het Westen ligt. De moderne samenleving heeft geen respect getoond voor de traditionele culturen en ze overrompeld door ze gedwongen deel te laten nemen aan de globale economie. Norberg-Hodge gaat hier niet diep in op de oorzaken, maar houdt het simpelweg op winstbelang van het Westen. Sachs daarentegen schetst nauwkeurig hoe dit volgens hem de afgelopen eeuw zo gekomen is. Het Westen is na\"ief geweest door blind te vertrouwen op de economische samenleving. Door andere culturen te beoordelen volgens economische normen en waarden leken ze ongelooflijk arm en het was de taak van het Westen om ze te ontwikkelen (en hiermee rijker te maken).

De grootste boosdoener zien Sachs en Norberg-Hodge in de Westerse manier van denken, de economische belangen die een grote rol spelen in de samenleving. Techniek is \'e\'en van de neven-effecten waar de traditionele culturen mee te maken krijgen. Sachs wijdt er behoorlijk over uit hoe de Derde Wereld met de techniek als het ware een Paard van Troje binnenhaalt en waarom Westerse techniek niet `gewoon werkt' in andere culturen. Norberg-Hodge noemt techniek vooral in voorbeelden en als hulpmiddel van het Westen om de kei-harde globale economie mogelijk te maken.

\paragraph{}

Beide schrijvers zijn het er over eens. Hoewel er nu veel mis gaat in traditionele culturen zijn zij daarvan niet zelf de schuldige, maar had het Westen de globale mars niet in moeten zetten. De moderne samenleving, met de Verenigde Staten voorop, heeft geen rekening gehouden de gevolgen van de globalisering voor wat we nu de Derde Wereld noemen. Sachs wil hier bovendien aan toevoegen dat het Westen hierdoor nu op de blaren moet zitten en vooral veel last heeft van wat er in de Derde Wereld op dit moment gebeurt.

\end{document}
