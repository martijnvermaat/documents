\documentclass[11pt]{article}
\usepackage[dutch]{babel}
\usepackage{a4}
\usepackage{latexsym}
\usepackage[
    colorlinks,
    pdftitle={Filosofie en Etiek van de Techniek college 4},
    pdfsubject={Modernity as the Universalization of Heresy}
    pdfauthor={Martijn Vermaat}
]{hyperref}

\title{Commentaar voor FeEvdT college 4}
\author{
    Martijn Vermaat, groep 11
}
\date{27 januari 2004}

\begin{document}
\maketitle

\begin{quote}
Commentaar op ``Modernity as the Universalization of Heresy'' (Berger, P.L.) uit ``The Heritical Imperative: Contemporary Possibilities of Religious Affirmation'' door Martijn Vermaat voor het college ``Pluralisme als kenmerk van moderne samenleving''.
\end{quote}

Het centrale thema in deze tekst van Berger is het kenmerken van moderniteit als de vervanging van het lot door de keuze. Beiden kunnen we niet absoluut zien--ook v\'o\'or de moderne samenleving waren er keuzes en ook nu bestaat er een lot--, maar als waarden die in een spectrum duidelijk uit elkaar liggen. De technologie ziet hij als \'e\'en van de belangrijkste drijfveren van de moderne situatie.

In eerste instantie lijken de keuzes die we tegenwoordig hebben een grote bevrijding te bieden. In tegenstelling tot de meeste mensen in de traditionele samenleving zitten we niet vast aan een bepaald geloof, aan een bepaalde stijl van leven en aan een bepaalde manier van denken. We worden niet louter gevormd door een eenduidige omgeving, maar komen in aanraking met zeer veel verschillende denkbeelden en kunnen van ieder kiezen wat ons aanspreekt.

Wat we echter niet over het hoofd moeten zien is de enige keuze die we hiermee juist verliezen: \emph{De keuze om te kiezen}. Konden mensen in vroeger tijden kiezen zich mee laten voeren met de gevestigde orde (`go with the flow') en kiezen voor iets anders; in de moderne samenleving hebben we deze keuze niet meer omdat `de gevestigde orde' in feite niet meer bestaat. We worden verplicht een keuze te maken tussen ordes die ieder even afwijkend zijn.

Deze verplichting te kiezen beperkt zich niet alleen tot het `kiezen' van een geloof, maar manifesteert zich voortdurend in het dagelijks leven. We kunnen kiezen wat we in onze vrije tijd doen, we kunnen kiezen wat we eten, etcetera, etcetera. Is een mens hier mentaal eigenlijk wel tegen bestand? Zijn we misschien niet veel gelukkiger zonder al deze keuzes, al lijken ze ons nog zo `vrij' te maken?

\end{document}
