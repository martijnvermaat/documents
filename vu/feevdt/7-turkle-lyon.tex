\documentclass[11pt]{article}
\usepackage[dutch]{babel}
\usepackage{a4}
\usepackage{latexsym}
\usepackage[
    colorlinks,
    pdftitle={Filosofie en Etiek van de Techniek college 8},
    pdfsubject={Internet en ethiek}
    pdfauthor={Martijn Vermaat}
]{hyperref}

\title{Commentaar voor FeEvdT college 8}
\author{
    Martijn Vermaat, groep 11
}
\date{23 februari 2004}

\begin{document}
\maketitle

\begin{quote}
Commentaar op ``Introduction: Identity in the Age of the Internet'' (Turkle, S.) uit ``Life on the Screen: Identity in the Age of the Internet'' door Martijn Vermaat voor het college ``Internet en ethiek''.
\end{quote}


In dit inleidend hoofdstuk legt Turkle de lezer uit dat het internet (en de computer in het algemeen) een compleet andere en steeds belangrijkere rol gaan spelen in de samenleving.
Was de computer in eerste instantie slechts een hulpmiddel bij ons dagelijks werk, nu zien we dat het gebruik van de computer hier ver voorbij gegroeid is.
Hiermee samen hangt de verschuiving van een modernistische visie op de computer naar een postmodernistische visie op de computer.
Origineel dacht men immers over de computer als een grote rekenmachine, de werking ervan was te ontrafelen en bovenal deterministisch.
Dat is tegenwoordig radicaal veranderd.
Men ziet ondeterministisch gedrag in de computer en stelt zich voor dat de computer typisch menselijk gedrag zou kunnen voortbrengen in de toekomst--of misschien zelfs nu al.

Turkle legt de nadruk op het simulerend gebruik van computers nu, tegenover het calculerend gebruik van computers vroeger.
Internet speelt hierbij een zeer belangrijke rol.
Als duidelijke voorbeelden noemt Turkle online simulatie spellen: spelers van over de hele wereld begeven zich via de computer met elkaar in een virtuele wereld, die de echte wereld simuleert.
Dit fenomeen blijkt zo krachtig te zijn dat het een grote invloed heeft op de mentale gesteldheid van de spelers.
De virtuele werelden worden soms zelfs gelijkwaardig geacht aan de fysieke wereld--emoties en sociale netwerken zijn er bijvoorbeeld niet vreemd.

Dit vervaagt de grens tussen mens en machine.
De mens groeit naar de machine (hierin ziet Turkle het modernisme) en de machine groeit naar de mens (hierin ziet Turkle het postmodernisme).
En inderdaad, zeker nu (al negen jaar nadat Turlke dit schreef), zien we dat mens en machine door elkaar gaat lopen.
Communicatie--een typisch menselijke bezigheid--gebeurt meer en meer slechts via machines (chatten, telefoneren, e-mailen).
Sterker nog, wanneer we via een machine communiceren wordt het verschil tussen mens-mens communicatie en mens-machine communicatie steeds onduidelijker.
Op chatkanalen communiceren mensen bijvoorbeeld niet alleen met andere mensen, maar ook met machines (bots) en \emph{op precies dezelfde manier}, via dezelfde interface.
Onderscheid kunnen we dan alleen nog maken door een stukje tekst op menselijkheid te beoordelen, iets dat steeds moeilijker kan worden, zeker in korte gesprekken.

In de praktijk zien we dat mensen er juist steeds minder moeite mee hebben dat deze grens verdwijnt.
Wat maakt het uit of je tegen een machine of tegen een mens praat?
Als je tevreden bent met de antwoorden--of liever met het hele gesprek--, waarom zou je je daar dan nog druk over maken?
Op deze manier stijgt de computer naar een menselijk niveau.
Verwachtingen van de computer worden steeds hoger.
De tendens is duidelijk dat wat eerst voor onmogelijk gehouden wordt, vervolgens als mogelijk bewezen of geloofd wordt.
Wat gebeurt er als de computer menselijker wordt dan de mens zelf?


\end{document}
