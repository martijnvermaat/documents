\documentclass[11pt]{report}
\usepackage[dutch]{babel}
\usepackage{a4}
\usepackage{latexsym}
\usepackage[
        colorlinks=true,
        linkcolor=zwart,
        citecolor=zwart,
        pdftitle={xxx},
        pdfsubject={xxx},
        pdfauthor={Martijn Vermaat}
]{hyperref}
\usepackage{graphicx}
\usepackage{color}
\definecolor{zwart}{rgb}{0,0,0}

\title{Technologisch determinisme \\ \small{Nachtmerrie, droom, of werkelijkheid?}}
\author{
    Martijn Vermaat \\ mvermaat@cs.vu.nl
}
\date{Amsterdam, Vrije Universiteit, 29 februari 2004}

\begin{document}

\maketitle


\begin{abstract}
Technologische ontwikkelingen beschouwen we in de westerse samenleving als een ideaal. De weg naar het paradijs is geplaveid met techniek. Maar zijn we zelf de drijvende kracht achter de techniek? Ontwikkeld de techniek zich zoals wij willen, of hebben we eigenlijk niets te zeggen en hollen we er maar achteraan? Dit is een belangrijke vraag, want het is duidelijk dat de invloed van techniek op de samenleving enorm is. Het wordt steeds duidelijker dat de samenleving gevormd wordt door technologie. Is dat de keuze die we gemaakt hebben?
\end{abstract}



\chapter{De invloed van technologie}

Er zijn in de westerse wereld nog maar weinig mensen die niet kunnen lezen. Sterker nog, de meesten van ons maken vrijwel de hele dag door gebruik van deze vaardigheid. We kunnen ons nog maar moeilijk een wereld voorstellen zonder letters. De invloed is groter en gaat verder dan we ons misschien realiseren.

\paragraph{}

De westerse maatschappij is volledig afhankelijk van het schrift. Bij alles wat gebeurt bestaan geschreven regels, beschrijvingen en adviezen. In het uitoefenen van ons dagelijks werk worden we bijgestaan door stapels papier. Zou van de ene op de andere dag niemand meer een woord kunnen lezen, dan breekt er een totale chaos uit. Niet alleen zijn we afhankelijk van tekst, onze persoonlijkheid wordt erdoor be\"invloed. Zonder de stapels boeken die we lezen zou het onmogelijk zijn de hoevelheid kennis te vergaren als we nu doen. We vormen onze meningen voor een groot deel door die van anderen te lezen. En andersom--anderen onze mening opdringen was nog nooit zo makkelijk.

Maar de invloed van tekst gaat nog verder dan onze gedachten. Fysiek passen we ons aan aan het leven in een wereld waarin lezen en schrijven gemeengoed zijn. Onze ogen zijn getraind in het herkennen van kleine patronen. Ook de manier waarop we denken is niet meer dezelfde. Tijdens gesprekken wordt door de \'e\'en maar al te vaak gevraagd of de ander die opmerking nog even op papier kan zetten zodat die nog even rustig na te lezen is. We zijn gewend informatie al lezende tot ons te nemen, waarbij we zelf ons tempo kunnen bepalen. Een korte afleiding is niet fataal--maar in een conventioneel gesprek is het niet wenselijk om af en toe die laatste zin nog even te laten herhalen.


\section{De technologie bepaalt}

De technologie bepaalt de aard van de samenleving. Het schrift is hier slechts een van de vele voorbeelden van. Niet alleen de laatste twee eeuwen, tijdens de industri\"ele revolutie en in het zogenaamde `informatie-tijdperk', maar vanaf het allereerste moment dat de mens zich bezig ging houden met techniek. Vanaf de agrarische revolutie bepaalde de landbouw (een techniek) hoe de samenleving eruit zag. Het gebruik van planten als medicijnen (een techniek) heeft nog eerder op veel samenlevingen invloed gehad. Wat betreft de invloed van techniek is er niets veranderd sinds we allemaal een computer op ons bureau hebben staan.

\paragraph{}

Een ander voorbeeld van een belangrijke technologie is televisie. De komst van de televisie heeft voor veel sociale veranderingen gezorgd. Een groot deel van de informatie die we dagelijks tot ons nemen komt tegenwoordig via dit medium. We kunnen vrijwel \emph{realtime} zien wat andere mensen meemaken, soms duizenden kilometers ver weg. De televisie maakt het mogelijk tegelijkertijd vele duizenden mensen aan te spreken. Informatievoorziening is hierdoor, na de komst van de radio, opnieuw in een andere dimensie geraakt.

De televisie heeft ook ons denkpatroon veranderd. We zijn gewend veel informatie in erg kleine brokjes snel achter elkaar te verwerken. Bovendien heeft televisie voor een \emph{push}-mentaliteit gezorgd: de informatie wordt ons letterlijk via het toestel gevoerd. Dit in tegenstelling tot de \emph{pull}-mentaliteit die nodig is voor het verzamelen van informatie uit gesprekken of teksten.

\paragraph{}

Het is dus duidelijk dat de aard van de samenleving door technologie bepaald wordt. Andere voorbeelden zijn de uitvinding van het schip, de uitvinding van de auto en de komst van electriciteitsnetwerken. In alles wat we doen hebben we ons aangepast aan de technieken waarmee we leven. We zijn afhankelijk van de techniek, we worden be\"invloed door te techniek en veranderen door de techniek.

Als de techniek zo machtig lijkt te zijn, is het interessant ons af te vragen waardoor de techniek gestuurd wordt. Bepaalt de mens waar de techniek heen gaat? Zijn wij het zelf die voor het zeggen hebben hoe de techniek zich ontwikkelt en dus ook hoe de samenleving er in de toekomst uit zal zien?



\chapter{Is de techniek autonoom?}


\section{Het grotere geheel zien}

Is de techniek autonoom? Om deze vraag te beantwoorden is een brede visie nodig. We moeten kijken naar wat de geschiedenis ons kan leren en wat er over heel de wereld gebeurt. Het beantwoorden voor het individuele geval is niet zo interessant: het is duidelijk dat iemand voor zichzelf, zij het met moeite, de invloed van technologie in zijn of haar leven min of meer kan bepalen. Deze mensen zijn er, maar ze zijn schaars en over het algemeen ge\"isoleerd van de samenleving. Laten we ons daarom beperken tot het groter geheel, in tijd en in sociaal opzicht.

\paragraph{}

Als we op zoek gaan naar duidelijke voorbeelden waarin techniek bestuurbaar lijkt, komen we uiteraard de Amish tegen. De Amish lijken er min of meer in te slagen moderne technieken buiten hun maatschappij te houden. Zou het mogelijk zijn als een collectief van de hele wereldbevolking hetzelfde te bereiken als de Amish? Waarschijnlijk niet. Er mag toch niet over het hoofd gezien worden dat ze slechts een zeer selecte groep van de wereldbevolking uitmaken en het is onwaarschijnlijk dat ze altijd blijven bestaan--velen van hen zwichtten reeds voor de moderne techniek.

\paragraph{}

Een ander voorbeeld is China rond het begin van de vijftiende eeuw. China liep op technologisch gebied ver voor op de rest van de wereld, totdat de staat relatief plotseling bepaalde dat de toekomst van China niet in verdere technologische ontwikkelingen lag. De techniek lijkt hier zeker niet autonoom geweest te zijn, als een kleine groep mensen de ontwikkelingen werkelijk een halt toe kon roepen.

Echter stonden, over heel de wereld gezien, de technologische ontwikkelingen niet stil. China mocht dan tijdelijk pas op de plaats maken, maar voor andere landen gold dit niet. Bovendien zou het slechts een kwestie van tijd zijn voor China weer aan zou haken.

\paragraph{}

Het lijkt het er op dat de techniek uiteindelijk, als we verder kijken, het laatste woord heeft. Individuen en kleine groepen kunnen met moeite stil blijven staan--en dat slechts tijdelijk. Het \emph{terugdraaien} van technologische ontwikkelingen op grote schaal lijkt al helemaal onmogelijk. Daar is zo snel geen voorbeeld van te verzinnen. En al was er \'e\'en, het feit dat het blijkbaar een uitzondering betreft zegt genoeg. De technologische mars lijkt moeilijk te stoppen.


\section{Regulering van techniek}

Het reguleren van techniek is in twee delen te splitsen. Op de eerste plaats kunnen we het hebben over het reguleren van technologische ontwikkelingen. Op de tweede plaats over het reguleren van gebruik van bestaande technologie. Zijn beiden niet te reguleren? Of gelden voor de \'e\'en andere wetten dan voor de ander?

\subsection*{Technologische ontwikkelingen}

Technologische ontwikkelingen worden slechts door twee factoren gestuurd: de economie en het toeval. Door hard op zoek te gaan naar nieuwe technologie\"en, stijgt de kans op ontwikkeling. Als we de geschiedenis erop na kijken zien we dat mensen onderzoek doen in tijden van economische bloei. De tweede factor, het toeval, speelt ook een belangrijke rol als het gaat om technologische ontdekkingen. Al met al is de technologische ontwikkeling moeilijk te sturen.

\subsection*{Gebruik van technologie}

Hebben we weinig invloed op de ontwikkeling van technologie; op het gebruik ervan hebben we totaal geen vat. De praktijk is wat dit betreft duidelijk. Geef de mens nieuwe technieken en hij zal ze gebruiken. Enige vorm van terughoudendheid lijkt ons vreemd. Voorbeelden zijn er genoeg. Neem onze auto's. De huidige technologie maakt het mogelijk dat bijna iedereen in een eigen auto rond rijdt. Dus \emph{rijdt} ook bijna iedereen in een eigen auto rond, andere (minder technische) oplossingen ten spijt. Een ander voorbeeld is het gebruik van plastic. Sinds de uitvinding ervan is het gebruik niet meer te stoppen. Worden we er als consument blij van? Worden we er beter van? Toch gebruiken we nu meer dan 120 ton plastic per jaar.

\paragraph{}

De enige beperking op ons gebruik van techniek lijkt dat een technologie significant moet zijn om voor ons interessant te zijn. Een voorbeeld is Betamax, een systeem dat een betere beeldkwaliteit had dan het bestaande VHS videosysteem. Toch bleek vrijwel niemand ge\"interesseerd in Betamax. Blijkbaar was het verschil met VHS niet significant genoeg.


\section{Effecten van techniek}

De enorme invloed van techniek op de samenleving lijkt vaak te komen in de vorm van neven-effecten: effecten van een techniek waar deze oorspronkelijk niet voor bedoeld was of voor gebruikt werd. Effecten die van te voren niet te voorspellen waren. Denk hierbij aan de enorme hoeveelheid luchtvervuiling van auto's, of het feit dat vrijwel iedere westerling met een eigen auto kan gaan waar hij of zij wil. Tijdens de ontwikkeling van de eerste auto's kon onmogelijk al rekening gehouden worden met deze gevolgen. Ook bijvoorbeeld de grondleggers van de huidige computers hebben nooit het idee gehad dat ze werkten aan een apparaat met de enorme impact als de computer op onze samenleving heeft gehad.

Deze constatering laat ons eigenlijk al zien dat de invloed van techniek moeilijk te sturen is. Als het gaat om directe, duidelijke gevolgen is het misschien zinvol de ontwikkeling van een bepaalde techniek te reguleren (als dat al mogelijk is), maar als het gaat om onvoorspelbare gevolgen: vergeet het maar. En nu blijken juist die gevolgen de belangrijkste oorzaak te zijn van het stempel dat de techniek op de samenleving drukt.


\section{De drijvende kracht achter de techniek}

We hebben gezien dat techniek met recht autonoom genoemd mag worden. Er zit een kracht achter de techniek en de technologische ontwikkelingen waar we niet tegenop kunnen. Maar waar komt deze kracht vandaan?

\paragraph{}

De mens heeft een natuurlijke drang naar meer. `Pakken wat je pakken kunt', een mens wil verder komen. Deze aard ligt ten grondslag aan de vele oorlogen, aan de groei van de wereldbevolking, aan het kapitalisme, enzovoorts. En het is deze aard van de mens die ook de kracht vormt achter de opmars van de techniek. De techniek is niet te stoppen, toch drijven we hem zelf aan.

Hoewel het de mens zelf is die verantwoordelijk gehouden kan worden voor de techniek, heeft de mens niets te beslissen. We hebben niet de keuze of we de technologische mars aandrijven--we \emph{moeten} wel. Op de \'e\'en of andere manier zit de techniek in ons denken als ideaal ingebakken. De eeuwige zoektocht naar het paradijs blijkt voor de meeste mensen toch steeds weer verenigbaar met technologische vooruitgang.



\chapter{Technologisch determinisme en het internet}

Om een idee te krijgen waar de wereld in de toekomst heen gaat, is het interessant te kijken naar het internet. Het internet zal een zeer grote--zo niet d\`e grootste--invloed hebben op de samenleving van de komende jaren.


\section{De impact van het internet}

Het is moeilijk ons nog iets voor te stellen dat een groter stempel op de samenleving drukt dan het internet. En de verwachting is niet dat de opmars van het internet binnen afzienbare tijd halt zal houden. De mogelijkheden die we hebben gekregen met het internet zijn overweldigend. Als we over een bepaald onderwerp iets op willen zoeken gaan we niet meer naar de bibliotheek, want vrijwel iedere burger heeft in een kwestie van minuten toegang tot de grootste bron van informatie die ooit bestaan heeft.

Langzaam maar zeker zien we ook dat het internet de rol van de televisie als medium gaat overnemen. Het wordt duidelijk dat ook--of misschien \emph{juist}--het internet een techniek is waarmee gemakkelijk grote groepen mensen te bereiken zijn. Een bewijs hiervoor is de vele reclame op het internet. Als we ergens veel reclame zien, kunnen we er vanuit gaan dat het bereik groot is.

Op sociaal gebied zijn de gevolgen zo mogelijk nog groter, maar ook moeilijker in kaart te brengen. Virtuele werelden (mogelijk gemaakt met het internet) vervagen de grens tussen de realiteit en cyberspace, tussen mens en machine. Chatbots zijn niet te onderscheiden van onze vrienden, waarmee we overigens meer chatten dan dat we ze daadwerkelijk ontmoeten.

Er zijn al mensen die meer tijd doorbrengen op het internet, al dan niet onder een alter-ego, dan in de re\"ele wereld. Maar `overleven' op het internet is een grote tegenstelling met het echte leven. Op het internet ben je anoniem, kun je weglopen voor je problemen, kun je niet gestraft worden voor asociaal gedrag en kun je niet achtervolg worden van plaats naar plaats. Het internet is een anarchie. Het is ieder voor zich en hoewel er regels zijn, is het naleven ervan slechts de verantwoordelijkheid van het individu.


\section{En ook dit is niet te stoppen?}

Ook het internet is een goed voorbeeld van een technologie waarvan de onvoorspelbare gevolgen juist de grootste invloed gehad hebben op de samenleving. Origineel slechts bedoeld voor militaire en academische doeleinden, had niemand kunnen bedenken dat binnen twintig jaar een meerderheid in de westerse wereld dagelijks met het internet te maken heeft.

De zware impact van het internet op de samenleving komt uit de techniek zelf, geen mens heeft dit bewust zo gewild en/of gecre\"eerd. De grote golf van internetgebruikers is niet te stoppen. Nu men zich eenmaal heeft aangepast aan een leven met het internet is het niet meer af te breken. Het is tegen de menselijke aard in een stap terug te doen.

\paragraph{}

Maar als de terugweg afgesloten is, waar gaan we dan heen? Zal er een dag komen dat de technologie \emph{ons} controleert, in plaats van andersom? Een dag dat de technologie ons net zo goed gebruikt als wij de technologie? Daar hoeven we niet op te wachten, want die dag is geschiedenis.

Ongetwijfeld zal met het internet deze verhouding tussen de mens en de techniek duidelijker worden, zichtbaarder. Maar in feite is het nooit anders geweest. Met de komst van kunstmatige intelligentie, virtuele werelden, sociale afzondering--allemaal dingen die juist met het internet werkelijkheid worden--, zal het steeds duidelijker worden dat we het niet voor het zeggen hebben. We houden de schijn op (of misschien is dit geen schijn), dat we er allemaal erg blij mee zijn, maar ondertussen hebben we geen andere keuze dan ons over te geven aan de techniek.


\end{document}
