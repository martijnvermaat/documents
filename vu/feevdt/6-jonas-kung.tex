\documentclass[11pt]{article}
\usepackage[dutch]{babel}
\usepackage{a4}
\usepackage{latexsym}
\usepackage[
    colorlinks,
    pdftitle={Filosofie en Etiek van de Techniek college 6},
    pdfsubject={Techniek en nieuwe dimensies van etiek}
    pdfauthor={Martijn Vermaat}
]{hyperref}

\title{Commentaar voor FeEvdT college 6}
\author{
    Martijn Vermaat, groep 11
}
\date{8 februari 2004}

\begin{document}
\maketitle

\begin{quote}
Commentaar op ``The Altered Nature of Human Action'' (Jonas, H.) uit ``The Imperative of Responsibility: In Search of an Ethics for the Technological Age'' door Martijn Vermaat voor het college ``Techniek en nieuwe dimensies van ethiek''.
\end{quote}


In deze tekst pleit Jonas voor het invoeren van een nieuwe ethiek, passend bij de mogelijkheden die de huidige technologie de mens biedt. Met de moderne techniek opent zich een heel nieuw scala aan mogelijke gevolgen van een bepaalde actie. Het gaat Jonas hierbij niet zozeer om de specifieke aard van een actie zelf in deze tijd, maar des te meer om de gevolgen buiten de directe omgeving van de actie. Hierbij moet gedacht worden aan afstand en aan tijd.

De traditionele ethiek, meent Jonas, beperkt het zicht op juist op die directe omgeving van een actie. Er hoeft niet duizenden jaren de toekomst in gekeken te worden; er hoeft niet duizenden kilometers om ons heen gekeken te worden; de gevolgen van een actie beperken zich tot de directe omgeving. Volgens Jonas is dit veranderd met de komst van de technologische vooruitgang. Het is nu mogelijk duizenden jaren de toekomst in, of duizenden kilometers ver weg, invloed uit te oefenen met een actie die slechts hier en nu plaats vindt. De ethiek moet zich daar aan aanpassen.

Nu zegt Jonas niet dat de traditionele ethiek waardeloos geworden is. Nee, deze moet worden uitgebreid of aangevuld met een nieuwe etiek die wel het nodige voor de moderne tijd in oogenschouw neemt. Bij het ontwikkelen van deze nieuwe ethiek staat Jonas vooral stil bij de gevolgen van onze daden die merkbaar zijn in de toekomst. Om vanuit een ethisch standpunt na te kunnen denken over dergelijke daden is het niet acceptabel de toekomst te negeren. We moeten ten eerste de kennis hebben om iets zinnigs te kunnen zeggen over de gevolgen voor de toekomst. Ten tweede zal vastgesteld moeten worden welke waarden in de toekomst spelen en zullen deze waarden verdedigd moeten worden. Het verdedigen van `de toekomst' is dus belangrijk, maar Jonas durft niet te zeggen wie dat op grote schaal zou moeten doen.

Dat de moderne technologie voor veel ethische problemen zorgt is duidelijk. We zullen moeten inzien dat we nog steeds onderworpen zijn als mensen. Niet langer doordat we weinig macht hebben, maar juist doordat we te veel macht hebben. Onze macht is groter dan ons vermogen de implicaties van het gebruik van deze macht te beoordelen. Het genitisch manipuleren van voedsel bijvoorbeeld is voor ons geen enkel probleem, maar wat voor gevolgen dit (al dan niet in de verre toekomst) kan hebben weten we niet. Daar zit een probleem dat aangepakt moet worden door een nieuwe ethiek en we zullen ons ervan bewust moeten worden dat dit probleem bestaat.


\end{document}
