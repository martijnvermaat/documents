\documentclass[a4paper,11pt]{article} % artikel3
\usepackage{listings}
\usepackage[dutch]{babel}
\usepackage{a4}
\usepackage{color, rotating}
\usepackage{latexsym}
\usepackage[
    colorlinks,
    pdftitle={Verslag van een gesprek met Martin Bravenboer},
    pdfsubject={Studie en Loopbaan},
    pdfauthor={Martijn Vermaat}
]{hyperref}


\title{Verslag van een gesprek met Martin Bravenboer}
\author{
    Martijn Vermaat\\
    mvermaat@cs.vu.nl
}
\date{9 november 2004}

\begin{document}
\maketitle


\section*{Inleiding}

\begin{quote}
Martin Bravenboer\footnote{http://www.cs.uu.nl/people/martin/} is ruim een
jaar geleden begonnen als PhD Student aan de Universiteit Utrecht, Institute
of Information and Computing Sciences, bij de groep Software
Technology\footnote{http://www.cs.uu.nl/groups/ST/}. Ik ken Martin al enkele
jaren via een aantal internet-fora en heb hem een keer eerder ontmoet in
2001.
\end{quote}

Aangezien ik er serieus over denk om na mijn master als bijvoorbeeld
AIO\footnote{In de praktijk wordt met de Nederlandse term `assistent in
  opleiding' hetzelfde bedoeld als met het Engelse `PhD Student'. In Nederland
  wordt steeds vaker de Engelse term gebruikt.} op de universiteit te gaan
werken, leek het me een goed idee een netwerkgesprek aan te gaan met Martin
Bravenboer. Ik heb hem daartoe opgebeld en hij stelde vrijwel direct voor een
afspraak te maken om een keer samen ergens in Utrecht te dineren. Dat hebben
we dus gedaan en de afspraak werd gemaakt voor maandagavond 8 november.

Wat ik met name met het gesprek duidelijk wilde krijgen is waaruit de
dagelijkse bezigheden van Martin bestaan en hoe hij in de positie van PhD
Student terecht is gekomen. Verder leek het me uiteraard interessant te horen
welke competenties je volgens hem nodig hebt als PhD Student.


\section*{Het gesprek}


\subsection*{Hoe kom je er?}

Het werd me in het gesprek al snel duidelijk dat Martin geen formele
sollicitatie procedure doorlopen heeft om in zijn huidige positie terecht te
komen. Tijdens de laatste twee jaar van zijn studie heeft hij een aantal
vakken gevolgd bij Eelco Visser\footnote{http://www.cs.uu.nl/people/visser/},
Assistent professor bij de groep Software Technology. Daar raakte hij zo
enthousiast, dat hij al snel wat extra werk deed rond de stof van deze vakken
en op deze manier wat vaker in contact kwam met Eelco Visser. Even later was
Martin student-assistent bij een van deze vakken.

Naarmate het einde van zijn studie naderde, bracht hij steeds meer tijd door
in de Software Technology groep, mede doordat hij zijn Master thesis onder
supervisie van Eelco Visser schreef. Toen deze na 6 tot 7 jaar studeren
afgerond werd (de studie duurt `officieel' 5 jaar), was het AIO worden niet
meer dan het regelen van de formaliteiten. In de praktijk was hij het al.

Behalve in zijn laatste studiejaar, heeft Martin geen vakken begeleid als
student-assistent. En hoe makkelijk hij ook in zijn positie gerold is, dat is
allemaal pas begonnen in de laatste twee jaar van zijn studie. Je zou dus wel
kunnen zeggen dat hij veel te danken heeft aan handig netwerken, maar daar
niet zijn hele studie mee bezig is geweest.


\subsection*{De dagelijkse bezigheden}

Het werk van Martin bestaat uit het ondersteunen van onderwijs, het
onderhouden en schrijven van software van de Software Technology groep en het
(voorbereiden van) het schrijven van papers. Hij vindt het soms vervelend dat
er erg veel tijd gaat zitten in onderwijs. Het voorbereiden van een college
kost veel tijd en \'e\'en vak heeft hij samen met drie anderen inhoudelijk
zelfs helemaal op moeten zetten.

In zijn perdiode als PhD Student heeft hij tot nu toe \'e\'en serieus paper
gepubliceerd, samen met Eelco Visser. Afgelopen week heeft hij het tijdens
OOPSLA\footnote{http://www.oopsla.org/} in Vancouver gepresenteerd. Het is erg
leuk om die reis te kunnen maken en veel vakgenoten te ontmoeten die je tot
dan toe alleen van naam kent. Maar natuurlijk kost het schrijven van een paper
ook enorm veel tijd en moeite. Vooral in de weken voor het insturen en
presenteren komt hij maar aan weinig slaap toe.

Er moet wel gezegd worden dat het niet normaal is voor een PhD Student om in
het eerste jaar een paper geaccepteerd te krijgen op een conferentie als
OOPSLA. Volgens Martin moeten velen het doen met kleinere conferenties of
workshops, of hebben ze \"uberhaupt moeite een paper te publiceren.

\paragraph{}

Dat maakte me benieuwd of Martin strak van 9 tot 5 werkt. En zoals verwacht,
dat is absoluut niet het geval. Meer van 8 tot 8, plus de weekenden en in de
bus. Hij is dus een echte work-a-holic. Maar volgens hem is dat toch eerder
uitzondering dan regel in zijn groep. Toevallig is zijn begeider Eelco Visser
ook iemand die dag en nacht met niets anders bezig is, maar het wordt absoluut
niet van je verwacht.


\subsection*{Een goede AIO is\ldots}

Volgens Martin is een goede AIO iemand die erg kritisch is. Het bekijken van
andermans werk doe je standaard met de instelling ``hier moet ergens iets niet
kloppen''. En andersom, als iemand negatief is over je idee\"en, maakt dat je
extra gemotiveerd zijn ongelijk te bewijzen. Verder is het belangrijk dat je
inventief bent en graag naar nieuwe en betere oplossingen zoekt.

Om je vier jaar als AIO succesvol te kunnen afronden moet je bovendien over
een flinke dosis doorzettingsvermogen beschikken. Je bent soms erg lang met
dezelfde onderwerpen bezig en met het schrijven van een paper lijk je
eigenlijk nooit klaar te zijn. Doordat je voornamelijk zelfstandig zult moeten
werken moet je uiteraard zelfstandig zijn, maar moet je er ook goed tegen
kunnen om alleen te werken. Hoewel er genoeg collega's direct om je heen te
vinden zijn, ben je vaak vrij ge\"isoleerd met je eigen werk bezig.

Bovendien is het erg handig als goed kunt uitleggen in verband met je taken in
het onderwijs. Jezelf kunnen verplaatsen in de ander is daarbij een goed
hulpmiddel.


\section*{Conclusie}

Als ik kijk naar wat Martin dagelijks doet, zie ik veel punten die mij
aanspreken. De zelfstandigheid en de vrijheid je werk naar eigen inzicht in te
delen zijn grote voordelen. Ook lijkt het me erg leuk met vernieuwende
technologie bezig te zijn. De omgeving van een universiteit garandeert
bovendien dat je tussen jonge mensen werkt en dat naast jou, ook je collega's
de onderzoekende en vernieuwende instelling hebben.

Het zal op sommige momenten moeilijk zijn je te motiveren, met name wanneer je
je voor de zoveelste dag op een onderwerp moet concentreren waar je al enkele
maanden mee bezig bent. Toch denk ik dat je daar mee om kunt gaan, met name
doordat je als het goed is juist met onderwerpen bezig bent die je zeer
interesseren.

\paragraph{}

Conclusies over hoe PhD Student te worden kan ik eigenlijk niet trekken uit
het gesprek. Bij Martin is het allemaal geleidelijk aan en min of meer als
vanzelf gegaan. Ik denk dat dit wel vaker zo gaat, dus ik kan mezelf slechts
aanraden zo veel mogelijk vakken te volgen waarvan ik enthousiast word en
contact met docenten van interessante vakken niet te mijden.


\end{document}
