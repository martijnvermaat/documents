\documentclass[a4paper,11pt]{artikel3} % article
\usepackage{listings}
\usepackage[dutch]{babel}
\usepackage{a4}
\usepackage{color, rotating}
\usepackage{latexsym}
\usepackage[
    colorlinks,
    pdftitle={Motivatie voor de master keuze},
    pdfsubject={Studie en Loopbaan},
    pdfauthor={Martijn Vermaat}
]{hyperref}


\title{Motivatie voor de master keuze}
\author{
    Martijn Vermaat\\
    mvermaat@cs.vu.nl
}
\date{10 november 2004}

\begin{document}
\maketitle


Als ik op dit moment voor een master opleiding moest kiezen, zou dat de master
`Formele methoden en software
verificatie'\footnote{http://www.cs.vu.nl/\~{}tcs/fmsv/fmsv-nl.html} zijn. Ik zal
nu proberen de belangrijkste redenen hiervoor duidelijk te maken.

\paragraph{}

Als ik van de afgelopen 2$^{1/2}$ jaar mijn leukste vakken op een rijtje zet,
ziet dat er ongeveer als volt uit:

\begin{itemize}
\item AI Kaleidoscoop
\item Formele Structuren
\item Inleiding Logica
\item Inleiding Theoretische Informatica
\item Voortgezette Logica
\end{itemize}

Deze vakken komen allemaal (op AI Kaleidoscoop na) uit de vakgroep
Theoretische Informatica\footnote{http://www.cs.vu.nl/\~{}tcs/}. De enige master
die deze vakgroep biedt is `Formele methoden en software verificatie'.

\paragraph{}

Al geruime tijd denk ik erover na mijn studie verder te gaan in het
onderzoek. De theoretische kant van de informatica interesseert mij enorm en
ik zou daar graag meer over leren. Uit de informatie van docenten begrijp ik
dat op de VU bij uitstek de master van de Theoretische Informatica vakgroep
veel studenten kent die later het onderzoek in gaan.

\paragraph{}

Een andere mogelijkheid die ik nog even open houd, is een master op een andere
universiteit. Met name de master `Software
Technology'\footnote{http://www.cs.uu.nl/groups/ST/Master/} aan de
Universiteit Utrecht zou ik nog iets beter willen bekijken. Inhoudelijk
spreekt deze master mij aan, maar ook juist het onderzoek dat in de vakgroep
Software Technology\footnote{http://www.cs.uu.nl/groups/ST/} gedaan wordt. Dat
laatste is met name interessant met het oog op de periode na mijn master, maar
ik sluit wat dat betreft natuurlijk niets uit wanneer ik toch een andere
master kies.

\end{document}
