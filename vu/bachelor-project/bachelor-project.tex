\documentclass[a4paper,11pt,draft]{article}
\usepackage[english]{babel}
\usepackage{a4}


\usepackage[T1]{fontenc}
\usepackage{ae,aecompl}


\usepackage{amsmath}
\usepackage{amssymb}
\usepackage{amsthm}


% Typesetting evaluation rules (http://cristal.inria.fr/~remy/latex/)
\usepackage{mathpartir}


% Don't add chapter to figure numbering
\renewcommand\thefigure{\arabic{figure}}


% Keywords like let,in,then,match,with
\newcommand{\kw}[1]{\mathtt{#1}}


% Theorems
\newtheorem{lemma}{\sffamily Lemma}
\newtheorem{theorem}{\sffamily Theorem}


% One other fancy thing is sansserif headings, we go through some trouble to
% get this in our document consistently...

% Use sansserif font in section headings
\usepackage{sectsty}
\allsectionsfont{\sffamily}

% Use sansserif font in abstract heading
\usepackage{abstract}
\renewcommand{\abstractnamefont}{\normalfont\small\bfseries\sffamily}
\setlength{\abstitleskip}{-1.5em}

% Use sansserif font in captions
% http://dcwww.camd.dtu.dk/~schiotz/comp/LatexTips/LatexTips.html#captfont
\newcommand{\captionfonts}{\sffamily \small}
\makeatletter
\long\def\@makecaption#1#2{%
  \vskip\abovecaptionskip
  \sbox\@tempboxa{{\captionfonts #1: #2}}%
  \ifdim \wd\@tempboxa >\hsize
    {\captionfonts #1: #2\par}
  \else
    \hbox to\hsize{\hfil\box\@tempboxa\hfil}%
  \fi
  \vskip\belowcaptionskip}
\makeatother

% Furthermore, we use \sffamily in theorem titles and description item titles


% Must be last in preamble
\usepackage[
  pdftex,
  colorlinks,
  citecolor=black,
  filecolor=black,
  linkcolor=black,
  urlcolor=black,
  pdfauthor={Martijn Vermaat},
  pdftitle={Verifying a CPS Transformation (draft)},
  pdfsubject={Verification of a transformation to continuation-passing style},
  pdfkeywords={lambda calculus, continuation-passing style, verification,
    functional programming},
  draft=false
]{hyperref}
\usepackage[figure]{hypcap}


% TODO: Remove draft everywhere
% TODO: Spacing in terms
% TODO: Placing of floating figures
% TODO: Don't use eqnarray
% TODO: Explain let binding in target language
% TODO: Conclusion or future work?


\title{\sffamily Verifying a CPS Transformation (draft)}

\author{\sffamily
  Martijn Vermaat\\[0.3em]
  \href{mailto:mvermaat@cs.vu.nl}{\texttt{mvermaat@cs.vu.nl}}
}
\date{\sffamily Bachelor Project, February 2008}


\begin{document}

\maketitle


\begin{abstract}
  This paper is a case study on the mechanical verification of a
  transformation to continuation-passing style (CPS) by Dargaye and Leroy.
  First, the concepts of CPS and transformation to CPS are introduced.
  We then define source and target languages, and a CPS transformation between
  them which is proved to preserve semantics.
  We conclude with a small overview of the mechanization of CPS
  transformations.
  A detailed proof of the semantics preservation theorem is attached as an
  appendix.
\end{abstract}


\section{Introduction}\label{sec:introduction}


Continuation-passing style (CPS) is a functional programming style in which control is
passed explicitely to every function in the form of a continuation. Functions
written in continuation-passing style, as opposed to functions written in direct
style, never return the result of their computation directly but instead pass it to a
continuation function. This continuation function is received as an extra argument.
For example, the unary function that returns its argument
substracted by $2$ is written $\lambda x. \, x - 2$ in direct style
$\lambda$-calculus. Written in continuation-passing style, this function becomes
$\lambda x. \, \lambda k. \, k \, (x - 2)$ where $k$ is the continuation function.

A main property of programs written in CPS is that their evaluation is independent
of the evaluation strategy. Evaluating a CPS program with a call-by-value strategy
yields the same evaluation steps as evaluating it with a call-by-need strategy
\cite{Plotkin-75}. This makes CPS programs attractive as a semantical target.

In the field of compiler construction, CPS is often used as an intermediate language
in compiling functional languages \cite{Appel-92,Orbit-86}. Many optimizing
transformations, such as inlining and memoization, are easier to perform on CPS
programs than on direct-style programs.

Programs written in direct style can be mechanically transformed to equivalent
programs in CPS \cite{Danvy-Filinski-92}. Many of such transformations are known,
the earliest of which is due to Plotkin \cite{Plotkin-75}. A major disadvantage of
Plotkin's original transformation is that it generates a lot of administrative redexes.
While theoretically harmless, in practice these redexes must be removed in a later
pass of an optimizig compiler and make proving certain properties of the transformation
less straightforward. Therefore, a number of alternative transformations to CPS have
been proposed, some of them yielding no administrative redexes at all. We will see two
examples of a CPS transformation in section~\ref{sec:transformations}.

\paragraph{}

Formal verification of programs is used to prove that a program does not have certain
defects or does have certain properties. Traditionally, program verification is done
by proving the source code of the program correct. Afterwards, the source code is
compiled to binary code which is then executed. It should be clear that this scheme
does not provide any guarantees on the correctness of the executed code, one of the
reasons being that the compiler could have altered the meaning of the program.

There are a number of ways to remedy this situation. The most obvious, and the one we
will focus on, being to verify the correctness of the compiler before compiling the
source code of the program. Other ways are verifying the result of the compilation
(translation validation) \cite{Translation-Validation-98,Necula-00} and using
proof-carrying code. While these alternatives do not require proving correctness of
the compiler, this requirement is shifted to the validator and the proof-checker,
respectively.

\paragraph{}

Dargaye and Leroy \cite{Dargaye-Leroy-07} have implemented and verified two CPS
transformations using the Coq proof assistant \cite{Coq-89,CoqArt-04}. Their work is
part of the larger project to implement a verified compiler for the mini-ML language,
a restricted part of the ML functional language rich enough to be used as a target
language for automatic extraction of programs from Coq specifications
\cite{Letouzey-02}.

We do a case study on the setting of Dargaye and Leroy, but simplified. Where their
transformations work on a realistic functional source language featuring $n$-ary
functions, recursive functions, algebraic data types and pattern-matching, we restrict
this list to $n$-ary functions only. Furthermore, we focus on their first
transformation, a straight-forward extension of Plotkin's original transformation,
ignoring their verification of an optimizing transformation. Finally, the
mechanization of the transformation and its verification will not be discussed.

Contributions of this text include a more detailed description of the correctness
proof of Plotkin's extended transformation and, hopefully, a simpler and more
approachable presentation of the matter at hand.


\section{CPS Transformations}\label{sec:transformations}

Let us now look at two transformations from programs in direct style to programs in
continuation-passing style. Plotkin's original call-by-value transformation on terms
in the untyped $\lambda$-calculus is defined as follows:
\begin{align*}
  [\![x]\!]               &= \lambda k. \, k \, x\\
  [\![\lambda x. \, M]\!] &= \lambda k. \, k \, (\lambda x. \, [\![M]\!])\\
  [\![M \, N]\!]          &= \lambda k. \, [\![M]\!] \, (\lambda m. \, [\![N]\!] \, (\lambda n. \, m \, n \, k))
\end{align*}
An entire program $M$ is now translated, using the initial continuation
$\lambda x. \, x$ (the identity function), as
$[\![M]\!] \, (\lambda x. \, x)$.

This transformation is conceptually easy, but unfortunately yields a lot of
administrative redexes. An administrative redex is a redex introduced by the CPS
transformation, thus not present in the original source term. Consider for example
the transformation of $(\lambda x. \, x) \, y$ and subsequent $\beta$-reduction
of several administrative redexes:
\begin{align*}
  [\![(\lambda x. \, x) \, y]\!] &= \lambda k. \, (\lambda k. \, k \, (\lambda x. \, (\lambda k. \, k \, x))) \, (\lambda m. \, (\lambda k. \, k \, y) \,
 (\lambda n. \, m \, n \, k))\\
                                 &\rightarrow_{\beta} \lambda k. \, (\lambda m. \, (\lambda k. \, k \, y) \, (\lambda n. \, m \, n \, k)) \, (\lambda x.
\, (\lambda k. \, k \, x))\\
                                 &\rightarrow_{\beta} \lambda k. \, (\lambda k. \, k \, y) \, (\lambda n. \, (\lambda x. \, (\lambda k. \, k \, x)) \, n
\, k)\\
                                 &\rightarrow_{\beta} \lambda k. \, (\lambda n. \, (\lambda x. \, (\lambda k. \, k \, x)) \, n \, k) \, y\\
                                 &\rightarrow_{\beta} \lambda k. \, (\lambda x. \, (\lambda k. \, k \, x)) \, y \, k
\end{align*}
Note that the term concluding the above development still contains a redex, but
this redex was already present in the source term.
Administrative redexes are to be avoided, because they complicate formal
reasoning about a transformation and add computation steps to a compilation
process.

The following variation on Plotkin's transformation generates fewer
administrative redexes. Instead of abstracting over the continuation function,
the continuation is turned into an additional argument of the transformation which
is thus defined as a binary function $[\![M]\!] \triangleright k$ on a source term
$M$ and a continuation term $k$:
\begin{align*}
  [\![x]\!] \triangleright k               &= k \, x\\
  [\![\lambda x. \, M]\!] \triangleright k &= k \, (\lambda x \, k. \, [\![M]\!] \triangleright k)\\
  [\![M \, N]\!] \triangleright k          &= [\![M]\!] \triangleright \lambda m. \, [\![N]\!] \triangleright \lambda n. \, m \, n \, k
\end{align*}
where the complete translation of a program $M$ becomes
$[\![M]\!] \triangleright (\lambda x. \, x)$.

Considering the term $(\lambda x. \, x) \, y$ again, transforming it will now
generate fewer administrative redexes:
\begin{align*}
  [\![(\lambda x. \, x) \, y]\!] \triangleright k &= (\lambda m. \, (\lambda n. \, m \, n \, k) \, y) \, (\lambda x \, k. \, k \, x)\\
                                                  &\rightarrow_{\beta} (\lambda n. \, (\lambda x \, k. \, k \, x) \, n \, k) \, y\\
                                                  &\rightarrow_{\beta} (\lambda x \, k. \, k \, x) \, y \, k
\end{align*}

\paragraph{}

These are just two of many known CPS transformations. For example, a
transformation that avoids generating any administrative redexes at all is
described by Danvy and Nielsen \cite{Danvy-Nielsen-03}. In this text, we
will consider an extension of the original transformation by Plotkin.


\section{Languages}\label{sec:languages}

We will now describe the syntax and semantics of two languages acting as the
source and target languages of the transformation, respectively.
They are direct simplifications of the languages used in
\cite{Dargaye-Leroy-07}, not containing recursive functions, algebraic
datatypes, and pattern matching.
The result is two simple extensions of untyped $\lambda$-calculus with $n$-ary
functions and $\kw{let}$-bindings.

In order to avoid difficulties with $\alpha$-conversion, de Bruijn indices
are used instead of named variables.
In this notation, variables are indexed by a natural number denoting the
number of abstractions in scope between the variable and its binding
abstraction.
$\alpha$-equality of terms now simply corresponds to syntactic equality.
As an example, consider the term
$\lambda x. \lambda y. \lambda z. x \, z \, (y \, z)$ in the
$\lambda$-calculus with named variables.
The same term written with de Bruijn indexed variables becomes
$\lambda \, \lambda \, \lambda \, x_{2} \, x_{0} \, (x_{1} \, x_{0})$.

The source and target languages differ only in that the target language
features two kinds of variables (independently indexed), the extra kind being
used for continuations and intermediate evaluation results introduced by the
CPS transformation.

\subsection{Source Language}

Terms $M$ in the source language are defined by the following grammar:
\begin{align*}
M ::=             &\; x_{n}
                  && \text{variable} \\
\llap{\textbar\:} &\; \lambda^{n}.M
                  && \text{abstraction of arity $n+1$} \\
\llap{\textbar\:} &\; M(M, \ldots, M)
                  && \text{function application} \\
\llap{\textbar\:} &\; \kw{let} \: M \: \kw{in} \: M
                  && \text{binding}
\end{align*}
where $n$ ranges over the natural numbers.

Variables $x_{i}$ are identified by their de Bruijn indices $i$.
An abstraction $\lambda^{n}.M$ binds variables $x_{n}, \ldots, x_{0}$ in $M$. The
$\kw{let}$-binding $\kw{let} \: M \: \kw{in} \: N$ binds $x_{0}$ to $M$ in $N$.

Simultaneous substitution of terms $N_{0}, \ldots, N_{n}$ for
$x_{0}, \ldots, x_{n}$ in $M$ is written $M\{N_{0}, \ldots, N_{n}\}$.
Lifting of free variables is denoted by the lifting operator: $\Uparrow^{n} M$
replaces all free $x_{i}$ in $M$ by $x_{i+n}$. A value $v$ is any term of the
form $\lambda^{n}.M$.

\paragraph{}

The big-step operational semantics of the source language is given in
Figure~\ref{fig:sourcesemantics}. The three rules define the evaluation relation
$M \Rightarrow v$ on terms $M$ and values $v$, reading ``$M$ evaluates to $v$''.
As can be seen from the $\kw{let}$-rule and the rule for function application,
arguments are evaluated first, making this a call-by-value semantics.

\paragraph{}

An example term is $\lambda^{1}. \, x_{0} \: (x_{1}, (\lambda^{0}. \, x_{0}))$, a
function expecting two arguments and returning the second argument applied to the
first argument and the identity function.

\begin{figure}
\begin{mathpar}
\infer*[left=Values]
       {\:}
       {\lambda^{n}.M \Rightarrow \lambda^{n}.M}
\and
\infer*[left=$\kw{let}$]
       {M \Rightarrow v_{1}
         \\ N\{v_{1}\} \Rightarrow v}
       {(\kw{let} \: M \: \kw{in} \: N) \Rightarrow v}
\and
\infer*[left=Function Application]
       {M \Rightarrow \lambda^{n}.P
         \\ N_{i} \Rightarrow v_{i}
         \\ P\{v_{n}, \ldots, v_{0}\} \Rightarrow v}
       {M(N_{0}, \ldots, N_{n}) \Rightarrow v}
\end{mathpar}
\caption{Source language semantics}
\label{fig:sourcesemantics}
\end{figure}

\subsection{Target Language}

The target language is a direct adaptation of the source language, adding
continuation variables $\kappa_{i}$ (we will refer to variables $x_{i}$
as source-level variables). The two types of variables are numbered
independently by their de Bruijn indices.

Terms $M'$ in the target language are defined by the following grammar:
\begin{align*}
M' ::=            &\; x_{n}
                  && \text{source-level variable} \\
\llap{\textbar\:} &\; \kappa_{n}
                  && \text{continuation variable} \\
\llap{\textbar\:} &\; \lambda^{n}.M'
                  && \text{abstraction of arity $n+1$} \\
\llap{\textbar\:} &\; M'(M', \ldots, M')
                  && \text{function application} \\
\llap{\textbar\:} &\; \kw{let} \: M' \: \kw{in} \: M'
                  && \text{binding}
\end{align*}
where $n$ ranges over the natural numbers.

An abstraction $\lambda^{n}.M'$ binds its first argument, the continuation,
to $\kappa_{0}$, and its remaining arguments to $x_{n-1}, \ldots, x_{0}$ in
$M'$.
$\kw{let}$-bindings behave analogous to the same construct in the source
language.

The double simultaneous substitution of terms $N'_{0}, \ldots, N'_{n}$ for
$\kappa_{0}, \ldots, \kappa_{n}$ and terms $P'_{0}, \ldots, P'_{m}$ for
$x_{0}, \ldots, x_{m}$ in $M'$ is written
$M'\{N'_{0}, \ldots, N'_{n}\}\{P'_{0}, \ldots, P'_{m}\}$.
We use the following notation to distinguish between lifting of
source-level variables and lifting of continuation variables:
$\Uparrow_{x}^{n} M$ lifts free $x_{i}$ by $n$ in $M$ and
$\Uparrow_{\kappa}^{n} M$ lifts free $\kappa_{i}$ by $n$ in $M$.

\paragraph{}

The big-step operational semantics for the target language is given in
Figure~\ref{fig:targetsemantics}.

\begin{figure}
\begin{mathpar}
\infer*[left=Values]
       {\:}
       {\lambda^{n}.M' \Rightarrow \lambda^{n}.M'}
\and
\infer*[left=$\kw{let}$]
       {M' \Rightarrow v_{1}
         \\ N'\{\}\{v_{1}\} \Rightarrow v}
       {(\kw{let} \: M' \: \kw{in} \: N') \Rightarrow v}
\and
\infer*[left=Function Application]
       {M' \Rightarrow \lambda^{n}.P'
         \\ N'_{i} \Rightarrow v_{i}
         \\ P'\{v_{0}\}\{v_{n}, \ldots, v_{1}\} \Rightarrow v}
       {M'(N'_{0}, \ldots, N'_{n}) \Rightarrow v}
\end{mathpar}
\caption{Target language semantics}
\label{fig:targetsemantics}
\end{figure}


\section{Transformation}\label{sec:transformation}

Again, we follow the presentation of \cite{Dargaye-Leroy-07}.
The transformation we will prove correct is a straight-forward extension of
Plotkin's call-by-value transformation, defined below by two mutually
recursive functions $\Psi$ on atoms and $[\![\cdot]\!]$ on terms. An atom $A$
is a variable or an abstraction:
\begin{equation*}
A ::= \; x_{n} \: \text{\textbar} \: \lambda^{n}.M
\end{equation*}
Note that $[\![\cdot]\!]$ really works on source language terms extended with
a $\kw{then}$ construct used in the transformation of function applications.
\begin{align*}
\Psi(x_{n}) &=
  x_{n} \\
\Psi(\lambda^{n}.M) &=
  \lambda^{n+1}.[\![M]\!](\kappa_{0}) \\[1em]
[\![A]\!] &=
  \lambda^{0}.\kappa_{0} (\Psi(A)) \\
[\![M(N_{1}, \ldots, N_{n})]\!] &=
  \lambda^{0}.[\![M . N_{1} \ldots N_{n} \: \kw{then} \:
  \kappa_{n}(\kappa_{n+1}, \kappa_{n-1}, \ldots, \kappa_{0})]\!] \\
[\![\kw{let} \: M \: \kw{in} \: N]\!] &=
  \lambda^{0}.[\![M]\!] (\lambda^{0}.\kw{let} \:
  \kappa_{0} \: \kw{in} \: [\![N]\!] (\kappa_{1})) \\
[\![M_{1} \ldots M_{n} \: \kw{then} \: N']\!] &=
  [\![M_{1}]\!] (\lambda^{0} \ldots [\![M_{n}]\!] (\lambda^{0}.N') \ldots )
\end{align*}

The transformation $[\![M]\!]$ of any source term $M$ is a one-argument
abstraction in the target language that will receive the current continuation
and bind it to the continuation variable $\kappa_{0}$. A function of arity $n$
in the source language is transformed to a function of arity $n+1$ in the
target language expecting a continuation as its first argument (bound
to $\kappa_{0}$) and $n$ regular arguments thereafter
(bound to $x_{n-1}, \ldots, x_{0}$).

The transformation of the term $M_{1} \ldots M_{n} \: \kw{then} \: N'$ first
evaluates the terms
$[\![M_{1}]\!], \ldots, [\![M_{n}]\!]$ and binds them to
$\kappa_{n-1}, \ldots, \kappa_{0}$ before evaluating $N'$. So in the
transformation of the function application
$M(N_{1}, \ldots, N_{n})$, we have $\kappa_{n}$ bound to $[\![M]\!]$,
$\kappa_{n-1}, \ldots, \kappa_{0}$ bound to $[\![N_{1}]\!], \ldots [\![N_{n}]\!]$,
and $\kappa_{n+1}$ bound to the outer continuation, after which
$\kappa_{n}(\kappa_{n+1}, \kappa_{n-1}, \ldots, \kappa_{0})$ is evaluated.

\paragraph{}

As an example, consider the transformation
$[\![(\lambda^{0}.x_{0})(\lambda^{1}.x_{0})]\!]$, yielding
\begin{align*}
  \lambda^{0}. \:& \bigl( \lambda^{0}.\kappa_{0}(\lambda^{1}. \: (\lambda^{0}.\kappa_{0}(x_{0})) \: (\kappa_{0}) \: ) \bigr)\\
  & \bigl( \lambda^{0}. \: (\lambda^{0}.\kappa_{0}(\lambda^{2}.(\lambda^{0}.\kappa_{0}(x_{0}))(\kappa_{0}))) \: (\lambda^{0}.\kappa_{1}(\kappa_{2},\kappa_{
    0})) \: \bigr) \text{ .}
\end{align*}

\paragraph{}

We say that a target language term is $\kappa$-closed if it does not contain
any free continuation variables $\kappa_{i}$. The following lemmas state some
important properties of the transformation. They are proved in
appendix~\ref{sec:proofs}.

\begin{lemma}\label{lem:kappaclosed}
  $[\![M]\!]$ and $\Psi(A)$ are $\kappa$-closed. As a corollary, transformed
  terms are invariant by substitution of $\kappa$-variables:
  \begin{equation*}
    [\![M]\!]\{\vec{N}\}\{\vec{P}\} = [\![M]\!]\{\}\{\vec{P}\}
  \end{equation*}
\end{lemma}

\begin{lemma}\label{lem:commuting}
  The transformation commutes with substitution of atoms for source-level variables:
  \begin{align*}
    [\![M\{A_{1}, \ldots, A_{n}\}]\!] &= [\![M]\!]\{\}\{\Psi(A_{1}), \ldots, \Psi(A_{n})\} \\
    \Psi(A\{A_{1}, \ldots, A_{n}\})   &= \Psi(A)\{\}\{\Psi(A_{1}), \ldots, \Psi(A_{n})\}
  \end{align*}
\end{lemma}


\section{Semantics Preservation}\label{sec:semanticpreservation}

We will now show that the transformation $[\![\cdot]\!]$ from source language
terms to target language terms preserves semantics. That is, if a term $M$ in
the source language evaluates to a value $v$, the translation $[\![M]\!]$ of $M$
applied to the initial continuation $\lambda^{0}.\kappa_{0}$, should evaluate to
the target language representation $\Psi(v)$ of $v$. This property is stated in
Theorem~\ref{thm:maintheorem}.

\begin{theorem}\label{thm:maintheorem}
If $M \Rightarrow v$ in the source language, then
$[\![M]\!] (\lambda^{0}.\kappa_{0}) \Rightarrow \Psi(v)$ in the target language.
\end{theorem}

For a proof by induction to go through we need to generalize this result. The
result we will prove is given as lemma~\ref{lem:mainlemma} and generalizes the
use of the initial continuation to any $\kappa$-closed, one-argument
abstraction.

\begin{lemma}\label{lem:mainlemma}
  Let $K = \lambda^{0}.P$ be a $\kappa$-closed, one-argument
  abstraction of the target language. If $M \Rightarrow v$ in the
  source language, and $P\{\Psi(v)\}\{\} \Rightarrow v'$ in the target
  language, then $[\![M]\!](K) \Rightarrow v'$ in the target language.
\end{lemma}

\begin{proof}
We will sketch the outline of a proof, a more elaborate version can be found in
appendix~\ref{sec:proofs}. The proof is by induction on the evaluation derivation
of $M \Rightarrow v$ and case analysis over the term $M$.
\begin{description}
\item[\sffamily Base case $M = \lambda^{n}.M_{1}$]\hfill

  We need to show
  $(\lambda^{0}.\kappa_{0}(\Psi(\lambda^{n}.M_{1}))) (K) \Rightarrow v'$.
  This is trivial.

\item[\sffamily Inductive case $M = M_{1}(N_{0}, \ldots, N_{n})$]\hfill

  Premises are $M_{1} \Rightarrow \lambda^{n}.Q$, $N_{i} \Rightarrow v_{i}$,
  and $Q\{v_{n}, \ldots, v_{0}\} \Rightarrow v$. Our goal is to show
  \begin{equation*}
    [\![M_{1}]\!]
    (\lambda^{0}.[\![N_{0}]\!](\lambda^{0}. \cdots
    [\![N_{n}]\!](\lambda^{0}.\kappa_{n+1}(K, \kappa_{n},
    \ldots, \kappa_{0})) \cdots ))
    \Rightarrow v' \text{ .}
  \end{equation*}
  We apply the induction hypothesis to the first premise and our new goal is
  \begin{equation*}
    [\![N_{0}]\!]
    (\lambda^{0}. \cdots [\![N_{n}]\!](\lambda^{0}.\Psi(\lambda^{n}.Q)
    (K, \kappa_{n}, \ldots, \kappa_{0})) \cdots )
    \Rightarrow v' \text{ .}
  \end{equation*}
  Repeatedly applying the induction hypothesis to premises
  $N_{i} \Rightarrow v_{i}$ (starting with $i=0$) leaves us to show
  \begin{equation*}
    [\![Q\{v_{n}, \ldots, v_{0}\}]\!] (K) \Rightarrow v' \text{ ,}
  \end{equation*}
  which follows from the induction hypothesis applied to the last premise and
  the assumption $P\{\Psi(v)\}\{\} \Rightarrow v'$.

\item[\sffamily Inductive case $M = \kw{let} \: M_{1} \: \kw{in} \: M_{2}$]\hfill

  Premises are $M_{1} \Rightarrow v_{1}$ and $M_{2}\{v_{1}\} \Rightarrow v$. The
  induction hypothesis applied to the second premise gives us
  \begin{equation*}
    [\![M_{2}\{v_{1}\}]\!] (\Uparrow_{x}^{1} \lambda^{0}.P)
    \Rightarrow v' \text{ .}
  \end{equation*}
  Via the $\kw{let}$-evaluation rule, we can now show
  \begin{equation*}
    [\![M_{1}]\!]
    (\lambda^{0}. \: \kw{let} \: \kappa_{0} \: \kw{in} \: [\![M_{2}]\!]
    (\Uparrow_{x}^{1} \lambda^{0}.P)) \Rightarrow v'
  \end{equation*}
  by applying the induction hypothesis to the first premise.
  The assumption $P\{\Psi(v)\}\{\} \Rightarrow v'$ and evaluation rule for
  function application complete the proof.\qedhere
\end{description}
\end{proof}

\paragraph{}

Theorem~\ref{thm:maintheorem} is an instantiation of
lemma~\ref{lem:mainlemma}, so we can now prove it easily.

\begin{proof}[Proof of Theorem~\ref{thm:maintheorem}]
We take for $K$ in lemma~\ref{lem:mainlemma} the initial continuation
$\lambda^{0}.\kappa_{0}$, yielding
$[\![M]\!] (\lambda^{0}.\kappa_{0}) \Rightarrow \Psi(v)$.
\end{proof}


\section{Mechanically Verified Proofs}\label{sec:mechanized}

Results such as that of the previous section can be mechanized using formal
tools (e.g. a proof assistent).
Reasons for doing so are several.
First, a mechanically checked proof is often more precise and convincing than
a proof on paper.
Second, mechanical proofs as are often easier to be reused in many ways, e.g.
for composition, modification, and extension.
Most importantly, the ultimate goal in the verification of these program
transformations is of course obtaining correctness proofs of actual
implementations (which are used directly in an application) and not just of
transformations merely defined on paper.

However, compared to the traditional way of proving things, mechanical proofs
come with a price.
On paper, typically a lot of handwaving is done, leaving out a lot of the
trivial details.
Compare this to a formal system where one has to be precise, giving us
confidence in the correctness of proofs on the one hand, but often making them
tedious to write and hard to read on the other.
In addition to this, reasoning mechanically about programs brings technical
problems, such as formalizing the binding of variables, described in the
POPLmark challenge \cite{Poplmark-Challenge-05}.
Several approaches to these problems exist and we will discuss some of them
below.

\subsection{Mechanized Verification of CPS Transformations}

The transformation described in section~\ref{sec:transformation} and an
optimized transformation by Danvy and Nielsen \cite{Danvy-Nielsen-03} have
been implemented and verified by Dargaye and Leroy \cite{Dargaye-Leroy-07}
using the Coq proof assistent.
The complete Coq development consists of 9000 lines of code.
An implementation of the transformations in Caml is automatically extracted
from the Coq specifications.
The source and target languages are realistic functional languages,
much like the ones introduced in section~\ref{sec:languages}, but extended
with recursive functions, algebraic datatypes, and pattern matching.
This is particularly original in that earlier work mostly considers the pure
$\lambda$-calculus.

Binding of variables is represented using de Bruijn indices, hereby omitting
the issue of $\alpha$-equivalence between terms.
Unfortunately, another complexity is introduced by the required lifting of
indices in substitutions.
A large part of the Coq code deals with the properties of lifting and
substitution.
Also, terms with de Bruijn variables are arguably harder to deal with for
humans than terms with named variables.

To avoid much of the problems with lifting, two sorts of de Bruijn indices are
used -- one for source-level variables and one for continuation variables.
Lifting of source-level variables is therefore no longer necessitated by the
introduction of continuation variable bindings.

The transformations are verified using big-step operational semantics.
While small-step semantics are more expressive (they can describe
non-terminating computations), they tend to make program transformations
significantly harder to prove correct.
In this case, difficulties caused by administrative redexes are avoided by
using big-step semantics.
Also, the intended use of the transformation is in a compiler for
automatically extracted programs from Coq specifications, which are strongly
normalizing.

\subsection{Related Work}

Other verified mechanizations of transformations to CPS differ on several
aspects, most notably the logical framework used to implement the
verification, the way languages and semantics are represented, and the type
of operational semantics used.

An important decision for the representation of terms is how to formalize
binding of variables.
Named variables and $\alpha$-conversion as we are used to on paper bring many
difficulties to mechanical reasoning.
A well-known and often used alternative are de Bruijn indices, although they
bring the complexity of lifting.
Higher-order abstract syntax is another frequently used technique, in which
variables do not refer to their binder by name (i.e. as in first-order
abstract syntax), but point directly to the binding site.
The equivalence classes of Pitt's nominal logic provide us with yet another
way to model $\alpha$-equivalence of terms on a syntactical level, but are
unfortunately not yet well supported in logical frameworks.
Finally, a combination of nominal logic and de Bruijn indices is called a
``locally nameless'' representation, where free variables are named and bound
variables are represented by their de Bruijn index.

A correctness proof of a transformation is based on some sort of semantics for
the source and target languages.
Usually, correctness is intepreted as preservation of semantics or observable
behaviour.
A denotational semantics translates terms to mathematical structures in a
compositional way.
These mathematical structures, called denotations, describe the behaviour of
the language.
An operational semantics gives meaning to a program by describing a
computation.
Two main styles of operational semantics exist.
Traditional small-step semantics describe a series of computations formed by
repeatedly applying one-step rules.
Big-step semantics relate programs and their computational result directly.

\paragraph{}

Minamide and Okuma \cite{Minamide-Okuma-03} verified several CPS
transformations for $\lambda$-calculus in Isabelle/HOL.
Terms are represented using first-order abstract syntax and named variables.
This representation is choosen because it is both close to what is used on
paper and, so they argue, to intermediate languages used in compilers.

As noted above, a number of difficulties arise from the use of named
variables, one of them being that of $\alpha$-convertibility.
This is tackled by reformulating $\alpha$-equivalence as a syntax-directed
deductive system. Other problems require that variables introduced by the
transformation are fresh and that $\kw{let}$-bound variables are unique.

Minamide and Okuma use a call-by-value operational semantics in traditional
small-step style.

\paragraph{}

In \cite{Tian-06}, Tian implements and verifies a higher-order one-pass CPS
transformation by Danvy and Nielsen in the logical framework Twelf.
The source and target languages are formalized using higher-order abstract
syntax and are accompanied by a big-step and small-step operational semantics,
respectively.

Because the transformation is higher-order, it avoids problems with
administrative redexes.
The issues of binding and substitution are addressed by using higher-order
abstract syntax.

\paragraph{}

Chlipala \cite{Chlipala-07} used Coq to implement a certified compiler from
simply typed $\lambda$-calculus to assembly language.
The first two passes of the compiler implement a transformation to CPS.

Chlipala aims for as much automation in the proofs as possible.
De Bruijn indices are used to represent variables and terms are given
dependent types.
Preservation of types is trivially proved by the dependent types and enables
type-directed proof search to automate large parts of the semantics
preservation proofs. The languages are accompanied by a type-directed
denotational semantics.


\pagebreak


\appendix


\section{Proofs}\label{sec:proofs}


\subsection{Lemma~\ref{lem:kappaclosed}}

\begin{quote}
  $[\![M]\!]$ and $\Psi(A)$ are $\kappa$-closed. As a corollary, transformed
  terms are invariant by substitution of continuation variables:
  \begin{equation*}
    [\![M]\!]\{\vec{N}\}\{\vec{P}\} = [\![M]\!]\{\}\{\vec{P}\}
  \end{equation*}
\end{quote}

\begin{proof}
The proof is by structural induction over $M$ and case analysis over $A$.

\begin{description}
\item[\sffamily Base case $M = x_{i}$]\hfill

  $[\![M]\!] = \lambda^{0}.\kappa_{0}(x_{i})$, which is obviously $\kappa$-closed.

\item[\sffamily Inductive case $M = \lambda^{n}.M_{1}$]\hfill

  By the induction hypothesis, $[\![M_{1}]\!]$ is $\kappa$-closed. Therefore, also
  $[\![M]\!] = \lambda^{0}.\kappa_{0}(\lambda^{n+1}.[\![M_{1}]\!](\kappa_{0}))$ is
  $\kappa$-closed.

\item[\sffamily Inductive case $M = M_{1}(N_{0}, \ldots, N_{n})$]\hfill
  \begin{align*}
    [\![M]\!] &= \lambda^{0}.[\![M_{1}.N_{0} \ldots N_{n} \: \kw{then} \: \kappa_{n+1}(\kappa_{n+2}, \kappa_{n}, \ldots, \kappa_{0})]\!] \\
              &= \lambda^{0}.[\![M_{1}]\!](\lambda^{0}.[\![N_{0}]\!](\lambda^{0}. \ldots [\![N_{n}]\!](\lambda^{0}.\kappa_{n+1}(\kappa_{n+2}, \kappa_{n}, \ldots, \kappa_{0})) \cdots ))
  \end{align*}
  By the induction hypothesis, $[\![M_{1}]\!]$, $[\![N_{0}]\!], \ldots, [\![N_{n}]\!]$ are $\kappa$-closed.
  There are $n+3$ abstractions surrounding the $\kappa_{n+2}$, so $[\![M]\!]$ is $\kappa$-closed.

\item[\sffamily Inductive case $M = \kw{let} \: M_{1} \: \kw{in} \: M_{2}$]\hfill

  $[\![M]\!] = \lambda^{0}.[\![M_{1}]\!](\lambda^{0}. \: \kw{let} \: \kappa_{0} \: \kw{in} \: [\![M_{2}]\!](\kappa_{1}))$
  is $\kappa$-free because $[\![M_{1}]\!]$ and $[\![M_{2}]\!]$ are (by the induction hypothesis).
\end{description}

Now, $\kappa$-freeness of $\Psi(A)$ and the corollary follow trivially.
\end{proof}


\subsection{Lemma~\ref{lem:commuting}}

\begin{quote}
  The transformation commutes with substitution of atoms for source-level variables:
  \begin{align*}
    [\![M\{A_{1}, \ldots, A_{n}\}]\!] &= [\![M]\!]\{\}\{\Psi(A_{1}), \ldots, \Psi(A_{n})\} \\
    \Psi(A\{A_{1}, \ldots, A_{n}\})   &= \Psi(A)\{\}\{\Psi(A_{1}), \ldots, \Psi(A_{n})\}
  \end{align*}
\end{quote}

\begin{proof}
We omit lifting of continuation variables when there are none (implicitely applying
lemma~\ref{lem:kappaclosed}). Also, we use the fact $\Uparrow_{x}^{n} \Psi(A) = \Psi(\Uparrow^{n} A)$
several times. The first statement is proved by a structural induction argument over $M$.

\begin{description}
\item[\sffamily Base case $M = x_{i}$]\hfill

  If $i < n$, we have
  \begin{align*}
    [\![x_{i}\{A_{1}, \ldots, A_{n}\}]\!] &= [\![x_{i}]\!] \\
                                        &= \lambda^{0}.\kappa_{0}(x_{i}) \\
                                        &= \lambda^{0}.\kappa_{0}(x_{i})\{\}\{\Psi(A_{1}), \ldots, \Psi(A_{n})\} \\
                                        &= [\![x_{i}]\!]\{\}\{\Psi(A_{1}), \ldots, \Psi(A_{n})\} \text{ .}
  \end{align*}
  Otherwise,
  \begin{align*}
    [\![x_{i}\{A_{1}, \ldots, A_{n}\}]\!] &= [\![A_{i+1}]\!] \\
                                        &= \lambda^{0}.\kappa_{0}(\Psi(A_{i+1})) \\
%                                        &= \lambda^{0}.\kappa_{0}(\Uparrow_{\kappa}^{1}\Psi(A_{i+1})) && \text{by lemma~\ref{lem:kappaclosed}} \\
                                        &= \lambda^{0}.\kappa_{0}(x_{i})\{\}\{\Psi(A_{1}), \ldots, \Psi(A_{n})\} \\
                                        &= [\![x_{i}]\!]\{\}\{\Psi(A_{1}), \ldots, \Psi(A_{n})\} \text{ .}
  \end{align*}

\item[\sffamily Inductive case $M = \lambda^{m}.M_{1}$]\hfill

  We use the induction hypothesis in the third step below:
  \begin{eqnarray*} % TODO: figure out how to do this without eqnarray
    &   & [\![(\lambda^{m}.M_{1})\{A_{1}, \ldots, A_{n}\}]\!] \\
    & = & [\![\lambda^{m}.M_{1}\{x_{0}, \ldots, x_{m}, \Uparrow^{m+1}A_{1}, \ldots, \Uparrow^{m+1}A_{n}\}]\!] \\
    & = & \lambda^{0}.\kappa_{0}(\lambda^{m+1}.[\![M_{1}\{x_{0}, \ldots, x_{m}, \Uparrow^{m+1}A_{1}, \ldots, \Uparrow^{m+1}A_{n}\}]\!](\kappa_{0})) \\
    & = & \lambda^{0}.\kappa_{0}(\lambda^{m+1}.[\![M_{1}]\!] \\
    &   & \qquad \{\}\{x_{0}, \ldots, x_{m}, \Uparrow_{x}^{m+1}\Psi(A_{1}), \ldots, \Uparrow_{x}^{m+1}\Psi(A_{n})\}(\kappa_{0})) \\
%    & = & \lambda^{0}.\kappa_{0}(\lambda^{m+1}.[\![M_{1}]\!]\{\}\{x_{0}, \ldots, x_{m}, \Uparrow_{\kappa}^{2} \Uparrow_{x}^{m+1}\Psi(A_{1}), \ldots, \Uparrow_{\kappa}^{2} \Uparrow_{x}^{m+1}\Psi(A_{n})\}(\kappa_{0})) \\ % \ref{lem:kappaclosed}
    & = & (\lambda^{0}.\kappa_{0}(\lambda^{m+1}.[\![M_{1}]\!](\kappa_{0})))\{\}\{\Psi(A_{1}), \ldots, \Psi(A_{n})\} \\
    & = & [\![\lambda^{m}.M_{1}]\!]\{\}\{\Psi(A_{1}), \ldots, \Psi(A_{n})\} \text{ .}
  \end{eqnarray*}

\item[\sffamily Inductive case $M = M_{1}(N_{0}, \ldots, N_{m})$]\hfill

  In the fourth step below we use the induction hypothesis:
  \begin{eqnarray*}
    &   & [\![M_{1}(N_{0}, \ldots, N_{m})\{A_{1}, \ldots, A_{n}\}]\!] \\
    & = & [\![M_{1}\{A_{1}, \ldots, A_{n}\}(N_{0}\{A_{1}, \ldots, A_{n}\}, \ldots, N_{m}\{A_{1}, \ldots, A_{n}\})]\!] \\
    & = & \lambda^{0}.[\![M_{1}\{A_{1}, \ldots, A_{n}\}.N_{0}\{A_{1}, \ldots, A_{n}\} \ldots N_{m}\{A_{1}, \ldots, A_{n}\} \: \\
    &   & \qquad \kw{then} \: \kappa_{m+1}(\kappa_{m+2}, \kappa_{m}, \ldots, \kappa_{0})]\!] \\
    & = & \lambda^{0}.[\![M_{1}\{A_{1}, \ldots, A_{n}\}]\!](\lambda^{0}.[\![N_{0}\{A_{1}, \ldots, A_{n}\}]\!] \\
    &   & \qquad (\lambda^{0}. \ldots [\![N_{m}\{A_{1}, \ldots, A_{n}\}]\!](\lambda^{0}.\kappa_{m+1}(\kappa_{m+2}, \kappa_{m}, \ldots, \kappa_{0})) \cdots )) \\
    & = & \lambda^{0}.[\![M_{1}]\!]\{\}\{\Psi(A_{1}), \ldots, \Psi(A_{n})\}(\lambda^{0}.[\![N_{0}]\!]\{\}\{\Psi(A_{1}), \ldots, \Psi(A_{n})\} \\
    &   & \qquad (\lambda^{0}. \ldots [\![N_{m}]\!]\{\}\{\Psi(A_{1}), \ldots, \Psi(A_{n})\} \\
    &   & \qquad \qquad (\lambda^{0}.\kappa_{m+1}(\kappa_{m+2}, \kappa_{m}, \ldots, \kappa_{0})) \cdots )) \\ % IH
    & = & (\lambda^{0}.[\![M_{1}]\!](\lambda^{0}.[\![N_{0}]\!](\lambda^{0}. \ldots [\![N_{m}]\!](\lambda^{0}.\kappa_{m+1}(\kappa_{m+2}, \kappa_{m}, \ldots, \kappa_{0})) \cdots ))) \\
    &   & \qquad \{\}\{\Psi(A_{1}), \ldots, \Psi(A_{n})\} \\
    & = & (\lambda^{0}.[\![M_{1}.N_{0} \ldots N_{m} \: \kw{then} \: \kappa_{m+1}(\kappa_{m+2}, \kappa_{m}, \ldots, \kappa_{0})]\!]) \\
    &   & \qquad \{\}\{\Psi(A_{1}), \ldots, \Psi(A_{n})\} \\
    & = & [\![M_{1}(N_{0}, \ldots, N_{m})]\!]\{\}\{\Psi(A_{1}), \ldots, \Psi(A_{n})\} \text{ .}
  \end{eqnarray*}

\item[\sffamily Inductive case $M = \kw{let} \: M_{1} \: \kw{in} \: M_{2}$]\hfill

  Here the induction hypothesis is used in the third step:
  \begin{eqnarray*} % TODO: figure out how to do this without eqnarray
    &   & [\![(\kw{let} \: M_{1} \: \kw{in} \: M_{2})\{A_{1}, \ldots, A_{n}\}]\!] \\
    & = & [\![\kw{let} \: M_{1}\{A_{1}, \ldots, A_{n}\} \: \kw{in} \: M_{2}\{x_{0}, \Uparrow^{1} A_{1}, \ldots, \Uparrow^{1} A_{n}\}]\!] \\
    & = & \lambda^{0}.[\![M_{1}\{A_{1}, \ldots, A_{n}\}]\!] \\
    &   & \qquad (\lambda^{0}. \: \kw{let} \: \kappa_{0} \: \kw{in} \: [\![M_{2}\{x_{0}, \Uparrow^{1} A_{1}, \ldots, \Uparrow^{1} A_{n}\}]\!](\kappa_{1})) \\
    & = & \lambda^{0}.[\![M_{1}]\!]\{\}\{\Psi(A_{1}), \ldots, \Psi(A_{n})\} \\
    &   & \qquad (\lambda^{0}. \: \kw{let} \: \kappa_{0} \: \kw{in} \: [\![M_{2}]\!]\{\}\{x_{0}, \Uparrow_{x}^{1} \Psi(A_{1}), \ldots, \Uparrow_{x}^{1} \Psi(A_{n})\}(\kappa_{1})) \\
    & = & \lambda^{0}.[\![M_{1}]\!](\lambda^{0}. \: \kw{let} \: \kappa_{0} \: \kw{in} \: [\![M_{2}]\!](\kappa_{1}))\{\}\{\Psi(A_{1}), \ldots, \Psi(A_{n})\} \\
    & = & [\![\kw{let} \: M_{1} \: \kw{in} \: M_{2}]\!]\{\}\{\Psi(A_{1}), \ldots, \Psi(A_{n})\} \text{ .}
  \end{eqnarray*}
\end{description}

A case analysis over $A$ proves the second statement.

\begin{description}
\item[\sffamily Base case $A = x_{i}$]\hfill

  If $i < n$, we have
  \begin{align*}
    \Psi(x_{i}\{A_{1}, \ldots, A_{n}\}) &= \Psi(x_{i}) \\
                                      &= x_{i} \\
                                      &= x_{i}\{\}\{\Psi(A_{1}), \ldots, \Psi(A_{n})\} \\
                                      &= \Psi(x_{i})\{\}\{\Psi(A_{1}), \ldots, \Psi(A_{n})\} \text{ .}
  \end{align*}
  Otherwise,
  \begin{align*}
    \Psi(x_{i}\{A_{1}, \ldots, A_{n}\}) &= \Psi(A_{i+1}) \\
                                      &= x_{i}\{\}\{\Psi(A_{1}), \ldots, \Psi(A_{n})\} \\
                                      &= \Psi(x_{i})\{\}\{\Psi(A_{1}), \ldots, \Psi(A_{n})\} \text{ .}
  \end{align*}

\item[\sffamily Inductive case $A = \lambda^{m}.M_{1}$]\hfill
  \begin{eqnarray*}
    &   & \Psi((\lambda^{m}.M_{1})\{A_{1}, \ldots, A_{n}\}) \\
    & = & \Psi(\lambda^{m}.M_{1}\{x_{0}, \ldots, x_{m}, \Uparrow^{m+1} A_{1}, \ldots, \Uparrow^{m+1} A_{n}\}) \\
    & = & \lambda^{m+1}.[\![M_{1}\{x_{0}, \ldots, x_{m}, \Uparrow^{m+1} A_{1}, \ldots, \Uparrow^{m+1} A_{n}\}]\!](\kappa_{0}) \\
    & = & \lambda^{m+1}.[\![M_{1}]\!]\{\}\{\Uparrow_{x}^{m+1} \Psi(A_{1}), \ldots, \Uparrow_{x}^{m+1} \Psi(A_{n})\}(\kappa_{0}) \\
    & = & (\lambda^{m+1}.[\![M_{1}]\!](\kappa_{0}))\{\}\{\Psi(A_{1}), \ldots, \Psi(A_{n})\} \\
    & = & \Psi(\lambda^{m}.M_{1})\{\}\{\Psi(A_{1}), \ldots, \Psi(A_{n})\} \text{ .}
  \end{eqnarray*}
  Here we used the first statement of this lemma, which we already proved above.\qedhere
\end{description}
\end{proof}


\subsection{Lemma~\ref{lem:mainlemma}}

\begin{quote}
  Let $K = \lambda^{0}.P$ be a $\kappa$-closed, one-argument
  abstraction of the target language. If $M \Rightarrow v$ in the
  source language, and $P\{\Psi(v)\}\{\} \Rightarrow v'$ in the target
  language, then $[\![M]\!](K) \Rightarrow v'$ in the target language.
\end{quote}

\begin{proof}
We assume $M \Rightarrow v$ and proceed by induction on its derivation, also
assuming $P\{\Psi(v)\}\{\} \Rightarrow v'$.

\begin{description}
\item[\sffamily Base case $M = \lambda^{n}.M_{1}$]\hfill

  This is the base case, where $v = \lambda^{n}.M_{1}$.

  Because $[\![\lambda^{n}.M_{1}]\!] = \lambda^{0}.\kappa_{0}(\Psi(\lambda^{n}.M_{1}))$,
  we need to show
  \begin{equation}\label{eq:base}
    (\lambda^{0}.\kappa_{0}(\Psi(\lambda^{n}.M_{1}))) \: (K) \Rightarrow v' \text{ .}
  \end{equation}

  This is trivial, because according to the evaluation rule for
  function application we have
  \begin{equation*}
    K \: (\Psi(\lambda^{n}.M_{1})) \Rightarrow P\{\Psi(\lambda^{n}.M_{1})\}\{\} \text{ .}
  \end{equation*}
  Using the same evaluation rule and lemma~\ref{lem:kappaclosed}, we conclude
  \eqref{eq:base}.

\item[\sffamily Inductive case $M = M_{1}(N_{0}, \ldots, N_{n})$]\hfill

Premises of the evaluation rule for function application are
$M_{1} \Rightarrow \lambda^{n}.Q$, $N_{i} \Rightarrow v_{i}$,
and $Q\{v_{n}, \ldots, v_{0}\} \Rightarrow v$.

Working out the transformation,
\begin{align*}
  [\![M_{1}(N_{0}, \ldots, N_{n})]\!]
  &= \lambda^{0}.[\![M_{1}.N_{0} \ldots N_{n} \: \kw{then} \: \kappa_{n+1}(\kappa_{n+2}, \kappa_{n}, \ldots, \kappa_{0})]\!] \\
  &= \lambda^{0}.[\![M_{1}]\!](\lambda^{0}.[\![N_{0}]\!](\lambda^{0}. \cdots \\
  &  \qquad \cdots [\![N_{n}]\!](\lambda^{0}.\kappa_{n+1}(\kappa_{n+2}, \kappa_{n}, \ldots, \kappa_{0})) \cdots )) \text{ ,}
\end{align*}
we need to show
\begin{multline}\label{eq:faresult}
    (\lambda^{0}.[\![M_{1}]\!](\lambda^{0}.[\![N_{0}]\!](\lambda^{0}. \cdots \\
    \cdots [\![N_{n}]\!](\lambda^{0}.\kappa_{n+1}(\kappa_{n+2}, \kappa_{n}, \ldots, \kappa_{0})) \cdots )))(K) \Rightarrow v' \text{ .}
\end{multline}

From the induction hypothesis applied to the premise $Q\{v_{n}, \ldots, v_{0}\} \Rightarrow v$
and continuation $K$ we conclude
\begin{equation*}
  [\![Q\{v_{n}, \ldots, v_{0}\}]\!] (K) \Rightarrow v' \text{ ,}
\end{equation*}
and also, by lemma~\ref{lem:commuting} ($v_{i}$ are atoms),
\begin{equation}\label{eq:thirdpremise}
  [\![Q]\!]\{\}\{\Psi(v_{n}), \ldots, \Psi(v_{0})\} (K) \Rightarrow v' \text{ .}
\end{equation}

Furthermore, $\Psi(\lambda^{n}.Q) = \lambda^{n+1}.[\![Q]\!](\kappa_{0})$, so
the evaluation rule for function application gives us
\begin{equation}\label{eq:psiq}
  \Psi(\lambda^{n}.Q) \: (K, \Psi(v_{0}), \ldots, \Psi(v_{n})) \Rightarrow v'
\end{equation}
by \eqref{eq:thirdpremise} via lemma~\ref{lem:kappaclosed}.


Let $P_{1} = [\![N_{0}]\!](\lambda^{0}. \cdots [\![N_{n}]\!](\lambda^{0}.\kappa_{n+1}(K, \kappa_{n}, \ldots, \kappa_{0})) \cdots )$.
We have
\begin{equation}\label{eq:p1}
  P_{1}\{\Psi(\lambda^{n}.Q)\}\{\} \Rightarrow v'
\end{equation}
by \eqref{eq:psiq} and repeatedly (for $i = 0 \ldots n$) applying the
induction hypothesis to premises $N_{i} \Rightarrow v_{i}$ and continuations
\begin{equation*}
  \lambda^{0}. [\![N_{i+1}]\!] (\lambda^{0}. \cdots [\![N_{n}]\!](\lambda^{0}.\kappa_{n+1}(K, \kappa_{n}, \ldots, \kappa_{0})) \cdots ) \text{ .}
\end{equation*}

Using this to apply the induction hypothesis to our first
premise and continuation
$\lambda^{0}.P_{1}$ ($\kappa$-closed by lemma~\ref{lem:kappaclosed}), we get
\begin{equation*}
[\![M_{1}]\!](\lambda^{0}.P_{1}) \Rightarrow v' \text{ .}
\end{equation*}

Now, \eqref{eq:faresult} is proved using the evaluation rule for
function application.

\item[\sffamily Inductive case $M = \kw{let} \: M_{1} \: \kw{in} \: M_{2}$]\hfill

% TODO: renumber equations from 1?

The evaluation rule for $\kw{let}$ gives us premises $M_{1} \Rightarrow v_{1}$ and $M_{2}\{v_{1}\} \Rightarrow v$.

By definition of $[\![\cdot]\!]$, we have
\begin{equation*}
  [\![\kw{let} \: M_{1} \: \kw{in} \: M_{2}]\!]
  = \lambda^{0}.[\![M_{1}]\!](\lambda^{0}. \: \kw{let} \: \kappa_{0} \: \kw{in} \: [\![M_{2}]\!](\kappa_{1}))
\end{equation*}
and thus need to show
\begin{equation}\label{eq:letresult}
  (\lambda^{0}.[\![M_{1}]\!](\lambda^{0}. \: \kw{let} \: \kappa_{0} \: \kw{in} \: [\![M_{2}]\!](\kappa_{1})))(K)
  \Rightarrow v' \text{ .}
\end{equation}

Applying the induction hypothesis to the second premise and continuation $K$ gives us
$[\![M_{2}\{v_{1}\}]\!](K) \Rightarrow v'$, which is equivalent to
\begin{equation}\label{eq:letsecondprem}
  [\![M_{2}]\!]\{\}\{\Psi(v_{1})\}(K) \Rightarrow v'
\end{equation}
by lemma~\ref{lem:commuting} and the fact that $v_{1}$ is an atom.

Now consider the term $P_{1} = \: \kw{let} \: \kappa_{0} \: \kw{in} \: [\![M_{2}]\!](\Uparrow_{x}^{1} K)$. We
will show
\begin{equation}\label{eq:letp1}
  P_{1}\{\Psi(v_{1})\}\{\} \Rightarrow v' \text{ .}
\end{equation}

By lemma~\ref{lem:kappaclosed} there are no free continuation variables in
$[\![M_{2}]\!](\Uparrow_{x}^{1} K)$, so
\begin{equation*}
  P_{1}\{\Psi(v_{1})\}\{\} = \: \kw{let} \: \Psi(v_{1}) \: \kw{in} \: [\![M_{2}]\!](\Uparrow_{x}^{1} K) \text{ .}
\end{equation*}
The $\kw{let}$-evaluation rule states that \eqref{eq:letp1} can now be concluded if
$\Psi(v_{1})$ evaluates to itself, which it does, and
$([\![M_{2}]\!](\Uparrow_{x}^{1} K))\{\}\{\Psi(v_{1})\} \Rightarrow v'$, which is
equivalent to \eqref{eq:letsecondprem}.

By \eqref{eq:letp1}, the induction hypothesis applied to the first premise and
continuation $\lambda^{0}.P_{1}$ ($\kappa$-closed by lemma~\ref{lem:kappaclosed})
results in
\begin{equation*}
  [\![M_{1}]\!](\lambda^{0}.P_{1}) \Rightarrow v' \text{ .}
\end{equation*}
From this, the expected result \eqref{eq:letresult} follows via the evaluation rule for
function application.\qedhere
\end{description}
\end{proof}


\bibliographystyle{plain}
\bibliography{bachelor-project}


\end{document}
