\documentclass[notheorems]{beamer}

\usepackage[english]{babel}

\usepackage{amsmath}
\usepackage{amssymb}

% Typesetting evaluation rules (http://cristal.inria.fr/~remy/latex/)
\usepackage{mathpartir}

\usepackage[T1]{fontenc}
\usepackage{ae,aecompl}

\usepackage{beamerthemesplit}

\setbeamertemplate{background canvas}[vertical shading][bottom=red!10,top=blue!10]
\setbeamertemplate{navigation symbols}{}
\setbeamertemplate{headline}{}
\usetheme{Warsaw}
\useinnertheme{rectangles}

\colorlet{darkred}{red!80!black}
\colorlet{darkblue}{blue!80!black}
\colorlet{darkgreen}{green!80!black}

% Keywords like let,in,then,match,with
\newcommand{\kw}[1]{\mathtt{#1}}

% Theorems
\newtheorem{lemma}{\sffamily Lemma 1}
\newtheorem{maintheorem}{\sffamily Correcness Theorem}


\title{Verifying a CPS Transformation}

\author{Martijn Vermaat}
\institute{mvermaat@cs.vu.nl\\
http://www.cs.vu.nl/\~{}mvermaat/}
\date{Bachelor Project\\
December 20, 2007}


\begin{document}


\frame{\titlepage}


\frame{

  \frametitle{Verifying a CPS Transformation}

  \tableofcontents

}


\section{CPS Transformations}


\frame{

  \frametitle{Continuation-Passing Style}

  \begin{block}{Direct style}
    Function returns the result of its computation.

    Example: $\lambda x . \, x - 2$
  \end{block}

  \uncover<2->{
    \begin{block}{Continuation passing style}
      Function passes result to a {\em continuation}.

      Example: $\lambda x \, k . \, k \, (x - 2)$
      \uncover<3->{
        \begin{itemize}
        \item Order of evaluation is fixed
        \item Suitable for aggressive optimizations
        \end{itemize}
      }
    \end{block}
  }

}


\frame{

  \frametitle{Transforming to CPS}

  Programs in direct style can be mechanically transformed to equivalent
  prorgrams in CPS:

  \begin{itemize}
    \item Plotkin, 1975
    \item Danvy and Nielsen, 2003
    \item \ldots and many more
  \end{itemize}

}


\frame{

  \frametitle{Plotkin's Original Transformation}

  Plotkin, 1975:
  \begin{align*}
    [\![x]\!]               &= \lambda k. \, k \, x\\
    [\![\lambda x. \, M]\!] &= \lambda k. \, k \, (\lambda x. \, [\![M]\!])\\
    [\![M \, N]\!]          &= \lambda k. \, [\![M]\!] \, (\lambda m. \, [\![N]\!] \, (\lambda n. \, m \, n \, k))
  \end{align*}

  \uncover<2->{
    Generates many administrative redexes:
    \begin{align*}
      [\![(\lambda x. \, x) \, y]\!] &= \lambda k. \, (\lambda k. \, k \, (\lambda x. \, (\lambda k. \, k \, x))) \, (\lambda m. \, (\lambda k. \, k \, y) \, (\lambda n. \, m \, n \, k))\\
                                     &\rightarrow_{\beta} \lambda k. \, (\lambda m. \, (\lambda k. \, k \, y) \, (\lambda n. \, m \, n \, k)) \, (\lambda x. \, (\lambda k. \, k \, x))\\
                                     &\rightarrow_{\beta} \lambda k. \, (\lambda k. \, k \, y) \, (\lambda n. \, (\lambda x. \, (\lambda k. \, k \, x)) \, n \, k)\\
                                     &\rightarrow_{\beta} \lambda k. \, (\lambda n. \, (\lambda x. \, (\lambda k. \, k \, x)) \, n \, k) \, y\\
                                     &\rightarrow_{\beta} \lambda k. \, (\lambda x. \, (\lambda k. \, k \, x)) \, y \, k\\
    \end{align*}
  }

}


\frame{

  \frametitle{Plotkin Optimized}

  Slightly optimized version of Plotkin's original:

  \begin{align*}
    [\![x]\!] \triangleright k               &= k \, x\\
    [\![\lambda x. \, M]\!] \triangleright k &= k \, (\lambda x \, k. \, [\![M]\!] \triangleright k)\\
    [\![M \, N]\!] \triangleright k          &= [\![M]\!] \triangleright \lambda m. \, [\![N]\!] \triangleright \lambda n. \, m \, n \, k
  \end{align*}

  \uncover<2->{
    Generates fewer administrative redexes:
    \begin{align*}
      [\![(\lambda x. \, x) \, y]\!] \triangleright k &= (\lambda m. \, (\lambda n. \, m \, n \, k) \, y) \, (\lambda x \, k. \, k \, x)\\
                                                      &\rightarrow_{\beta} (\lambda n. \, (\lambda x \, k. \, k \, x) \, n \, k) \, y\\
                                                      &\rightarrow_{\beta} (\lambda x \, k. \, k \, x) \, y \, k
    \end{align*}
  }

}


\section{Correct Compilers}


\frame{

  \frametitle{Verifying a CPS Transformation}

  \tableofcontents[currentsection]

}


\frame{

  \frametitle{Program Compilation}

  Traditionally:

  \begin{enumerate}
    \item Source program is proven correct
    \item Source program is compiled to binary code
    \item Binary code is executed
  \end{enumerate}

  \uncover<2->{
    {\bf Observation:} No guarantees on correctness of executed program.
  }

}


\frame{

  \frametitle{Roads to Correct Compilers}

  Gap between correct source code and correct binary code.
  Ways to fill it:

  \begin{itemize}
    \item Prove correctness of the compiler  % correctness proof of compiler
    \item Validate compilation result        % correctness proof of validator
    \item Use proof-carrying code            % check proof
  \end{itemize}

  We will focus on the first.

}


\frame{

  \frametitle{Proving a Compiler Correct}

  Compiler is a function $C \, : \, \text{Source code} \rightarrow \text{Target code}$

  \begin{block}{Correctness of a compiler}
    Amounts to:
    \begin{itemize}
      \item Semantics($S$) $=$ Semantics($C(S)$)
      \item Observable behaviour($S$) $=$ Observable behaviour($C(S)$)
      \item $P(S) \: \Rightarrow \: P(C(S))$
      \item $P(S) \: \Rightarrow \: P'(C(S))$
      \item \ldots
    \end{itemize}
  \end{block}

  \uncover<2->{
    $C$ is composed of many stages, one of which may be a CPS transformation.
  }

}


\section{Our Setting}


\frame{

  \frametitle{Verifying a CPS Transformation}

  \tableofcontents[currentsection]

}


\frame{

  \frametitle{Mechanized Verification of CPS Transformations}

  Zaynah Dargaye and Xavier Leroy: {\em Mechanized verification of CPS transformations} (LPAR 2007)

  \begin{itemize}
    \item Part of the Compcert project (INRIA)
    \item Use Coq to implement and verify two CPS transformations
    \item Relatively interesting source and target languages
    \item No source code available (yet)
  \end{itemize}

  We will look at a simplified version of their setting.

}


\frame{

  \frametitle{Source Language}

  \begin{align*}
    M ::=             &\; x_{n}
      && \text{variable} \\
    \llap{\textbar\:} &\; \lambda^{n}.M
      && \text{abstraction of arity $n+1$} \\
    \llap{\textbar\:} &\; M(M, \ldots, M)
      && \text{function application} \\
    \llap{\textbar\:} &\; \kw{let} \: M \: \kw{in} \: M
      && \text{binding}
  \end{align*}

  Example term: $(\lambda^{0}. \, x_{0}) \, (\lambda^{1}. \, x_{0})$

  \begin{itemize}
    \item de Bruijn indices
    \item $\lambda^{n}. \, M$ binds $x_{n}, \ldots, x_{0}$ in $M$
    \item $\kw{let} \: M \: \kw{in} \: N$ binds $x_{0}$ to $M$ in $N$
    \item Big-step operational semantics (next slide)
  \end{itemize}

}


\frame{

  \frametitle{Source Language Semantics}

  \begin{mathpar}
    \infer*
           {\:}
           {\lambda^{n}.M \Rightarrow \lambda^{n}.M}
    \and
    \infer*
           {M \Rightarrow v_{1}
             \\ N\{v_{1}\} \Rightarrow v}
           {(\kw{let} \: M \: \kw{in} \: N) \Rightarrow v}
    \and
    \infer*
           {M \Rightarrow \lambda^{n}.P
             \\ N_{i} \Rightarrow v_{i}
             \\ P\{v_{n}, \ldots, v_{0}\} \Rightarrow v}
           {M(N_{0}, \ldots, N_{n}) \Rightarrow v}
  \end{mathpar}

  Example evaluation:
  \begin{equation*}
    (\lambda^{0}. \, x_{0}) \, (\lambda^{1}. \, x_{0}) \: \Rightarrow \: (\lambda^{1}. \, x_{0})
  \end{equation*}

}


\frame{

  \frametitle{Target Language}

  \begin{align*}
    M' ::=            &\; x_{n}
      && \text{source-level variable} \\
    \llap{\textbar\:} &\; \kappa_{n}
      && \text{continuation variable} \\
    \llap{\textbar\:} &\; \lambda^{n}.M'
      && \text{abstraction of arity $n+1$} \\
    \llap{\textbar\:} &\; M'(M', \ldots, M')
      && \text{function application} \\
    \llap{\textbar\:} &\; \kw{let} \: M' \: \kw{in} \: M'
      && \text{binding}
  \end{align*}

  Example term: $\lambda^{0}. \, \kappa_{0}$ (the initial continuation)

  Example term: $\lambda^{0}. \, (\lambda^{1}. \, \kappa_{0} \, (x_{0})) \, (\kappa_{0}, (\lambda^{2}. \, \kappa_{0} \, (x_{0})))$

  \begin{itemize}
    \item Two sorts of variables (indepenently numbered)
    \item $\lambda^{n}. \, M$ binds $\kappa_{0}, x_{n-1}, \ldots, x_{0}$ in $M$
  \end{itemize}

}


\frame{

  \frametitle{Target Language Semantics}

  \begin{mathpar}
    \infer*
           {\:}
           {\lambda^{n}.M' \Rightarrow \lambda^{n}.M'}
    \and
    \infer*
           {M' \Rightarrow v_{1}
             \\ N'\{\}\{v_{1}\} \Rightarrow v}
           {(\kw{let} \: M' \: \kw{in} \: N') \Rightarrow v}
    \and
    \infer*
           {M' \Rightarrow \lambda^{n}.P'
             \\ N'_{i} \Rightarrow v_{i}
             \\ P'\{v_{0}\}\{v_{n}, \ldots, v_{1}\} \Rightarrow v}
           {M'(N'_{0}, \ldots, N'_{n}) \Rightarrow v}
  \end{mathpar}

  Example evaluation:
  \begin{equation*}
    \lambda^{0}. \, (\lambda^{1}. \, \kappa_{0} \, (x_{0})) \, (\kappa_{0}, (\lambda^{2}. \, \kappa_{0} \, (x_{0})))
    \: \Rightarrow \: \lambda^{0}. \, \kappa_{0} \, (\lambda^{2}. \, \kappa_{0} \, (x_{0}))
  \end{equation*}

}


\frame{

  \frametitle{CPS Transformation}

  $[\![\cdot]\!]$ is an extension of Plotkin's original transformation:

  \begin{align*}
    \Psi(x_{n}) &=
      x_{n} \\
    \Psi(\lambda^{n}.M) &=
      \lambda^{n+1}.[\![M]\!](\kappa_{0}) \\[1em]
    [\![A]\!] &=
      \lambda^{0}.\kappa_{0} (\Psi(A)) \\
    [\![M(N_{1}, \ldots, N_{n})]\!] &=
      \lambda^{0}.[\![M . N_{1} \ldots N_{n} \: \kw{then} \:
      \kappa_{n}(\kappa_{n+1}, \kappa_{n-1}, \ldots, \kappa_{0})]\!] \\
    [\![\kw{let} \: M \: \kw{in} \: N]\!] &=
      \lambda^{0}.[\![M]\!] (\lambda^{0}.\kw{let} \:
      \kappa_{0} \: \kw{in} \: [\![N]\!] (\kappa_{1})) \\
    [\![M_{1} \ldots M_{n} \: \kw{then} \: N']\!] &=
      [\![M_{1}]\!] (\lambda^{0} \ldots [\![M_{n}]\!] (\lambda^{0}.N') \ldots )
  \end{align*}

}


\frame{

  \frametitle{CPS Transformation Example}

  On a slide like this, just $[\![(\lambda^{0}.x_{0})(\lambda^{1}.x_{0})]\!]$ is quite a term:
  \begin{align*}
    \lambda^{0}. \:& \bigl( \lambda^{0}.\kappa_{0}(\lambda^{1}. \: (\lambda^{0}.\kappa_{0}(x_{0})) \: (\kappa_{0}) \: ) \bigr)\\
    & \bigl( \lambda^{0}. \: (\lambda^{0}.\kappa_{0}(\lambda^{2}.(\lambda^{0}.\kappa_{0}(x_{0}))(\kappa_{0}))) \: (\lambda^{0}.\kappa_{1}(\kappa_{2},\kappa_{0})) \: \bigr)
  \end{align*}

  \uncover<2->{
    It evaluates to
    \begin{equation*}
      \lambda^{0}. \kappa_{0}(\lambda^{2}.\kappa_{0}(x_{0}))
    \end{equation*}
  }

}


\section{Proving Correctness of \texorpdfstring{$[\![\cdot]\!]$}{the Transformation}}


\frame{

  \frametitle{Verifying a CPS Transformation}

  \tableofcontents[currentsection]

}


\frame{

  \frametitle{Correctness of $[\![\cdot]\!]$}

  What does it mean for $[\![\cdot]\!]$ to be correct?

  \uncover<2->{
    \begin{itemize}
      \item $[\![\cdot]\!]$ preserves semantics
      \item \uncover<3->{
        How do we express this?

        (This will not work: $[\![M]\!] \Rightarrow v$ whenever $M \Rightarrow v$)
      }
    \end{itemize}
  }

  \uncover<4->{
    \begin{maintheorem}
      If $M \Rightarrow v$ in the source language, then
      $[\![M]\!] (\lambda^{0}.\kappa_{0}) \Rightarrow \Psi(v)$ in the target language.
    \end{maintheorem}
  }

}


\frame{

  \frametitle{Proof of Correctness Theorem}

  \begin{maintheorem}
    If $M \Rightarrow v$ in the source language, then
    $[\![M]\!] (\lambda^{0}.\kappa_{0}) \Rightarrow \Psi(v)$ in the target language.
  \end{maintheorem}

  More general result needed for proof by induction:

  \begin{lemma}
    Let $K = \lambda^{0}.P$ be a $\kappa$-closed, one-argument
    abstraction of the target language. If $M \Rightarrow v$ in the
    source language, and $P\{\Psi(v)\}\{\} \Rightarrow v'$ in the target
    language, then $[\![M]\!](K) \Rightarrow v'$ in the target language.
  \end{lemma}

}


\frame{

  \frametitle{Proof of Lemma 1}

  \begin{lemma}
    Let $K = \lambda^{0}.P$ be a $\kappa$-closed, one-argument
    abstraction of the target language. If $M \Rightarrow v$ in the
    source language, and $P\{\Psi(v)\}\{\} \Rightarrow v'$ in the target
    language, then $[\![M]\!](K) \Rightarrow v'$ in the target language.
  \end{lemma}

  \begin{proof}
    By induction on the evaluation derivation of $M \Rightarrow v$ and
    case analysis over the term $M$:
    \begin{description}
    \item[Base case $M = \lambda^{n}.M_{1}$]
    \item[Inductive case $M = M_{1}(N_{0}, \ldots, N_{n})$]\hfill

      Premises: $M_{1} \Rightarrow \lambda^{n}.Q$, $N_{i} \Rightarrow v_{i}$,
      $Q\{v_{n}, \ldots, v_{0}\} \Rightarrow v$
    \item[Inductive case $M = \kw{let} \: M_{1} \: \kw{in} \: M_{2}$]\hfill

      Premises: $M_{1} \Rightarrow v_{1}$, $M_{2}\{v_{1}\} \Rightarrow v$\qedhere
    \end{description}
  \end{proof}

}


\frame{

  \frametitle{Coq Development}

  Zaynah Dargaye and Xavier Leroy: {\em Mechanized verification of CPS transformations} (LPAR 2007)

  \begin{itemize}
    \item Mechanization of proof in Coq
    \item Extraction of executable Caml code
    \item Also: recursive functions, pattern matching, optimized transformation
    \item $\approx$ 9000 lines of Coq
    \item Goal: verified mini-ML to Cminor compiler (part of Compcert)
  \end{itemize}

}


\section{Discussion and Related Work}


\frame{

  \frametitle{Verifying a CPS Transformation}

  \tableofcontents[currentsection]

}


\frame{

  \frametitle{Design Choices}

  \begin{itemize}
    \item de Bruijn indices
    \item Two-sorted de Bruijn indices
    \item Big-step operational semantics
    \item Coq
    \item \ldots
  \end{itemize}

}






\begin{frame}

  \frametitle{Questions and Further Reading}

  Any questions?\\[4em]

  \begin{block}{Further Reading}
    \texttt{http://metaborg.org}
  \end{block}

\end{frame}


\end{document}
