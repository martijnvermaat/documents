\documentclass[a4paper,11pt]{article}
\usepackage[english]{babel}
\usepackage{a4,fullpage}
\usepackage{amsmath,amsfonts,amssymb}


\renewcommand{\familydefault}{\sfdefault}


\title{Knowledge Systems - Assignment 1: Classification}

\author{Laurens Bronwasser\\
1363956\\
lmbronwa@cs.vu.nl
\and
Martijn Vermaat\\
1362917\\
mvermaat@cs.vu.nl}

\date{March 2005}


\begin{document}


\maketitle


\renewcommand{\abstractname}{Introduction}
\begin{abstract}
  The problem to be solved by our classification system is to determine why a
  desktop computer cannot access an http server on the Internet. Causes can
  be, among others, failing hardware (modem, router, NIC, or communication
  wires) and improper configuration of certain devices or software like the
  browser or firewall. The classification problem involved here is to
  classify the reported problem into possible causes.
\end{abstract}


\section*{Classification Method}

We found MC4 to be the best solution for this knowledge system. Because the
system should be able to perform well enough regardless of the classification
method used for our purposes (ignoring extremely na\"ive approaches of
course), the computational properties of different classification methods were
less important in choosing a method than some other properties like required
efforts for the user.

Not all possibly needed data is available at the beginning. Because
acquisition of data takes considerable effort for the user (i.e. it might not
be clear from the start what data is relevant), MC1 is not a good
solution. For the same reason, we did not choose MC3. In addition, MC3 may
also request unnecessary data.

MC2 would be a good candidate, as it tries not to ask the user for irrelevant
data. However, we also see some possibilities to model the solution space
hierarchically. For instance, if it is known that some part of the complete
system is working, it can be inferred that the problem is not related to the
subparts of this part. Also, if some device (e.g. a router) is not part of a
particular system, all problems associated with this device can be
eliminated. This hierarchy yields solution MC4, which is basically an
extension of MC2. It extends MC2 by using hierarchical solution discrimination
techniques as applied in MC3.

MC4, like MC2, requires initial data to determine the relevant classes to be
considered and does not request irrelevant data. First, it applies initial data
abstraction and establishes abstract `seed classes'. Then, it descends the
hierarchy, starting at the seed classes and rules out classes by requesting
additional data and dismissing classes not explaining the data.


\section*{Initial Data}

To determine the seed classes, the system asks for some initial data. The
initial data concerns the following questions:
\begin{enumerate}
\item Is the system connected by means of a 56k modem, ADSL modem or cable
  modem?
\item Is the modem directly connected to the desktop computer, or is an
  intermediate network involved?
\end{enumerate}

The first question is to rule out all problems concerning the modems that are
not present. The second question is to eliminate all problems concerning local
area networking. All further questions depend on the outcome of these two, and
will therefore be considered after the seed classes have been determined.


\section*{Reasoning}

Reasoning is done by finding the covering relations of the solutions in the
top of the hierarchy that have not yet been ruled out. Data is requested to
check these covering relations.

\paragraph{Form of Solution: Singleton Solution}

In principle, every cause of failure corresponds to one problem to be
solved. To achieve a system that can always return a single solution, every
solution class has some discriminating properties, i.e. covering relations.

\paragraph{Termination Criteria}

The system does not immediately terminate after a possible solution is found,
but tries to strengthen the arguments for the solution and find competing
possible solutions. The two termination criteria used by the system are:
\begin{enumerate}
\item There are no more rules to apply that change the solution space.
\item The user interferes the system (``This is taking too long, I'll just
  have my six-year-old son look at this problem'').
\end{enumerate}

\paragraph{Inclusion Criteria}

In order not to ignore possible solutions, the system uses conservative
inclusion as the inclusion criterium. Compared to positive coverage, the
advantage of conservative inclusion is that all consistent problem causes are
considered, even if some of them don't follow directly from user data.

The disadvantage is of course that the system needs a good ranking algorithm
and enough related requests for user input for all possible solutions in order
to come up with a high-quality set of solutions. If this turns out to be a
problem in practice, a certain probability treshold could be added to the
inclusion criteria.

\paragraph{Completeness of Search: Exhaustive Search}

If data that is necessary to discriminate certain classes cannot be obtained,
the system provides a list of all possible solutions. This list can be ranked
by likelyhood of the cause of failure (e.g. based on statistical data).

\paragraph{Ranking Criteria}

For ranking generated solutions, a number of methods are used. Of course,
reasoning by elimination always seems like an obvious technique to
employ--inconsistent solutions should be discarded in favour of consistent
solutions. However, this still leaves room for more than one solution.

The solutions still assumed valid at this stage are ranked by a ranking
function. This ranking function is based on statistical and historical data.
For example, if the system cannot determine whether (1) the Ethernet cable or
(2) the Ethernet card is out of order, it reports these failure classes in
exactly this order, because an Ethernet cable is more likely to break than an
Ethernet card.


\section*{Todo}

Wat hieronder staat moet nog gedaan worden.

voorbeeld scenario.

\end{document}
