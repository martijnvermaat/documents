\documentclass[a4paper,11pt]{article}
\usepackage[dutch]{babel}
\usepackage{a4,fullpage}
\usepackage{amsmath,amsfonts,amssymb}
\usepackage{graphicx}

%\renewcommand{\familydefault}{\sfdefault}


\title{Uitwerking van opgave 16c\\
\normalsize{bij paragraaf 5.3 van Huth\&Ryan}}
%\date{Martijn Vermaat, 1 oktober 2009}
\date{}


\begin{document}

\maketitle


We zoeken een eigenschap van de toegankelijkheidsrelatie die correspondeert
met het formuleschema $\Diamond \Box \phi \rightarrow \Box \Diamond \phi$,
dat wil zeggen:

\begin{quote}
  Het schema $\Diamond \Box \phi \rightarrow \Box \Diamond \phi$ is geldig in
  een frame
  $\mathcal{F} = (W, R)$ \\
  $\Longleftrightarrow$ \\
  $R$ heeft de gezochte eigenschap
\end{quote}
De gezochte eigenschap wordt wel de `diamond property' genoemd en is als
volgt gedefinieerd:
\begin{table}[!hm]
  \begin{minipage}[t]{0.65\linewidth}\centering
    \vspace{10pt}
  $\forall xyz \; [ \; Rxy \wedge Rxz \; \rightarrow \; \exists t \; (
    Ryt \wedge Rzt ) \; ]$
  \end{minipage}
  \hfill
  \begin{minipage}[t]{0.3\linewidth}\centering
    \vspace{0pt}\raggedright
  \includegraphics[totalheight=3cm]{diamond}
  \end{minipage}
\end{table}

Er rest ons nog te bewijzen dat de diamond property inderdaad correspondeert
met het gegeven formuleschema.

\begin{description}

\item{\bf (a)}
We bewijzen eerst dat het schema $\Diamond \Box \phi \rightarrow \Box \Diamond \phi$ geldig
is in een frame als de relatie van het frame de diamond property heeft.

We nemen aan dat $R$ de diamond property heeft. Neem nu een labeling functie $L$ en een
verzameling werelden $W$ zodat $\mathcal{M} = (W, R, L)$ een model is. We
laten zien dat $\mathcal{M} \Vdash \Diamond \Box \phi \rightarrow \Box \Diamond \phi$.

Kies een willekeurige $x$ uit $W$ en neem aan dat $x \Vdash \Diamond \Box \phi$. Er
is dus een $y$ met $Rxy$ en $y \Vdash \Box \phi$. De diamond property zegt nu dat
voor iedere $z$ met $Rxz$ er een $t$ bestaat met $Ryt$ en $Rzt$. Dat betekent dat
$t \Vdash \phi$ (immers, $y \Vdash \Box \phi$). Maar dan hebben we voor iedere $z$ met
$Rxz$ dus $z \Vdash \Diamond \phi$. Dit geeft precies $x \Vdash \Box \Diamond \phi$.

Hiermee hebben we laten zien dat $\Diamond \Box \phi \rightarrow \Box \Diamond \phi$
waar is in iedere wereld van ieder model op ieder frame met de diamond property en dus
geldig in al deze frames.

\item{\bf (b)}
We bewijzen vervolgens dat het schema $\Diamond \Box \phi \rightarrow \Box \Diamond \phi$
niet geldig is in een frame als de relatie van het frame niet de diamond property heeft.

Neem een willekeurig frame $\mathcal{F} = (W, R)$ zonder de diamond property. Dan
zijn er dus drie werelden $x$, $y$ en $z$ in $W$ zodat $Rxy$ en $Rxz$ zonder dat er
een wereld $t$ bestaat met $Ryt$ en $Rzt$.

We laten nu zien dat $\Diamond \Box \phi \rightarrow \Box \Diamond \phi$ niet geldig is in
$\mathcal{F}$. Daartoe kiezen we een labeling functie $L$ zodat
$\mathcal{M} = (W, R, L)$ een model is met:
\begin{quote}
$p$ is waar in alle werelden $u$ met $Ryu$ en nergens anders
\end{quote}

Nu hebben we $x \Vdash \Diamond \Box p$, want $y$ is toegankelijk vanuit $x$ en alle
werelden vanuit $y$ toegankelijk maken $p$ waar.

Maar we hebben {\em niet} $x \Vdash \Box \Diamond p$, immers, $\Diamond p$ zou waar
moeten zijn in alle werelden bereikbaar vanuit $x$, waaronder $z$. Maar
$z \not \Vdash \Diamond p$ omdat er geen wereld bereikbaar is vanuit $z$ die $p$
waar maakt. De enige werelden die $p$ waar maken zijn bereikbaar vanuit $y$ en er
is geen wereld $t$ met $Ryt$ en $Rzt$.

Dit betekent dat $x \not \Vdash \Diamond \Box p \rightarrow \Box \Diamond p$.
Dit is een instantie van het schema en dus weten we ook dat
$x \not \Vdash \Diamond \Box \phi \rightarrow \Box \Diamond \phi$.

Hiermee hebben we laten zien dat er een wereld in een model op $\mathcal{F}$
is waar $\Diamond \Box \phi \rightarrow \Box \Diamond \phi$ niet waar is, dus dat
$\Diamond \Box \phi \rightarrow \Box \Diamond \phi$ niet geldig is in $\mathcal{F}$.

\end{description}


\end{document}
