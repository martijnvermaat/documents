\documentclass[a4paper,11pt]{article}
\usepackage[dutch]{babel}
\usepackage{a4,fullpage}
\usepackage{amsmath,amsfonts,amssymb}
\usepackage{amsthm}

%\renewcommand{\familydefault}{\sfdefault}


\title{Logic and Structure\\
\normalsize{Solutions to Selected Exercises}}
%\date{Martijn Vermaat, 30 november 2005}
\date{}


\begin{document}

\maketitle


\section{Propositional Logic}


\subsection{Propositions and Connectives}


\begin{description}

\item{1.}
$p_{1}, p_{2}, p_{3}, \neg p_{2}, \neg p_{3}, p_{1} \leftrightarrow p_{2}, p_{3} \vee (p_{1} \leftrightarrow p_{2}),
\neg p_{2} \rightarrow (p_{3} \vee (p_{1} \leftrightarrow p_{2})),\\
(\neg p_{2} \rightarrow (p_{3} \vee (p_{1} \leftrightarrow p_{2}))) \wedge \neg p_{3}$.

$p_{1}, p_{2}, p_{4}, p_{7}, \bot, \neg \bot, p_{7} \rightarrow \neg \bot, \neg p_{2}, p_{4} \leftrightarrow \neg p_{2},
(p_{4} \leftrightarrow \neg p_{2}) \rightarrow p_{1},\\
(p_{7} \rightarrow \neg \bot) \leftrightarrow ((p_{4} \leftrightarrow \neg p_{2}) \rightarrow p_{1})$.

$p_{1}, p_{2}, p_{1} \rightarrow p_{2}, (p_{1} \rightarrow p_{2}) \rightarrow p_{1},
((p_{1} \rightarrow p_{2}) \rightarrow p_{1}) \rightarrow p_{2},
(((p_{1} \rightarrow p_{2}) \rightarrow p_{1}) \rightarrow p_{2}) \rightarrow p_{1}$.

\item{2.}
Suppose $((\neg \in X$ and $X$ satisfies (i), (ii), and (iii) of Definition
1.1.2. Now $Y = X - \{((\neg\}$ is smaller than $X$ while still satisfying
those rules making $X$ not the smallest set satisfying them, hence $((\neg$
cannot be in $PROP$. (By the same reasoning as in the example on page 8.)

\item{3.}
Let $\varphi, \psi$ be formulas. We use induction on $\chi$ to show
$\varphi \in Sub(\psi) \And \psi \in Sub(\chi) \Rightarrow \varphi \in Sub(\chi)$
for all $\chi$.
($Sub$ as in Definition 1.1.7.)
\begin{description}
\item{(i)} $\chi$ is atomic, therefore $Sub(\chi) = \{\chi\}$. Suppose
$\varphi \in Sub(\psi)$ and $\psi \in \{\chi\}$. Now $\psi = \chi$ and
$Sub(\psi) = \{\chi\}$. So we also have $\varphi = \chi$ and thus
$\varphi \in Sub(\chi)$.

\item{(ii)} $\chi = \chi_{1} \square \chi_{2}$. By definition,
$Sub(\chi) = Sub(\chi_{1}) \cup Sub(\chi_{2}) \cup \{\chi_{1} \square \chi_{2}\}$.
Suppose $\varphi \in Sub(\psi)$ and $\psi \in Sub(\chi)$. Now we have three
cases:
  \begin{description}
    \item{(1)} $\psi \in Sub(\chi_{1})$. By induction we may assume
      $\varphi \in Sub(\chi_{1})$. Hence $\varphi \in Sub(\chi)$.
    \item{(2)} $\psi \in Sub(\chi_{2})$. Like (1).
    \item{(3)} $\psi = \chi_{1} \square \chi_{2}$. Now $\psi = \chi$
      and trivially $\varphi \in Sub(\chi)$.
  \end{description}

\item{(iii)} $\chi = \neg \chi_{1}$. Again by definition,
$Sub(\chi) = Sub(\chi_{1}) \cup \{\neg \chi_{1}\}$. Suppose $\varphi \in Sub(\psi)$
and $\psi \in Sub(\chi)$. We have two cases:
  \begin{description}
    \item{(1)} $\psi \in Sub(\chi_{1})$. Same as (1) in (ii).
    \item{(2)} $\psi = \neg \chi_{1}$. Because $\psi = \chi$, also
      $\varphi \in Sub(\chi)$.
  \end{description}

\end{description}

By the Induction Principle we now have
$\varphi \in Sub(\psi) \And \psi \in Sub(\chi) \Rightarrow \varphi \in Sub(\chi)$
for all formulas $\chi$.

\item{4.}
Let $\varphi \in Sub(\psi)$ and suppose $\psi_{0}, \ldots, \psi_{n}$ is a
formation sequence with $\psi_{n} = \psi$. We show
$\varphi \in \{\psi_{0}, \ldots, \psi_{n}\}$ by induction on $n$:
\begin{description}
\item{$n = 0$:} $\psi$ must be atomic, so $Sub(\psi) = \{\psi\}$. This
means $\varphi = \psi$ and thus $\varphi \in \{\psi_{0}, \ldots, \psi_{n}\}$.

\item{$n > 0$:} We consider possible structures for $\psi_{n}$:
  \begin{description}
    \item{(i)} $\psi_{n}$ is atomic. Same as $n = 0$ case above.
    \item{(ii)} $\psi = \psi_{a} \square \psi_{b}$. Note that $\psi_{a}$ and
      $\psi_{b}$ are in $\{\psi_{0}, \ldots, \psi_{n-1}\}$ (by Definition 1.1.4).
      Let $n_{a}, n_{b}$ be such that $\psi_{n_{a}} = \psi_{a}$ and
      $\psi_{n_{b}} = \psi_{b}$. Also,
      $Sub(\psi) = Sub(\psi_{a}) \cup Sub(\psi_{b}) \cup \{\psi_{a} \square \psi_{b}\}$
      which leaves three cases for $\varphi$:
        \begin{description}
          \item{(1)} $\varphi \in Sub(\psi_{a})$. By induction,
            $\varphi \in \{\psi_{0}, \ldots, \psi_{n_{a}}\}$ and thus
            $\varphi \in \{\psi_{0}, \ldots, \psi_{n}\}$.
          \item{(2)} $\varphi \in Sub(\psi_{a})$. Same as (1).
          \item{(3)} $\varphi = \psi_{a} \square \psi_{b}$. Now $\varphi = \psi$,
            so $\varphi \in \{\psi_{0}, \ldots, \psi_{n}\}$.
        \end{description}
    \item{(iii)} $\psi = \neg \psi_{a}$. Analogous to (ii).
  \end{description}
\end{description}

\item{5.} By induction on $\psi$ we show that $\varphi \in Sub(\psi)$ if $\varphi$ is in a shortest
  formation sequence of $\varphi$.
\begin{description}
\item{(i)} Atomic $\psi$ has a shortest formation sequence $\psi$. If $\varphi$ is
  in this sequence, $\varphi = \psi$ and $\varphi \in Sub(\psi)$ (for
  $Sub(\psi) = \{\psi\}$).
\item{(ii)} Suppose $\varphi$ is in a shortest formation sequence of $\psi_{0} \square \psi_{1}$, then
  clearly $\varphi$ is in a shortest formation sequence of either $\psi_{0}$ or $\psi_{1}$, or
  $\varphi = \psi_{0} \square \psi_{1}$. We consider all three cases:
  \begin{description}
    \item{(1)} By induction $\varphi \in Sub(\psi_{0})$. Now,
      $Sub(\psi_{0} \square \psi_{1} = Sub(\psi_{0}) \cup Sub(\psi_{1}) \cup \{\psi_{0} \square \psi_{1}\}$
      and thus $\varphi \in Sub(\psi_{0} \square \psi_{1})$.
    \item{(2)} Same as case (1).
    \item{(3)} A formula is always a subformula of itself.
  \end{description}
\item{(iii)} Suppose $\varphi$ is in a shortest formation sequence of $\neg \psi$, then $\varphi$ is in
  a shortest formation sequence of $\psi$ or $\varphi = \neg \psi$. The two cases are handled like we did
  in (ii).
\end{description}

\item{6.}

\item{7.}

\item{8.}

\item{9.}

\item{10.}

\item{11.}


\end{description}


\subsection{Semantics}


\end{document}
