\documentclass[a4paper,11pt]{article}
\usepackage[dutch]{babel}
\usepackage{a4}
\usepackage{latexsym}

\title{Opdrachten wetenschapsgeschiedenis deel 2\\[0.6em]
\normalsize{Probabilistische Revolutie\\
Statistiek en Levensverzekeringen in Nederland\\
Het veranderende Imago van de Computer}}
\author{
    Martijn Vermaat
    \footnote{Discussiegroep gevormd met Vijay Rao (varao@cs.vu.nl)}\\
    mvermaat@cs.vu.nl
}
\date{7 februari 2005}

\begin{document}
\maketitle


\section*{Opdracht 1}


\begin{quote}
\emph{Waarvoor werd een sterftetafel gebruikt en wat was nodig om een
  sterftetafel op te stellen?}
\end{quote}


Een sterftetafel zoals deze in de zeventiende eeuw werd opgesteld door Granut
werd door bepaalde groepen van de bevolking gebruikt om een indicatie te
krijgen van het besmettingsgevaar tijdens een grote pestepidemie in de
stad. Gaf de sterftetafel aan dat er wel erg veel mensen stierven aan de pest,
dan was het wellicht beter (tijdelijk) buiten de stad te gaan wonen om niet
zelf ook slachtoffer van de epidemie te worden.

Verder, zoals ook de gebroeders Huygens in hun brieven laten zien, is er op
basis van een sterftetafel iets te zeggen over de verwachting van iemands te
bereiken leeftijd. Als later de statistiek een grotere rol gaat spelen
gebruikt onder andere Johan de Witt deze gegevens om de waarde van lijfrentes
te kunnen bepalen.

De basis van een sterftetafel bestond zowel uit een empirische als uit een
speculatieve. In de eerste plaats waren er gegevens nodig over sterftecijfers,
liefst zo compleet en nauwkeurig mogelijk. Deze vormden de empirische
basis. Om uit deze gegevens een sterftetafel op te kunnen stellen waren enkele
aannamen nodig. Graunt nam bijvoorbeeld aan dat na het zesde jaar de kans om
te sterven altijd gelijk blijft. Deze aannamen vormden een speculatieve basis
die tezamen met de empirische basis voldoende was om een sterftetafel op te
stellen.


\begin{quote}
\emph{Waarom is in het voorbeeld van het artikel de keuze voor de mediaan als
  `midden' meer voor de hand liggend dan de keuze van het gemiddelde? Waarom
  is dat nu anders?}
\end{quote}


Vanwege de hoogte kindersterfte in die tijd was de sterftecijfer behoorlijk
asymmetrisch verdeeld over de verschillende leeftijden. Wanneer je in een
dergelijke verdeling de mediaan bepaalt zal deze een stuk lager liggen dan het
gemiddelde. Welke je gebruikt is afhankelijk van het beoogde doel. De
verwachting van iemands te bereiken leeftijd krijg je door het gemiddelde te
nemen. De mediaan geeft je echter de leeftijd die iemand met even grote kans
wel of niet zal halen.

Bij de weddenschap uit de brievenwisseling zou de mediaan dus gekozen moeten
worden. Intuitief gezien geeft de mediaan ook een beter beeld van de
levensverwachting bij een hoge kindersterfte. Strict gezien ligt de
levensverwachting inderdaad hoger, maar er zijn slechts weinig mensen die deze
leeftijd zullen halen. Tegenwoordig kennen we geen hoge kindersterfte meer,
waardoor het sterftecijfer veel gelijker verdeeld is over de leeftijden. In
dat geval liggen het gemiddelde en de mediaan veel dichter bij elkaar en is
intuitief gezien het gemiddelde een minstens zo goede maat voor de
levensverwachting.


\section*{Opdracht 2}


\begin{quote}
\emph{Dit artikel begint met een citaat over de `geschiedenis' van het gebruik
van de kansverdeling die wij nu de normale verdeling noemen: eerst in de
astronomie, daarna in de sociologie, daarna in de natuurkunde en daarna
overal. Geef voor minstens twee vakgebieden (wel of niet behandeld in dit
artikel) aan hoe deze kansverdeling daar voor het eerst werd gebruikt en hoe
dat kwam.}
\end{quote}


Laten we, omdat ik niet kan kiezen, maar gewoon de twee eerstgenoemde
vakgebieden behandelen: de astronomie en de sociologie.

In de achtiende eeuw hield men zich in de astronomie onder andere bezig met
het bepalen van de complexe banen van enkele planeten en de maan. Bovendien
was er vanuit de praktijk een groeiende behoefte aan nauwkeurige gegevens over
de afmetingen en vorm van de aarde. Met deze doeleinden in het hoofd werden er
erg veel metingen verricht, waardoor men onvermijdelijk met het verschijnsel
van de meetfout in aanraking kwam. Na enige tijd merkte men op dat de
oogenschijnlijk toevallige meetfouten een regelmaat vertoonden. Onder andere
Bernoulli, de Laplace en Gauss gingen op zoek naar de aard van deze regelmaat
en men kwam er achter dat de kansverdeling uit de centrale limietstelling hier
ook aanwezig was. Dit heeft ervoor gezorgd dat de waarschijnlijkheidsrekening
ook een theorie van meetfouten werd en deze heeft een belangrijke rol gespeeld
in de verdere ontwikkeling van de sterrenkunde.

Ondertussen had zich in Duitsland al in de zeventiende eeuw het vakgebied van
de `Statistik' ontwikkeld. Na de oversteek naar Groot-Brittanni\"e werd dit
vakgebied opgenomen in de zogenaamde politieke rekenkunde, waar men zich bezig
hield met feiten en getallen die inzicht geven in de aard van de
samenleving. In de achtiende eeuw zag de Belg Quetelet hoe de
waarschijnlijkheidsrekening toegepast werd in de astrologie. Vervolgens kreeg
hij het idee de statistiek en kansrekening toe te passen op de gegevens uit de
`Statistik', omdat hij hierin de mogelijkheid zag de fysieke en morele
toestand van de burger te bestuderen. Hij deed dit op basis van de l'Homme
Moyen, een denkbeeldige mens met alle eigenschappen gemiddeld van de
onderzochte bevolkingsgroep. Hierbij gebruikte hij de normale verdeling zoals
hij die gezien had in de astrologie. Dit idee kreeg een enorme aanhang en
zorgde voor nieuwe toepassingen van de kansrekening. Poisson (en eerder al
Bernoulli) formuleerde hierop de wet van de grote getallen waarna veelvuldig
verwezen zou worden in de `Statistik'.


\section*{Opdracht 3}


\begin{quote}
\emph{In de conclusie zegt de auteur dat vanuit een statistisch perspectief
  Pearson het sterkst uit het behandelde debat naar voren komt. Ben je het
  ermee eens of niet? Beredeneer je antwoord, waarin je ook de sterke en
  zwakke kanten van de argumenten van de andere betrokkenen betrekt.}
\end{quote}


Met zijn `First Study' was in 1910 Pearson een van de eersten die op basis
van een statistische methode stelling nam tegen de op dat moment heersende
overtuiging dat alcoholisme van een of meerdere ouders een negatieve invloed
heeft op de fysiek van kinderen. Uit het artikel blijkt dat Pearson zich er
ter dege van bewust was dat hij hier een sterke afkeurende reactie op zou
krijgen en daar kreeg hij gelijk in. De gevestigde `Temperance' beweging
verkondigde de grote gevaren van alcohol en voerde daarbij als \'e\'en van de
meest aansprekende argumenten de invloed van alcoholisme op het kind aan.

Na de publicatie van de `First Study' reageerden aanhangers van de
`Temperance' beweging dan ook afkeurend op de conclusies van het
onderzoek. Een eerste reactie was niet inhoudelijk, maar stelde dat
statistische methoden niet geschikt waren voor het onderwerp van de studie en
deed de conclusie af als onbelangrijk. Pearson reageerde hierop ge\"irriteerd
over de onwetenschappelijke houding en stelde dat de criticus zijn artikel
niet eens gelezen kon hebben.

De eerste serieuze reactie kwam van de econoom Marshall. Hij wond zich op over
een aanname over het loon van drinkende en niet-drinkende arbeiders. Volgens
hem was het onjuist te concluderen dat alcoholisme geen economische gevolgen
had voor de arbeider. Hij stelde dat Pearson zijn onderzochte groep mensen
beter had kunnen samenstellen. Marshall's kritiek richtte zich niet op de
basis van het artikel, maar op een bijzaak. Pearson reageerde met de opmerking
dat Marshall zich beter op de feiten kon concentreren dan wat vermoedens uit
te spreken over de studie. Daarop herhaalde Marshall zijn eerdere opmerkingen,
verwees hij naar een ongepubliceerd artikel van Keynes en trok hij zich terug
uit het debat.

Vervolgens bekritiseerden enkele aanhangers van de `Temperance' beweging de
`First Study'. Ze hadden geen achtergrond in de statistiek en vonden het dus
moeilijk een punt te vinden waarop ze de studie aan konden vallen. Wanneer ze
dat deden, kwam Pearson vrij snel met een statistisch tegenargument. Zeer
felle reacties kwamen van Horsley en door zijn vijandige houding lijkt het
erop dat hij vooral uit overtuiging de conclusies van de studie veroordeelde
en niet zo zeer op inhoudelijke gronden. Alle verschillende argumenten uit de
`Temperance' hoek werden vervolgens \'e\'en voor \'e\'en weerlegd door
Pearson, wel of niet op statistische gronden.

\paragraph{}

Al direct in de `First Study' noemt Pearson enkele zwakke punten van zijn
onderzoek en geeft hij daarbij aan waarom de conclusie toch te rechtvaardigen
is. Gedurende het debat hierna is er slechts weinig echt inhoudelijk
gereageerd en laten enkelen zich verleiden tot een emotioneel argument. De
critici konden de statistische aanpak van Pearson op dit onderwerp niet
accepteren. Het bij voorbaat al hebben van deze instelling heeft het debat
geen goed gedaan. De enkele keer dat er inhoudelijke kritiek kwam wist Pearson
deze direct te verdedigen.

De conclusie is dat de critici niet alleen inhoudelijk, maar ook qua vorm het
debat verliezen van Pearson. In die tijd was de mening van de critici echter
wel de gangbare mening onder de intellectuelen, dus in de praktijk zal Pearson
de meeste lezers niet hebben weten te overtuigen. Pas later werd een
dergelijke toepassing van statistiek meer geaccepteerd.


\section*{Opdracht 4}


\begin{quote}
\emph{Het artikel van Stamhuis, De Gans en Van den Bogaard gaat over
  `statistische expert knowledge'. Wat wordt in dit artikel gezegd over
  experts uit andere disciplines? In het vorige artikel, dat van Stigler,
  namen verschillende soorten experts deel aan het debat. Welke? Welke rol
  speelden de verschillende soorten expertises in het debat?}
\end{quote}


todo


\section*{Opdracht 5}


\begin{quote}
\emph{Porters artikel laat zien dat, tenminste in de periode die en het land
  dat hij behandelt, standaardoplossingen bij (levens)verzekeringen niet
  mogelijk werden geacht door betrokken deskundigen, meestal
  actuarissen. Standaardisatie was dus niet mogelijk en transparantie niet
  bereikbaar. Welke argumenten voor deze visie worden aangedragen door Porter
  en/of de actuarissen die hij aan het woord laat?}
\end{quote}


todo


\section*{Opdracht 6}


\begin{quote}
\emph{Welke fasen onderscheidt Mahoney in de ontwikkeling van computers vanaf
  de jaren 1960 tot 2000?}
\end{quote}


todo


\section*{Opmerkingen naar aanleiding van de discussie}


todo


\end{document}
