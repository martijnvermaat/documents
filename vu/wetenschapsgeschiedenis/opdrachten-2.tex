\documentclass[a4paper,11pt]{article}
\usepackage[dutch]{babel}
\usepackage{a4}
\usepackage{latexsym}

\title{Opdrachten wetenschapsgeschiedenis deel 2\\[0.6em]
\normalsize{Probabilistische Revolutie\\
Statistiek en Levensverzekeringen in Nederland\\
Het veranderende Imago van de Computer}}
\author{
    Martijn Vermaat
    \footnote{Discussiegroep gevormd met Vijay Rao (varao@cs.vu.nl)}\\
    mvermaat@cs.vu.nl
}
\date{7 februari 2005}

\begin{document}
\maketitle


\section*{Opdracht 1}


\begin{quote}
\emph{Waarvoor werd een sterftetafel gebruikt en wat was nodig om een
  sterftetafel op te stellen?}
\end{quote}


Een sterftetafel zoals deze in de zeventiende eeuw werd opgesteld door Granut
werd door bepaalde groepen van de bevolking gebruikt om een indicatie te
krijgen van het besmettingsgevaar tijdens een grote pestepidemie in de
stad. Gaf de sterftetafel aan dat er wel erg veel mensen stierven aan de pest,
dan was het wellicht beter (tijdelijk) buiten de stad te gaan wonen om niet
zelf ook slachtoffer van de epidemie te worden.

Verder, zoals ook de gebroeders Huygens in hun brieven laten zien, is er op
basis van een sterftetafel iets te zeggen over de verwachting van iemands te
bereiken leeftijd. Als later de statistiek een grotere rol gaat spelen
gebruikt onder andere Johan de Witt deze gegevens om de waarde van lijfrentes
te kunnen bepalen.

De basis van een sterftetafel bestond zowel uit een empirische als uit een
speculatieve. In de eerste plaats waren er gegevens nodig over sterftecijfers,
liefst zo compleet en nauwkeurig mogelijk. Deze vormden de empirische
basis. Om uit deze gegevens een sterftetafel op te kunnen stellen waren enkele
aannamen nodig. Graunt nam bijvoorbeeld aan dat na het zesde jaar de kans om
te sterven altijd gelijk blijft. Deze aannamen vormden een speculatieve basis
die tezamen met de empirische basis voldoende was om een sterftetafel op te
stellen.


\begin{quote}
\emph{Waarom is in het voorbeeld van het artikel de keuze voor de mediaan als
  `midden' meer voor de hand liggend dan de keuze van het gemiddelde? Waarom
  is dat nu anders?}
\end{quote}


Vanwege de hoogte kindersterfte in die tijd was de het sterftecijfer
behoorlijk asymmetrisch verdeeld over de verschillende leeftijden. Wanneer je
in een dergelijke verdeling de mediaan bepaalt zal deze een stuk lager liggen
dan het gemiddelde. Welke je gebruikt is afhankelijk van het beoogde doel. De
verwachting van iemands te bereiken leeftijd krijg je door het gemiddelde te
nemen. De mediaan geeft je echter de leeftijd die iemand met even grote kans
wel of niet zal halen.

Bij de weddenschap uit de brievenwisseling zou de mediaan dus gekozen moeten
worden. Intuitief gezien geeft de mediaan ook een beter beeld van de
levensverwachting bij een hoge kindersterfte. Strict gezien ligt de
levensverwachting inderdaad hoger, maar er zijn slechts weinig mensen die deze
leeftijd zullen halen. Tegenwoordig kennen we geen hoge kindersterfte meer,
waardoor het sterftecijfer veel gelijker verdeeld is over de leeftijden. In
dat geval liggen het gemiddelde en de mediaan veel dichter bij elkaar en is
intuitief gezien het gemiddelde een minstens zo goede maat voor de
levensverwachting.


\section*{Opdracht 2}


\begin{quote}
\emph{Dit artikel begint met een citaat over de `geschiedenis' van het gebruik
van de kansverdeling die wij nu de normale verdeling noemen: eerst in de
astronomie, daarna in de sociologie, daarna in de natuurkunde en daarna
overal. Geef voor minstens twee vakgebieden (wel of niet behandeld in dit
artikel) aan hoe deze kansverdeling daar voor het eerst werd gebruikt en hoe
dat kwam.}
\end{quote}


todo


\section*{Opdracht 3}


\begin{quote}
\emph{In de conclusie zegt de auteur dat vanuit een statistisch perspectief
  Pearson het sterkst uit het behandelde debat naar voren komt. Ben je het
  ermee eens of niet? Beredeneer je antwoord, waarin je ook de sterke en
  zwakke kanten van de argumenten van de andere betrokkenen betrekt.}
\end{quote}





\section*{Opdracht 4}


\begin{quote}
\emph{Het artikel van Stamhuis, De Gans en Van den Bogaard gaat over
  `statistische expert knowledge'. Wat wordt in dit artikel gezegd over
  experts uit andere disciplines? In het vorige artikel, dat van Stigler,
  namen verschillende soorten experts deel aan het debat. Welke? Welke rol
  speelden de verschillende soorten expertises in het debat?}
\end{quote}


todo


\section*{Opdracht 5}


\begin{quote}
\emph{Porters artikel laat zien dat, tenminste in de periode die en het land
  dat hij behandelt, standaardoplossingen bij (levens)verzekeringen niet
  mogelijk werden geacht door betrokken deskundigen, meestal
  actuarissen. Standaardisatie was dus niet mogelijk en transparantie niet
  bereikbaar. Welke argumenten voor deze visie worden aangedragen door Porter
  en/of de actuarissen die hij aan het woord laat?}
\end{quote}


todo


\section*{Opdracht 6}


\begin{quote}
\emph{Welke fasen onderscheidt Mahoney in de ontwikkeling van computers vanaf
  de jaren 1960 tot 2000?}
\end{quote}


todo


\section*{Opmerkingen naar aanleiding van de discussie}


todo


\end{document}
