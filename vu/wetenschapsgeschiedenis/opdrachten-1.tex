\documentclass[a4paper,11pt]{article}
\usepackage[dutch]{babel}
\usepackage{a4}
\usepackage{latexsym}

\title{Opdrachten wetenschapsgeschiedenis deel 1\\[0.6em]
\normalsize{Oudheid\\
Wereldbeelden\\
Wetenschappelijke Revolutie}}
\author{
    Martijn Vermaat
    \footnote{Discussiegroep gevormd met Vijay Rao (varao@cs.vu.nl)}\\
    mvermaat@cs.vu.nl
}
\date{30 januari 2005}

\begin{document}
\maketitle


\section*{Opdracht 1}


\begin{quote}
\emph{Behandel de belangrijkste aspecten van een wereldbeeld uit de oudheid
  die Lindbergh wel behandelt, maar die op college niet aan de orde is
  geweest.}
\end{quote}


In de tekst van Lindbergh wordt het wereldbeeld van Aristotels
behandeld. Aristoteles baseerde dit wereldbeeld op dat van Callippus, een
lichte modificatie van Eudoxus' wereldbeeld. Eudoxus modelleerde het heelal
als een serie van concentrische bollen, waarvan de middelste de aarde was en
de overige plaats boden aan de planeten. Om de schijnbaar onregelmatige
bewegingen van de planeten te kunnen verklaren, voegde Eudoxus aan iedere
planeet enkele bollen toe, waarbij de rotaties van deze bollen tezamen de
beweging van de planeet beschreven.

Callippus voegde aan het model van Eudoxus nog een aantal bollen toe om de
bewegingen van de zon, de maan en enkele planeten beter te kunnen
verklaren. Het resultaat was uiteraard een complex wiskundig model van vele
bollen.

Aristoteles nam dit model als uitgangspunt, maar waar Eudoxus en Callippus
het als een zuiver wiskundig model zagen, maakte hij zich zorgen om de
fysische eigenschappen van de bollen zoals ze volgens het model de
hemellichaamen droegen. Hij stelde zich namelijk de bollen als werkelijke
materie voor en dus ontstond het probleem dat de bewegingen van twee draaiende
concentrische bollen elkaar onvermijdelijk zouden be\"invloeden. Als oplossing
hiervoor verdubbelde hij bijna het aantal bollen door er tussen iedere twee
bestaande \'e\'en toe te voegen. Deze extra bollen namen de bewegingen van de
overige bollen op zodat deze bewegingen geen invloed meer hadden op andere
bollen. Uiteraard werd het model hier niet eenvoudiger op, in totaal beschreef
Aristoteles meer dan vijfenvijftig bollen in zijn wereldbeeld.


\begin{quote}
\emph{Waarom vindt Lindbergh dat Aristarchus geen goede redenen had voor zijn
  heliocentrische wereldbeeld?}
\end{quote}


Lindbergh merkt op hoe opvallend (en schijnbaar prijzenswaardig) het is dat
Aristarchus reeds in zijn tijd de zon in het midden van zijn heelal zette,
juist omdat pas 1500 jaar later Copernicus hier mee kwam. Ook is het
verleidelijk de wetenschappers in de tussenliggende periode niet erg serieus
te nemen omdat zei de idee\"en van Aristarchus genegeerd hebben.

Dit zou volgens Lindbergh echter een te gemakkelijke conclusie zijn. Het mag
dan zo zijn dat er in onze tijd alle aanwijzingen zijn voor een heliocentrisch
wereldbeeld, in de tijd van Aristarchus was dat heel anders. In de eerste
plaats was het volgens de religie (in die tijd een authoriteit) ondenkbaar de
aarde uit het middelpunt van het heelal te halen. Ten tweede ging het idee in
tegen alle traditionele astronomie. In de tijd van Aristarchus ging de
wetenschap er blind vanuit dat de aarde het middelpunt vormde. Verder is idee
niet te verantwoorden door het gebruik van gezond verstand en bovendien zou
het bepaalde waarnemingen aan de hemel voorspellen die men echter niet waar
nam. Al met al heeft Aristarchus weinig of geen argumenten voor zijn
wereldbeeld en kunnen we het niet anders zien dan als een merkwaardig idee
voor zijn tijd.


\begin{quote}
\emph{Wat waren volgens Lindbergh redenen voor Ptolemeus om bij het ontwerpen
  van zijn wereldbeeld van cirkelvormige bewegingen uit te gaan?}
\end{quote}


Hiervoor geeft Lindbergh in de tekst een aantal redenen. Volgens Lindbergh is
het niet meer dan logische dat we duidelijke kenmerken uit de hellenistische
wiskunde terug zien in het wereldbeeld van Ptolemeus. Ptolemeus stelde zich,
geheel in de hellenistische traditie, als doel de beweging van de
hemellichamen te beschrijven in termen van de meest elementaire terugkerende
beweging: de eenparige cirkelbeweging.

Een andere reden is dat de opgebouwde wiskundige theorie van zijn tijd weinig
kon aanvangen met bewegingen naast de eenparige cirkelbeweging. Wilde hij dus
ook maar iets hebben aan zijn resultaten (hij had zich verder inderdaad als
doel gesteld toekomstige standen van hemellichamen te kunnen voorspellen), dan
kon hij niet anders dan zich conformeren aan deze elementaire beweging.

Verder deden de waarnemingen van periodieke bewegingen van de hemellichamen
toch wel sterk vermoeden dat er hierachter eenparige cirkelbewegingen
schuilgingen.

Als laatste noemt Lindbergh dat het in de cultuur van het hellenisme eigenlijk
niet anders kon dan dat juist de hemel opgebouwd was uit de meest perfecte
vormen (cirkel) en bewegingen (eenparige cirkelbeweging).


\begin{quote}
\emph{Zijn er volgens jou redenen te bedenken die Ptolemeus zou kunnen hebben
  gehad om van de cirkelvormige beweging af te wijken? Zoja, welke? Zo nee,
  waarom niet?}
\end{quote}


Dergelijke redenen zouden gezocht kunnen worden in het feit dat Ptolemeus met
zijn perfecte eenparige cirkelbewegingen een wel heel ingewikkeld model nodig
had, of in het feit dat zelfs zijn model niet alle waarnemingen kon
verklaren.

Toch zijn dit voor Ptolemeus geen redenen geweest. Dat hij zich aan het
laatste gebrek van zijn wereldbeelden niet enorm stoorde blijkt wel uit zijn
opvatting dat een wereldbeeld `gekozen moest worden op basis van de wiskundige
eenvoud', blijkbaar zonder lang stil te staan bij eigenschappen die niet
overeenkwamen met de waargenomen realiteit.

Verder ligt het geheel in de traditie van het dogmatische Griekse denken om
alles terug te willen brengen in de meest elementaire vormen en
bewegingen. Voor Ptolemeus was er geen alternatief dan het verklaren van de
planetaire bewegingen in termen van de meest elementaire beweging.


\begin{quote}
\emph{Welk probleem werd in de Oudheid in de `Wetenschap van het Evenwicht'
  behandeld? Wat is de oplossing van dit probleem in het boek `Mechanical
  Problems' toegeschreven aan Aristoteles en in `On the Equilibrium of Planes'
  van Archimedes?}
\end{quote}


Onder de `Wetenschap van het Evenwicht' werd verstaan het zoeken van een
verklaring voor (onder andere) het feit dat een balans zijn evenwicht bewaart
wanneer de gewichten aan beide zijden omgekeerd evenredig zijn met hun afstand
tot het evenwichtspunt van de balans.

Een zogenaamd `dynamisch' bewijs voor dit verschijnsel werd door Aristoteles
gegeven in `Mechanical Problems'. Het bewijs is gebaseerd op de beweging die
de balans krijgt wanneer er aan \'e\'en kant een gewicht op geplaatst
wordt. Wordt er aan de andere kant ook een gewicht geplaatst, dan staat de
balans stil, dan en slechts dan, als de twee resulterende bewegingen elkaar
opheffen.

Archimedes geeft in `On the Equilibrium of Planes' een `statisch' bewijs voor
hetzelfde verschijnsel. Hierin weet hij het probleem terug te brengen tot een
puur geometrisch probleem, waarbij de fysische eigenschappen (behalve het
gewicht van de gewichten) niet van belang zijn. Het bewijs bestaat uit twee
premissen en een daaruit volgende afleiding, zoals gebruikelijk in de Griekse
wiskunde.


\section*{Opdracht 2}


\begin{quote}
\emph{Bennett zet tegenover elkaar waarom vanuit hedendaags standpunt Kepler
  een belangrijke astronoom was en wat Kepler zelf de belangrijkste
  doelstelling en het belangrijkste resultaat van zijn werk vond. Breng het
  hedendaagse standpunt en Keplers eigen doelstelling onder woorden in 10-20
  regels.}
\end{quote}


Gedurende vele eeuwen heeft men voor het verklaren van de beweging van de
hemellichaam slechts perfecte bollen en circelbewegingen gebruikt. Kepler was
de eerste die hier van af week door de elliptische baan van de planeet rond de
zon te introduceren. Deze en de andere twee `wetten van Kepler' zoals wij ze
vandaag de dag kennen zijn belangrijke aspecten uit de moderne astronomie die
er voor de tijd van Kepler niet waren. Onder andere Newton heeft later op
basis van deze wetten ons huidige wereldbeeld vorm gegeven.

Kepler zelf zag deze wetten niet als peilers van zijn werk. In tegendeel, ze
komen slechts als \'e\'en van de vele resultaten verspreid over zijn teksten
voor. Wat Kepler zelf als doel had was het ontrafelen van de structuur van
God's schepping, het heelal. Als overtuigd Lutheraan was hij ge\"obsedeerd
door de Goddelijke schoonheid die schuil moest gaan achter de bewegingen van
de planeten. Deze Goddelijke schoonheid probeerde hij te begrijpen, voor zover
dat volgens hem mogelijk was als `gewone mens', volgens een wiskundig en
tegelijkertijd fysisch model.


\section*{Opdracht 3}


\begin{quote}
\emph{Crosby behandelt in zijn boek het fenomeen dat in de Westerse
  samenleving tussen 1250-1600, op velerlei gebied kwalitatieve benaderingen
  zijn vervangen door kwantitatieve. Op blz. 17 schrijft hij het volgende:
  ``The West's distinctive intellectual accomplishment was to bring
  mathematics and measurement together \ldots marriage?''. Op welke
  `mathematics' en `measurement' doelt hij hier? Waarom karakteriseert hij het
  samengaan als een `shotgun marriage'?}
\end{quote}


Crosby doelt met `mathematics' op de theoretische en filosofische
wetenschappen. Volgens de Platoonse school kon men via deze weg achter
werkelijke waarheden komen en niet via waarnemingen van verschijnselen in de
wereld om ons heen.

In groot contrast daarmee staan de praktische toepassingen van wetenschap, het
ontdekken via de waarneming en het in de praktijk brengen van nuttige
ontdekkingen. Dit is wat Crosby hier beschrijft als `measurement'.

Volgens Crosby was een groot resultaat van de Westerse cultuur van 1300 tot
1600 het bij elkaar brengen van deze `mathematics' en
`measurement'. Historische gezien gingen zij altijd moeilijk samen, met name
door de Platoonse kijk op de fysische werkelijkheid zoals men die ervaart. De
`echte wiskundige' verdeed zijn tijd niet met het bestuderen hiervan en
bovendien is er lang weinig behoefte geweest aan het rijmen van de
theoretische wetenschap met de werkelijkheid, simpelweg omdat men er niet zo
zeer in geloofde dat dat mogelijk was. Het lijkt daarom een onmogelijke opgave
deze twee polen samen te brengen, maar volgens Crosby is het Westen daar
uitstekend in geslaagd. Het resultaat is de ongekende wetenschappelijke
vooruitgang vanaf de veertiende eeuw.


\begin{quote}
\emph{In zijn slothoofdstuk behandelt Crosby welke kenmerken de Westerse
  zestiende eeuwse samenleving had waardoor ze zo machtig en invloedrijk heeft
  kunnen worden. Welke kenmerken zijn dat? Licht je antwoord toe.}
\end{quote}

Hoi hoi.


\section*{Opdracht 4}


\begin{quote}
\emph{Wat zijn volgens Vermij de redenen dat in de zestiende eeuw de status en
  de rol van de wiskunde en van de wiskundige veranderde? Waaruit bestond die
  verandering?}
\end{quote}


Ja hoor.


\begin{quote}
\emph{Zijn er parallellen met de tekst van Crosby? Zo ja, welke?}
\end{quote}


Nee hoor.


\end{document}
