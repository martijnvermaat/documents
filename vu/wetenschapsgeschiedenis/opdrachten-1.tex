\documentclass[a4paper,11pt]{article}
\usepackage[dutch]{babel}
\usepackage{a4}
\usepackage{latexsym}

\title{Opdrachten wetenschapsgeschiedenis deel 1\\
Oudheid; Wereldbeelden; Wetenschappelijke Revolutie}
\author{
    Martijn Vermaat
    \footnote{Discussiegroep gevormd met Vijay Rao (varao@cs.vu.nl)}\\
    mvermaat@cs.vu.nl
}
\date{30 januari 2005}

\begin{document}
\maketitle


\section*{Opdracht 1}


\begin{quote}
\emph{Behandel de belangrijkste aspecten van een wereldbeeld uit de oudheid
  die Lindbergh wel behandelt, maar die op college niet aan de orde is
  geweest.}
\end{quote}


bla bla.


\begin{quote}
\emph{Waarom vindt Lindbergh dat Aristarchus geen goede redenen had voor zijn
  heliocentrische wereldbeeld?}
\end{quote}


bla ble.


\begin{quote}
\emph{Wat waren volgens Lindbergh redenen voor Ptolemeus om bij het ontwerpen
  van zijn wereldbeeld van cirkelvormige bewegingen uit te gaan?}
\end{quote}


ho ho.


\begin{quote}
\emph{Zijn er volgens jou redenen te bedenken die Ptolemeus zou kunnen hebben
  gehad om van de cirkelvormige beweging af te wijken? Zoja, welke? Zo nee,
  waarom niet?}
\end{quote}


hoi hoi.


\begin{quote}
\emph{Welk probleem werd in de Oudheid in de `Wetenschap van het Evenwicht'
  behandeld? Wat is de oplossing van dit probleem in het boek `Mechanical
  Problems' toegeschreven aan Aristoteles en in `On the Equilibrium of Planes'
  van Archimedes?}
\end{quote}


Nou nou!


\section*{Opdracht 2}


\begin{quote}
\emph{Bennett zet tegenover elkaar waarom vanuit hedendaags standpunt Kepler
  een belangrijke astronoom as en wat Kepler zelf de belangrijkste
  doelstelling en het belangrijkste resultaat van zijn werk vond. Breng het
  hedendaagse standpunt en Keplers eigen doelstelling onder woorden in 10-20
  regels.}
\end{quote}


Bla bla.


\section*{Opdracht 3}


\begin{quote}
\emph{Crosby behandelt in zijn boek het fenomeen dat in de Westerse
  samenleving tussen 1250-1600, op velerlei gebied kwalitatieve benaderingen
  zijn vervangen door kwantitatieve. Op blz. 17 schrijft hij het volgende:
  ``The West's distinctive intellectual accomplishment was to bring
  mathematics and measurement together \ldots marriage?''. Op welke
  `mathematics' en `measurement' doelt hij hier? Waarom karakteriseert hij het
  samengaan als een `shotgun marriage'?}
\end{quote}


Zo zo.


\begin{quote}
\emph{In zijn slothoofdstuk behandelt Crosby welke kenmerken de Westerse
  zestiende eeuwse samenleving had waardoor ze zo machtig en invloedrijk heeft
  kunnen worden. Welke kenmerken zijn dat? Licht je antwoord toe.}
\end{quote}

Hoi hoi.


\section*{Opdracht 4}


\begin{quote}
\emph{Wat zijn volgens Vermij de redenen dat in de zestiende eeuw de status en
  de rol van de wiskunde en van de wiskundige veranderde? Waaruit bestond die
  verandering?}
\end{quote}


Ja hoor.


\begin{quote}
\emph{Zijn er parallellen met de tekst van Crosby? Zo ja, welke?}
\end{quote}


Nee hoor.


\end{document}
