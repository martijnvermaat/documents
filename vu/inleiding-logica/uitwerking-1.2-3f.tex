\documentclass[a4paper,11pt]{article}
\usepackage[dutch]{babel}
\usepackage{a4,fullpage}
\usepackage{amsmath,amsfonts,amssymb}
\usepackage{fitch} % http://folk.uio.no/johanw/FitchSty.html

%\renewcommand{\familydefault}{\sfdefault}


\title{Uitwerking van opgave 3f\\
\normalsize{bij paragraaf 1.2 van Huth\&Ryan}}
%\date{Martijn Vermaat, 13 september 2005}
\date{}


\begin{document}

\maketitle


We bewijzen $p \rightarrow q \vdash \neg p \vee q$ volgens natuurlijke deductie zonder de regel LEM:

\begin{equation*}
\begin{fitch}
p \rightarrow q                        & premisse                      \\ % 1
\fh \neg ( \neg p \vee q )             & assumptie                     \\ % 2
\fa \fh p                              & assumptie                     \\ % 3
\fa \fa q                              & $\rightarrow_{\text{e}}$ 1, 3 \\ % 4
\fa \fa \neg p \vee q                  & $\vee_{\text{i$_{2}$}}$ 4     \\ % 5
\fa \fa \bot                           & $\neg_{\text{e}}$ 2, 5        \\ % 6
\fa \neg p                             & $\neg_{\text{i}}$ 3-6         \\ % 7
\fa \neg p \vee q                      & $\vee_{\text{i$_{1}$}}$ 7     \\ % 8
\fa \bot                               & $\neg_{\text{e}}$ 2, 8        \\ % 9
\neg p \vee q                          & PBC (r.a.a.) 2-9              \\ % 10
\end{fitch}
\end{equation*}
Deze afleiding komt in structuur voor een groot deel overeen met de afleiding voor de regel LEM zoals
die gegeven is in het boek op pagina 25. Dat is natuurlijk geen toeval!

\paragraph{}

Met de regel LEM is een afleiding makkelijker:

\begin{equation*}
\begin{fitch}
p \rightarrow q                        & premisse                      \\ % 1
p \vee \neg p                          & LEM                           \\ % 2
\fh p                                  & assumptie                     \\ % 3
\fa q                                  & $\rightarrow_{\text{e}}$ 1, 3 \\ % 4
\fa \neg p \vee q                      & $\vee_{\text{i$_{2}$}}$ 4     \\ % 5
\fh \neg p                             & assumptie                     \\ % 6
\fa \neg p \vee q                      & $\vee_{\text{i$_{1}$}}$ 6     \\ % 7
\neg p \vee q                          & $\vee_{\text{e}}$ 2, 3-5, 6-7 \\ % 8
\end{fitch}
\end{equation*}


\end{document}
