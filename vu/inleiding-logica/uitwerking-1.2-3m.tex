\documentclass[a4paper,11pt]{article}
\usepackage[dutch]{babel}
\usepackage{a4,fullpage}
\usepackage{amsmath,amsfonts,amssymb}
\usepackage{fitch} % http://folk.uio.no/johanw/FitchSty.html

%\renewcommand{\familydefault}{\sfdefault}


\title{Uitwerking van opgave 3m\\
\normalsize{bij paragraaf 1.2 van Huth\&Ryan}}
%\date{Martijn Vermaat, 27 september 2007}
\date{}


\begin{document}

\maketitle


We geven een afleiding in natuurlijke decuctie voor de sequent $p \wedge q \rightarrow r \vdash (p \rightarrow r) \vee (q \rightarrow r)$:

\begin{equation*}
\begin{fitch}
p \wedge q \rightarrow r                      & premisse                        \\ % 1
p \vee \neg p                                 & LEM                             \\ % 2
\fh p                                         & assumptie                       \\ % 3
\fa \fh q                                     & assumptie                       \\ % 4
\fa \fa p \wedge q                            & $\wedge_{\text{i}}$ 3,4         \\ % 5
\fa \fa r                                     & $\rightarrow_{\text{e}}$ 5,1    \\ % 6
\fa q \rightarrow r                           & $\rightarrow_{\text{i}}$ 4-6    \\ % 7
\fa (p \rightarrow r) \vee (q \rightarrow r)  & $\vee_{\text{i$_{2}$}}$ 7       \\ % 8
\fh \neg p                                    & assumptie                       \\ % 9
\fa \fh p                                     & assumptie                       \\ % 10
\fa \fa \bot                                  & $\neg_{\text{e}}$ 10,9          \\ % 11
\fa \fa r                                     & $\bot_{\text{e}}$ 11            \\ % 12
\fa p \rightarrow r                           & $\rightarrow_{\text{i}}$ 10-12  \\ % 13
\fa (p \rightarrow r) \vee (q \rightarrow r)  & $\vee_{\text{i$_{1}$}}$ 13      \\ % 14
(p \rightarrow r) \vee (q \rightarrow r)      & $\vee_{\text{e}}$ 2,3-8,9-14    \\ % 15
\end{fitch}
\end{equation*}


\end{document}
