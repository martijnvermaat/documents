\documentclass[a4paper,11pt]{article}
\usepackage[dutch]{babel}
\usepackage{a4,fullpage}
\usepackage{amsmath,amsfonts,amssymb}

%\renewcommand{\familydefault}{\sfdefault}


\title{Uitwerkingen van opgaven 6 a en d\\
\normalsize{bij paragraaf 5.2 van Huth\&Ryan}}
%\date{Martijn Vermaat, 19 oktober 2006}
\date{}


\begin{document}

\maketitle


\begin{description}


\item{\bf (a)}
We moeten laten zien dat $\models \Box(\phi \wedge \psi) \leftrightarrow (\Box \phi \wedge \Box \psi)$ geldt. Dat
wil zeggen, dat deze formule waar is in iedere wereld in ieder model. We bekijken daarvoor een willekeurige wereld
$x$ in een willekeurig model $\mathcal{M} = (W, R, L)$.

Volgens definitie 5.4 geldt $\mathcal{M}, x \Vdash \Box(\phi \wedge \psi) \leftrightarrow (\Box \phi \wedge \Box \psi)$
dan en slechts dan als $\mathcal{M}, x \Vdash \Box(\phi \wedge \psi) \Leftrightarrow \mathcal{M}, x \Vdash (\Box \phi \wedge \Box \psi)$.
We beschouwen beide richtingen:

\begin{description}

\item{\bf $\Rightarrow$}
Wegens $\mathcal{M}, x \Vdash \Box(\phi \wedge \psi)$ geldt $\mathcal{M}, y \Vdash \phi \wedge \psi$ voor alle
werelden $y \in W$ met $R(x,y)$ (naar de betekenis van $\Box$, zie steeds definitie 5.4). De betekenis van
conjunctie zegt ons dat dan ook $\mathcal{M}, y \Vdash \phi$
voor alle $y \in W$ met $R(x,y)$ en $\mathcal{M}, y \Vdash \psi$ voor alle $y \in W$ met $R(x,y)$. Dit geeft
samen $\mathcal{M}, x \Vdash \Box \phi$ en $\mathcal{M}, x \Vdash \Box \psi$ en dus
$\mathcal{M}, x \Vdash (\Box \phi \wedge \Box \psi)$.

\item{\bf $\Leftarrow$}
Gegeven is nu dat $\mathcal{M}, x \Vdash (\Box \phi \wedge \Box \psi)$ en dus hebben we
$\mathcal{M}, x \Vdash \Box \phi$ en $\mathcal{M}, x \Vdash \Box \psi$ volgens de betekenis van conjunctie.
Maar dan geldt ook dat $\mathcal{M}, y \Vdash \phi$ voor alle werelden $y \in W$ met $R(x,y)$ en
$\mathcal{M}, y \Vdash \psi$ voor alle werelden $y \in W$ met $R(x,y)$. Samen geeft dat $\mathcal{M}, y \Vdash \phi \wedge \psi$
voor alle $y \in W$ met $R(x,y)$ en dat maakt precies dat $\mathcal{M}, x \Vdash \Box(\phi \wedge \psi)$.

\end{description}


\item{\bf (b)}
We moeten laten zien dat $\Diamond \perp \leftrightarrow \perp$ geldig is, dus dat dezelfde werelden
$\Diamond \perp$ en $\perp$ waar maken. Op de eerste plaats is er geen enkele wereld die $\perp$ waar
maakt (zie wederom steeds definitie 5.4). Kijken we naar $\Diamond \perp$, dan zien we dat deze formule
waar gemaakt wordt door precies de werelden van waaruit een wereld toegankelijk is die $\perp$ waar maakt.
Zoals al opgemerkt bestaat een dergelijke wereld niet, dus is er ook geen enkele wereld die $\Diamond \perp$
waar maakt. Hieruit volgt dat $\Diamond \perp \leftrightarrow \perp$ geldig is.


\end{description}


\end{document}
