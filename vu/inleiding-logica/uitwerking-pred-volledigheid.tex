\documentclass[a4paper,11pt]{article}
\usepackage[dutch]{babel}
\usepackage{a4,fullpage}
\usepackage{amsmath,amsfonts,amssymb}
\usepackage{amsthm}

%\renewcommand{\familydefault}{\sfdefault}


\title{Uitwerking van extra opgave\\
\normalsize{bij volledigheid van predikatenlogica}}
%\date{Martijn Vermaat, 30 november 2005}
\date{}


\begin{document}

\maketitle


\begin{quote}
  Bewijs dat een verzameling formules $\{\phi_{1}, \ldots, \phi_{n}\}$ een model
  heeft (``consistent'' of ``satisfiable'' is), precies dan als het niet
  mogelijk is er $\bot$ uit af te leiden. Dus:
  \begin{equation*}
    \{\phi_{1}, \ldots, \phi_{n}\} \text{ consistent}
    \quad \Leftrightarrow \quad
    \{\phi_{1}, \ldots, \phi_{n}\} \not \vdash \bot
  \end{equation*}
\end{quote}

\begin{proof}

\begin{description}

\item{$\Rightarrow$:} Laat $\mathcal{M}$ een model zijn voor
$\{\phi_{1}, \ldots, \phi_{n}\}$. We weten dat $\mathcal{M} \not \models \bot$
($\bot$ is in geen enkel model waar) en dus dat
$\{\phi_{1}, \ldots, \phi_{n}\} \not \models \bot$ ($\mathcal{M}$ is een
tegenmodel).

Hier uit volgt dat
$\{\phi_{1}, \ldots, \phi_{n}\} \not \vdash \bot$ (correctheidsstelling).

\item{$\Leftarrow$:} (Bewijs uit het ongerijmde.) Stel er is geen $\mathcal{M}$
met $\mathcal{M} \models \{\phi_{1}, \ldots, \phi_{n}\}$. Dan
\mbox{$\{\phi_{1}, \ldots, \phi_{n}\} \models \bot$} (een tegenmodel zou
$\{\phi_{1}, \ldots, \phi_{n}\}$ waar moeten maken) en dus
$\{\phi_{1}, \ldots, \phi_{n}\} \vdash \bot$ (volledigheidsstelling).
Tegenspraak.

We concluderen dat $\{\phi_{1}, \ldots, \phi_{n}\}$ wel een model heeft.

\end{description}

\end{proof}


\end{document}
