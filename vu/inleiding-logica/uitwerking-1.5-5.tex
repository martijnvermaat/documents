\documentclass[a4paper,11pt]{article}
\usepackage[dutch]{babel}
\usepackage{a4,fullpage}
\usepackage{amsmath,amsfonts,amssymb}
\usepackage{fitch} % http://folk.uio.no/johanw/FitchSty.html

%\renewcommand{\familydefault}{\sfdefault}


\title{Uitwerking van opgave 5\\
\normalsize{bij paragraaf 1.5 van Huth\&Ryan}}
%\date{Martijn Vermaat, 20 september 2005}
\date{}


\begin{document}

\maketitle


Te bewijzen is dat de relatie $\equiv$ (semantische equivalentie) een
equivalentierelatie is. Dat wil zeggen, dat $\equiv$ reflexief, symmetrisch
en transitief is. Er is gedefini\"eerd dat:

\begin{eqnarray*}
\phi \equiv \psi & \Longleftrightarrow & \phi \models \psi \\
                 &                     & \mbox{\`en} \\
                 &                     & \psi \models \phi
\end{eqnarray*}


\begin{description}

\item{\bf (a) Reflexiviteit}
We willen laten zien dat voor iedere $\phi$ geldt: $\phi \models \phi$ (en andersom, wat
in dit geval niet uit maakt). Dit is zo, wanneer voor iedere waardetoekenning waarbij
$\phi$ waar is ook $\phi$ waar is en dat is natuurlijk het geval.

\item{\bf (b) Symmetrie}
De relatie $\equiv$ is symmetrisch als $\phi \equiv \psi \, \Rightarrow \, \psi \equiv \phi$.

We nemen aan dat $\phi \equiv \psi$ en dus weten we dat $\phi \models \psi$ en
$\psi \models \phi$. Maar dan weten we ook dat $\psi \equiv \phi$.

\item{\bf (c) Transitiviteit}
Ten slotte is $\equiv$ transitief wanneer gegeven $\phi \equiv \psi$ en $\psi \equiv \eta$ ook
$\phi \equiv \eta$ waar is.

We weten dat voor iedere waardetoekenning waarbij $\phi$ waar is ook $\psi$ waar is (want
$\phi \models \psi$). Maar voor iedere waardetoekenning waarbij $\psi$ waar is, is ook $\eta$
waar (want $\psi \models \eta$). Dus is bij iedere waardetoekenning die $\phi$ waar maakt ook
$\eta$ waar en hebben we $\phi \models \eta$. Op dezelfde manier kunnen we beredeneren dat
$\eta \models \phi$ en dus is $\phi \equiv \eta$ waar.

\end{description}


\end{document}
