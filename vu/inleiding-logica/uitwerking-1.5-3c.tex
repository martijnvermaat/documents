\documentclass[a4paper,11pt]{article}
\usepackage[dutch]{babel}
\usepackage{a4,fullpage}
\usepackage{amsmath,amsfonts,amssymb,amsthm}

%\renewcommand{\familydefault}{\sfdefault}

\newtheorem{lemma}{Lemma}
\newtheorem*{theorem}{Stelling}


\title{Uitwerking van opgave 3c\\
\normalsize{bij paragraaf 1.5 van Huth\&Ryan}}
%\date{Martijn Vermaat, 4 september 2008}
\date{}


\begin{document}

\maketitle


\begin{theorem}\label{thm:incomplete}
  De verzameling connectieven $\{\neg, \leftrightarrow\}$ is niet functioneel
    volledig.
\end{theorem}

We bewijzen eerst het volgende lemma. Wanneer we in het vervolg
schrijven dat een formule $\phi$ verandert, bedoelen we dat de waarde van
$\phi$ onder een gegeven waardetoekenning verandert.

\begin{lemma}\label{lem:linear}
  Laat $\phi$ een formule zijn met uitsluitend connectieven uit
  $\{\neg, \leftrightarrow\}$.
  Voor iedere atomaire letter geldt dat bij een verandering van de
  waardetoekenning aan deze letter \`ofwel $\phi$ altijd verandert,
  \`ofwel $\phi$ nooit verandert.
\end{lemma}

\begin{proof}
  Het bewijs verloopt via inductie naar de structuur van $\phi$.

  \begin{description}
  \item[Basisgeval $\phi = p$]\hfill

    Triviaal.

  \item[Inductiestap $\phi = \neg \psi$]\hfill

    Direct via de inductiehypothese, $\phi$ verandert precies wanneer
    $\neg \psi$ dat doet.

  \item[Inductiestap $\phi = \psi \leftrightarrow \chi$]\hfill

    Volgens de inductiehypothesen kunnen we de volgende gevallen
    onderscheiden bij het veranderen van de waardetoekenning aan een atomaire
    letter:
    \begin{itemize}
      \item $\psi$ en $\chi$ veranderen altijd, $\phi$ verandert nooit.
      \item $\psi$ en $\chi$ veranderen nooit, $\phi$ verandert nooit.
      \item $\psi$ verandert altijd en $\chi$ verandert nooit, $\phi$ verandert altijd.
      \item $\psi$ verandert nooit en $\chi$ verandert altijd, $\phi$ verandert altijd.\qedhere
    \end{itemize}
\end{description}
\end{proof}

\begin{proof}[Bewijs van stelling]
  Stel $\{\neg, \leftrightarrow\}$ is functioneel volledig.
  Laat $\phi \equiv p \rightarrow q$ met alleen connectieven uit
  $\{\neg, \leftrightarrow\}$ in $\phi$.
  Verandering van de waardetoekenning aan $q$ geeft soms een verandering van
  $\phi$ en soms niet. Contradictie met lemma~\ref{lem:linear}.
\end{proof}


\end{document}
