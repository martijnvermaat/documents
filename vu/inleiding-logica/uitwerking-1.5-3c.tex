\documentclass[a4paper,11pt]{article}
\usepackage[dutch]{babel}
\usepackage{a4,fullpage}
\usepackage{amsmath,amsfonts,amssymb,amsthm}

%\renewcommand{\familydefault}{\sfdefault}

\newtheorem{lemma}{Lemma}
\newtheorem*{theorem}{Stelling}


\title{Uitwerking van opgave 3c\\
\normalsize{bij paragraaf 1.5 van Huth\&Ryan}}
%\date{Martijn Vermaat, 4 september 2008}
\date{}


\begin{document}

\maketitle


\begin{theorem}\label{thm:incomplete}
  De verzameling connectieven $\{\neg, \leftrightarrow\}$ is niet functioneel
    volledig.
\end{theorem}

\paragraph{Notatie}

Laat $v : \mathtt{atoms} \rightarrow \{\mathtt{T}, \mathtt{F}\}$ een
waardetoekenning zijn.

We schrijven $[\![ \phi ]\!]_{v}$ voor de waarde van $\phi$ onder $v$.
Met $v \diamond p$ bedoelen we de waardetoekenning $v$ met alleen de waarde voor $p$
veranderd:
\begin{equation*}
  v \diamond p \: (q) =
  \begin{cases}
    v(q)     & \mbox{als }q \neq p \\
    v(q)^{-1} & \mbox{anders}
  \end{cases}
\end{equation*}

\renewcommand{\labelitemi}{}
\begin{lemma}\label{lem:linear}
  Laat $\phi$ een welgevormde formule zijn over $\{\mathtt{atoms}, \neg, \leftrightarrow\}$.
  Voor iedere $p \in \mathtt{atoms}$ geldt
  \begin{itemize}
    \item $[\![ \phi ]\!]_{v} \neq [\![ \phi ]\!]_{v \diamond p}$ voor iedere waardetoekenning $v$, \`of
    \item $[\![ \phi ]\!]_{v} = [\![ \phi ]\!]_{v \diamond p}$ voor iedere waardetoekenning $v$.
  \end{itemize}
\end{lemma}

\begin{proof}
  Via inductie naar de structuur van $\phi$:

  \begin{description}
  \item[Basisgeval $\phi = q$]\hfill

    Triviaal.

  \item[Inductiestap $\phi = \neg \psi$]\hfill

    Direct via de inductiehypothese en $[\![ \phi ]\!]_{v} \neq [\![ \neg \psi ]\!]_{v}$
    voor willekeurige $v$.

  \item[Inductiestap $\phi = \psi \leftrightarrow \chi$]\hfill

    Beschouw $p \in \mathtt{atoms}$. De inductiehypothesen geven vier gevallen:
    \begin{enumerate}
      \item $[\![ \psi ]\!]_{u} = [\![ \psi ]\!]_{u \diamond p}$ en $[\![ \chi ]\!]_{w} = [\![ \chi ]\!]_{w \diamond p}$ voor alle $u,w$
        $\Rightarrow$
        $[\![ \phi ]\!]_{v} = [\![ \phi ]\!]_{v \diamond p}$ voor alle $v$
      \item $[\![ \psi ]\!]_{u} \neq [\![ \psi ]\!]_{u \diamond p}$ en $[\![ \chi ]\!]_{w} \neq [\![ \chi ]\!]_{w \diamond p}$ voor alle $u,w$
        $\Rightarrow$
        $[\![ \phi ]\!]_{v} = [\![ \phi ]\!]_{v \diamond p}$ voor alle $v$
      \item $[\![ \psi ]\!]_{u} = [\![ \psi ]\!]_{u \diamond p}$ en $[\![ \chi ]\!]_{w} \neq [\![ \chi ]\!]_{w \diamond p}$ voor alle $u,w$
        $\Rightarrow$
        $[\![ \phi ]\!]_{v} \neq [\![ \phi ]\!]_{v \diamond p}$ voor alle $v$
      \item $[\![ \psi ]\!]_{u} \neq [\![ \psi ]\!]_{u \diamond p}$ en $[\![ \chi ]\!]_{w} = [\![ \chi ]\!]_{w \diamond p}$ voor alle $u,w$
        $\Rightarrow$
        $[\![ \phi ]\!]_{v} \neq [\![ \phi ]\!]_{v \diamond p}$ voor alle $v$\qedhere
    \end{enumerate}
\end{description}
\end{proof}

\begin{proof}[Bewijs van stelling]
  Stel $\{\neg, \leftrightarrow\}$ is functioneel volledig.
  Laat $\phi$ een welgevormde formule zijn over $\{\mathtt{atoms}, \neg, \leftrightarrow\}$
  met $\phi \equiv p \rightarrow q$. Dan $[\![ \phi ]\!]_{v} = [\![ \phi ]\!]_{v \diamond p}$
  voor iedere $v$ met $v(q) = \mathtt{T}$ en
  $[\![ \phi ]\!]_{v} \neq [\![ \phi ]\!]_{v \diamond p}$
  voor iedere $v$ met $v(q) = \mathtt{F}$.
  Dit is in tegenspraak met lemma~\ref{lem:linear}.
\end{proof}


\end{document}
