\documentclass[a4paper,11pt]{article}
\usepackage[dutch]{babel}
\usepackage{a4,fullpage}
\usepackage{amsmath,amsfonts,amssymb}
\usepackage{fitch} % http://folk.uio.no/johanw/FitchSty.html

%\renewcommand{\familydefault}{\sfdefault}


\title{Uitwerking van opgave 2\\
\normalsize{bij paragraaf 1.5 van Huth\&Ryan}}
%\date{Martijn Vermaat, 20 september 2005}
\date{}


\begin{document}

\maketitle


Om te bepalen welke van de gegeven formules \textbf{(a)-(d)} semantisch equivalent
zijn aan de formule $p \rightarrow (q \vee r)$ maken we eerst de waarheidstafel voor
$p \rightarrow (q \vee r)$:

\paragraph{}

\begin{tabular}{c|c|c|c|c}
$p$   & $q$   & $r$   & $q \vee r$ & $p \rightarrow (q \vee r)$ \\
\hline
\tt T & \tt T & \tt T & \tt T      & \tt T \\
\tt F & \tt T & \tt T & \tt T      & \tt T \\
\tt T & \tt F & \tt T & \tt T      & \tt T \\
\tt F & \tt F & \tt T & \tt T      & \tt T \\
\tt T & \tt T & \tt F & \tt T      & \tt T \\
\tt F & \tt T & \tt F & \tt T      & \tt T \\
\tt T & \tt F & \tt F & \tt F      & \tt F \\
\tt F & \tt F & \tt F & \tt F      & \tt T
\end{tabular}

\paragraph{}

\begin{description}

\item{\bf (a)} $q \vee (\neg p \vee r)$

De waarheidstafel voor $q \vee (\neg p \vee r)$ ziet er als volgt uit:

\paragraph{}

\begin{tabular}{c|c|c|c|c}
$p$   & $q$   & $r$   & $\neg p \vee r$ & $q \vee (\neg p \vee r)$ \\
\hline
\tt T & \tt T & \tt T & \tt T           & \tt T \\
\tt F & \tt T & \tt T & \tt T           & \tt T \\
\tt T & \tt F & \tt T & \tt T           & \tt T \\
\tt F & \tt F & \tt T & \tt T           & \tt T \\
\tt T & \tt T & \tt F & \tt F           & \tt T \\
\tt F & \tt T & \tt F & \tt T           & \tt T \\
\tt T & \tt F & \tt F & \tt F           & \tt F \\
\tt F & \tt F & \tt F & \tt T           & \tt T
\end{tabular}

\paragraph{}

We zien dat dit dezelfde waarheidstafel is, dus concluderen we dat \textbf{(a)}
inderdaad semantisch equivalent is aan de gegeven formule.

\item{\bf (b)} $q \wedge \neg r \rightarrow p$

Bekijk de volgende waardetoekenning: $p$ := \texttt{F}, $q$ := \texttt{T}, $r$ := \texttt{F}.

De bijbehorende regel in de waarheidstafel wijkt af van die voor de gegeven formule, dus
concluderen we direct dat \textbf{(b)} niet semantisch equivalent is.

\begin{tabular}{c|c|c|c|c|c}
$p$   & $q$   & $r$   & $\neg r$ & $q \wedge \neg r$ & $q \wedge \neg r \rightarrow p$ \\
\hline
\tt F & \tt T & \tt F & \tt T    & \tt T             & \tt F
\end{tabular}

\item{\bf (c)} $p \wedge \neg r \rightarrow q$

We maken de waarheidstafel voor $p \wedge \neg r \rightarrow q$:

\paragraph{}

\begin{tabular}{c|c|c|c|c|c}
$p$   & $q$   & $r$   & $\neg r$ & $p \wedge \neg r$ & $p \wedge \neg r \rightarrow q$\\
\hline
\tt T & \tt T & \tt T & \tt F    & \tt F             & \tt T \\
\tt F & \tt T & \tt T & \tt F    & \tt F             & \tt T \\
\tt T & \tt F & \tt T & \tt F    & \tt F             & \tt T \\
\tt F & \tt F & \tt T & \tt F    & \tt F             & \tt T \\
\tt T & \tt T & \tt F & \tt T    & \tt T             & \tt T \\
\tt F & \tt T & \tt F & \tt T    & \tt F             & \tt T \\
\tt T & \tt F & \tt F & \tt T    & \tt T             & \tt F \\
\tt F & \tt F & \tt F & \tt T    & \tt F             & \tt T
\end{tabular}

\paragraph{}

Deze waarheidstafel is gelijk aan die van de gegeven formule, dus is \textbf{(c)}
semantisch equivalent.

\item{\bf (d)} $\neg q \wedge \neg r \rightarrow \neg p$

Formule \textbf{(d)} is semantisch equivalent aan de gegeven formule. We bekijken hiervoor
weer de waarheidstafel:

\paragraph{}

\begin{tabular}{c|c|c|c|c|c|c|c}
$p$   & $q$   & $r$   & $\neg p$ & $\neg q$ & $\neg r$ & $\neg q \wedge \neg r$ & $\neg q \wedge \neg r \rightarrow \neg p$\\
\hline
\tt T & \tt T & \tt T & \tt F    & \tt F    & \tt F    & \tt F                  & \tt T \\
\tt F & \tt T & \tt T & \tt T    & \tt F    & \tt F    & \tt F                  & \tt T \\
\tt T & \tt F & \tt T & \tt F    & \tt T    & \tt F    & \tt F                  & \tt T \\
\tt F & \tt F & \tt T & \tt T    & \tt T    & \tt F    & \tt F                  & \tt T \\
\tt T & \tt T & \tt F & \tt F    & \tt F    & \tt T    & \tt F                  & \tt T \\
\tt F & \tt T & \tt F & \tt T    & \tt F    & \tt T    & \tt F                  & \tt T \\
\tt T & \tt F & \tt F & \tt F    & \tt T    & \tt T    & \tt T                  & \tt F \\
\tt F & \tt F & \tt F & \tt T    & \tt T    & \tt T    & \tt T                  & \tt T
\end{tabular}

\end{description}


\end{document}
