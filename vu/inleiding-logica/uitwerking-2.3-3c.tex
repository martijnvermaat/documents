\documentclass[a4paper,11pt]{article}
\usepackage[dutch]{babel}
\usepackage{a4,fullpage}
\usepackage{amsmath,amsfonts,amssymb}
\usepackage{amsthm}

%\renewcommand{\familydefault}{\sfdefault}


\title{Uitwerking van opgave 3 c\\
\normalsize{bij paragraaf 2.3 van Huth\&Ryan}}
%\date{Martijn Vermaat, 30 november 2005}
\date{}


\begin{document}

\maketitle


\begin{quote}
  Bewijs dat er geen predikatenlogische formule $\phi$ bestaat zo dat geldt:
  \begin{equation*}
    \mathcal{M} \models \phi
    \quad \Leftrightarrow \quad
    \text{het domein van $\mathcal{M}$ is eindig}
  \end{equation*}
\end{quote}

\begin{proof}
Stel dat dergelijke $\phi$ wel bestaat. Voor iedere $n \in \mathbb{N}$
($\geq 1$) is er een model voor $\phi$ met $n$ elementen in het domein. Volgens
de Skolem-L\"owenheim stelling heeft $\phi$ een model $\mathcal{M}$ met een
oneindig groot domein. Tegenspraak (de definitie van $\phi$ zegt dat het domein
van $\mathcal{M}$ eindig is).

Er bestaat dus geen formule $\phi$ volgens bovenstaande definitie.
\end{proof}


\begin{quote}
  Bewijs dat er ook geen verzameling predikatenlogische formules $\Sigma$
  bestaat zo dat geldt:
  \begin{equation*}
    \mathcal{M} \models \Sigma
    \quad \Leftrightarrow \quad
    \text{het domein van $\mathcal{M}$ is eindig}
  \end{equation*}
\end{quote}

De Skolem-L\"owenheim stelling wordt in Huth\&Ryan niet gegeven voor het
algemenere geval van een verzameling formules. Na bestudering van het bewijs bij
deze stelling kunnen we concluderen dat de stelling net zo goed op gaat voor het
algemenere geval en eenzelfde bewijs geven als bij de vorige opgave.

Instructiever is wellicht het volgende bewijs met de compactheidsstelling (dat
overigens ook een bewijs is voor de vorige opgave).

\begin{proof}
Stel dat dergelijke verzameling $\Sigma$ wel bestaat. De formule
$\lambda_{n} \overset{\text{\tiny{def}}}{=} \exists x_{1} \ldots \exists x_{n} \bigwedge_{i \ne j} x_{i} \ne x_{j}$
zegt dat er ten minste $n$ elementen in een model zijn ($n > 1$). Laat
$\Gamma = \Sigma \cup \{\lambda_{n} | n>1\}$.

We bekijken een willekeurige eindige deelverzameling $\Delta$ van $\Gamma$. Er
bestaat een $m$ groter dan alle $n$ met $\lambda_{n}$ in $\Delta$ ($\Delta$ is
immers eindig). Ieder model met ten minste $m$ elementen is nu een model voor
$\Delta$ (het is immers een model voor $\Sigma$ en voor alle $\lambda_{n}$ in
$\Delta$).

De compactheidsstelling zegt dat $\Gamma$ een model $\mathcal{M}$ heeft.
Volgens de formules $\lambda_{n}$ is het domein van $\mathcal{M}$ oneindig
groot. Maar omdat $\mathcal{M}$ ook een model voor $\Sigma$ is, heeft dit domein
per definitie van $\Sigma$ eindig veel elementen. Dit is een tegenspraak, dus
kan de verzameling $\Sigma$ niet bestaan.
\end{proof}


\end{document}
