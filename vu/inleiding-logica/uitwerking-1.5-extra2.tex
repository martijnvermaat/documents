\documentclass[a4paper,11pt]{article}
\usepackage[dutch]{babel}
\usepackage{a4,fullpage}
\usepackage{amsmath,amsfonts,amssymb}
\usepackage{fitch} % http://folk.uio.no/johanw/FitchSty.html

%\renewcommand{\familydefault}{\sfdefault}


\title{Uitwerking van extra opgave 2\\
\normalsize{bij paragraaf 1.5 van Huth\&Ryan}}
%\date{Martijn Vermaat, 20 september 2005}
\date{}


\begin{document}

\maketitle


Vergelijk \textbf{A} en \textbf{B}:

\begin{description}

\item{\bf A:}
$\models \phi \rightarrow \psi$

\item{\bf B:}
Als $\models \phi$ dan $\models \psi$

\end{description}

Geldt \textbf{A} $\Rightarrow$ \textbf{B}? En \textbf{B} $\Rightarrow$ \textbf{A}?

\paragraph{}

We proberen een waarheidstafel voor $\phi \rightarrow \psi$ te maken:

\paragraph{}

\begin{tabular}{c|c|c}
$\phi$   & $\psi$   & $\phi \rightarrow \psi$ \\
\hline
\tt T    & \tt T    & \tt T \\
\tt F    & \tt T    & \tt T \\
\tt T    & \tt F    & \tt F \\
\tt F    & \tt F    & \tt T
\end{tabular}

\paragraph{}

Wanneer we nu \textbf{A} aannemen weten we dat $\phi \rightarrow \psi$
een tautologie is en de waarheidstafel dus geen regels bevat
die \texttt{F} geven. Regel drie uit bovenstaande tafel komt hierin
dus niet voor (er is blijkbaar geen waardentoekenning mogelijk die
$\phi$ waar maakt en $\psi$ niet waar).

Stel nu dat $\models \phi$. Dan vervallen ook regels twee en vier
uit de waarheidstafel en blijft regel \'e\'en over. Maar dan zien
we dat ook $\psi$ een tautologie is en dus hebben we $\models \psi$.

Hiermee hebben we laten zien dat \textbf{A} $\Rightarrow$ \textbf{B} geldt.

\paragraph{}

Maar \textbf{B} $\Rightarrow$ \textbf{A} geldt niet. We geven een tegenvoorbeeld.
Neem als instanties $\phi := p \vee q$ en $\psi := p$.

Nu is stelling \textbf{B} waar, omdat $\phi$ geen tautologie is. Maar \textbf{A}
is niet waar want $p \vee q \rightarrow p$ is geen tautologie. Dus geldt
niet \textbf{B} $\Rightarrow$ \textbf{A}.


\end{document}
