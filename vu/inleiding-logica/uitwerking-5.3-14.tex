\documentclass[a4paper,11pt]{article}
\usepackage[dutch]{babel}
\usepackage{a4,fullpage}
\usepackage{amsmath,amsfonts,amssymb}

%\renewcommand{\familydefault}{\sfdefault}


\title{Uitwerking van opgave 14\\
\normalsize{bij paragraaf 5.3 van Huth\&Ryan}}
%\date{Martijn Vermaat, 18 oktober 2005}
\date{}


\begin{document}

\maketitle


Een voorbeeld van een frame met een reflexieve, transitieve, maar niet
symmetrische toegankelijkheidsrelatie is het het frame $\mathcal{F} = (W, R)$
met
\begin{align*}
W &= \{x, y\} \\
R &= \{(x, x), (x, y), (y, y)\}
\end{align*}
(Teken het frame en controleer de eigenschappen van $R$!)

\paragraph{}

We laten zien dat $p \rightarrow \Box \Diamond p$ niet geldig is in
$\mathcal{F}$ \footnote{We weten dat dit zo moet zijn omdat $R$ niet reflexief
is (zie opgave 7 van \S5.3).}.

Daartoe kiezen we een labeling $L$ met $L(x) = \{p\}$ en $L(y) = \{\}$ en
bekijken we of $x \Vdash p \rightarrow \Box \Diamond p$. We zien dat
$x \Vdash p$, maar $x \not \Vdash \Box \Diamond p$. Er is namelijk een wereld
$y$ bereikbaar vanuit $x$ met $y \not \Vdash \Diamond p$ (de enige wereld
bereikbaar vanuit $y$ maakt $p$ niet waar). Hieruit volgt dat
$p \rightarrow \Box \Diamond p$ niet waar is in $x$ en dus niet geldig in
$\mathcal{F}$.

\paragraph{}

Gevraagd wordt nu of we een labeling op $\mathcal{F}$ en een wereld in $W$
kunnen vinden zodat deze wereld $p \rightarrow \Box \Diamond p$ waar maakt. We
gebruiken dezelfde labeling $L$ als voorheen en bekijken wereld $y$. Hierin
is $p$ niet waar en dus hebben we gemakkelijk
$y \Vdash p \rightarrow \Box \Diamond$.


\end{document}
