\documentclass[a4paper,11pt]{article}
\usepackage[dutch]{babel}
\usepackage{a4,fullpage}
\usepackage{amsmath,amsfonts,amssymb}

%\renewcommand{\familydefault}{\sfdefault}


\title{Uitwerking van opgave 7\\
\normalsize{bij paragraaf 5.3 van Huth\&Ryan}}
%\date{Martijn Vermaat, 18 oktober 2005}
\date{}


\begin{document}

\maketitle


We bewijzen stelling 5.14 voor geval B (symmetrie), dat wil zeggen:

\begin{quote}
  Het schema $\phi \rightarrow \Box \Diamond \phi$ is geldig in een frame
  $\mathcal{F} = (W, R)$ \\
  $\Longleftrightarrow$ \\
  $R$ is symmetrisch
\end{quote}


\begin{description}

\item{\bf (a)}
We bewijzen eerst dat het schema $\phi \rightarrow \Box \Diamond \phi$ geldig
is in een frame als de relatie van het frame symmetrisch is.

We nemen aan dat $R$ symmetrisch is. Neem nu een labeling functie $L$ en een
verzameling werelden $W$ zodat $\mathcal{M} = (W, R, L)$ een model is. We
laten zien dat $\mathcal{M} \Vdash \phi \rightarrow \Box \Diamond \phi$.

Kies een willekeurige $x$ uit $W$ en neem aan dat $x \Vdash \phi$. We zien nu
dat $y \Vdash \Diamond \phi$ voor iedere $y$ met $R(x, y)$:
\begin{quote}
  Omdat $R$ symmetrisch is, hebben we ook $R(y, x)$. En omdat $x \Vdash \phi$
  (onze aanname) volgt hieruit dat $y \Vdash \Diamond \phi$.
\end{quote}
Hieruit volt $x \Vdash \phi \rightarrow \Box \Diamond \phi$ en omdat $x$
willekeurig is ook $\mathcal{M} \Vdash \phi \rightarrow \Box \Diamond \phi$.

Merk op dat we niets aangenomen hebben over $\mathcal{M}$, behalve dat $R$
symmetrisch is. Dus is $\phi \rightarrow \Box \Diamond \phi$ waar in ieder
model op ieder frame met een symmetrische relatie en dus geldig in al deze
frames.

\item{\bf (b)}
We bewijzen vervolgens dat het schema $\phi \rightarrow \Box \Diamond \phi$
niet geldig is in een frame als de relatie van het frame niet symmetrisch is.

Neem een willekeurig frame $\mathcal{F} = (W, R)$ dat niet symmetrisch is. Dan
zijn er dus twee werelden $x$ en $y$ in $W$ zodat $R(x, y)$ en niet $R(y, x)$
\footnote{Merk op dat dit niet de enige werelden in $W$ en paren in $R$ hoeven
te zijn. We weten echter wel dat tenminste $x$ en $y$ bestaan, dat $R(x, y)$
en dat zeker niet $R(y, x)$.}.

We laten nu zien dat $\phi \rightarrow \Box \Diamond \phi$ niet geldig is in
$\mathcal{F}$. Daartoe kiezen we een labeling functie $L$ zodat
$\mathcal{M} = (W, R, L)$ een model is met:
\begin{quote}
$p$ is waar in wereld $x$ (en nergens anders)
\end{quote}

Nu hebben we $x \Vdash p$, maar ook $y \not \Vdash \Diamond p$. Want stel dat
$y \Vdash \Diamond p$, dan moet er een wereld $z$ zijn met $R(y, z)$ en
$z \Vdash p$. Dit kan echter niet, want de enige wereld waar $p$ waar is is
wereld $x$ en we weten dat niet $R(y, x)$.

Hieruit volgt dat we niet voor alle werelden $u$ met $R(x, u)$
$u \Vdash \Diamond p$ hebben en dus $x \not \Vdash \Box \Diamond p$. Samen met
$x \Vdash p$ betekent dit dat $x \not \Vdash p \rightarrow \Box \Diamond p$.
Dit is een instantie van het schema en dus weten we ook dat
$x \not \Vdash \phi \rightarrow \Box \Diamond \phi$.

Hiermee hebben we laten zien dat er een wereld in een model op $\mathcal{F}$
is waar $\phi \rightarrow \Box \Diamond \phi$ niet waar is dus dat
$\phi \rightarrow \Box \Diamond \phi$ niet geldig is in $\mathcal{F}$.

\end{description}


\end{document}
