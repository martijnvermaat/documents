\documentclass[a4paper,11pt]{article}
\usepackage[dutch]{babel}
\usepackage{a4,fullpage}
\usepackage{amsmath,amsfonts,amssymb}
\usepackage{fitch}

\renewcommand{\familydefault}{\sfdefault}


\title{Inleiding Theoretische Informatica -- Equationele Specificaties\\
\normalsize{Antwoorden bij extra opgaven}}
\author{Martijn Vermaat (mvermaat@cs.vu.nl)}
\date{5 april 2006}


\begin{document}

\maketitle


\section*{Opgave 1}

\begin{description}

\item{\bf a.}
\begin{equation*}
\begin{fitch}
\texttt{add(x,0) = x}                            & \texttt{[E1]} \\ %1
\texttt{add(add(succ(z),0),0) = add(succ(z),0)}  & subst, 1      \\ %2
\texttt{add(succ(z),0) = succ(z)}                & subst, 1      \\ %3
\texttt{add(add(succ(z),0),0) = succ(z)}         & trans, 2, 3   \\ %4

\texttt{add(x,succ(y)) = succ(add(x,y))}         & \texttt{[E2]} \\ %5
\texttt{add(z,succ(0)) = succ(add(z,0))}         & subst, 5      \\ %6
\texttt{add(z,0) = z}                            & subst, 1      \\ %7
\texttt{succ(add(z,0)) = succ(z)}                & congr, 7      \\ %8
\texttt{add(z,succ(0)) = succ(z)}                & trans, 6, 8   \\ %9
\texttt{succ(z) = add(z,succ(0))}                & symm, 9       \\ %10

\texttt{add(add(succ(z),0),0) = add(z,succ(0))}  & trans, 4, 10     %11
\end{fitch}
\end{equation*}

\item{\bf b.}
\begin{verbatim}
module Spec

  sorts object

  functions
    0    :                 -> object
    succ : object          -> object
    pred : object          -> object
    add  : object # object -> object
    sub  : object # object -> object

  equations
    [E1] : add(x,0)       = x
    [E2] : add(x,succ(y)) = succ(add(x,y))
    [E3] : add(x,pred(y)) = pred(add(x,y))
    [E4] : succ(pred(x))  = x
    [E5] : pred(succ(x))  = x
    [E6] : sub(x,0)       = x
    [E7] : sub(x,succ(y)) = pred(sub(x,y))
    [E8] : sub(x,pred(y)) = succ(sub(x,y))

end 
\end{verbatim}

\end{description}


\section*{Opgave 2}

\begin{description}

\item{\bf a.}
\begin{equation*}
\begin{fitch}
\texttt{s(h(x)) = s(x)}                 & \texttt{[E2]}  \\ % 1
\texttt{s(s(h(x))) = s(s(x))}           & congr, 1       \\ % 2
\texttt{s(s(h(a))) = s(s(a))}           & subst, 2       \\ % 3
\texttt{s(h(s(a))) = s(s(a))}           & subst, 1       \\ % 4
\texttt{s(s(a)) = s(h(s(a)))}           & symm, 4        \\ % 5
\texttt{s(s(h(a))) = s(h(s(a)))}        & trans, 3, 5       % 6
\end{fitch}
\end{equation*}

\item{\bf b.}
De algebra $\mathfrak{K}$ is geen model voor de specificatie $\texttt{Spec}$
omdat de vergelijking $\texttt{[E1]}$ hierin niet waar is. Kies bijvoorbeeld
de assignment $\theta$ met $\theta(x) = 3$. We zien dan dat
\begin{align*}
\bar \theta(\texttt{h(h(x))}) &= 3+2 \\
                              &= 5
\end{align*}
terwijl
\begin{align*}
\bar \theta(\texttt{x}) &= 3.
\end{align*}

\item{\bf c.}
We bekijken voor de overige algebra's of ze initiele modellen zijn voor $\texttt{Spec}$.

\paragraph{De algebra $\mathfrak{L}$}

Deze algebra is geen initieel model voor $\texttt{Spec}$, omdat er junk
is. Voor alle getallen ongelijk aan 0 geldt namelijk dat ze niet de
interpretatie zijn van een gesloten term.

Bovendien bevat deze algebra confusion, omdat iedere gesloten term
ge\"interpreteerd wordt als het getal 0. Dit betekent dat iedere vergelijking
van gesloten termen waar is in $\mathfrak{L}$, terwijl bijvoorbeeld
$\texttt{s(x) = x}$ niet afleidbaar is in $\texttt{Spec}$.

\paragraph{De algebra $\mathfrak{M}$}

Deze algebra is geen initieel model voor $\texttt{Spec}$, omdat er confusion
aanwezig is. De termen $\texttt{a}$ en $\texttt{h(a)}$ worden namelijk gelijk
ge\"interpreteerd, terwijl ze niet als gelijk kunnen worden afgeleid in de
specificatie $\texttt{Spec}$.

\paragraph{De algebra $\mathfrak{N}$}

Deze algebra is geen initieel model voor $\texttt{Spec}$, omdat er
confusion aanwezig is. Bijvoorbeeld de termen $\texttt{x}$ en $\texttt{h(x)}$
worden ge\"identificeerd, terwijl deze gelijkheid niet afleidbaar is in
$\texttt{Spec}$.

\end{description}


\end{document}
