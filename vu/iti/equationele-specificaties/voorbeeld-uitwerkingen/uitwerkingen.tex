\documentclass[a4paper,11pt]{article}
\usepackage[dutch]{babel}
\usepackage{a4,fullpage}
\usepackage{amsmath,amsfonts,amssymb}
\usepackage{fitch}

\renewcommand{\familydefault}{\sfdefault}


\title{Inleiding Theoretische Informatica -- Equationele Specificaties\\
\normalsize{Uitwerkingen bij opgaven 3.1, 3.4, 5.1, 5.5, 6.1}}
\author{Martijn Vermaat (mvermaat@cs.vu.nl)}
\date{5 april 2006}


\begin{document}

\maketitle


\section*{Opgave 3.1}

\begin{description}

\item{\bf (a)}
Een afleiding in de specificate $\texttt{Stack-Of-Data}$ van de vergelijking
\begin{quote}
\begin{verbatim}
pop(pop(pop(empty))) = empty
\end{verbatim}
\end{quote}
\begin{equation*}
\begin{fitch}
\texttt{pop(empty) = empty}                        & \texttt{[1]}  \\ % 1
\texttt{pop(pop(empty)) = pop(empty)}              & congr, 1      \\ % 2
\texttt{pop(pop(pop(empty))) = pop(pop(empty))}    & congr, 2      \\ % 3
\texttt{pop(pop(empty))) = empty}                  & trans, 1, 2   \\ % 4
\texttt{pop(pop(pop(empty))) = empty}              & trans, 3, 4      % 5
\end{fitch}
\end{equation*}

\item{\bf (b)}
Een afleiding in de specificatie \texttt{Stack-Of-Data} van de vergelijking
\begin{quote}
\begin{verbatim}
pop(push(x, pop(push(x, empty)))) = empty
\end{verbatim}
\end{quote}
\begin{equation*}
\begin{fitch}
\texttt{pop(push(x, s)) = s}                                     & \texttt{[3]}  \\ % 1
\texttt{pop(push(x, empty)) = empty}                             & subs, 1       \\ % 2
\texttt{pop(push(x, pop(push(x, empty)))) = pop(push(x, empty))} & subs, 1       \\ % 3
\texttt{pop(push(x, pop(push(x, empty)))) = empty}               & trans, 3, 2      % 4
\end{fitch}
\end{equation*}

\end{description}


\section*{Opgave 3.4}

Om te laten zien dat $\sim$ een equivalentierelatie is, laten we zien dat $\sim$
transitief, symmetrisch en reflexief is.

\begin{description}

\item{Transitiviteit}

  Stel $t \sim u$ en $u \sim v$, dan is volgens de definitie $E \vdash t = u$ en
  $E \vdash u = v$. Wegens de regel \emph{transitiviteit} van de
  afleidbaarheid, is dan ook $E \vdash t = v$. Met de definitie zien we dan
  weer dat ook $t \sim v$ waar is en dus is $\sim$ transitief.

\item{Symmetrie}

  Stel $t \sim u$, dan is volgens de definitie $E \vdash t = u$. De
  afleidingsregel \emph{symmetrie} zegt dat dan ook $E \vdash u = t$ en
  volgens de definitie zien we dan weer dat ook $u \sim t$. En dus is $\sim$
  symmetrisch.

\item{Reflexiviteit}

  Voor iedere term $t$ geldt volgens de afleidingsregel \emph{reflexiviteit}
  dat $E \vdash t = t$. Onze definitie zegt dan dat ook $t \sim t$ en dus is
  $\sim$ reflexief.

\end{description}

Omdat $\sim$ transitief, symmetrisch en reflexief is, is $\sim$ een
equivalentierelatie.


\section*{Opgave 5.1}

De algebra $\mathfrak{P}(V)$ (met $V$ een verzameling) heeft als drager de
machtsverzameling van $V$. De functies worden ge\"interpreteerd als
\begin{align*}
\texttt{true}_{\mathfrak{P}(V)}     &= V , \\
\texttt{false}_{\mathfrak{P}(V)}    &= \{\} , \\
\texttt{not}_{\mathfrak{P}(V)}(X)   &= V - X , \\
\texttt{and}_{\mathfrak{P}(V)}(X,Y) &= X \cap Y , \\
\texttt{or}_{\mathfrak{P}(V)}(X,Y)  &= X \cup Y .
\end{align*}

We constateren eerst dat $\mathfrak{P}(V)$ een algebra voor
$\texttt{Booleans}$ is omdat de functietypes kloppen. Vervolgens laten we zien
dat $\mathfrak{P}(V)$ zelfs een model is voor $\texttt{Booleans}$ door te
laten zien dat alle vergelijkingen ($\texttt{[B1]} \ldots \texttt{[B5]}$) waar
zijn in $\mathfrak{P}(V)$.

\paragraph{}

Laat $\theta$ een assignment zijn van elementen uit $\mathcal{P}(V)$ aan
de variabelen $x$ en $y$, volgens
\begin{align*}
\theta    &: \{x,y\} \rightarrow \mathcal{P}(V), \\
\theta(x) &= X, \\
\theta(y) &= Y.
\end{align*}

We laten nu zien dat voor iedere vergelijking uit $\texttt{Booleans}$ de
rechter kant identiek is aan de linker kant onder $\bar \theta$.

\begin{itemize}

\item{$\texttt{[B1] and(true,x) = x}$}
\begin{align*}
\bar \theta(\texttt{and}(\texttt{true}, x)) &= \texttt{and}_{\mathfrak{P}(V)}(\bar \theta(\texttt{true}), \bar \theta(x)) \\
                                            &= \texttt{and}_{\mathfrak{P}(V)}(\texttt{true}_{\mathfrak{P}(V)}, \theta(x)) \\
                                            &= \texttt{and}_{\mathfrak{P}(V)}(V,X) \\
                                            &= V \cap X \\
                                            &= X
\end{align*}

\begin{align*}
\bar \theta(x) &= \theta(x) \\
               &= X
\end{align*}

\item{$\texttt{[B2] and(false,x) = false}$}
\begin{align*}
\bar \theta(\texttt{and}(\texttt{false}, x)) &= \texttt{and}_{\mathfrak{P}(V)}(\bar \theta(\texttt{false}), \bar \theta(x)) \\
                                             &= \texttt{and}_{\mathfrak{P}(V)}(\texttt{false}_{\mathfrak{P}(V)}, \theta(x)) \\
                                             &= \texttt{and}_{\mathfrak{P}(V)}(\{\},X) \\
                                             &= \{\} \cap X \\
                                             &= \{\}
\end{align*}

\begin{align*}
\bar \theta(\texttt{false}) &= \texttt{false}_{\mathfrak{P}(V)} \\
                            &= \{\}
\end{align*}

\item{$\texttt{[B3] not(true) = false}$}
\begin{align*}
\bar \theta(\texttt{not(true)}) &= \texttt{not}_{\mathfrak{P}(V)}(\bar \theta(\texttt{true})) \\
                                &= \texttt{not}_{\mathfrak{P}(V)}(\texttt{true}_{\mathfrak{P}(V)}) \\
                                &= \texttt{not}_{\mathfrak{P}(V)}(V) \\
                                &= V - V \\
                                &= \{\}
\end{align*}

\begin{align*}
\bar \theta(\texttt{false}) &= \texttt{false}_{\mathfrak{P}(V)} \\
                            &= \{\}
\end{align*}

\item{$\texttt{[B4] not(false) = true}$}
\begin{align*}
\bar \theta(\texttt{not(false)}) &= \texttt{not}_{\mathfrak{P}(V)}(\bar \theta(\texttt{false})) \\
                                 &= \texttt{not}_{\mathfrak{P}(V)}(\texttt{false}_{\mathfrak{P}(V)}) \\
                                 &= \texttt{not}_{\mathfrak{P}(V)}(\{\}) \\
                                 &= V - \{\} \\
                                 &= V
\end{align*}

\begin{align*}
\bar \theta(\texttt{true}) &= \texttt{true}_{\mathfrak{P}(V)} \\
                           &= V
\end{align*}

\item{$\texttt{[B5] or(x,y) = not(and(not(x),not(y)))}$}
\begin{align*}
\bar \theta(\texttt{or}(x,y)) &= \texttt{or}_{\mathfrak{P}(V)}(\bar \theta(x), \bar \theta(y)) \\
                              &= \texttt{or}_{\mathfrak{P}(V)}(\theta(x), \theta(y)) \\
                              &= \texttt{or}_{\mathfrak{P}(V)}(X,Y) \\
                              &= X \cup Y
\end{align*}

\begin{align*}
\bar \theta(\texttt{not}(\texttt{and}(\texttt{not}(x), \texttt{not}(y))))
      &= \texttt{not}_{\mathfrak{P}(V)}(\bar \theta(\texttt{and}(\texttt{not}(x), \texttt{not}(y)))) \\
      &= \texttt{not}_{\mathfrak{P}(V)}(\texttt{and}_{\mathfrak{P}(V)}(\bar \theta(\texttt{not}(x)), \bar \theta(\texttt{not}(y)))) \\
      &= \texttt{not}_{\mathfrak{P}(V)}(\texttt{and}_{\mathfrak{P}(V)}(\texttt{not}_{\mathfrak{P}(V)}(\bar \theta(x)), \texttt{not}_{\mathfrak{P}(V)}(\bar \theta(y)))) \\
      &= \texttt{not}_{\mathfrak{P}(V)}(\texttt{and}_{\mathfrak{P}(V)}(\texttt{not}_{\mathfrak{P}(V)}(\theta(x)), \texttt{not}_{\mathfrak{P}(V)}(\theta(y)))) \\
      &= \texttt{not}_{\mathfrak{P}(V)}(\texttt{and}_{\mathfrak{P}(V)}(\texttt{not}_{\mathfrak{P}(V)}(X), \texttt{not}_{\mathfrak{P}(V)}(Y))) \\
      &= \texttt{not}_{\mathfrak{P}(V)}(\texttt{and}_{\mathfrak{P}(V)}(V - X, V - Y)) \\
      &= \texttt{not}_{\mathfrak{P}(V)}((V - X) \cap (V - Y)) \\
      &= V - ((V - X) \cap (V - Y)) \\
      &= X \cup Y
\end{align*}

\end{itemize}

Hiermee hebben we laten zien dat iedere vergelijking in $\texttt{Booleans}$
waar is in $\mathfrak{P}(V)$ en dus dat $\mathfrak{P}(V)$ een model is voor de
specificatie $\texttt{Booleans}$.


\section*{Opgave 5.5}

Om te laten zien dat een bepaalde vergelijking niet afleidbaar is uit een
specificatie kunnen we een tegenmodel construeren. Dit is een model voor de
specificatie waarin de gegeven vergelijking niet waar is. Volgens de
correctheidsstelling is deze vergelijking dan ook niet afleidbaar uit de
specificatie.

We volgen de hint op die bij de opgave gegeven wordt. We beschouwen de algebra
$\mathfrak{M}$ voor de specificatie $\texttt{NatBool}$ met drager $A$ en
interpretaties als we gewend zijn, behalve voor interpretaties hier onder
gedefini\"eerd.

\begin{align*}
A_{\texttt{nat}} &= \{ \omega, 0, 1, 2, 3, \ldots \} \\
A_{\texttt{bool}} &= \{T, F\}
\end{align*}

Voor alle $m,n \in \mathbb{N}$ defini\"eren we interpretaties als volgt.

\begin{align*}
\texttt{0}_{\mathfrak{M}}                  &= 0 \\
\texttt{succ}_{\mathfrak{M}}(n)            &= n+1 \\
\texttt{succ}_{\mathfrak{M}}(\omega)       &= \omega \\
\texttt{add}_{\mathfrak{M}}(m,n)           &= m+n \\
\texttt{add}_{\mathfrak{M}}(\omega,n)      &= \omega \\
\texttt{add}_{\mathfrak{M}}(n,\omega)      &= \omega \\
\texttt{add}_{\mathfrak{M}}(\omega,\omega) &= \omega \\
\texttt{mul}_{\mathfrak{M}}(m,n)           &= m*n \\
\texttt{mul}_{\mathfrak{M}}(\omega,0)      &= 0 \\
\texttt{mul}_{\mathfrak{M}}(\omega,n+1)    &= \omega \\
\texttt{mul}_{\mathfrak{M}}(n,\omega)      &= \omega \\
\texttt{mul}_{\mathfrak{M}}(\omega,\omega) &= \omega \\
\texttt{even}_{\mathfrak{M}}(n)            &= \begin{cases}
  T & \text{als $n$ even is} \\
  F & \text{als $n$ oneven is}
\end{cases} \\
\texttt{even}_{\mathfrak{M}}(\omega)       &= T \\
\texttt{odd}_{\mathfrak{M}}(n)             &= \begin{cases}
  F & \text{als $n$ even is} \\
  T & \text{als $n$ oneven is}
\end{cases} \\
\texttt{odd}_{\mathfrak{M}}(\omega)        &= T
\end{align*}

We hebben dus een algebra geconstrueerd waarin we een extra element toegevoegd
hebben dat `zowel even als oneven' is. Om alle vergelijkingen nog waar te
maken was er wat puzzelwerk nodig (je kunt $\omega$ opvatten als nuldeler voor
zowel optelling als vermeningvuldiging als je er iets in wilt zien).

Nu moeten we laten zien dat $\mathfrak{M}$ een model is voor de specificatie
$\texttt{NatBool}$. We laten de vergelijkingen $\texttt{[B1] \ldots [B5]}$
voor wat ze zijn; onze toevoeging van $\omega$ heeft hier geen invloed
op. Van de overige vergelijkingen uit $\texttt{NatBool}$ laten we voor
$\texttt{[A1]}$, $\texttt{[M2]}$ en $\texttt{[E2]}$ zien dat ze waar zijn in
$\mathfrak{M}$, de rest gaat op vergelijkbare wijze.

\paragraph{}

Laat nu $\theta$ een assignment zijn van elementen uit $A$ aan de variabelen
$x$ en $y$. We stellen dat $\theta(x) = a$ en $\theta(y) = b$.

\begin{itemize}

\item{$\texttt{[A1] add(x,0) = x}$}
\begin{align*}
\bar \theta(\texttt{add}(x,\texttt{0})) &= \texttt{add}_{\mathfrak{M}}(\bar \theta(x), \bar \theta(\texttt{0})) \\
                                        &= \texttt{add}_{\mathfrak{M}}(\theta(x), 0_{\mathfrak{M}}) \\
                                        &= \texttt{add}_{\mathfrak{M}}(a, 0) \\
                                        &= a
\end{align*}

\begin{align*}
\bar \theta(x) &= \theta(x) \\
               &= a
\end{align*}

\item{$\texttt{[M2] mul(x,succ(y)) = add(mul(x,y),x)}$}

We schrijven eerst de linkerkant uit.
\begin{align*}
\bar \theta(\texttt{mul}(x,\texttt{succ}(y))) &= \texttt{mul}_{\mathfrak{M}}(\bar \theta(x), \bar \theta(\texttt{succ}(y))) \\
                                              &= \texttt{mul}_{\mathfrak{M}}(\theta(x), \texttt{succ}_{\mathfrak{M}}(\bar \theta(y))) \\
                                              &= \texttt{mul}_{\mathfrak{M}}(a, \texttt{succ}_{\mathfrak{M}}(\theta(y))) \\
                                              &= \texttt{mul}_{\mathfrak{M}}(a, \texttt{succ}_{\mathfrak{M}}(b))
\end{align*}

We onderscheiden nu verschillende gevallen voor $a$ en $b$.

\begin{itemize}

\item
Wanneer $a = b = \omega$ hebben we:
\begin{align*}
\bar \theta(\texttt{mul}(x,\texttt{succ}(y))) &= \texttt{mul}_{\mathfrak{M}}(\omega, \texttt{succ}_{\mathfrak{M}}(\omega)) \\
                                              &= \texttt{mul}_{\mathfrak{M}}(\omega, \omega) \\
                                              &= \omega
\end{align*}

\item
Wanneer $a,b \in \mathbb{N}$ hebben we:

\begin{align*}
\bar \theta(\texttt{mul}(x,\texttt{succ}(y))) &= \texttt{mul}_{\mathfrak{M}}(a, b+1) \\
                                              &= a*(b+1)
\end{align*}

\item
Wanneer $a = \omega$ en $b \in \mathbb{N}$ hebben we:
\begin{align*}
\bar \theta(\texttt{mul}(x,\texttt{succ}(y))) &= \texttt{mul}_{\mathfrak{M}}(\omega, b+1) \\
                                              &= \omega
\end{align*}

\item
Wanneer $a \in \mathbb{N}$ en $b = \omega$ hebben we:
\begin{align*}
\bar \theta(\texttt{mul}(x,\texttt{succ}(y))) &= \texttt{mul}_{\mathfrak{M}}(a, \texttt{succ}_{\mathfrak{M}}(\omega)) \\
                                              &= \texttt{mul}_{\mathfrak{M}}(a, \omega) \\
                                              &= \omega
\end{align*}

\end{itemize}

En vervolgens schrijven we de rechterkant uit.
\begin{align*}
\bar \theta(\texttt{add}(\texttt{mul}(x,y),x)) &= \texttt{add}_{\mathfrak{M}}(\bar \theta(\texttt{mul}(x,y),x)) \\
                                               &= \texttt{add}_{\mathfrak{M}}(\texttt{mul}_{\mathfrak{M}}(\bar \theta(x),\bar \theta(y)),\bar \theta(x)) \\
                                               &= \texttt{add}_{\mathfrak{M}}(\texttt{mul}_{\mathfrak{M}}(\theta(x),\theta(y)),\theta(x)) \\
                                               &= \texttt{add}_{\mathfrak{M}}(\texttt{mul}_{\mathfrak{M}}(a,b),a)
\end{align*}

We onderscheiden nu weer dezelfde gevallen voor $a$ en $b$.

\begin{itemize}

\item
Wanneer $a = b = \omega$ hebben we:
\begin{align*}
\bar \theta(\texttt{add}(\texttt{mul}(x,y),x)) &= \texttt{add}_{\mathfrak{M}}(\texttt{mul}_{\mathfrak{M}}(\omega,\omega),\omega) \\
                                               &= \omega
\end{align*}

\item
Wanneer $a,b \in \mathbb{N}$ hebben we:
\begin{align*}
\bar \theta(\texttt{add}(\texttt{mul}(x,y),x)) &= \texttt{add}_{\mathfrak{M}}(\texttt{mul}_{\mathfrak{M}}(a,b),a) \\
                                               &= \texttt{add}_{\mathfrak{M}}(a*b,a) \\
                                               &= (a*b)+a \\
                                               &= a*(b+1)
\end{align*}

\item
Wanneer $a = \omega$ en $b \in \mathbb{N}$ hebben we:
\begin{align*}
\bar \theta(\texttt{add}(\texttt{mul}(x,y),x)) &= \texttt{add}_{\mathfrak{M}}(\texttt{mul}_{\mathfrak{M}}(\omega,b),\omega) \\
                                               &= \texttt{add}_{\mathfrak{M}}(\ldots,\omega) \\
                                               &= \omega
\end{align*}

\item
Wanneer $a \in \mathbb{N}$ en $b = \omega$ hebben we:
\begin{align*}
\bar \theta(\texttt{add}(\texttt{mul}(x,y),x)) &= \texttt{add}_{\mathfrak{M}}(\texttt{mul}_{\mathfrak{M}}(a,\omega),a) \\
                                               &= \texttt{add}_{\mathfrak{M}}(\omega,a) \\
                                               &= \omega
\end{align*}

\end{itemize}

\item{$\texttt{[E2] even(succ(x)) = odd(x)}$}

We schrijven weer eerst de linkerkant uit.
\begin{align*}
\bar \theta(\texttt{even}(\texttt{succ}(x))) &= \texttt{even}_{\mathfrak{M}}(\bar \theta(\texttt{succ}(x))) \\
                                             &= \texttt{even}_{\mathfrak{M}}(\texttt{succ}_{\mathfrak{M}}(\bar \theta(x)))) \\
                                             &= \texttt{even}_{\mathfrak{M}}(\texttt{succ}_{\mathfrak{M}}(\theta(x)))) \\
                                             &= \texttt{even}_{\mathfrak{M}}(\texttt{succ}_{\mathfrak{M}}(a))
\end{align*}

We onderscheiden twee gevallen voor $a$.

\begin{itemize}

\item
Wanneer $a \in \mathbb{N}$ hebben we:
\begin{align*}
\bar \theta(\texttt{even}(\texttt{succ}(x))) &= \texttt{even}_{\mathfrak{M}}(\texttt{succ}_{\mathfrak{M}}(a)) \\
                                             &= \texttt{even}_{\mathfrak{M}}(a+1) \\
                                             &= \texttt{even}_{\mathfrak{M}}(a+1) \\
                                             &= \begin{cases}
  F & \text{als $a$ even is} \\
  T & \text{als $a$ oneven is}
\end{cases}
\end{align*}

\item
Wanneer $a = \omega$ hebben we:
\begin{align*}
\bar \theta(\texttt{even}(\texttt{succ}(x))) &= \texttt{even}_{\mathfrak{M}}(\texttt{succ}_{\mathfrak{M}}(\omega)) \\
                                             &= \texttt{even}_{\mathfrak{M}}(\omega) \\
                                             &= T
\end{align*}

\end{itemize}

En vervolgens schrijven we de rechterkant uit.
\begin{align*}
\bar \theta(\texttt{odd}(x)) &= \texttt{odd}_{\mathfrak{M}}(\bar \theta(x)) \\
                             &= \texttt{odd}_{\mathfrak{M}}(\theta(x)) \\
                             &= \texttt{odd}_{\mathfrak{M}}(a)
\end{align*}

We onderscheiden weer twee gevallen voor $a$.

\begin{itemize}

\item
Wanneer $a \in \mathbb{N}$ hebben we:
\begin{align*}
\bar \theta(\texttt{odd}(x)) &= \texttt{odd}_{\mathfrak{M}}(a) \\
                             &= \begin{cases}
  F & \text{als $a$ even is} \\
  T & \text{als $a$ oneven is}
\end{cases}
\end{align*}

\item
Wanneer $a = \omega$ hebben we:
\begin{align*}
\bar \theta(\texttt{odd}(x)) &= \texttt{odd}_{\mathfrak{M}}(a) \\
                             &= \texttt{odd}_{\mathfrak{M}}(\omega) \\
                             &= T
\end{align*}

\end{itemize}

\end{itemize}

\paragraph{}

We hebben nu laten zien dat $\mathfrak{M}$ inderdaad een model is voor de
specificatie $\texttt{NatBool}$. Nu bekijken we de vergelijking
\begin{displaymath}
\texttt{even(x) = not(odd(x))}
\end{displaymath}
en zien dat deze niet waar is in $\mathfrak{M}$. Beschouw bijvoorbeeld de
assignment $\theta$ met $\theta(x) = \omega$. We hebben dan
\begin{align*}
\bar \theta(\texttt{even}(x)) &= \texttt{even}_{\mathfrak{M}}(\bar \theta(x)) \\
                              &= \texttt{even}_{\mathfrak{M}}(\theta(x)) \\
                              &= \texttt{even}_{\mathfrak{M}}(\omega) \\
                              &= T
\end{align*}
en
\begin{align*}
\bar \theta(\texttt{not}(\texttt{odd}(x))) &= \texttt{not}_{\mathfrak{M}}(\bar \theta(\texttt{odd}(x))) \\
                                           &= \texttt{not}_{\mathfrak{M}}(\texttt{odd}_{\mathfrak{M}}(\bar \theta(x))) \\
                                           &= \texttt{not}_{\mathfrak{M}}(\texttt{odd}_{\mathfrak{M}}(\theta(x))) \\
                                           &= \texttt{not}_{\mathfrak{M}}(\texttt{odd}_{\mathfrak{M}}(\omega)) \\
                                           &= \texttt{not}_{\mathfrak{M}}(T) \\
                                           &= F.
\end{align*}

Uit alle vergelijkingen van $\texttt{NatBool}$ volgt deze vergelijking dus
niet semantisch. Volgens de correctheid van afleidbaarheid, is deze
vergelijking dan ook niet afleidbaar in de specificatie $\texttt{NatBool}$.


\section*{Opgave 6.1}

\begin{description}

\item{\bf (a)} % model voor Booleans met junk zonder confusion
We hebben bij opgave 5.1 gezien dat iedere verzamelingsalgebra
$\mathfrak{P}(V)$ over een verzameling $V$ (als in voorbeeld 4.3) een model is
voor de specificatie $\texttt{Booleans}$. Nu beschouwen we de algebra
$\mathfrak{P}(F)$ over de verzameling $F=\{1,2,3,4,5,6\}$ met als drager
$\mathcal{P}(F)$.

\paragraph{Wel junk}

Iedere gesloten term uit $\texttt{Booleans}$ wordt in $\mathfrak{P}(F)$
ge\"interpreteerd als \`ofwel heel F, \`ofwel de lege verzameling $\{\}$. Het
bewijs hiervan verloopt via inductie naar de structuur van de term en laten we
hier achterwege.

Laten we nu het element $\{2,4,5\}$ uit $\mathfrak{P}(F)$ bekijken. Dit is
niet $F$, ook niet $\{\}$ en dus niet de interpretatie van een gesloten
term. Hieruit volgt dat $\mathfrak{P}(F)$ junk bevat.

\paragraph{Geen confusion}

Voor iedere gesloten term $t$ uit $\texttt{Booleans}$ geldt dat
\begin{align*}
\vdash t = \texttt{true} \quad \text{of} \quad \vdash t = \texttt{false}. &&\text{(volgens voorbeeld 6.3)}
\end{align*}
Laat nu $s$ en $t$ gesloten termen zijn met $\not \vdash t = s$. Dan moet
\`ofwel $\vdash s = \texttt{true}$ en $\vdash t = \texttt{false}$, \`ofwel
$\vdash s = \texttt{false}$ en $\vdash t = \texttt{true}$. In beide gevallen
worden $s$ en $t$ als verschillende elementen van $\mathfrak{P}(F)$
ge\"interpreteerd en dus bevat $\mathfrak{P}(F)$ geen confusion.

\item{\bf (b)} % model voor Booleans zonder junk met confusion
Laten we nu analoog aan onderdeel {\bf (a)} de verzamelingsalgebra
$\mathfrak{P}(\{\})$ over de lege verzameling bekijken met in de drager als
enige element $\{\}$.

\paragraph{Geen junk}

Volgens de definitie van de verzamelingsalgebra interpreteren we nu
$\texttt{true}$ als $\{\}$. Hieruit volgt direct dat $\mathfrak{P}(\{\})$ geen
junk bevat, want het enige element uit de drager is de interpretatie van een
gesloten term.

\paragraph{Wel confusion}

In de algebra $\mathfrak{P}(\{\})$ worden de interpretaties van de termen
$\texttt{true}$ en $\texttt{false}$ ge\"identificeerd. Toch is de vergelijking
$\texttt{true} = \texttt{false}$ niet afleidbaar uit de specificatie
$\texttt{Booleans}$ (bovenstaande algebra $\mathfrak{P}(F)$ is bijvoorbeeld
een tegenmodel) en dus bevat $\mathfrak{P}(\{\})$ confusion.

\item{\bf (c)} % model voor Booleans met junk met confusion
We bekijken de algebra $\mathfrak{B}_{3}$ voor de specificatie
$\texttt{Booleans}$ met als drager $\{A,B,C\}$ en interpretaties
\begin{align*}
\texttt{true}_{\mathfrak{B}_{3}}     &= B, \\
\texttt{false}_{\mathfrak{B}_{3}}    &= B, \\
\texttt{and}_{\mathfrak{B}_{3}}(x,y) &= B, \\
\texttt{or}_{\mathfrak{B}_{3}}(x,y)  &= B, \\
\texttt{not}_{\mathfrak{B}_{3}}(x)   &= B.
\end{align*}
Dat $\mathfrak{B}_{3}$ een model is voor $\texttt{Booleans}$ mag duidelijk
zijn (iedere term evalueert naar $B$ en dus is iedere vergelijking waar).

\paragraph{Wel junk}

Het is niet moeilijk in te zien dat iedere gesloten term uit
$\texttt{Booleans}$ wordt in $\mathfrak{B}_{3}$ ge\"interpreteerd als $B$. Dit
betekent dat de elementen $A$ en $C$ niet de interpretatie van een gesloten
term zijn en dus dat $\mathfrak{B}_{3}$ junk bevat.

\paragraph{Wel confusion}

In de algebra $\mathfrak{B}_{3}$ worden de interpretaties van de termen
$\texttt{true}$ en $\texttt{false}$ ge\"identificeerd. Toch is de vergelijking
$\texttt{true} = \texttt{false}$ niet afleidbaar uit de specificatie
$\texttt{Booleans}$ (bovenstaande algebra $\mathfrak{P}(F)$ is bijvoorbeeld
een tegenmodel) en dus bevat $\mathfrak{B}_{3}$ confusion.

\end{description}


\end{document}
