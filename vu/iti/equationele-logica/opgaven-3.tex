\documentclass[a4paper,11pt]{article}
\usepackage[dutch]{babel}
\usepackage{amsfonts}
\usepackage{a4}
\usepackage{latexsym}
\usepackage{fitch} % http://folk.uio.no/johanw/FitchSty.html

% niet te gewichtig willen doen met veel ruimte
\setlength{\textwidth}{16cm}
\setlength{\textheight}{23.0cm}
\setlength{\topmargin}{0cm}
\setlength{\oddsidemargin}{0.2mm}
\setlength{\evensidemargin}{0.2mm}
\setlength{\parindent}{0cm}

% standaard enumerate met abc
\renewcommand\theenumi{\alph{enumi}}


\begin{document}


{\bf Uitwerkingen bij Inleiding Theoretische Informatica\\
Deel 1: Equationele Logica -- Afleidbaarheid}\\[2em]


{\bf Opgave 3.1}

\begin{enumerate}

\item % afleiding voor pop(pop(pop(empty))) = empty
Een afleiding in de specificate \verb|Stack-Of-Data| van de vergelijking
\begin{verbatim}
pop(pop(pop(empty))) = empty
\end{verbatim}

\begin{equation*}
\begin{fitch}
pop(empty) = empty                             & [1]       \\
\neg K \neg K p \to K p                        & PROP, 1   \\
K(p \to q) \to (K p \to K q)                   & DB        \\
K(p \to (p \vee q)) \to (K p \to K (p \vee q)) & SUB, 3    \\
p \to (p \vee q)                               & CT        \\
K(p \to (p \vee q))                            & UG, 5     \\
K p \to K (p \vee q)                           & MP, 4,6   \\
\neg K \neg K p \to K (p \vee q)               & \dag, 2,7
\end{fitch}
\end{equation*}

\item % afleiding voor pop(push(x, pop(push(x, empty)))) = empty

\end{enumerate}


\end{document}
