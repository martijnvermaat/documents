\documentclass[a4paper,11pt]{article}
\usepackage[dutch]{babel}
\usepackage{amsfonts}
\usepackage{a4}
\usepackage{latexsym}
\usepackage{fitch} % http://folk.uio.no/johanw/FitchSty.html

% niet te gewichtig willen doen met veel ruimte
\setlength{\textwidth}{16cm}
\setlength{\textheight}{23.0cm}
\setlength{\topmargin}{0cm}
\setlength{\oddsidemargin}{0.2mm}
\setlength{\evensidemargin}{0.2mm}
\setlength{\parindent}{0cm}

% standaard enumerate met abc
\renewcommand\theenumi{\alph{enumi}}


\begin{document}


{\bf Uitwerkingen bij Inleiding Theoretische Informatica\\
Deel 1: Equationele Logica -- Afleidbaarheid}\\[2em]


{\bf Opgave 3.1}

\begin{enumerate}

\item % afleiding voor pop(pop(pop(empty))) = empty
Een afleiding in de specificate \verb|Stack-Of-Data| van de vergelijking
\begin{verbatim}
pop(pop(pop(empty))) = empty
\end{verbatim}

\begin{equation*}
\begin{fitch}
\verb|pop(empty) = empty|                        & [1]           \\ % 1
\verb|pop(pop(empty)) = pop(empty)|              & congr, 1      \\ % 2
\verb|pop(pop(pop(empty))) = pop(pop(empty))|    & congr, 2      \\ % 3
\verb|pop(pop(empty))) = empty|                  & trans, 1, 2   \\ % 4
\verb|pop(pop(pop(empty))) = empty|              & trans, 3, 4      % 5
\end{fitch}
\end{equation*}

\item % afleiding voor pop(push(x, pop(push(x, empty)))) = empty
Een afleiding in de specificatie \verb|Stack-Of-Data| van de vergelijking
\begin{verbatim}
pop(push(x, pop(push(x, empty)))) = empty
\end{verbatim}

\begin{equation*}
\begin{fitch}
\verb|pop(push(x, s)) = s|                       & [3]           \\ % 1
\verb|pop(push(x, empty)) = empty|               & subs, 1       \\ % 2
\verb|pop(push(x, pop(push(x, empty)))) = pop(push(x, empty))| & subs, 1 \\ % 3
\verb|pop(push(x, pop(push(x, empty)))) = empty| & trans, 3, 2      % 4
\end{fitch}
\end{equation*}

Wellicht is het een beetje verwarrend dat in de af te leiden vergelijking twee
keer dezelfde variabele \verb|x| voor komt (terwijl dat niet belangrijk is in
de opgave) en bovendien \verb|x| ook gebruikt wordt in het axioma \verb|[3]|
(bij elkaar een wat ongelukkig toeval).\\[2em]

\end{enumerate}


{\bf Opgave 3.2}

\begin{enumerate}

\item % afleiding voor and(not(x), not(y)) = not(or(x, y))
Todo.

\item % afleiding voor mul(succ(0), add(0, succ(0))) = succ(0)
We geven een afleiding in de specificatie \verb|Naturals| voor de vergelijking
\begin{verbatim}
mul(succ(0), add(0, succ(0))) = succ(0)
\end{verbatim}

\begin{equation*}
\begin{fitch}
\verb|add(x, succ(y)) = succ(add(x, y))|              & [N2]           \\ % 1
\verb|add(0, succ(0)) = succ(add(0, 0))|              & subs, 1        \\ % 2
\verb|add(x, 0) = x|                                  & [N1]           \\ % 3
\verb|add(0, 0) = 0|                                  & subs, 3        \\ % 4
\verb|succ(add(0, 0)) = succ(0)|                      & congr, 4       \\ % 5
\verb|add(0, succ(0)) = succ(0)|                      & trans, 2, 5    \\ % 6
\verb|succ(0) = succ(0)|                              & refl           \\ % 7
\verb|mul(succ(0), add(0, succ(0))) = mul(succ(0), succ(0))|  & congr, 6, 7 \\ % 8
\verb|mul(x, succ(y)) = add(mul(x, y), x)|            & [N2]           \\ % 9
\verb|mul(succ(0), succ(0)) = add(mul(succ(0), 0), succ(0))| & subs, 9 \\ % 10
\verb|mul(x, 0) = 0|                                  & [N1]           \\ % 11
\verb|mul(succ(0), 0) = 0|                            & subs, 11       \\ % 12
\verb|add(mul(succ(0), 0), succ(0)) = add(0, succ(0))| & congr, 11, 7  \\ % 13
\verb|add(mul(succ(0), 0), succ(0)) = succ(0)|        & trans, 13, 6   \\ % 14
\verb|mul(succ(0), succ(0)) = succ(0)|                & trans, 10, 14  \\ % 15
\verb|mul(succ(0), add(0, succ(0))) = succ(0)|        & trans, 8, 15      % 16
\end{fitch}
\end{equation*}
Misschien kan ik deze afleiding nog een beetje omvormen, zodat de strategie
wat duidelijker wordt...\\[2em]

\end{enumerate}


{\bf Opgave 3.3}

We bewijzen voor iedere gesloten term $t$ dat deze afleidbaar gelijk is aan
een term van de vorm $succ^{n}(0)$. Dit doen we volgens inductie naar de
structuur van de term $t$:

\begin{description}

\item{Basisgeval:}
  Het basisgeval is wanneer $t$ een constante is, in dit geval is de enige
  mogelijkheid de constante $0$. En omdat $succ^0(0) = 0$ zien we dat de
  bewering waar is in het basisgeval. (Om precies te zijn: $t$ is afleidbaar
  gelijk aan zichzelf wegens reflexiviteit.)

\item{Inductiestap:}
  We onderscheiden drie mogelijkheden voor de structuur van $t$ in onze
  inductiestap:

  \begin{description}

  \item{$t$ is van de vorm $succ(u)$}

    We mogen volgens inductie aannemen dat $u$ afleidbaar gelijk is aan een
    term van de vorm $succ^{n}(0)$. Wegens congruentie is dan $t$ afleidbaar
    gelijk aan een term van de vorm $succ(succ^{n}(0))$ en dat is natuurlijk
    $succ^{n+1}(0)$.

  \item{$t$ is van de vorm $add(u, v)$}

    De aannamen zijn dat $u$ en $v$ afleidbaar gelijk zijn aan respectievelijk
    een term van de vorm $succ^{n}(0)$ en een term van de vorm
    $succ^{m}(0)$. Volgens congruentie is $t$ dan afleidbaar gelijk aan een
    term van de vorm $add(succ^{n}(0), succ^{m}(0))$.

    Als $m=0$ dan is $t$ afleidbaar gelijk aan een term van de vorm
    $succ^{n}(0)$ (volgens $[N1]$) en zijn we klaar. Als $m>0$, dan is $t$
    afleidbaar gelijk aan een term van de vorm $succ(add(succ^{n}(0),
    succ^{m-1}(0)))$ (volgens $[N2]$). (Kunnen we terminatie hiervan
    intu\"itief laten zien, of moet dat nog formeel?)

  \item{$t$ is van de vorm $mul(u, v)$}

    Wederom mogen we aannemen dat $u$ en $v$ afleidbaar gelijk zijn aan
    respectievelijk een term van de vorm $succ^{n}(0)$ en een term van de vorm
    $succ^{m}(0)$. Wegens congruentie is $t$ dan afleidbaar gelijk aan een
    term van de vorm $mul(succ^{n}(0), succ^{m}(0))$.

    Weer $m=0$ of $m>0$ en dan $[N1]$ of $[N2]$.

  \end{description}

\end{description}

Hiermee hebben we bewezen dat de bewering klopt. We hoeven geen rekening te
houden met variabelen, omdat deze niet voor komen in een gesloten term.\\[2em]


{\bf Opgave 3.4} % laat zien dat ~ een eq relatie is

Om te laten zien dat $\sim$ een equivalentierelatie is, laten we zien dat $\sim$
transitief, symmetrisch en reflexief is.

\begin{description}

\item{Transitiviteit}

  Stel $t \sim u$ en $u \sim v$, dan is volgens de definitie $E \vdash t = u$ en
  $E \vdash u = v$. Wegens de regel \emph{transitiviteit} van de
  afleidbaarheid, is dan ook $E \vdash t = v$. Met de definitie zien we dan
  weer dat ook $t \sim v$ waar is en dus is $\sim$ transitief.

\item{Symmetrie}

  Todo.

\item{Reflexiviteit}

  Todo.

\end{description}

Omdat $\sim$ transitief, symmetrisch en reflexief is, is $\sim$ een
equivalentierelatie.


\end{document}
