\documentclass[a4paper,twocolumn,10pt]{article}


\usepackage[landscape,pdftex,top=.5in,right=.5in,left=.5in,bottom=.5in]{geometry}
%\usepackage[cm]{fullpage}
\usepackage[dutch]{babel}
\usepackage[T1]{fontenc}
\usepackage{ae,aecompl}
\usepackage{amsmath}
\usepackage{amssymb}


\setlength{\columnsep}{.8in}


\pagestyle{empty}


\makeatletter
\renewcommand{\section}{\@startsection{section}{1}{0mm}%
                                {2.5ex plus -.5ex minus -.2ex}%
                                {1.5ex plus .2ex}%x
                                {\normalfont\large\bfseries}}
\renewcommand{\subsection}{\@startsection{subsection}{2}{0mm}%
                                {-1explus -.5ex minus -.2ex}%
                                {0.5ex plus .2ex}%
                                {\normalfont\small\bfseries}}
\renewcommand\subsubsection{\@startsection{subsubsection}{3}{0mm}%
                                {-1ex plus -.5ex minus -.2ex}%
                                {1ex plus .2ex}%
                                {\normalfont\small\bfseries}}
\makeatother


\setcounter{secnumdepth}{0}


\setlength{\parindent}{0pt}
\setlength{\parskip}{0pt plus 0.5ex}


\usepackage[
  pdftex,
  colorlinks,
  citecolor=black,
  filecolor=black,
  linkcolor=black,
  urlcolor=black,
  pdfauthor={Martijn Vermaat},
  pdftitle={Equationele Logica beknopt},
  pdfsubject={Equationele Logica},
]{hyperref}


\begin{document}


\footnotesize


\begin{center}
     \Large{\textbf{Equationele Logica beknopt}} \\
\end{center}


\section{Syntax}


\subsection{Signatuur \texorpdfstring{$(S, \Sigma)$}{(S, Sigma)}}

\begin{tabular}{@{}ll@{}}
$S$       &  Verzameling soortnamen. \\
$\Sigma$  &  Verzameling functiesymbolen met types over $S$.
\end{tabular}


\subsection{Termen \texorpdfstring{$Ter_{\Sigma}(X)$}{TerSigma(X)} over \texorpdfstring{$(S, \Sigma)$}{(S, Sigma)}}

\begin{tabular}{@{}ll@{}}
$(S, \Sigma)$      &  Signatuur. \\
$X$                &  Verzameling variabelen met types uit $S$.
\end{tabular}

De verzameling $Ter_{\Sigma}(X)$ van termen met vrije
variabelen uit $X$ is inductief gedefinieerd als volgt waarbij
steeds de types gerespecteerd worden:
\begin{itemize}
\item $f \in \Sigma$ en $t_{1}, \ldots, t_{n} \in Ter_{\Sigma}(X)$ $\Rightarrow$ $f(t_{1}, \ldots, t_{n}) \in Ter_{\Sigma}(X)$
\item $x \in Ter_{\Sigma}(X)$ voor alle $x \in X$
\end{itemize}


\subsection{Specificatie \texorpdfstring{$((S, \Sigma), E)$}{((S, Sigma), E)}}

\begin{tabular}{@{}ll@{}}
$(S, \Sigma)$  &  Signatuur. \\
$E$            &  Verzameling vergelijkingen $l_{i} = r_{i}$ met
                  $l_{i}$ en $r_{i}$ termen uit $Ter_{\Sigma}(X)$.
\end{tabular}


\subsection{Substitutie \texorpdfstring{$\bar{\theta} : Ter_{\Sigma}(X) \rightarrow Ter_{\Sigma}(X)$}{theta : TerSigma(X) -> TerSigma(X)}}

\begin{tabular}{@{}ll@{}}
$\theta : X \rightarrow Ter_{\Sigma}(X)$                    &  Substitutie van termen voor variabelen. \\
$\bar{\theta} : Ter_{\Sigma}(X) \rightarrow Ter_{\Sigma}(X)$  &  Uitbreiding op termen.
\end{tabular}

Gegeven een substitutie $\theta$ wordt de uitbreiding daarvan
inductief gedefinieerd als:
\begin{align*}
  \bar{\theta}(x) &= \theta(x) \\
  \bar{\theta}(f(t_{1}, \ldots, t_{n})) &= f(\bar{\theta}(t_{1}), \ldots, \bar{\theta}(t_{n}))
\end{align*}


\section{Semantiek}


\subsection{\texorpdfstring{$\Sigma$}{Sigma}-algebra \texorpdfstring{$\mathfrak{A} = (A, I)$}{A = (A, I)}}

\begin{tabular}{@{}ll@{}}
$(S, \Sigma)$  &  Signatuur. \\
$A$            &  Drager, $S$-soortig. \\
$I$            &  Interpretatie.
\end{tabular}

Voor iedere $f \in \Sigma$ geeft $I(f)$ een interpretatie $f_{\mathfrak{A}}$
op de drager, waarbij steeds alle typen kloppen.


\subsection{Assignment \texorpdfstring{$\bar{\theta} : Ter_{\Sigma}(X) \rightarrow A$}{theta : TerSigma(X) -> A}}

\begin{tabular}{@{}ll@{}}
$\mathfrak{A} = (A, I)$                       &  $\Sigma$-algebra. \\
$\theta : X \rightarrow A$                    &  Assignment van $a \in A$ aan $x \in X$. \\
$\bar{\theta} : Ter_{\Sigma}(X) \rightarrow A$  &  Uitbreiding op termen.
\end{tabular}

Gegeven een assignment $\theta$ wordt de uitbreiding daarvan
inductief gedefinieerd als:
\begin{align*}
  \bar{\theta}(x) &= \theta(x) \\
  \bar{\theta}(f(t_{1}, \ldots, t_{n})) &= f_\mathfrak{A}(\bar{\theta}(t_{1}), \ldots, \bar{\theta}(t_{n}))
\end{align*}


\subsection{Waarheid \texorpdfstring{$\mathfrak{A} \models t_{1} = t_{2}$}{A |= t1 = t2}}

$t_{1} = t_{2}$ is waar in $\mathfrak{A}$ wanneer $\bar{\theta}(t_{1}) = \bar{\theta}(t_{2})$
voor iedere assignment $\theta$.
Voor een verzameling vergelijkingen $E$ zeggen we dat $\mathfrak{A} \models E$ wanneer
$\mathfrak{A} \models t_{1} = t_{2}$ voor alle $t_{1} = t_{2}$ in $E$.


\subsection{Semantisch gevolg \texorpdfstring{$E \models t_{1} = t_{2}$}{E |= t1 = t2}}

$t_{1} = t_{2}$ volgt semantisch uit $E$ wanneer $\mathfrak{A} \models t_{1} = t_{2}$
voor iedere algebra $\mathfrak{A}$ met $\mathfrak{A} \models E$.


\section{Equationele logica}


\subsection{Afleidbaarheid \texorpdfstring{$E \vdash t_{1} = t_{2}$}{E |- t1 = t2}}

De verzameling vergelijkingen afleidbaar uit $E$ is inductief gedefinieerd:
\begin{itemize}
\item als $t_{1} = t_{2} \in E$ (het is een axioma), dan $E \vdash t_{1} = t_{2}$ ,
\item $E \vdash t = t$ voor alle $t$ (reflexiviteit) ,
\item als $E \vdash t_{1} = t_{2}$, dan $E \vdash t_{2} = t_{1}$ (symmetrie) ,
\item als $E \vdash t_{1} = t_{2}$ en $E \vdash t_{2} = t_{3}$, dan $E \vdash t_{1} = t_{3}$ (transitiviteit) ,
\item als $E \vdash t_{i} = u_{i}$ voor $i = 1 \ldots n$, dan $f(t_{i}, \ldots, t_{n}) = f(u_{1}, \ldots, u_{n})$ (congruentie) ,
\item als $E \vdash t_{1} = t_{2}$, dan $E \vdash \bar{\theta}(t_{1}) = \bar{\theta}(t_{2})$ voor alle substituties $\theta$ .
\end{itemize}


\subsection{Volledigheid van \texorpdfstring{$\vdash$}{|-}}

\begin{tabular}{@{}ll@{}}
  Correctheid   &  $E \vdash t_{1} = t_{2} \quad \Longrightarrow \quad E \models t_{1} = t_{2}$. \\
  Volledigheid  &  $E \models t_{1} = t_{2} \quad \Longrightarrow \quad E \vdash t_{1} = t_{2}$.
\end{tabular}



\section{Modellen}


\subsection{Model \texorpdfstring{$\mathfrak{A}$}{A} voor \texorpdfstring{$((S, \Sigma), E)$}{((S, Sigma), E)}}

Een $\Sigma$-algebra $\mathfrak{A}$ is een model voor de specificatie $((S, \Sigma), E)$
wanneer $\mathfrak{A} \models E$.


\subsection{Initi\"ele modellen}

\begin{tabular}{@{}ll@{}}
$((S, \Sigma), E)$       &  Specificatie. \\
$\mathfrak{A} = (A, I)$  &  Model voor $((S, \Sigma), E)$. \\
\end{tabular}

Een element $a \in A$ is junk wanneer het niet de interpretatie
is van een gesloten term.

Gesloten termen $t_{1}$ en $t_{2}$ vormen confusion wanneer
$\mathfrak{A} \models t_{1} = t_{2}$ terwijl $E \not\models t_{1} = t_{2}$.

Een model is initieel wanneer het geen junk en geen confusion bevat.


\subsection{Termmodel \texorpdfstring{$\mathfrak{Ter}_{\Sigma}/\negmedspace\sim$}{TerSigma/~}}

Gegeven een specificatie $((S, \Sigma), E)$ bestaat het termmodel
$\mathfrak{Ter}_{\Sigma}/\negmedspace\sim$ uit
\begin{itemize}
\item drager $Ter_{\Sigma}/\negmedspace\sim$ van equivalentieklassen $[t]$ van termen $t \in Ter_{\Sigma}$ ,
\item voor alle $f \in \Sigma$, een interpretatie $f_{\mathfrak{Ter}_{\Sigma}/\negmedspace\sim}([t_{1}], \ldots, [t_{n}]) \equiv [f(t_{1}, \ldots, t_{2})]$ .
\end{itemize}

Hierbij is de equivalentieklasse $[t]$ van $t$ gedefinieerd als $\{u \in Ter_{\Sigma} \mid E \vdash t = u\}$.

Het termmodel $\mathfrak{Ter}_{\Sigma}/\negmedspace\sim$ is initieel voor $((S, \Sigma), E)$.


\section{Isomorfie}


\subsection{Homomorfisme \texorpdfstring{$\phi$}{phi} van \texorpdfstring{$\mathfrak{A}$}{A} naar \texorpdfstring{$\mathfrak{B}$}{B}}

\begin{tabular}{@{}ll@{}}
$\mathfrak{A} = (A, I_{\mathfrak{A}})$ en $\mathfrak{B} = (B, I_{\mathfrak{B}})$  &  $\Sigma$-algebra's. \\
$\phi : A \rightarrow B$                                                   &  Afbeelding van $A$ naar $B$.
\end{tabular}

De afbeelding $\phi$ is een homomorfisme van $\mathfrak{A}$ naar $\mathfrak{B}$ wanneer
\begin{equation*}
  \phi(f_{\mathfrak{A}}(a_{1}, \ldots, a_{n})) \equiv f_{\mathfrak{B}}(\phi(a_{1}, \ldots, a_{n}))
\end{equation*}
voor alle $f \in \Sigma$ en alle $a_{i} \in A$ (types respecterend).


\subsection{Isomorfie \texorpdfstring{$\cong$}{~=}}

Een bijectief homomorfisme van $\mathfrak{A}$ naar $\mathfrak{B}$ is een isomorfisme.
Als er een isomorfisme bestaat tussen twee algebra's $\mathfrak{A}$ en $\mathfrak{B}$
noemen we ze isomorf ($\mathfrak{A} \cong \mathfrak{B}$).

Alle initi\"ele modellen van een specificatie zijn isomorf.


\paragraph{}
\noindent


\rule{0.3\linewidth}{0.25pt}
\scriptsize

Martijn Vermaat (\href{mailto:mvermaat@cs.vu.nl}{\texttt{mvermaat@cs.vu.nl}}), maart 2008. \\
Gebaseerd op het dictaat \emph{Equationele specificatie van datatypen}.


\end{document}
