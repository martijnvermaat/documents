\documentclass[a4paper,11pt]{article}
\usepackage[dutch]{babel}
\usepackage{amsfonts}
\usepackage{a4}
\usepackage{latexsym}
\usepackage{fitch} % http://folk.uio.no/johanw/FitchSty.html

% niet te gewichtig willen doen met veel ruimte
\setlength{\textwidth}{16cm}
\setlength{\textheight}{23.0cm}
\setlength{\topmargin}{0cm}
\setlength{\oddsidemargin}{0.2mm}
\setlength{\evensidemargin}{0.2mm}
\setlength{\parindent}{0cm}

% standaard enumerate met abc
\renewcommand\theenumi{\alph{enumi}}


\begin{document}


{\bf Uitwerkingen bij Inleiding Theoretische Informatica\\
Deel 1: Equationele Logica -- Algebra's}\\[2em]


{\bf Opgave 4.1}

\begin{enumerate}

\item % laat zien dat \x->x een homomorphisme is van N naar Z
Hiertoe moeten we laten zien dat $\phi$ over alle functies voldoet aan de
voorwaarden van een homomorphisme:

\begin{itemize}

\item{de constante $\verb|0|_{\mathfrak{N}}$}

  \begin{eqnarray*}
    \phi(\verb|0|_{\mathfrak{N}}) & = & \phi(0) \\
                                  & = & 0 \\
                                  & = & \verb|0|_{\mathfrak{Z}}
  \end{eqnarray*}

\item{de functie $\verb|succ|_{\mathfrak{N}}$}

  Voor willekeurige $x$ uit $\{0,1,2,\ldots\}$:

  \begin{eqnarray*}
    \phi(\verb|succ|_{\mathfrak{N}}(x)) & = & \phi(x+1) \\
                                        & = & x+1 \\
                                        & = & \verb|succ|_{\mathfrak{Z}}(x) \\
                                        & = & \verb|succ|_{\mathfrak{Z}}(\phi(x))
  \end{eqnarray*}

\item{de functie $\verb|add|_{\mathfrak{N}}$}

  Voor willekeurige $x,y$ uit $\{0,1,2,\ldots\}$:

  \begin{eqnarray*}
    \phi(\verb|add|_{\mathfrak{N}}(x,y)) & = & \phi(x+y) \\
                                         & = & x+y \\
                                         & = & \verb|add|_{\mathfrak{Z}}(x,y) \\
                                         & = &
                                         \verb|add|_{\mathfrak{Z}}(\phi(x),\phi(y))
  \end{eqnarray*}

\item{de functie $\verb|mul|_{\mathfrak{N}}$}

  Voor willekeurige $x,y$ uit $\{0,1,2,\ldots\}$:

  \begin{eqnarray*}
    \phi(\verb|mul|_{\mathfrak{N}}(x,y)) & = & \phi(x*y) \\
                                         & = & x*y \\
                                         & = & \verb|mul|_{\mathfrak{Z}}(x,y) \\
                                         & = & \verb|mul|_{\mathfrak{Z}}(\phi(x),\phi(y))
  \end{eqnarray*}

\end{itemize}

Hiermee hebben we laten zien dat $\phi$ aan alle voorwaarden van een
homomorphisme voldoet.\\[2em]

\end{enumerate}


{\bf Opgave 4.2}


\end{document}
