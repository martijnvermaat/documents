\documentclass[a4paper,11pt]{article}
\usepackage[dutch]{babel}
\usepackage{amsfonts}
\usepackage{a4}
\usepackage{latexsym}
\usepackage{fitch} % http://folk.uio.no/johanw/FitchSty.html

% niet te gewichtig willen doen met veel ruimte
\setlength{\textwidth}{16cm}
\setlength{\textheight}{23.0cm}
\setlength{\topmargin}{0cm}
\setlength{\oddsidemargin}{0.2mm}
\setlength{\evensidemargin}{0.2mm}
\setlength{\parindent}{0cm}

% standaard enumerate met abc
\renewcommand\theenumi{\alph{enumi}}


\begin{document}


{\bf Uitwerkingen bij Inleiding Theoretische Informatica\\
Deel 1: Equationele Logica -- Algebra's}\\[2em]


{\bf Opgave 4.1}

\begin{enumerate}

\item % laat zien dat \x->x een homomorphisme is van N naar Z
Hiertoe moeten we laten zien dat $\phi$ over alle functies voldoet aan de
voorwaarden van een homomorphisme:

\begin{itemize}

\item{de constante $\texttt{0}_{\mathfrak{N}}$}

  \begin{eqnarray*}
    \phi(\texttt{0}_{\mathfrak{N}}) & = & \phi(0) \\
                                  & = & 0 \\
                                  & = & \texttt{0}_{\mathfrak{Z}}
  \end{eqnarray*}

\item{de functie $\texttt{succ}_{\mathfrak{N}}$}

  Voor willekeurige $x$ uit $\{0,1,2,\ldots\}$:

  \begin{eqnarray*}
    \phi(\texttt{succ}_{\mathfrak{N}}(x)) & = & \phi(x+1) \\
                                        & = & x+1 \\
                                        & = & \texttt{succ}_{\mathfrak{Z}}(x) \\
                                        & = & \texttt{succ}_{\mathfrak{Z}}(\phi(x))
  \end{eqnarray*}

\item{de functie $\texttt{add}_{\mathfrak{N}}$}

  Voor willekeurige $x,y$ uit $\{0,1,2,\ldots\}$:

  \begin{eqnarray*}
    \phi(\texttt{add}_{\mathfrak{N}}(x,y)) & = & \phi(x+y) \\
                                         & = & x+y \\
                                         & = & \texttt{add}_{\mathfrak{Z}}(x,y) \\
                                         & = &
                                         \texttt{add}_{\mathfrak{Z}}(\phi(x),\phi(y))
  \end{eqnarray*}

\item{de functie $\texttt{mul}_{\mathfrak{N}}$}

  Voor willekeurige $x,y$ uit $\{0,1,2,\ldots\}$:

  \begin{eqnarray*}
    \phi(\texttt{mul}_{\mathfrak{N}}(x,y)) & = & \phi(x*y) \\
                                         & = & x*y \\
                                         & = & \texttt{mul}_{\mathfrak{Z}}(x,y) \\
                                         & = & \texttt{mul}_{\mathfrak{Z}}(\phi(x),\phi(y))
  \end{eqnarray*}

\end{itemize}

Hiermee hebben we laten zien dat $\phi$ aan alle voorwaarden van een
homomorphisme voldoet.

\item % bestaan er meer homomorphismes van N naar Z?

Nee?

\item % laat zien dat er geen homomorphismes bestaan van Z naar N

Stel, er is een homomorphisme $\phi$ van $\mathfrak{Z}$ naar
$\mathfrak{N}$. We bekijken $\phi(-1)=n$ en weten dat $n \in \mathbb{N}$ ($n
\ge 0$). Nu hebben we

\begin{displaymath}
\phi(\texttt{succ}_{\mathfrak{Z}}(-1)) \, = \, \phi(0) \, = \, 0
\end{displaymath}

omdat $\phi(\texttt{0}_{\mathfrak{Z}}=0)$ gelijk moet zijn aan
$\texttt{0}_{\mathfrak{N}} = 0$.

Vervolgens zien we dat

\begin{displaymath}
\texttt{succ}_{\mathfrak{N}}(\phi(-1)) = \texttt{succ}_{\mathfrak{N}}(n) = n+1 \ne 0
\end{displaymath}

en dus voldoet $\phi$ niet aan de voorwaarden van een homomorphisme. Dit is in
tegenspraak met de aanname dat $\phi$ een homomorphisme is, dus kan $\phi$ niet
bestaan.\\[2em]

\end{enumerate}


{\bf Opgave 4.2}

\begin{enumerate}

\item % is \n->n+3 een homomorphisme in N+

  $\phi(n) = n+3$ is geen homomorphisme in $\mathcal{N}^{+}$, want we zien
  bijvoorbeeld dat:

  \begin{eqnarray*}
    \phi(\texttt{add}(2,5)) & = & \phi(7) \\
                          & = & 10 \\
                          & \ne & 13 \\
                          & = & \texttt{add}(5,8) \\
                          & = & \texttt{add}(\phi(2),\phi(5))
  \end{eqnarray*}

\item % is \n->2n een homomorphisme in N+

  $\phi(n) = 2n$ is wel een homomorphisme in $\mathcal{N}^{+}$, want voor
  willekeurige $x,y$ uit $\mathbb{N}$:

  \begin{eqnarray*}
    \phi(\texttt{0}_{\mathbb{N}^{+}}) & = & \phi(0) \\
                                    & = & 0 \\
                                    & = & \texttt{0}_{\mathbb{N}^{+}}
  \end{eqnarray*}

  \begin{eqnarray*}
    \phi(\texttt{add}(x,y)) & = & \phi(x+y) \\
                          & = & 2(x+y) \\
                          & = & 2x \, + \, 2y \\
                          & = & \texttt{add}(2x, 2y) \\
                          & = & \texttt{add}(\phi(x), \phi(y))
  \end{eqnarray*}

  Hiermee voldoet $\phi$ aan alle voorwaarden.\\[2em]

\end{enumerate}


{\bf Opgave 4.3}

Todo.\\[2em]


{\bf Opgave 4.4}

De vraag
\begin{quote}
``Laat zien dat er een \ldots bestaat.''
\end{quote}
betekent in de praktijk (bijna) altijd:
\begin{quote}
``Geef een \ldots en laat zien dat het een \ldots is.''
\end{quote}

We geven het homomorphisme $\phi$ van $\mathfrak{A}$ naar $\mathfrak{B}$
gedefini\"eerd als

\begin{displaymath}
\phi(x) = \begin{cases}
  \square & \text{als $x \in A_{\texttt{data}}$;} \\
  |x|     & \text{als $x \in A_{\texttt{stack}}$.}
\end{cases}
\end{displaymath}

Voor alle constanten $\texttt{di}_{\mathfrak{A}}$ hebben we

\begin{eqnarray*}
\phi(\texttt{di}_{\mathfrak{A}}) & = & \phi(a_{i}) \\
                                 & = & \square \\
                                 & = & \texttt{di}_{\mathfrak{B}},
\end{eqnarray*}

en voor $\texttt{error}_{\mathfrak{A}}$ hebben we

\begin{eqnarray*}
\phi(\texttt{error}_{\mathfrak{A}}) & = & \phi(\bot) \\
                                    & = & \square \\
                                    & = & \texttt{di}_{\mathfrak{B}}.
\end{eqnarray*}

Ook $\texttt{push}$, $\texttt{pop}$ en $\texttt{top}$ voldoen aan de
voorwaarden, want voor alle $d \in A_{\texttt{data}}$, $s \in
A_{\texttt{stack}}$ is

\begin{eqnarray*}
\phi(\texttt{push}_{\mathfrak{A}}(d,s)) & = & \phi(conc(d,s)) \\
                                        & = & |conc(d,s)| \\
                                        & = & |s| + 1 \\
                                        & = & \texttt{push}_{\mathfrak{B}}(\square,|s|) \\
                                        & = & \texttt{push}_{\mathfrak{B}}(\phi(d),\phi(s)),
\end{eqnarray*}

\begin{eqnarray*}
\phi(\texttt{pop}_{\mathfrak{A}}(s)) & = & \phi(tail(s)) \\
                                     & = & |tail(s)| \\
                                     & = & max(|s| - 1, 0) \\
                                     & = & \texttt{pop}_{\mathfrak{B}}(|s|) \\
                                     & = & \texttt{pop}_{\mathfrak{B}}(\phi(s)),
\end{eqnarray*}

en

\begin{eqnarray*}
\phi(\texttt{top}_{\mathfrak{A}}(s)) & = & \phi(head(s)) \\
                                     & = & \phi(e) \quad \text{met $e \in A_{\texttt{data}}$} \\
                                     & = & \square \\
                                     & = & \texttt{top}_{\mathfrak{B}}(|s|) \\
                                     & = & \texttt{top}_{\mathfrak{B}}(\phi(s)).
\end{eqnarray*}

Hieruit volgt dat $\phi$ een homomorphisme van $\mathfrak{A}$ naar
$\mathfrak{B}$ is en dus bestaat een dergelijk homomorphisme.\\[2em]


{\bf Opgave 4.5} % beschrijf termalgebra's van Naturals, Booleans en NatBool

De termalgebra $\mathfrak{Ter}_{\texttt{Naturals}}$ heeft als drager de
verzameling gesloten termen $Ter_{\Sigma}$ uit $\texttt{Naturals}$. De
interpretaties van de functiesymbolen uit $\texttt{Naturals}$ zijn als volgt:

\begin{eqnarray*}
  \texttt{0}_{\mathfrak{Ter}_{\texttt{Naturals}}}        & = & \texttt{0} \\
  \texttt{succ}_{\mathfrak{Ter}_{\texttt{Naturals}}}(t)  & = & \texttt{succ}(t) \\
  \texttt{add}_{\mathfrak{Ter}_{\texttt{Naturals}}}(t,u) & = & \texttt{add}(t,u) \\
  \texttt{mul}_{\mathfrak{Ter}_{\texttt{Naturals}}}(t,u) & = & \texttt{mul}(t,u)
\end{eqnarray*}

Voor de termalgebra $\mathfrak{Ter}_{\texttt{Booleans}}$ van
$\texttt{Booleans}$ gebruiken we als drager de verzameling gesloten termen
$Ter_{\Sigma}$ uit $\texttt{Booleans}$. De interpretaties van de
functiesymbolen zijn als volgt:

\begin{eqnarray*}
  \texttt{true}_{\mathfrak{Ter}_{\texttt{Naturals}}}     & = & \texttt{true} \\
  \texttt{false}_{\mathfrak{Ter}_{\texttt{Naturals}}}    & = & \texttt{false} \\
  \texttt{and}_{\mathfrak{Ter}_{\texttt{Naturals}}}(t,u) & = & \texttt{and}(t,u) \\
  \texttt{or}_{\mathfrak{Ter}_{\texttt{Naturals}}}(t,u)  & = & \texttt{or}(t,u) \\
  \texttt{not}_{\mathfrak{Ter}_{\texttt{Naturals}}}(t)   & = & \texttt{not}(t)
\end{eqnarray*}

De drager van de termalgebra $\mathfrak{Ter}_{\texttt{NatBool}}$ is de
verzameling gesloten termen $Ter_{\Sigma}$ uit $\texttt{NatBool}$. De
interpretaties van de functiesymbolen zijn dezelfde als in
$\mathfrak{Ter}_{\texttt{Naturals}}$ en $\mathfrak{Ter}_{\texttt{Booleans}}$
met daar aan toegevoegd de volgende interpretaties:

\begin{eqnarray*}
  \texttt{even}_{\mathfrak{Ter}_{\texttt{NatBool}}}(t)   & = & \texttt{even}(t) \\
  \texttt{odd}_{\mathfrak{Ter}_{\texttt{NatBool}}}(t)   & = & \texttt{odd}(t)
\end{eqnarray*}


{\bf Opgave 4.6}

Todo.\\[2em]


{\bf Opgave 4.7}

Todo.\\[2em]


\end{document}
