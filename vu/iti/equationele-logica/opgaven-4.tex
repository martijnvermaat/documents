\documentclass[a4paper,11pt]{article}
\usepackage[dutch]{babel}
\usepackage{amsfonts}
\usepackage{a4}
\usepackage{latexsym}
\usepackage{fitch} % http://folk.uio.no/johanw/FitchSty.html

% niet te gewichtig willen doen met veel ruimte
\setlength{\textwidth}{16cm}
\setlength{\textheight}{23.0cm}
\setlength{\topmargin}{0cm}
\setlength{\oddsidemargin}{0.2mm}
\setlength{\evensidemargin}{0.2mm}
\setlength{\parindent}{0cm}

% standaard enumerate met abc
\renewcommand\theenumi{\alph{enumi}}


\begin{document}


{\bf Uitwerkingen bij Inleiding Theoretische Informatica\\
Deel 1: Equationele Logica -- Algebra's}\\[2em]


{\bf Opgave 4.1}

\begin{enumerate}

\item % laat zien dat \x->x een homomorphisme is van N naar Z
Hiertoe moeten we laten zien dat $\phi$ over alle functies voldoet aan de
voorwaarden van een homomorphisme:

\begin{itemize}

\item{de constante $\verb|0|_{\mathfrak{N}}$}

  \begin{eqnarray*}
    \phi(\verb|0|_{\mathfrak{N}}) & = & \phi(0) \\
                                  & = & 0 \\
                                  & = & \verb|0|_{\mathfrak{Z}}
  \end{eqnarray*}

\item{de functie $\verb|succ|_{\mathfrak{N}}$}

  Voor willekeurige $x$ uit $\{0,1,2,\ldots\}$:

  \begin{eqnarray*}
    \phi(\verb|succ|_{\mathfrak{N}}(x)) & = & \phi(x+1) \\
                                        & = & x+1 \\
                                        & = & \verb|succ|_{\mathfrak{Z}}(x) \\
                                        & = & \verb|succ|_{\mathfrak{Z}}(\phi(x))
  \end{eqnarray*}

\item{de functie $\verb|add|_{\mathfrak{N}}$}

  Voor willekeurige $x,y$ uit $\{0,1,2,\ldots\}$:

  \begin{eqnarray*}
    \phi(\verb|add|_{\mathfrak{N}}(x,y)) & = & \phi(x+y) \\
                                         & = & x+y \\
                                         & = & \verb|add|_{\mathfrak{Z}}(x,y) \\
                                         & = &
                                         \verb|add|_{\mathfrak{Z}}(\phi(x),\phi(y))
  \end{eqnarray*}

\item{de functie $\verb|mul|_{\mathfrak{N}}$}

  Voor willekeurige $x,y$ uit $\{0,1,2,\ldots\}$:

  \begin{eqnarray*}
    \phi(\verb|mul|_{\mathfrak{N}}(x,y)) & = & \phi(x*y) \\
                                         & = & x*y \\
                                         & = & \verb|mul|_{\mathfrak{Z}}(x,y) \\
                                         & = & \verb|mul|_{\mathfrak{Z}}(\phi(x),\phi(y))
  \end{eqnarray*}

\end{itemize}

Hiermee hebben we laten zien dat $\phi$ aan alle voorwaarden van een
homomorphisme voldoet.

\item % bestaan er meer homomorphismes van N naar Z?

Nee?

\item % laat zien dat er geen homomorphismes bestaan van Z naar N

Stel, er is een homomorphisme $\phi$ van $\mathfrak{Z}$ naar
$\mathfrak{N}$. We bekijken $\phi(-1)=n$ en weten dat $n \in \mathcal{N}$ ($n
\ge 0$). Nu hebben we

\begin{displaymath}
\phi(\verb|succ|_{\mathfrak{Z}}(-1)) \, = \, \phi(0) \, = \, 0
\end{displaymath}

omdat $\phi(\verb|0|_{\mathfrak{Z}}=0)$ gelijk moet zijn aan
$\verb|0|_{\mathfrak{N}} = 0$.

Vervolgens zien we dat

\begin{displaymath}
\verb|succ|_{\mathfrak{N}}(\phi(-1)) = \verb|succ|_{\mathfrak{N}}(n) = n+1 \ne 0
\end{displaymath}

en dus voldoet $\phi$ niet aan de voorwaarden van een homomorphisme. Dit is in
tegenspraak met de aanname dat $\phi$ een homomorphisme is, dus kan $\phi$ niet
bestaan.\\[2em]

\end{enumerate}


{\bf Opgave 4.2}

\begin{enumerate}

\item % is \n->n+3 een homomorphisme in N+

  $\phi(n) = n+3$ is geen homomorphisme in $\mathcal{N}^{+}$, want we zien
  bijvoorbeeld dat:

  \begin{eqnarray*}
    \phi(\verb|add|(2,5)) & = & \phi(7) \\
                          & = & 10 \\
                          & \ne & 13 \\
                          & = & \verb|add|(5,8) \\
                          & = & \verb|add|(\phi(2),\phi(5))
  \end{eqnarray*}

\item % is \n->2n een homomorphisme in N+

  $\phi(n) = 2n$ is wel een homomorphisme in $\mathcal{N}^{+}$, want voor
  willekeurige $x,y$ uit $\mathbb{N}$:

  \begin{eqnarray*}
    \phi(\verb|0|_{\mathbb{N}^{+}}) & = & \phi(0) \\
                                    & = & 0 \\
                                    & = & \verb|0|_{\mathbb{N}^{+}}
  \end{eqnarray*}

  \begin{eqnarray*}
    \phi(\verb|add|(x,y)) & = & \phi(x+y) \\
                          & = & 2(x+y) \\
                          & = & 2x \, + \, 2y \\
                          & = & \verb|add|(2x, 2y) \\
                          & = & \verb|add|(\phi(x), \phi(y))
  \end{eqnarray*}

  Hiermee voldoet $\phi$ aan alle voorwaarden.\\[2em]

\end{enumerate}


{\bf Opgave 4.3}

Todo.


{\bf Opgave 4.4}

Todo.


{\bf Opgave 4.5}

Todo.


{\bf Opgave 4.6}

Todo.


{\bf Opgave 4.7}

Todo.


\end{document}
