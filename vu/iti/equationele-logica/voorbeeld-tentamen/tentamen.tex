\documentclass[a4paper,11pt]{article}
\usepackage[dutch]{babel}
\usepackage{a4,fullpage}
\usepackage{amsmath,amsfonts,amssymb}
\usepackage{fitch}

\renewcommand{\familydefault}{\sfdefault}


\title{Voorbeeld tentamenvragen Equationele Logica}
\date{5 maart 2005}


\begin{document}

\maketitle


\section*{Opgave 1}

Beschouw de volgende specificatie $\texttt{Spec}$:

\begin{quote}
\begin{verbatim}
module Spec

  sorts object

  functions
    0    :                 -> object
    succ : object          -> object
    add  : object # object -> object

  equations
    [E1] : add(x,0)       = x
    [E2] : add(x,succ(y)) = succ(add(x,y))

end 
\end{verbatim}
\end{quote}

\begin{description}

\item{\bf a.}
Geef een formele afleiding in de specificatie $\texttt{Spec}$ van de
vergelijking
\begin{displaymath}
\texttt{add(add(succ(z),0),0) = add(z,succ(0))}.
\end{displaymath}

\item{\bf b.}
De specificatie beschrijft de natuurlijke getallen met $\texttt{succ}$ als de
opvolger functie en $\texttt{add}$ als optelling. Om alle gehele getallen te
kunnen beschrijven wordt de specificatie uitgebreid met de volgende
functiesymbolen:

\begin{quote}
\begin{verbatim}
    pred : object          -> object
    sub  : object # object -> object
\end{verbatim}
\end{quote}

Hiermee worden respectievelijk de functies voorganger en aftrekken
bedoeld. Breid de specificatie uit met geschikte vergelijkingen die het gedrag
van de nieuwe verzameling functiesymbolen op de gehele getallen beschrijven.

\end{description}


\section*{Opgave 2}

We werken in de signatuur $(S,\Sigma)$ met \'e\'en soort: $\texttt{nat}$; en
alleen de constante $\texttt{0}$ en het binaire functiesymbool
$\texttt{add}$.
$\mathcal{N}^{+}$ is de $\Sigma$-algebra met als domein de natuurlijke
getallen en de voor de hand liggende interpretaties: $\texttt{0}$ als het
getal $0$ en $\texttt{add}$ als de gewone optelling. Welke van de volgende
twee afbeeldingen zijn homomorfismes? Toelichting is alleen nodig wanneer een
afbeelding geen homomorfisme is.

\begin{description}

\item{\bf a.}
$\phi$ gegeven door $\phi(n) = n+3$.

\item{\bf b.}
$\phi$ gegeven door $\phi(n) = 2n$.

\end{description}


\section*{Opgave 3}

Gegeven is de volgende specificatie $\texttt{Spec}$:

\begin{quote}
\begin{verbatim}
module Spec

  sorts object

  functions
    a :        -> object
    h : object -> object
    s : object -> object

  equations
    [E1] : h(h(x)) = x
    [E2] : s(h(x)) = s(x)

end Spec
\end{verbatim}
\end{quote}

Voor deze specificatie beschouwen we de volgende $\Sigma$-algebra's
$\mathfrak{K}$, $\mathfrak{L}$, $\mathfrak{M}$ en $\mathfrak{N}$.

\begin{align*}
\mathfrak{K} &: & &K = \mathbb{N}, & &a_{\mathfrak{K}} = 0, & &h_{\mathfrak{K}}(n) = n+1, & &s_{\mathfrak{K}}(n) = n^{2}. \\
\mathfrak{L} &: & &L = \mathbb{Z}, & &a_{\mathfrak{L}} = 0, & &h_{\mathfrak{L}}(z) = -z, & &s_{\mathfrak{L}}(z) = z^{2}. \\
\mathfrak{M} &: & &M = \mathbb{Z}, & &a_{\mathfrak{M}} = 0, & &h_{\mathfrak{M}}(z) = -z, & &s_{\mathfrak{M}}(z) = |z|+1. \\
\mathfrak{N} &: & &N = \{2,4,16,\ldots\}, & &a_{\mathfrak{N}} = 2, & &h_{\mathfrak{N}}(x) = x, & &s_{\mathfrak{N}}(x) = x^{2}.
\end{align*}

\begin{description}

\item{\bf a.}
Geef een afleiding voor de volgende vergelijking:
\begin{displaymath}
\texttt{s(s(h(a))) = s(h(s(a)))}.
\end{displaymath}

\item{\bf b.}
Precies \'e\'en van de algebra's $\mathfrak{K}$, $\mathfrak{L}$,
$\mathfrak{M}$, $\mathfrak{N}$ is \emph{geen} model voor de specificatie
$\texttt{Spec}$. Geef aan welke dat is en waarom.

\item{\bf c.}
Ga voor elk van de drie algebra's van $\mathfrak{K}$, $\mathfrak{L}$,
$\mathfrak{M}$, $\mathfrak{N}$ die een model zijn voor $\texttt{Spec}$ na of
het een initieel model is voor $\texttt{Spec}$. Alleen negatieve antwoorden
moeten worden gemotiveerd door het aangeven van junk en/of confusion.

\item{\bf d.}
Geef een kanonieke term algebra voor de specificatie $\texttt{Spec}$.

\end{description}


\end{document}
