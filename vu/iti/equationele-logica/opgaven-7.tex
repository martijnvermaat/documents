\documentclass[a4paper,11pt]{article}
\usepackage[dutch]{babel}
\usepackage{amsfonts}
\usepackage{a4}
\usepackage{latexsym}
\usepackage{fitch} % http://folk.uio.no/johanw/FitchSty.html

% niet te gewichtig willen doen met veel ruimte
\setlength{\textwidth}{16cm}
\setlength{\textheight}{23.0cm}
\setlength{\topmargin}{0cm}
\setlength{\oddsidemargin}{0.2mm}
\setlength{\evensidemargin}{0.2mm}
\setlength{\parindent}{0cm}

% standaard enumerate met abc
\renewcommand\theenumi{\alph{enumi}}


\begin{document}


{\bf Uitwerkingen bij Inleiding Theoretische Informatica\\
Deel 1: Equationele Logica -- Het termmodel}\\[2em]


{\bf Opgave 7.1}

Een mogelijkheid voor een initieel correcte specificatie voor de algebra
$\mathfrak{B}$ is de volgende:
\begin{verbatim}
module Spec

  sorts obj

  functions
    0    :           -> obj
    succ : obj       -> obj
    add  : obj # obj -> obj
    mul  : obj # obj -> obj

  equations
    [E1] succ(succ(x)) = x
    [E2] add(x,succ(x)) = succ(0)
    [E3] add(x,x) = 0
    [E4] mul(x,succ(x)) = 0
    [E5] mul(x,x) = x

end Spec
\end{verbatim}


{\bf Opgave 7.2}

\begin{enumerate}

\item % initieel correcte specificatie voor Z

Een initieel correcte specificatie voor de algebra $\mathfrak{Z}$ is de
volgende:
\begin{verbatim}
module Integers

  sorts int

  functions
    succ : int -> int
    pred : int -> int

  equations
    [E1] succ(pred(x)) = x
    [E2] pred(succ(x)) = x

end Integers
\end{verbatim}
Het lijkt logisch ook een constante ($\texttt{0}$) toe te voegen met als
interpretatie het getal 0.

\item % uitbreiden naar initieeel correcte specificatie voor Z+

We breiden $\texttt{Integers}$ uit naar een initieel correcte specificatie
voor $\mathfrak{Z}^{+}$ door de volgende vergelijking toe te voegen:
\begin{verbatim}
    [E3] neg(succ(x)) = pred(neg(x))
    [E4] neg(pred(x)) = succ(neg(x))
\end{verbatim}
Wanneer ook de constante $\texttt{0}$ toegevoegd is, hebben we ook de volgende
vergelijking nog nodig:
\begin{verbatim}
    [E5] neg(0) = 0
\end{verbatim}

\end{enumerate}


{\bf Opgave 7.3}

\begin{enumerate}

\item % kta voor specificatie Booleans

Er zijn twee equivalentieklassen onder afleidbare gelijkheid van gesloten
termen. Uit beide klassen kiezen we \'e\'en representant voor de drager van de
kanonieke term algebra $\mathfrak{A}_{1}$:
\begin{align*}
A_{1} &= \{\texttt{true}, \texttt{false}\}
\end{align*}
De functiesymbolen uit $\texttt{Booleans}$ interpreteren we als volgt:
\begin{align*}
\texttt{true}_{\mathfrak{A}_{1}}     &= \texttt{true} \\
\texttt{false}_{\mathfrak{A}_{1}}    &= \texttt{false} \\
\texttt{and}_{\mathfrak{A}_{1}}(t,s) &= \begin{cases}
  \texttt{true}  & \text{als $t = s = \texttt{true}$} \\
  \texttt{false} & \text{in de overige gevallen}
\end{cases} \\
\texttt{or}_{\mathfrak{A}_{1}}(t,s) &= \begin{cases}
  \texttt{true}  & \text{als $t = \texttt{true}$} \\
  \texttt{true}  & \text{als $s = \texttt{true}$} \\
  \texttt{false} & \text{in de overige gevallen}
\end{cases} \\
\texttt{not}_{\mathfrak{A}_{1}}(t) &= \begin{cases}
  \texttt{true}  & \text{als $t = \texttt{false}$} \\
  \texttt{false} & \text{als $t = \texttt{true}$}
\end{cases}
\end{align*}

\item % nog een

We kunnen ook andere representanten voor de drager kiezen, bijvoorbeeld zoals
in deze algebra $\mathfrak{A}_{2}$:
\begin{align*}
A_{2} &= \{\texttt{true}, \texttt{not(true)}\}
\end{align*}
De functiesymbolen uit $\texttt{Booleans}$ interpreteren we nu als volgt:
\begin{align*}
\texttt{true}_{\mathfrak{A}_{2}}     &= \texttt{true} \\
\texttt{false}_{\mathfrak{A}_{2}}    &= \texttt{not(true)} \\
\texttt{and}_{\mathfrak{A}_{2}}(t,s) &= \begin{cases}
  \texttt{true}      & \text{als $t = s = \texttt{true}$} \\
  \texttt{not(true)} & \text{in de overige gevallen}
\end{cases} \\
\texttt{or}_{\mathfrak{A}_{2}}(t,s) &= \begin{cases}
  \texttt{true}      & \text{als $t = \texttt{true}$} \\
  \texttt{true}      & \text{als $s = \texttt{true}$} \\
  \texttt{not(true)} & \text{in de overige gevallen}
\end{cases} \\
\texttt{not}_{\mathfrak{A}_{2}}(t) &= \begin{cases}
  \texttt{true}      & \text{als $t = \texttt{not(true)}$} \\
  \texttt{not(true)} & \text{als $t = \texttt{true}$}
\end{cases}
\end{align*}

\item % nog een

De derde (en laatste) mogelijkheid voor de drager van een kta bij deze
specificatie is de volgende algebra $\mathfrak{A}_{3}$:
\begin{align*}
A_{3} &= \{\texttt{not(false)}, \texttt{false}\}
\end{align*}
De functiesymbolen uit $\texttt{Booleans}$ interpreteren we nu als volgt:
\begin{align*}
\texttt{true}_{\mathfrak{A}_{3}}     &= \texttt{not(false)} \\
\texttt{false}_{\mathfrak{A}_{3}}    &= \texttt{false} \\
\texttt{and}_{\mathfrak{A}_{3}}(t,s) &= \begin{cases}
  \texttt{not(false)} & \text{als $t = s = \texttt{not(false)}$} \\
  \texttt{false}      & \text{in de overige gevallen}
\end{cases} \\
\texttt{or}_{\mathfrak{A}_{3}}(t,s) &= \begin{cases}
  \texttt{not(false)} & \text{als $t = \texttt{not(false)}$} \\
  \texttt{not(false)} & \text{als $s = \texttt{not(false)}$} \\
  \texttt{false}      & \text{in de overige gevallen}
\end{cases} \\
\texttt{not}_{\mathfrak{A}_{3}}(t) &= \begin{cases}
  \texttt{not(false)} & \text{als $t = \texttt{false}$} \\
  \texttt{false}      & \text{als $t = \texttt{not(false)}$}
\end{cases}
\end{align*}

\end{enumerate}


{\bf Opgave 7.4}

\begin{enumerate}

\item % initieel correcte specificatie E voor B

Een initieel correcte specificatie $E$ voor $\mathfrak{B}$ bestaat uit de
volgende vergelijkingen:
\begin{verbatim}
[E1] f(f(f(x))) = f(f(c))
[E2] f(a) = c
\end{verbatim}

\item % kta voor E

Een kta voor $E$ is de algebra $\mathfrak{E}$ met als drager de verzameling
\begin{align*}
\{\texttt{a}, \, \texttt{c}, \, \texttt{f(c)}, \, \texttt{f(f(c))}\}
\end{align*}
en als interpretaties
\begin{align*}
\texttt{a}_{\mathfrak{E}}          &= \texttt{a}, \\
\texttt{c}_{\mathfrak{E}}          &= \texttt{c}, \\
\texttt{f}_{\mathfrak{E}}(a)       &= \texttt{c}, \\
\texttt{f}_{\mathfrak{E}}(c)       &= \texttt{f(c)}, \\
\texttt{f}_{\mathfrak{E}}(f(c))    &= \texttt{f(f(c))}, \\
\texttt{f}_{\mathfrak{E}}(f(f(c))) &= \texttt{f(f(c))}.
\end{align*}

\end{enumerate}


{\bf Opgave 7.5}

\begin{enumerate}

\item % laat zien dat afleidbaar is in(a,in(b,in(a,e)))=in(a,in(b,e))

\item % zijn K,L,M modellen van Abc?

\item % initiele modellen?

\item % beschrijf initieel model met uit elke equivalentieklasse een element

\item % is dit een kta? zo nee, pas het aan tot een kta

\end{enumerate}


\end{document}
