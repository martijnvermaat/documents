\documentclass[a4paper,11pt]{article}
\usepackage[dutch]{babel}
\usepackage{amsfonts}
\usepackage{a4}
\usepackage{latexsym}
\usepackage{fitch} % http://folk.uio.no/johanw/FitchSty.html

% niet te gewichtig willen doen met veel ruimte
\setlength{\textwidth}{16cm}
\setlength{\textheight}{23.0cm}
\setlength{\topmargin}{0cm}
\setlength{\oddsidemargin}{0.2mm}
\setlength{\evensidemargin}{0.2mm}
\setlength{\parindent}{0cm}

% standaard enumerate met abc
\renewcommand\theenumi{\alph{enumi}}


\begin{document}


{\bf Uitwerkingen bij Inleiding Theoretische Informatica\\
Deel 1: Equationele Logica -- Het termmodel}\\[2em]


{\bf Opgave 7.1}

Een mogelijkheid voor een initieel correcte specificatie voor de algebra
$\mathfrak{B}$ is de volgende:
\begin{verbatim}
module Spec

  sorts obj

  functions
    0    :           -> obj
    succ : obj       -> obj
    add  : obj # obj -> obj
    mul  : obj # obj -> obj

  equations
    [E1] succ(succ(x)) = x
    [E2] add(x,succ(x)) = succ(0)
    [E3] add(x,x) = 0
    [E4] mul(x,succ(x)) = 0
    [E5] mul(x,x) = x

end Spec
\end{verbatim}


{\bf Opgave 7.2}

\begin{enumerate}

\item % initieel correcte specificatie voor Z

Een initieel correcte specificatie voor de algebra $\mathfrak{Z}$ is de
volgende:
\begin{verbatim}
module Integers

  sorts int

  functions
    succ : int -> int
    pred : int -> int

  equations
    [E1] succ(pred(x)) = x
    [E2] pred(succ(x)) = x

end Integers
\end{verbatim}
Het lijkt logisch ook een constante ($\texttt{0}$) toe te voegen met als
interpretatie het getal 0.

\item % uitbreiden naar initieeel correcte specificatie voor Z+

We breiden $\texttt{Integers}$ uit naar een initieel correcte specificatie
voor $\mathfrak{Z}^{+}$ door de volgende vergelijking toe te voegen:
\begin{verbatim}
    [E3] neg(succ(x)) = pred(neg(x))
    [E4] neg(pred(x)) = succ(neg(x))
\end{verbatim}
Wanneer ook de constante $\texttt{0}$ toegevoegd is, hebben we ook de volgende
vergelijking nog nodig:
\begin{verbatim}
    [E5] neg(0) = 0
\end{verbatim}

\end{enumerate}


{\bf Opgave 7.3}

\begin{enumerate}

\item % kta voor specificatie Booleans

Als drager de termen $\texttt{true}$ en $\texttt{false}$.

\item % nog een

Als drager de termen $\texttt{true}$ en $\texttt{not(true)}$.

\item % nog een

Als drager de termen $\texttt{not(false)}$ en $\texttt{false}$.

\end{enumerate}


\end{document}
