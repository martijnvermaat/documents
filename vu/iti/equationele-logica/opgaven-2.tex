\documentclass[a4paper,11pt]{article}
\usepackage[dutch]{babel}
\usepackage{amsfonts}
\usepackage{a4}
\usepackage{latexsym}

% niet te gewichtig willen doen met veel ruimte
\setlength{\textwidth}{16cm}
\setlength{\textheight}{23.0cm}
\setlength{\topmargin}{0cm}
\setlength{\oddsidemargin}{0.2mm}
\setlength{\evensidemargin}{0.2mm}
\setlength{\parindent}{0cm}

% standaard enumerate met abc
\renewcommand\theenumi{\alph{enumi}}


\begin{document}


{\bf Uitwerkingen bij Inleiding Theoretische Informatica\\
Deel 1: Equationele Logica -- Abstracte specificaties}\\[2em]


{\bf Opgave 2.1}

\begin{enumerate}

\item % eindige signatuur voor gehele getallen
Signatuur $(S, \Sigma)$ met
\begin{eqnarray*}
S & = & \{int\} \\
\Sigma_{\rightarrow int} & = & \{0\} \\
\Sigma_{int \rightarrow int} & = & \{succ, pred\}.
\end{eqnarray*}

Of in ASF notatie.

\item % stelsel vergelijkingen bij signatuur
\begin{verbatim}
[E1] succ(pred(x)) = x
[E2] pred(succ(x)) = x
\end{verbatim}

\item % extra
Optioneel ook \verb|add| en \verb|mul| toevoegen, bijvoorbeeld zo:

\begin{verbatim}
functions
  add: int # int -> int
  mul: int # int -> int

equations
  [A1] : add(x, 0)       = x
  [A2] : add(x, succ(y)) = succ(add(x, y))
  [A3] : add(x, pred(y)) = pred(add(x, y))
  [S1] : sub(x, 0)       = x
  [S2] : sub(x, succ(y)) = pred(sub(x, y))
  [S3] : sub(x, pred(y)) = succ(sub(x, y))
  [M1] : mul(x, 0)       = 0
  [M2] : mul(x, succ(y)) = add(mul(x, y), x)
  [M3] : mul(x, pred(y)) = sub(mul(x, y), x)
\end{verbatim}

(Klopt dit allemaal?)

\end{enumerate}


{\bf Opgave 2.2} % FIFO queue

Specificatie voor een FIFO queue:

\begin{verbatim}
module FIFO

  sorts
    data, queue

  functions
    peek    : queue        -> data
    dequeue : queue        -> queue
    enqueue : data # queue -> queue
    empty   :              -> queue
    error   :              -> data

  equations
    [P1] peek(empty)                        = error
    [P2] peek(enqueue(d, empty)             = x
    [P3] peek(enqueue(d, enqueue(e, q)))    = peek(enqueue(e, q))
    [D1] dequeue(empty)                     = empty
    [D2] dequeue(enqueue(d, empty))         = empty
    [D3] dequeue(enqueue(d, enqueue(e, q))) = enqueue(d, dequeue(enqueue(e, q)))

end FIFO
\end{verbatim}


{\bf Opgave 2.3} % specificatie voor: d(x,y) <-> x is een deler van y

We breiden \verb|NatBool| uit met de functie \verb|d|, de intentie is:

\begin{displaymath}
\verb|d(m, n)| \quad \Longleftrightarrow \quad \mbox{$m$ is een deler van $n$} \quad \Longleftrightarrow \quad \mbox{$m \neq 0$ en er is een $k$ zo dat $m k = n$}
\end{displaymath}

\begin{verbatim}
functions
  d : Nat # Nat -> Bool

equations
  [D1] d(0, x)                           = false
  [D2] d(x, add(x, y))                   = d(x, y)
  [D3] d(succ(x), 0)                     = true
  [D4] d(add(succ(y), succ(x)), succ(y)) = false
\end{verbatim}

Het algoritme dat hier gebruikt wordt om de uitkomst van \verb|d(m,n)| te
bepalen is als volgt:

\begin{description}

\item{\verb|m=0|:}
Delen door nul kan niet, uitkomst is dus \verb|false|.

\item{\verb|m>0|:}
Trek \verb|m| net zo lang van \verb|n| af tot \verb|n| gelijk is aan \verb|0| (uitkomst is
\verb|true|), of \verb|m| groter is dan \verb|n| (uitkomst is \verb|false|).

\end{description}


\end{document}
