\documentclass[a4paper,11pt]{article}
\usepackage[dutch]{babel}
\usepackage{a4,fullpage}
\usepackage{amsmath,amsfonts,amssymb}
\usepackage{fitch}

\renewcommand{\familydefault}{\sfdefault}


\title{Voorbeeld uitwerkingen Equationele Logica\\
\normalsize{bij opgaven 3.1, 3.4, 4.1, 4.4, 4.6, 5.1, 5.5, 6.1, 6.2, 7.3 en 7.4}}
\date{5 maart 2005}


\begin{document}

\maketitle


\section*{Opgave 3.1}

\begin{description}

\item{\bf (a)}
Een afleiding in de specificate $\texttt{Stack-Of-Data}$ van de vergelijking
\begin{quote}
\begin{verbatim}
pop(pop(pop(empty))) = empty
\end{verbatim}
\end{quote}
\begin{equation*}
\begin{fitch}
\texttt{pop(empty) = empty}                        & \texttt{[1]}  \\ % 1
\texttt{pop(pop(empty)) = pop(empty)}              & congr, 1      \\ % 2
\texttt{pop(pop(pop(empty))) = pop(pop(empty))}    & congr, 2      \\ % 3
\texttt{pop(pop(empty))) = empty}                  & trans, 1, 2   \\ % 4
\texttt{pop(pop(pop(empty))) = empty}              & trans, 3, 4      % 5
\end{fitch}
\end{equation*}

\item{\bf (b)}
Een afleiding in de specificatie \texttt{Stack-Of-Data} van de vergelijking
\begin{quote}
\begin{verbatim}
pop(push(x, pop(push(x, empty)))) = empty
\end{verbatim}
\end{quote}
\begin{equation*}
\begin{fitch}
\texttt{pop(push(x, s)) = s}                                     & \texttt{[3]}  \\ % 1
\texttt{pop(push(x, empty)) = empty}                             & subs, 1       \\ % 2
\texttt{pop(push(x, pop(push(x, empty)))) = pop(push(x, empty))} & subs, 1       \\ % 3
\texttt{pop(push(x, pop(push(x, empty)))) = empty}               & trans, 3, 2      % 4
\end{fitch}
\end{equation*}

\end{description}


\section*{Opgave 3.4}

Om te laten zien dat $\sim$ een equivalentierelatie is, laten we zien dat $\sim$
transitief, symmetrisch en reflexief is.

\begin{description}

\item{Transitiviteit}

  Stel $t \sim u$ en $u \sim v$, dan is volgens de definitie $E \vdash t = u$ en
  $E \vdash u = v$. Wegens de regel \emph{transitiviteit} van de
  afleidbaarheid, is dan ook $E \vdash t = v$. Met de definitie zien we dan
  weer dat ook $t \sim v$ waar is en dus is $\sim$ transitief.

\item{Symmetrie}

  Stel $t \sim u$, dan is volgens de definitie $E \vdash t = u$. De
  afleidingsregel \emph{symmetrie} zegt dat dan ook $E \vdash u = t$ en
  volgens de definitie zien we dan weer dat ook $u \sim t$. En dus is $\sim$
  symmetrisch.

\item{Reflexiviteit}

  Voor iedere term $t$ geldt volgens de afleidingsregel \emph{reflexiviteit}
  dat $E \vdash t = t$. Onze definitie zegt dan dat ook $t \sim t$ en dus is
  $\sim$ reflexief.

\end{description}

Omdat $\sim$ transitief, symmetrisch en reflexief is, is $\sim$ een
equivalentierelatie.


\section*{Opgave 4.1}

\begin{description}

\item{\bf (a)}
Hiertoe moeten we laten zien dat $\phi$ over alle functies voldoet aan de
voorwaarden van een homomorphisme:

\begin{itemize}

\item{de constante $\texttt{0}_{\mathfrak{N}}$}
  \begin{align*}
    \phi(\texttt{0}_{\mathfrak{N}}) &= \phi(0) \\
                                    &= 0 \\
                                    &= \texttt{0}_{\mathfrak{Z}}
  \end{align*}

\item{de functie $\texttt{succ}_{\mathfrak{N}}$}

  Voor willekeurige $x \in \mathbb{N}$ hebben we:
  \begin{align*}
    \phi(\texttt{succ}_{\mathfrak{N}}(x)) &= \phi(x+1) \\
                                          &= x+1 \\
                                          &= \texttt{succ}_{\mathfrak{Z}}(x) \\
                                          &= \texttt{succ}_{\mathfrak{Z}}(\phi(x))
  \end{align*}

\item{de functie $\texttt{add}_{\mathfrak{N}}$}

  Voor willekeurige $x,y \in \mathbb{N}$ hebben we:
  \begin{align*}
    \phi(\texttt{add}_{\mathfrak{N}}(x,y)) &= \phi(x+y) \\
                                           &= x+y \\
                                           &= \texttt{add}_{\mathfrak{Z}}(x,y) \\
                                           &= \texttt{add}_{\mathfrak{Z}}(\phi(x),\phi(y))
  \end{align*}

\item{de functie $\texttt{mul}_{\mathfrak{N}}$}

  Voor willekeurige $x,y \in \mathbb{N}$ hebben we:
  \begin{align*}
    \phi(\texttt{mul}_{\mathfrak{N}}(x,y)) &= \phi(x*y) \\
                                           &= x*y \\
                                           &= \texttt{mul}_{\mathfrak{Z}}(x,y) \\
                                           &= \texttt{mul}_{\mathfrak{Z}}(\phi(x),\phi(y))
  \end{align*}

\end{itemize}

Hiermee hebben we laten zien dat $\phi$ aan alle voorwaarden van een
homomorphisme voldoet.

\item{\bf (b)} % bestaan er meer homomorphismes van N naar Z?
Nee, de interpretatie van $\texttt{0}$ in $\mathfrak{N}$ moet op de
interpretatie van $\texttt{0}$ in $\mathfrak{Z}$ afgebeeld worden. Met de
voorwaarden van een homomorfisme over de successor functie ligt daarmee heel
de afbeelding vast.

\item{\bf (c)} % laat zien dat er geen homomorphismes bestaan van Z naar N
Stel, er is een homomorphisme $\phi$ van $\mathfrak{Z}$ naar
$\mathfrak{N}$. We bekijken $\phi(-1)=n$ en weten dat $n \in \mathbb{N}$ ($n
\ge 0$). Nu hebben we
\begin{displaymath}
\phi(\texttt{succ}_{\mathfrak{Z}}(-1)) \, = \, \phi(0) \, = \, 0
\end{displaymath}
omdat $\phi(\texttt{0}_{\mathfrak{Z}}=0)$ gelijk moet zijn aan
$\texttt{0}_{\mathfrak{N}} = 0$.

Vervolgens zien we dat
\begin{displaymath}
\texttt{succ}_{\mathfrak{N}}(\phi(-1)) = \texttt{succ}_{\mathfrak{N}}(n) = n+1 \ne 0
\end{displaymath}
en dus voldoet $\phi$ niet aan de voorwaarden van een homomorphisme. Dit is in
tegenspraak met de aanname dat $\phi$ een homomorphisme is, dus kan $\phi$ niet
bestaan.

\end{description}


\section*{Opgave 4.4}

De vraag
\begin{quote}``Laat zien dat er een \ldots bestaat.''\end{quote}
betekent in de praktijk (bijna) altijd:
\begin{quote}``Geef een \ldots en laat zien dat het een \ldots is.''\end{quote}

We geven het homomorphisme $\phi$ van $\mathfrak{A}$ naar $\mathfrak{B}$
gedefini\"eerd als
\begin{align*}
\phi(x) &= \begin{cases}
  \square & \text{als $x \in A_{\texttt{data}}$;} \\
  |x|     & \text{als $x \in A_{\texttt{stack}}$,}
\end{cases}
\end{align*}
waarbij $|x|$ de lengte van de string $x$ geeft.

Nu laten we zien dat $\phi$ daadwerkelijk een homomorfisme is. Voor alle
constanten $\texttt{di}_{\mathfrak{A}}$ hebben we
\begin{align*}
\phi(\texttt{di}_{\mathfrak{A}}) &= \phi(a_{i}) \\
                                 &= \square \\
                                 &= \texttt{di}_{\mathfrak{B}},
\end{align*}
en voor $\texttt{error}_{\mathfrak{A}}$ hebben we
\begin{align*}
\phi(\texttt{error}_{\mathfrak{A}}) &= \phi(\bot) \\
                                    &= \square \\
                                    &= \texttt{di}_{\mathfrak{B}}.
\end{align*}

Ook over $\texttt{push}$, $\texttt{pop}$ en $\texttt{top}$ voldoet $\phi$ aan
de voorwaarden, want voor alle $d \in A_{\texttt{data}}$ en $s \in
A_{\texttt{stack}}$ is
\begin{align*}
\phi(\texttt{push}_{\mathfrak{A}}(d,s)) &= \phi(conc(d,s)) \\
                                        &= |conc(d,s)| \\
                                        &= |s| + 1 \\
                                        &= \texttt{push}_{\mathfrak{B}}(\square,|s|) \\
                                        &= \texttt{push}_{\mathfrak{B}}(\phi(d),\phi(s)),
\end{align*}
\begin{align*}
\phi(\texttt{pop}_{\mathfrak{A}}(s)) &= \phi(tail(s)) \\
                                     &= |tail(s)| \\
                                     &= max(|s| - 1, \, 0) \\
                                     &= \texttt{pop}_{\mathfrak{B}}(|s|) \\
                                     &= \texttt{pop}_{\mathfrak{B}}(\phi(s)),
\end{align*}
en
\begin{align*}
\phi(\texttt{top}_{\mathfrak{A}}(s)) &= \phi(head(s)) \\
                                     &= \phi(e) &&\text{met $e \in A_{\texttt{data}}$} \\
                                     &= \square \\
                                     &= \texttt{top}_{\mathfrak{B}}(|s|) \\
                                     &= \texttt{top}_{\mathfrak{B}}(\phi(s)).
\end{align*}

Hieruit volgt dat $\phi$ een homomorphisme van $\mathfrak{A}$ naar
$\mathfrak{B}$ is en dus bestaat een dergelijk homomorphisme.


\section*{Opgave 4.6}

\begin{description}

\item{\bf (a)} % laat zien dat de eigenschap homomorphisme gesloten is onder compositie
Gegeven een homomorphisme van $\mathfrak{A}$ naar $\mathfrak{B}$
\begin{align*}
\psi &: A \rightarrow B
\end{align*}
en een homomorphisme van $\mathfrak{B}$ naar $\mathfrak{C}$
\begin{align*}
\omega &: B \rightarrow C
\end{align*}
krijgen we met de compositie
\begin{align*}
\phi &: A \rightarrow C \\
\phi &= \omega \circ \psi
\end{align*}
een homomorphisme van $\mathfrak{A}$ naar $\mathfrak{C}$.

We hebben namelijk voor alle constanten $c$
\begin{align*}
\phi(c_{\mathfrak{A}}) &= \omega(\psi(c_{\mathfrak{A}})) \\
                       &= \omega(c_{\mathfrak{B}}) \\
                       &= c_{\mathfrak{C}}
\end{align*}
en voor alle functiesymbolen $f$ (waarbij $n$ het aantal argumenten van $f$
is)
\begin{align*}
\phi(f_{\mathfrak{A}}(a_{1}, \ldots, a_{n})) &=
                                             \omega(\psi(f_{\mathfrak{A}}(a_{1},
                                             \ldots, a_{n}))) \\
                                             &=
                                             \omega(f_{\mathfrak{B}}(\psi(a_{1}),
                                             \ldots, \psi(a_{n}))) \\
                                             &=
                                             f_{\mathfrak{C}}(\omega(\psi(a_{1})),
                                             \ldots, \omega(\psi(a_{n}))) \\
                                             &=
                                             f_{\mathfrak{C}}(\phi(a_{1}),
                                             \ldots, \phi(a_{n})).
\end{align*}

\item{\bf (b)} % laat zien dat de eigenschap isomorfisme gesloten is onder conversie
Gegeven een isomorfisme van $\mathfrak{A}$ naar $\mathfrak{B}$
\begin{align*}
\psi &: A \rightarrow B
\end{align*}
construeren we een isomorfisme van $\mathfrak{B}$ naar $\mathfrak{A}$ als
\begin{align*}
\phi &: B \rightarrow A \\
\phi &= \psi^{-1}.
\end{align*}

De inverse van $\psi$ bestaat omdat $\psi$ een bijectie is. Bovendien is de
inverse van een bijectie ook weer een bijectie, dus is $\phi$ een bijectie.

We laten nu zien dat $\phi$ een homomorphisme is van $\mathfrak{B}$ naar
$\mathfrak{A}$. Voor alle constanten $c$ hebben we
\begin{align*}
\phi(c_{\mathfrak{B}}) &= \psi^{-1}(c_{\mathfrak{B}}) \\
                       &= c_{\mathfrak{A}}
\end{align*}
en voor alle functiesymbolen $f$ hebben we
\begin{align*}
\phi(f_{\mathfrak{B}}(a_{1}, \ldots, a_{n})) &= \psi^{-1}(f_{\mathfrak{B}}(a_{1}, \ldots, a_{n})) \\
                                             &=
                                             f_{\mathfrak{A}}(\phi(a_{1}),
                                             \ldots, \phi(a_{n})).
\end{align*}
\begin{quote}
Ter verduidelijking: deze laatste gelijkheid volgt uit het feit dat $\psi$ een
homomorphisme is, want daarmee is
\begin{align*}
\psi(f_{\mathfrak{A}}(\phi(a_{1}), \ldots, \phi(a_{n}))) &= f_{\mathfrak{B}}(\psi(\phi(a_{1})), \ldots, \psi(\phi(a_{n}))) \\
                                                         &= f_{\mathfrak{B}}(a_{1}, \ldots, a_{n}).
\end{align*}
\end{quote}

We hebben nu laten zien dat $\phi$ een bijectief homomorphisme is en dus een isomorphisme.

\item{\bf (c)} % laat zien dat isomorphie een equivalentierelatie is
Een relatie is een equivalentierelatie als deze reflexief, symmetrisch en
transitief is.

\paragraph{Reflexiviteit}

Gegeven een algebra $\mathfrak{A}$ bekijken we het isomorphisme
\begin{align*}
\phi(x) &= x.
\end{align*}

Dat $\phi$ bijectief is moge duidelijk zijn. We laten kort zien dat $\phi$ een
homomorphisme is en daarmee dus ook een isomorphisme. Voor constanten $c$ en
functiesymbolen $f$ hebben we
\begin{align*}
\phi(c_{\mathfrak{A}})                       &= c_{\mathfrak{A}} \\
\phi(f_{\mathfrak{A}}(a_{1}, \ldots, a_{n})) &= f_{\mathfrak{A}}(a_{1}, \ldots, a_{n}) \\
                                             &= f_{\mathfrak{A}}(\phi(a_{1}), \ldots, \phi(a_{n})).
\end{align*}

Hiermee is $\phi$ een isomorphisme van $\mathfrak{A}$ naar $\mathfrak{A}$ en
hebben we dus
\begin{align*}
\mathfrak{A} &\cong \mathfrak{A}.
\end{align*}

\paragraph{Symmetrie}

Gegeven twee algebra's $\mathfrak{A}$ en $\mathfrak{B}$ met
\begin{align*}
\mathfrak{A} &\cong \mathfrak{B}.
\end{align*}

Er bestaat dus een isomorphisme van $\mathfrak{A}$ naar $\mathfrak{B}$. We
hebben bij {\bf (b)} gezien dat de inverse van dit isomorphisme een
isomorphisme is van $\mathfrak{B}$ naar $\mathfrak{A}$ en dus hebben we ook
\begin{align*}
\mathfrak{B} &\cong \mathfrak{A}.
\end{align*}

\paragraph{Transitiviteit}

Gegeven drie algebra's $\mathfrak{A}$, $\mathfrak{B}$ en $\mathfrak{C}$ met
\begin{align*}
\mathfrak{A} &\cong \mathfrak{B} \\
\mathfrak{B} &\cong \mathfrak{C}.
\end{align*}

Er bestaan dus isomorphismes $\phi$ van $\mathfrak{A}$ naar $\mathfrak{B}$ en
$\psi$ van $\mathfrak{B}$ naar $\mathfrak{C}$. We construeren nu
\begin{align*}
\omega &= \psi \circ \phi
\end{align*}
welke een homomorphisme van $\mathfrak{A}$ naar $\mathfrak{C}$ is. De
compositie van twee bijecties geeft weer een bijectie en dus is $\omega$ een
isomorphisme van $\mathfrak{A}$ naar $\mathfrak{C}$ en hebben we
\begin{align*}
\mathfrak{A} &\cong \mathfrak{C}.
\end{align*}

Hiermee hebben we laten zien dat $\cong$ een equivalentierelatie is.\\[2em]

\end{description}


\section*{Opgave 5.1}

De algebra $\mathfrak{P}(V)$ (met $V$ een verzameling) heeft als drager de
machtsverzameling van $V$. De functies worden ge\"interpreteerd als
\begin{align*}
\texttt{true}_{\mathfrak{P}(V)}     &= V , \\
\texttt{false}_{\mathfrak{P}(V)}    &= \{\} , \\
\texttt{not}_{\mathfrak{P}(V)}(X)   &= V - X , \\
\texttt{and}_{\mathfrak{P}(V)}(X,Y) &= X \cap Y , \\
\texttt{or}_{\mathfrak{P}(V)}(X,Y)  &= X \cup Y .
\end{align*}

We constateren eerst dat $\mathfrak{P}(V)$ een algebra voor
$\texttt{Booleans}$ is omdat de functietypes kloppen. Vervolgens laten we zien
dat $\mathfrak{P}(V)$ zelfs een model is voor $\texttt{Booleans}$ door te
laten zien dat alle vergelijkingen ($\texttt{[B1]} \ldots \texttt{[B5]}$) waar
zijn in $\mathfrak{P}(V)$.

\paragraph{}

Laat $\theta$ een assignment zijn van elementen uit $\mathcal{P}(V)$ aan
de variabelen $x$ en $y$, volgens
\begin{align*}
\theta    &: \{x,y\} \rightarrow \mathcal{P}(V), \\
\theta(x) &= X, \\
\theta(y) &= Y.
\end{align*}

We laten nu zien dat voor iedere vergelijking uit $\texttt{Booleans}$ de
rechter kant identiek is aan de linker kant onder $\bar \theta$.

\begin{itemize}

\item{$\texttt{[B1] and(true,x) = x}$}
\begin{align*}
\bar \theta(\texttt{and}(\texttt{true}, x)) &= \texttt{and}_{\mathfrak{P}(V)}(\bar \theta(\texttt{true}), \bar \theta(x)) \\
                                            &= \texttt{and}_{\mathfrak{P}(V)}(\texttt{true}_{\mathfrak{P}(V)}, \theta(x)) \\
                                            &= \texttt{and}_{\mathfrak{P}(V)}(V,X) \\
                                            &= V \cap X \\
                                            &= X
\end{align*}

\begin{align*}
\bar \theta(x) &= \theta(x) \\
               &= X
\end{align*}

\item{$\texttt{[B2] and(false,x) = false}$}
\begin{align*}
\bar \theta(\texttt{and}(\texttt{false}, x)) &= \texttt{and}_{\mathfrak{P}(V)}(\bar \theta(\texttt{false}), \bar \theta(x)) \\
                                             &= \texttt{and}_{\mathfrak{P}(V)}(\texttt{false}_{\mathfrak{P}(V)}, \theta(x)) \\
                                             &= \texttt{and}_{\mathfrak{P}(V)}(\{\},X) \\
                                             &= \{\} \cap X \\
                                             &= \{\}
\end{align*}

\begin{align*}
\bar \theta(\texttt{false}) &= \texttt{false}_{\mathfrak{P}(V)} \\
                            &= \{\}
\end{align*}

\item{$\texttt{[B3] not(true) = false}$}
\begin{align*}
\bar \theta(\texttt{not(true)}) &= \texttt{not}_{\mathfrak{P}(V)}(\bar \theta(\texttt{true})) \\
                                &= \texttt{not}_{\mathfrak{P}(V)}(\texttt{true}_{\mathfrak{P}(V)}) \\
                                &= \texttt{not}_{\mathfrak{P}(V)}(V) \\
                                &= V - V \\
                                &= \{\}
\end{align*}

\begin{align*}
\bar \theta(\texttt{false}) &= \texttt{false}_{\mathfrak{P}(V)} \\
                            &= \{\}
\end{align*}

\item{$\texttt{[B4] not(false) = true}$}
\begin{align*}
\bar \theta(\texttt{not(false)}) &= \texttt{not}_{\mathfrak{P}(V)}(\bar \theta(\texttt{false})) \\
                                 &= \texttt{not}_{\mathfrak{P}(V)}(\texttt{false}_{\mathfrak{P}(V)}) \\
                                 &= \texttt{not}_{\mathfrak{P}(V)}(\{\}) \\
                                 &= V - \{\} \\
                                 &= V
\end{align*}

\begin{align*}
\bar \theta(\texttt{true}) &= \texttt{true}_{\mathfrak{P}(V)} \\
                           &= V
\end{align*}

\item{$\texttt{[B5] or(x,y) = not(and(not(x),not(y)))}$}
\begin{align*}
\bar \theta(\texttt{or}(x,y)) &= \texttt{or}_{\mathfrak{P}(V)}(\bar \theta(x), \bar \theta(y)) \\
                              &= \texttt{or}_{\mathfrak{P}(V)}(\theta(x), \theta(y)) \\
                              &= \texttt{or}_{\mathfrak{P}(V)}(X,Y) \\
                              &= X \cup Y
\end{align*}

\begin{align*}
\bar \theta(\texttt{not}(\texttt{and}(\texttt{not}(x), \texttt{not}(y))))
      &= \texttt{not}_{\mathfrak{P}(V)}(\bar \theta(\texttt{and}(\texttt{not}(x), \texttt{not}(y)))) \\
      &= \texttt{not}_{\mathfrak{P}(V)}(\texttt{and}_{\mathfrak{P}(V)}(\bar \theta(\texttt{not}(x)), \bar \theta(\texttt{not}(y)))) \\
      &= \texttt{not}_{\mathfrak{P}(V)}(\texttt{and}_{\mathfrak{P}(V)}(\texttt{not}_{\mathfrak{P}(V)}(\bar \theta(x)), \texttt{not}_{\mathfrak{P}(V)}(\bar \theta(y)))) \\
      &= \texttt{not}_{\mathfrak{P}(V)}(\texttt{and}_{\mathfrak{P}(V)}(\texttt{not}_{\mathfrak{P}(V)}(\theta(x)), \texttt{not}_{\mathfrak{P}(V)}(\theta(y)))) \\
      &= \texttt{not}_{\mathfrak{P}(V)}(\texttt{and}_{\mathfrak{P}(V)}(\texttt{not}_{\mathfrak{P}(V)}(X), \texttt{not}_{\mathfrak{P}(V)}(Y))) \\
      &= \texttt{not}_{\mathfrak{P}(V)}(\texttt{and}_{\mathfrak{P}(V)}(V - X, V - Y)) \\
      &= \texttt{not}_{\mathfrak{P}(V)}((V - X) \cap (V - Y)) \\
      &= V - ((V - X) \cap (V - Y)) \\
      &= X \cup Y
\end{align*}

\end{itemize}

Hiermee hebben we laten zien dat iedere vergelijking in $\texttt{Booleans}$
waar is in $\mathfrak{P}(V)$ en dus dat $\mathfrak{P}(V)$ een model is voor de
specificatie $\texttt{Booleans}$.


\section*{Opgave 5.5}

Om te laten zien dat een bepaalde vergelijking niet afleidbaar is uit een
specificatie kunnen we een tegenmodel construeren. Dit is een model voor de
specificatie waarin de gegeven vergelijking niet waar is. Volgens de
correctheidsstelling is deze vergelijking dan ook niet afleidbaar uit de
specificatie.

We volgen de hint op die bij de opgave gegeven wordt. We beschouwen de algebra
$\mathfrak{M}$ voor de specificatie $\texttt{NatBool}$ met drager $A$ en
interpretaties als we gewend zijn, behalve voor interpretaties hier onder
gedefini\"eerd.

\begin{align*}
A_{\texttt{nat}} &= \{ \omega, 0, 1, 2, 3, \ldots \} \\
A_{\texttt{bool}} &= \{T, F\}
\end{align*}

Voor alle $m,n \in \mathbb{N}$ defini\"eren we interpretaties als volgt.

\begin{align*}
\texttt{0}_{\mathfrak{M}}                  &= 0 \\
\texttt{succ}_{\mathfrak{M}}(n)            &= n+1 \\
\texttt{succ}_{\mathfrak{M}}(\omega)       &= \omega \\
\texttt{add}_{\mathfrak{M}}(m,n)           &= m+n \\
\texttt{add}_{\mathfrak{M}}(\omega,n)      &= \omega \\
\texttt{add}_{\mathfrak{M}}(n,\omega)      &= \omega \\
\texttt{add}_{\mathfrak{M}}(\omega,\omega) &= \omega \\
\texttt{mul}_{\mathfrak{M}}(m,n)           &= m*n \\
\texttt{mul}_{\mathfrak{M}}(\omega,0)      &= 0 \\
\texttt{mul}_{\mathfrak{M}}(\omega,n+1)    &= \omega \\
\texttt{mul}_{\mathfrak{M}}(n,\omega)      &= \omega \\
\texttt{mul}_{\mathfrak{M}}(\omega,\omega) &= \omega \\
\texttt{even}_{\mathfrak{M}}(n)            &= \begin{cases}
  T & \text{als $n$ even is} \\
  F & \text{als $n$ oneven is}
\end{cases} \\
\texttt{even}_{\mathfrak{M}}(\omega)       &= T \\
\texttt{odd}_{\mathfrak{M}}(n)             &= \begin{cases}
  F & \text{als $n$ even is} \\
  T & \text{als $n$ oneven is}
\end{cases} \\
\texttt{odd}_{\mathfrak{M}}(\omega)        &= T
\end{align*}

We hebben dus een algebra geconstrueerd waarin we een extra element toegevoegd
hebben dat `zowel even als oneven' is. Om alle vergelijkingen nog waar te
maken was er wat puzzelwerk nodig (je kunt $\omega$ opvatten als nuldeler voor
zowel optelling als vermeningvuldiging als je er iets in wilt zien).

Nu moeten we laten zien dat $\mathfrak{M}$ een model is voor de specificatie
$\texttt{NatBool}$. We laten de vergelijkingen $\texttt{[B1] \ldots [B5]}$
voor wat ze zijn; onze toevoeging van $\omega$ heeft hier geen invloed
op. Van de overige vergelijkingen uit $\texttt{NatBool}$ laten we voor
$\texttt{[A1]}$, $\texttt{[M2]}$ en $\texttt{[E2]}$ zien dat ze waar zijn in
$\mathfrak{M}$, de rest gaat op vergelijkbare wijze.

\paragraph{}

Laat nu $\theta$ een assignment zijn van elementen uit $A$ aan de variabelen
$x$ en $y$. We stellen dat $\theta(x) = a$ en $\theta(y) = b$.

\begin{itemize}

\item{$\texttt{[A1] add(x,0) = x}$}
\begin{align*}
\bar \theta(\texttt{add}(x,\texttt{0})) &= \texttt{add}_{\mathfrak{M}}(\bar \theta(x), \bar \theta(\texttt{0})) \\
                                        &= \texttt{add}_{\mathfrak{M}}(\theta(x), 0_{\mathfrak{M}}) \\
                                        &= \texttt{add}_{\mathfrak{M}}(a, 0) \\
                                        &= a
\end{align*}

\begin{align*}
\bar \theta(x) &= \theta(x) \\
               &= a
\end{align*}

\item{$\texttt{[M2] mul(x,succ(y)) = add(mul(x,y),x)}$}

We schrijven eerst de linkerkant uit.
\begin{align*}
\bar \theta(\texttt{mul}(x,\texttt{succ}(y))) &= \texttt{mul}_{\mathfrak{M}}(\bar \theta(x), \bar \theta(\texttt{succ}(y))) \\
                                              &= \texttt{mul}_{\mathfrak{M}}(\theta(x), \texttt{succ}_{\mathfrak{M}}(\bar \theta(y))) \\
                                              &= \texttt{mul}_{\mathfrak{M}}(a, \texttt{succ}_{\mathfrak{M}}(\theta(y))) \\
                                              &= \texttt{mul}_{\mathfrak{M}}(a, \texttt{succ}_{\mathfrak{M}}(b))
\end{align*}

We onderscheiden nu verschillende gevallen voor $a$ en $b$.

\begin{itemize}

\item
Wanneer $a = b = \omega$ hebben we:
\begin{align*}
\bar \theta(\texttt{mul}(x,\texttt{succ}(y))) &= \texttt{mul}_{\mathfrak{M}}(\omega, \texttt{succ}_{\mathfrak{M}}(\omega)) \\
                                              &= \texttt{mul}_{\mathfrak{M}}(\omega, \omega) \\
                                              &= \omega
\end{align*}

\item
Wanneer $a,b \in \mathbb{N}$ hebben we:

\begin{align*}
\bar \theta(\texttt{mul}(x,\texttt{succ}(y))) &= \texttt{mul}_{\mathfrak{M}}(a, b+1) \\
                                              &= a*(b+1)
\end{align*}

\item
Wanneer $a = \omega$ en $b \in \mathbb{N}$ hebben we:
\begin{align*}
\bar \theta(\texttt{mul}(x,\texttt{succ}(y))) &= \texttt{mul}_{\mathfrak{M}}(\omega, b+1) \\
                                              &= \omega
\end{align*}

\item
Wanneer $a \in \mathbb{N}$ en $b = \omega$ hebben we:
\begin{align*}
\bar \theta(\texttt{mul}(x,\texttt{succ}(y))) &= \texttt{mul}_{\mathfrak{M}}(a, \texttt{succ}_{\mathfrak{M}}(\omega)) \\
                                              &= \texttt{mul}_{\mathfrak{M}}(a, \omega) \\
                                              &= \omega
\end{align*}

\end{itemize}

En vervolgens schrijven we de rechterkant uit.
\begin{align*}
\bar \theta(\texttt{add}(\texttt{mul}(x,y),x)) &= \texttt{add}_{\mathfrak{M}}(\bar \theta(\texttt{mul}(x,y),x)) \\
                                               &= \texttt{add}_{\mathfrak{M}}(\texttt{mul}_{\mathfrak{M}}(\bar \theta(x),\bar \theta(y)),\bar \theta(x)) \\
                                               &= \texttt{add}_{\mathfrak{M}}(\texttt{mul}_{\mathfrak{M}}(\theta(x),\theta(y)),\theta(x)) \\
                                               &= \texttt{add}_{\mathfrak{M}}(\texttt{mul}_{\mathfrak{M}}(a,b),a)
\end{align*}

We onderscheiden nu weer dezelfde gevallen voor $a$ en $b$.

\begin{itemize}

\item
Wanneer $a = b = \omega$ hebben we:
\begin{align*}
\bar \theta(\texttt{add}(\texttt{mul}(x,y),x)) &= \texttt{add}_{\mathfrak{M}}(\texttt{mul}_{\mathfrak{M}}(\omega,\omega),\omega) \\
                                               &= \omega
\end{align*}

\item
Wanneer $a,b \in \mathbb{N}$ hebben we:
\begin{align*}
\bar \theta(\texttt{add}(\texttt{mul}(x,y),x)) &= \texttt{add}_{\mathfrak{M}}(\texttt{mul}_{\mathfrak{M}}(a,b),a) \\
                                               &= \texttt{add}_{\mathfrak{M}}(a*b,a) \\
                                               &= (a*b)+a \\
                                               &= a*(b+1)
\end{align*}

\item
Wanneer $a = \omega$ en $b \in \mathbb{N}$ hebben we:
\begin{align*}
\bar \theta(\texttt{add}(\texttt{mul}(x,y),x)) &= \texttt{add}_{\mathfrak{M}}(\texttt{mul}_{\mathfrak{M}}(\omega,b),\omega) \\
                                               &= \texttt{add}_{\mathfrak{M}}(\ldots,\omega) \\
                                               &= \omega
\end{align*}

\item
Wanneer $a \in \mathbb{N}$ en $b = \omega$ hebben we:
\begin{align*}
\bar \theta(\texttt{add}(\texttt{mul}(x,y),x)) &= \texttt{add}_{\mathfrak{M}}(\texttt{mul}_{\mathfrak{M}}(a,\omega),a) \\
                                               &= \texttt{add}_{\mathfrak{M}}(\omega,a) \\
                                               &= \omega
\end{align*}

\end{itemize}

\item{$\texttt{[E2] even(succ(x)) = odd(x)}$}

We schrijven weer eerst de linkerkant uit.
\begin{align*}
\bar \theta(\texttt{even}(\texttt{succ}(x))) &= \texttt{even}_{\mathfrak{M}}(\bar \theta(\texttt{succ}(x))) \\
                                             &= \texttt{even}_{\mathfrak{M}}(\texttt{succ}_{\mathfrak{M}}(\bar \theta(x)))) \\
                                             &= \texttt{even}_{\mathfrak{M}}(\texttt{succ}_{\mathfrak{M}}(\theta(x)))) \\
                                             &= \texttt{even}_{\mathfrak{M}}(\texttt{succ}_{\mathfrak{M}}(a))
\end{align*}

We onderscheiden twee gevallen voor $a$.

\begin{itemize}

\item
Wanneer $a \in \mathbb{N}$ hebben we:
\begin{align*}
\bar \theta(\texttt{even}(\texttt{succ}(x))) &= \texttt{even}_{\mathfrak{M}}(\texttt{succ}_{\mathfrak{M}}(a)) \\
                                             &= \texttt{even}_{\mathfrak{M}}(a+1) \\
                                             &= \texttt{even}_{\mathfrak{M}}(a+1) \\
                                             &= \begin{cases}
  F & \text{als $a$ even is} \\
  T & \text{als $a$ oneven is}
\end{cases}
\end{align*}

\item
Wanneer $a = \omega$ hebben we:
\begin{align*}
\bar \theta(\texttt{even}(\texttt{succ}(x))) &= \texttt{even}_{\mathfrak{M}}(\texttt{succ}_{\mathfrak{M}}(\omega)) \\
                                             &= \texttt{even}_{\mathfrak{M}}(\omega) \\
                                             &= T
\end{align*}

\end{itemize}

En vervolgens schrijven we de rechterkant uit.
\begin{align*}
\bar \theta(\texttt{odd}(x)) &= \texttt{odd}_{\mathfrak{M}}(\bar \theta(x)) \\
                             &= \texttt{odd}_{\mathfrak{M}}(\theta(x)) \\
                             &= \texttt{odd}_{\mathfrak{M}}(a)
\end{align*}

We onderscheiden weer twee gevallen voor $a$.

\begin{itemize}

\item
Wanneer $a \in \mathbb{N}$ hebben we:
\begin{align*}
\bar \theta(\texttt{odd}(x)) &= \texttt{odd}_{\mathfrak{M}}(a) \\
                             &= \begin{cases}
  F & \text{als $a$ even is} \\
  T & \text{als $a$ oneven is}
\end{cases}
\end{align*}

\item
Wanneer $a = \omega$ hebben we:
\begin{align*}
\bar \theta(\texttt{odd}(x)) &= \texttt{odd}_{\mathfrak{M}}(a) \\
                             &= \texttt{odd}_{\mathfrak{M}}(\omega) \\
                             &= T
\end{align*}

\end{itemize}

\end{itemize}

\paragraph{}

We hebben nu laten zien dat $\mathfrak{M}$ inderdaad een model is voor de
specificatie $\texttt{NatBool}$. Nu bekijken we de vergelijking
\begin{displaymath}
\texttt{even(x) = not(odd(x))}
\end{displaymath}
en zien dat deze niet waar is in $\mathfrak{M}$. Beschouw bijvoorbeeld de
assignment $\theta$ met $\theta(x) = \omega$. We hebben dan
\begin{align*}
\bar \theta(\texttt{even}(x)) &= \texttt{even}_{\mathfrak{M}}(\bar \theta(x)) \\
                              &= \texttt{even}_{\mathfrak{M}}(\theta(x)) \\
                              &= \texttt{even}_{\mathfrak{M}}(\omega) \\
                              &= T
\end{align*}
en
\begin{align*}
\bar \theta(\texttt{not}(\texttt{odd}(x))) &= \texttt{not}_{\mathfrak{M}}(\bar \theta(\texttt{odd}(x))) \\
                                           &= \texttt{not}_{\mathfrak{M}}(\texttt{odd}_{\mathfrak{M}}(\bar \theta(x))) \\
                                           &= \texttt{not}_{\mathfrak{M}}(\texttt{odd}_{\mathfrak{M}}(\theta(x))) \\
                                           &= \texttt{not}_{\mathfrak{M}}(\texttt{odd}_{\mathfrak{M}}(\omega)) \\
                                           &= \texttt{not}_{\mathfrak{M}}(T) \\
                                           &= F.
\end{align*}

Uit alle vergelijkingen van $\texttt{NatBool}$ volgt deze vergelijking dus
niet semantisch. Volgens de correctheid van afleidbaarheid, is deze
vergelijking dan ook niet afleidbaar in de specificatie $\texttt{NatBool}$.


\section*{Opgave 6.1}

\begin{description}

\item{\bf (a)} % model voor Booleans met junk zonder confusion
We hebben bij opgave 5.1 gezien dat iedere verzamelingsalgebra
$\mathfrak{P}(V)$ over een verzameling $V$ (als in voorbeeld 4.3) een model is
voor de specificatie $\texttt{Booleans}$. Nu beschouwen we de algebra
$\mathfrak{P}(F)$ over de verzameling $F=\{1,2,3,4,5,6\}$ met als drager
$\mathcal{P}(F)$.

\paragraph{Wel junk}

Iedere gesloten term uit $\texttt{Booleans}$ wordt in $\mathfrak{P}(F)$
ge\"interpreteerd als \`ofwel heel F, \`ofwel de lege verzameling $\{\}$. Het
bewijs hiervan verloopt via inductie naar de structuur van de term en laten we
hier achterwege.

Laten we nu het element $\{2,4,5\}$ uit $\mathfrak{P}(F)$ bekijken. Dit is
niet $F$, ook niet $\{\}$ en dus niet de interpretatie van een gesloten
term. Hieruit volgt dat $\mathfrak{P}(F)$ junk bevat.

\paragraph{Geen confusion}

Voor iedere gesloten term $t$ uit $\texttt{Booleans}$ geldt dat
\begin{align*}
\vdash t = \texttt{true} \quad \text{of} \quad \vdash t = \texttt{false}. &&\text{(volgens voorbeeld 6.3)}
\end{align*}
Laat nu $s$ en $t$ gesloten termen zijn met $\not \vdash t = s$. Dan moet
\`ofwel $\vdash s = \texttt{true}$ en $\vdash t = \texttt{false}$, \`ofwel
$\vdash s = \texttt{false}$ en $\vdash t = \texttt{true}$. In beide gevallen
worden $s$ en $t$ als verschillende elementen van $\mathfrak{P}(F)$
ge\"interpreteerd en dus bevat $\mathfrak{P}(F)$ geen confusion.

\item{\bf (b)} % model voor Booleans zonder junk met confusion
Laten we nu analoog aan onderdeel {\bf (a)} de verzamelingsalgebra
$\mathfrak{P}(\{\})$ over de lege verzameling bekijken met in de drager als
enige element $\{\}$.

\paragraph{Geen junk}

Volgens de definitie van de verzamelingsalgebra interpreteren we nu
$\texttt{true}$ als $\{\}$. Hieruit volgt direct dat $\mathfrak{P}(\{\})$ geen
junk bevat, want het enige element uit de drager is de interpretatie van een
gesloten term.

\paragraph{Wel confusion}

In de algebra $\mathfrak{P}(\{\})$ worden de interpretaties van de termen
$\texttt{true}$ en $\texttt{false}$ ge\"identificeerd. Toch is de vergelijking
$\texttt{true} = \texttt{false}$ niet afleidbaar uit de specificatie
$\texttt{Booleans}$ (bovenstaande algebra $\mathfrak{P}(F)$ is bijvoorbeeld
een tegenmodel) en dus bevat $\mathfrak{P}(\{\})$ confusion.

\item{\bf (c)} % model voor Booleans met junk met confusion
We bekijken de algebra $\mathfrak{B}_{3}$ voor de specificatie
$\texttt{Booleans}$ met als drager $\{A,B,C\}$ en interpretaties
\begin{align*}
\texttt{true}_{\mathfrak{B}_{3}}     &= B, \\
\texttt{false}_{\mathfrak{B}_{3}}    &= B, \\
\texttt{and}_{\mathfrak{B}_{3}}(x,y) &= B, \\
\texttt{or}_{\mathfrak{B}_{3}}(x,y)  &= B, \\
\texttt{not}_{\mathfrak{B}_{3}}(x)   &= B.
\end{align*}
Dat $\mathfrak{B}_{3}$ een model is voor $\texttt{Booleans}$ mag duidelijk
zijn (iedere term evalueert naar $B$ en dus is iedere vergelijking waar).

\paragraph{Wel junk}

Het is niet moeilijk in te zien dat iedere gesloten term uit
$\texttt{Booleans}$ wordt in $\mathfrak{B}_{3}$ ge\"interpreteerd als $B$. Dit
betekent dat de elementen $A$ en $C$ niet de interpretatie van een gesloten
term zijn en dus dat $\mathfrak{B}_{3}$ junk bevat.

\paragraph{Wel confusion}

In de algebra $\mathfrak{B}_{3}$ worden de interpretaties van de termen
$\texttt{true}$ en $\texttt{false}$ ge\"identificeerd. Toch is de vergelijking
$\texttt{true} = \texttt{false}$ niet afleidbaar uit de specificatie
$\texttt{Booleans}$ (bovenstaande algebra $\mathfrak{P}(F)$ is bijvoorbeeld
een tegenmodel) en dus bevat $\mathfrak{B}_{3}$ confusion.

\end{description}


\section*{Opgave 6.2}

\begin{description}

\item{\bf (a)} % construeer het termmodel voor Booleans
We construeren het termmodel $\mathfrak{Ter}_{\Sigma}/\negmedspace\sim$ voor de
specificatie $\texttt{Booleans}$ met behulp van de equivalentierelatie $\sim$
op termen uit $Ter_{\Sigma}$:
\begin{displaymath}
s \sim t \, \Longleftrightarrow \, E \vdash s = t
\end{displaymath}
waarbij $E$ de verzameling vergelijkingen in $\texttt{Booleans}$ is.

Als drager van het termmodel nemen we nu de equivalentieklassen van de
gesloten termen onder $\sim$. We hebben eerder gezien dat iedere gesloten term
in $\texttt{Booleans}$ afleidbaar gelijk is aan \`ofwel $\texttt{true}$
\`ofwel $\texttt{false}$, dus hebben we als drager genoeg aan de twee
equivalentieklassen $[\texttt{true}]$ en $[\texttt{false}]$ van de termen
$\texttt{true}$ en $\texttt{false}$.

De interpretaties van de constanten en functiesymbolen kiezen we in het
termmodel als volgt:
\begin{align*}
\texttt{true}_{\mathfrak{Ter}_{\Sigma}/\sim}         &= [\texttt{true}] \\
\texttt{false}_{\mathfrak{Ter}_{\Sigma}/\sim}        &= [\texttt{false}] \\
\texttt{and}_{\mathfrak{Ter}_{\Sigma}/\sim}([t],[s]) &= [\texttt{and}(t,s)] \\
\texttt{or}_{\mathfrak{Ter}_{\Sigma}/\sim}([t],[s])  &= [\texttt{or}(t,s)] \\
\texttt{not}_{\mathfrak{Ter}_{\Sigma}/\sim}([t])     &= [\texttt{not}(t)] \\
\end{align*}

\item{\bf (b)} % laat zien dat het termmodel isomorf is met B2
We bekijken nu de functie $\phi : Ter_{\Sigma}/\negmedspace\sim \,\,
\rightarrow \, A$ die equivalentieklassen afbeelt op de drager $A$ van
$\mathfrak{B}_{2}$ en gedefini\"eerd is als
\begin{align*}
\phi([\texttt{true}])  &= T, \\
\phi([\texttt{false}]) &= F.
\end{align*}

Nu laten we zien dat $\phi$ een isomorfisme is van
$\mathfrak{Ter}_{\Sigma}/\negmedspace\sim$ naar $\mathfrak{B}_{2}$.

\paragraph{Homomorfisme}

Eerst bekijken we of $\phi$ een homomorfisme is. Voor de constante
$\texttt{true}$ hebben we:
\begin{align*}
\phi(\texttt{true}_{\mathfrak{Ter}_{\Sigma}/\sim}) &= \phi([\texttt{true}]) \\
                                                   &= T \\
                                                   &= \texttt{true}_{\mathfrak{B}_{2}}
\end{align*}
Voor de constante $\texttt{false}$ gaat dit op gelijke wijze. We bekijken nu
het functiesymbool $\texttt{not}$ toegepast op het element $x$. Voor de waarde
van $x$ zijn er twee mogelijkheden, we beperken ons hier tot
$[\texttt{true}]$:
\begin{align*}
\phi(\texttt{not}_{\mathfrak{Ter}_{\Sigma}/\sim}(x)) &= \phi(\texttt{not}_{\mathfrak{Ter}_{\Sigma}/\sim}([\texttt{true}])) \\
                                                     &= \phi([\texttt{not(true)}]) \\
                                                     &= \phi([\texttt{false}]) \\
                                                     &= F
\end{align*}
\begin{align*}
\texttt{not}_{\mathfrak{B}_{2}}(\phi(x)) &= \texttt{not}_{\mathfrak{B}_{2}}(\phi([\texttt{true}])) \\
                                         &= \texttt{not}_{\mathfrak{B}_{2}}(T) \\
                                         &= F
\end{align*}
Vervolgens bekijken we het functiesymbool $\texttt{and}$ toegepast op de
elementen $x$ en $y$. Ook hier zijn er voor $x$ en $y$ beiden twee mogelijke
waarden, we behandelen alleen het geval dat $x = [\texttt{true}]$ en $y =
[\texttt{false}]$:
\begin{align*}
\phi(\texttt{and}_{\mathfrak{Ter}_{\Sigma}/\sim}(x,y)) &= \phi(\texttt{and}_{\mathfrak{Ter}_{\Sigma}/\sim}([\texttt{true}],[\texttt{false}])) \\
                                                       &= \phi([\texttt{and}(\texttt{true},\texttt{false})]) \\
                                                       &= \phi([\texttt{false}]) \\
                                                       &= F
\end{align*}
\begin{align*}
\texttt{and}_{\mathfrak{B}_{2}}(\phi(x),\phi(y)) &= \texttt{and}_{\mathfrak{B}_{2}}(\phi([\texttt{true}]),\phi([\texttt{false}])) \\
                                                 &= \texttt{and}_{\mathfrak{B}_{2}}(T,F) \\
                                                 &= F
\end{align*}
Hetzelfde verhaal voor het functiesymbool $\texttt{or}$ zullen we achterwege
laten. Hiermee hebben we laten zien dat $\phi$ een homomorfisme is van
$\mathfrak{Ter}_{\Sigma}/\negmedspace\sim$ naar $\mathfrak{B}_{2}$.

\paragraph{Surjectief}

De drager van $\mathfrak{B}_{2}$ bestaat uit de elementen $T$ en $F$. In onze
definitie van $\phi$ zien we duidelijk dat beiden het beeld zijn van een
element uit $\mathfrak{Ter}_{\Sigma}/\negmedspace\sim$ en dus is $\phi$ surjectief.

\paragraph{Injectief}

Evenzo is gemakkelijk te zien dat de twee verschillende elementen
$[\texttt{true}]$ en $[\texttt{false}]$ met $\phi$ ook twee verschillende
beelden hebben, namelijk $T$ respectievelijk $F$. En dus is $\phi$ injectief.

\paragraph{}

Nu we hebben laten zien dat er een isomorfisme bestaat van
$\mathfrak{Ter}_{\Sigma}/\negmedspace\sim$ naar $\mathfrak{B}_{2}$ ($\phi$ is een
injectief en surjectief homomorfisme) weten we dus ook dat deze twee algebra's
isomorf zijn.

\end{description}


\section*{Opgave 7.3}

\begin{description}

\item{\bf (a)} % kta voor specificatie Booleans
Er zijn twee equivalentieklassen onder afleidbare gelijkheid van gesloten
termen. Uit beide klassen kiezen we \'e\'en representant voor de drager van de
kanonieke term algebra $\mathfrak{A}_{1}$:
\begin{align*}
A_{1} &= \{\texttt{true}, \texttt{false}\}.
\end{align*}
De functiesymbolen uit $\texttt{Booleans}$ interpreteren we als volgt:
\begin{align*}
\texttt{true}_{\mathfrak{A}_{1}}     &= \texttt{true}, \\
\texttt{false}_{\mathfrak{A}_{1}}    &= \texttt{false}, \\
\texttt{and}_{\mathfrak{A}_{1}}(t,s) &= \begin{cases}
  \texttt{true}  & \text{als $t = s = \texttt{true}$;} \\
  \texttt{false} & \text{in de overige gevallen},
\end{cases} \\
\texttt{or}_{\mathfrak{A}_{1}}(t,s) &= \begin{cases}
  \texttt{true}  & \text{als $t = \texttt{true}$ of $s = \texttt{true}$;} \\
  \texttt{false} & \text{in de overige gevallen,}
\end{cases} \\
\texttt{not}_{\mathfrak{A}_{1}}(t) &= \begin{cases}
  \texttt{true}  & \text{als $t = \texttt{false}$;} \\
  \texttt{false} & \text{als $t = \texttt{true}$.}
\end{cases}
\end{align*}

\item{\bf (b)} % nog een
We kunnen ook andere representanten voor de drager kiezen, bijvoorbeeld zoals
in deze algebra $\mathfrak{A}_{2}$:
\begin{align*}
A_{2} &= \{\texttt{true}, \texttt{not(true)}\}.
\end{align*}
De functiesymbolen uit $\texttt{Booleans}$ interpreteren we nu als volgt:
\begin{align*}
\texttt{true}_{\mathfrak{A}_{2}}     &= \texttt{true}, \\
\texttt{false}_{\mathfrak{A}_{2}}    &= \texttt{not(true)}, \\
\texttt{and}_{\mathfrak{A}_{2}}(t,s) &= \begin{cases}
  \texttt{true}      & \text{als $t = s = \texttt{true}$;} \\
  \texttt{not(true)} & \text{in de overige gevallen},
\end{cases} \\
\texttt{or}_{\mathfrak{A}_{2}}(t,s) &= \begin{cases}
  \texttt{true}      & \text{als $t = \texttt{true}$ of $s = \texttt{true}$;} \\
  \texttt{not(true)} & \text{in de overige gevallen,}
\end{cases} \\
\texttt{not}_{\mathfrak{A}_{2}}(t) &= \begin{cases}
  \texttt{true}      & \text{als $t = \texttt{not(true)}$;} \\
  \texttt{not(true)} & \text{als $t = \texttt{true}$.}
\end{cases}
\end{align*}

\item{\bf (c)} % nog een
De derde mogelijkheid voor de drager van een kta bij deze
specificatie is de volgende algebra $\mathfrak{A}_{3}$:
\begin{align*}
A_{3} &= \{\texttt{not(false)}, \texttt{false}\}.
\end{align*}
De functiesymbolen uit $\texttt{Booleans}$ interpreteren we nu als volgt:
\begin{align*}
\texttt{true}_{\mathfrak{A}_{3}}     &= \texttt{not(false)}, \\
\texttt{false}_{\mathfrak{A}_{3}}    &= \texttt{false}, \\
\texttt{and}_{\mathfrak{A}_{3}}(t,s) &= \begin{cases}
  \texttt{not(false)} & \text{als $t = s = \texttt{not(false)}$;} \\
  \texttt{false}      & \text{in de overige gevallen,}
\end{cases} \\
\texttt{or}_{\mathfrak{A}_{3}}(t,s) &= \begin{cases}
  \texttt{not(false)} & \text{als $t = \texttt{not(false)}$ of $s = \texttt{not(false)}$;} \\
  \texttt{false}      & \text{in de overige gevallen,}
\end{cases} \\
\texttt{not}_{\mathfrak{A}_{3}}(t) &= \begin{cases}
  \texttt{not(false)} & \text{als $t = \texttt{false}$;} \\
  \texttt{false}      & \text{als $t = \texttt{not(false)}$.}
\end{cases}
\end{align*}

\end{description}

Merk overigens op dat er geen andere mogelijkheden zijn voor een kanonieke
term algebra voor de specificatie $\texttt{Booleans}$. Wanneer we twee andere
representanten uit de twee equivalentieklassen kiezen voor de drager van een
kta kunnen we niet voldoen aan de eis dat iedere subterm van een element uit
de drager ook een element uit de drager is.
Dit is alleen te voorkomen door meer dan twee elementen in de drager te
plaatsen, maar dan voldoen we niet meer aan de eis dat iedere
equivalentieklasse precies \'e\'en representant heeft in de drager.


\section*{Opgave 7.4}

\begin{description}

\item{\bf (a)} % initieel correcte specificatie E voor B
Een initieel correcte specificatie $E$ voor $\mathfrak{B}$ bestaat uit de
volgende vergelijkingen:
\begin{quote}
\begin{verbatim}
[E1] f(f(f(x))) = f(f(c))
[E2] f(a) = c
\end{verbatim}
\end{quote}

\item{\bf (b)} % kta voor E
Een kta voor $E$ is de algebra $\mathfrak{E}$ met als drager de verzameling
\begin{align*}
\{\texttt{a}, \, \texttt{c}, \, \texttt{f(c)}, \, \texttt{f(f(c))}\}
\end{align*}
en als interpretaties
\begin{align*}
\texttt{a}_{\mathfrak{E}}          &= \texttt{a}, \\
\texttt{c}_{\mathfrak{E}}          &= \texttt{c}, \\
\texttt{f}_{\mathfrak{E}}(a)       &= \texttt{c}, \\
\texttt{f}_{\mathfrak{E}}(c)       &= \texttt{f(c)}, \\
\texttt{f}_{\mathfrak{E}}(f(c))    &= \texttt{f(f(c))}, \\
\texttt{f}_{\mathfrak{E}}(f(f(c))) &= \texttt{f(f(c))}.
\end{align*}

\end{description}


\end{document}
