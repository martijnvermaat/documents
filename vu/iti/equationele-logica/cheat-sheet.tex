\documentclass[a4paper,twocolumn,10pt]{article}


\usepackage[landscape,pdftex]{geometry}
\usepackage[cm]{fullpage}
\usepackage[dutch]{babel}
\usepackage{amsmath}
\usepackage{amssymb}


\setlength{\columnsep}{.8in}


\pagestyle{empty}


\makeatletter
\renewcommand{\section}{\@startsection{section}{1}{0mm}%
                                {2.5ex plus -.5ex minus -.2ex}%
                                {1.5ex plus .2ex}%x
                                {\normalfont\large\bfseries}}
\renewcommand{\subsection}{\@startsection{subsection}{2}{0mm}%
                                {-1explus -.5ex minus -.2ex}%
                                {0.5ex plus .2ex}%
                                {\normalfont\small\bfseries}}
\renewcommand\subsubsection{\@startsection{subsubsection}{3}{0mm}%
                                {-1ex plus -.5ex minus -.2ex}%
                                {1ex plus .2ex}%
                                {\normalfont\small\bfseries}}
\makeatother


\setcounter{secnumdepth}{0}


\setlength{\parindent}{0pt}
\setlength{\parskip}{0pt plus 0.5ex}


\begin{document}


\footnotesize


\begin{center}
     \Large{\textbf{Equationele Logica beknopt}} \\
\end{center}


\section{Syntax}


\subsection{Signatuur $(S, \Sigma)$}

\begin{tabular}{@{}ll@{}}
$S$       &  Verzameling soortnamen. \\
$\Sigma$  &  Verzameling functiesymbolen met types over $S$.
\end{tabular}


\subsection{Termen $Ter_{\Sigma}(X)$ over $(S, \Sigma)$}

\begin{tabular}{@{}ll@{}}
$(S, \Sigma)$      &  Signatuur. \\
$X$                &  Verzameling variabelen met types uit $S$.
\end{tabular}

De verzameling $Ter_{\Sigma}(X)$ van termen met vrije
variabelen uit $X$ is inductief gedefinieerd als volgt waarbij
steeds de types gerespecteerd worden:
\begin{itemize}
\item $f \in \Sigma$ en $t_{1}, \ldots, t_{n} \in Ter_{\Sigma}(X)$ $\Rightarrow$ $f(t_{1}, \ldots, t_{n}) \in Ter_{\Sigma}(X)$
\item $x \in Ter_{\Sigma}(X)$ voor alle $x \in X$
\end{itemize}


\subsection{Specificatie $((S, \Sigma), E)$}

\begin{tabular}{@{}ll@{}}
$(S, \Sigma)$  &  Signatuur. \\
$E$            &  Verzameling vergelijkingen $l_{i} = r_{i}$ met
                  $l_{i}$ en $r_{i}$ termen uit $Ter_{\Sigma}(X)$.
\end{tabular}


\subsection{Substitutie $\bar{\theta} : Ter_{\Sigma}(X) \rightarrow Ter_{\Sigma}$}

\begin{tabular}{@{}ll@{}}
$\theta : X \rightarrow Ter_{\Sigma}(X)$                      &  Substitutie van termen voor variabelen. \\
$\bar{\theta} : Ter_{\Sigma}(X) \rightarrow Ter_{\Sigma}(X)$  &  Uitbreiding op termen.
\end{tabular}

Gegeven een substitutie $\theta$ wordt de uitbreiding daarvan
inductief gedefinieerd als:
\begin{align*}
  \bar{\theta}(x) &= \theta(x) \\
  \bar{\theta}(f(t_{1}, \ldots, t_{n})) &= f(\bar{\theta}(t_{1}), \ldots, \bar{\theta}(t_{n}))
\end{align*}


\section{Equationele logica}


\subsection{Afleidbaarheid $E \vdash t_{1} = t_{2}$}

De verzameling vergelijkingen afleidbaar uit $E$ is inductief gedefinieerd:
\begin{itemize}
\item als $t_{1} = t_{2} \in E$ (het is een axioma), dan $E \vdash t_{1} = t_{2}$ ,
\item $E \vdash t = t$ voor alle $t$ (reflexiviteit) ,
\item als $E \vdash t_{1} = t_{2}$, dan $E \vdash t_{2} = t_{1}$ (symmetrie) ,
\item als $E \vdash t_{1} = t_{2}$ en $E \vdash t_{2} = t_{3}$, dan $E \vdash t_{1} = t_{3}$ (transitiviteit) ,
\item als $E \vdash t_{i} = u_{i}$ voor $i = 1 \ldots n$, dan $f(t_{i}, \ldots, t_{n}) = f(u_{1}, \ldots, u_{n})$ (congruentie) ,
\item als $E \vdash t_{1} = t_{2}$, dan $E \vdash \bar{\theta}(t_{1}) = \bar{\theta}(t_{2})$ voor alle substituties $\theta$ .
\end{itemize}


\section{Semantiek}


\subsection{$\Sigma$-algebra $\mathfrak{A} = (A, I)$}

\begin{tabular}{@{}ll@{}}
$(S, \Sigma)$  &  Signatuur. \\
$A$            &  Drager, $S$-soortig. \\
$I$            &  Interpretatie.
\end{tabular}

Voor iedere $f \in \Sigma$ geeft $I(f)$ een interpretatie $f_{\mathfrak{A}}$
op de drager, waarbij steeds alle typen kloppen.


\subsection{Assignment $\bar{\theta} : Ter_{\Sigma}(X) \rightarrow A$}

\begin{tabular}{@{}ll@{}}
$\mathfrak{A} = (A, I)$                         &  $\Sigma$-algebra. \\
$\theta : X \rightarrow A$                      &  Assignment van $a \in A$ aan $x \in X$. \\
$\bar{\theta} : Ter_{\Sigma}(X) \rightarrow A$  &  Uitbreiding op termen.
\end{tabular}

Gegeven een assignment $\theta$ wordt de uitbreiding daarvan
inductief gedefinieerd als:
\begin{align*}
  \bar{\theta}(x) &= \theta(x) \\
  \bar{\theta}(f(t_{1}, \ldots, t_{n})) &= f_\mathfrak{A}(\bar{\theta}(t_{1}), \ldots, \bar{\theta}(t_{n}))
\end{align*}


\subsection{Waarheid $\mathfrak{A} \models t_{1} = t_{2}$}

$t_{1} = t_{2}$ is waar in $\mathfrak{A}$ wanneer $\bar{\theta}(t_{1}) = \bar{\theta}(t_{2})$
voor iedere assignment $\theta$.
Voor een verzameling vergelijkingen $E$ zeggen we dat $\mathfrak{A} \models E$ wanneer
$\mathfrak{A} \models t_{1} = t_{2}$ voor alle $t_{1} = t_{2}$ in $E$.


\subsection{Semantisch gevolg $E \models t_{1} = t_{2}$}

$t_{1} = t_{2}$ volgt semantisch uit $E$ wanneer $\mathfrak{A} \models t_{1} = t_{2}$
voor iedere algebra $\mathfrak{A}$ met $\mathfrak{A} \models E$.


\section{Modellen}


\subsection{Model $\mathfrak{A}$ voor $((S, \Sigma), E)$}

Een $\Sigma$-algebra $\mathfrak{A}$ is een model voor de specificatie $((S, \Sigma), E)$
wanneer iedere vergelijking uit $E$ waar is in $\mathfrak{A}$.


\subsection{Initi\"ele modellen}

\begin{tabular}{@{}ll@{}}
$((S, \Sigma), E)$       &  Specificatie. \\
$\mathfrak{A} = (A, I)$  &  Model voor $((S, \Sigma), E)$. \\
\end{tabular}

Een element $a \in A$ is junk wanneer het niet de interpretatie
is van een gesloten term.

Gesloten termen $t_{1}$ en $t_{2}$ vormen confusion wanneer
$\mathfrak{A} \models t_{1} = t_{2}$ terwijl $E \not\models t_{1} = t_{2}$.

Een model is initieel wanneer het geen junk en confusion bevat.


\subsection{Termmodel}

Gegeven een specificatie $((S, \Sigma), E)$ bestaat het termmodel
$\mathfrak{Ter}_{\Sigma}/\negmedspace\sim$ uit
\begin{itemize}
\item drager $Ter_{\Sigma}/\negmedspace\sim$ van equivalentieklassen $[t]$ van termen $t \in Ter_{\Sigma}$ ,
\item voor alle $f \in \Sigma$, een interpretatie $f_{\mathfrak{Ter}_{\Sigma}/\negmedspace\sim}([t_{1}], \ldots, [t_{n}]) \equiv [f(t_{1}, \ldots, t_{2})]$ .
\end{itemize}

Hierbij is de equivalentieklasse $[t]$ van $t$ gedefinieerd als $\{u \in Ter_{\Sigma} \mid E \vdash t = u\}$.

% semantisch gevolg, correctheid, volledigheid


\rule{0.3\linewidth}{0.25pt}
\scriptsize

Maart 2008, Martijn Vermaat


\end{document}
