\documentclass[a4paper,11pt]{article}
\usepackage[dutch]{babel}
\usepackage{amsfonts}
\usepackage{a4}
\usepackage{latexsym}
\usepackage{fitch} % http://folk.uio.no/johanw/FitchSty.html

% niet te gewichtig willen doen met veel ruimte
\setlength{\textwidth}{16cm}
\setlength{\textheight}{23.0cm}
\setlength{\topmargin}{0cm}
\setlength{\oddsidemargin}{0.2mm}
\setlength{\evensidemargin}{0.2mm}
\setlength{\parindent}{0cm}

% standaard enumerate met abc
\renewcommand\theenumi{\alph{enumi}}


\begin{document}


{\bf Uitwerkingen bij Inleiding Theoretische Informatica\\
Deel 1: Equationele Logica -- Initi\"ele modellen}\\[2em]


{\bf Opgave 6.1}

\begin{enumerate}

\item % model voor Booleans met junk zonder confusion

We hebben bij opgave 5.1 gezien dat iedere verzamelingsalgebra
$\mathfrak{P}(V)$ over een verzameling $V$ (als in voorbeeld 4.3) een model is
voor de specificatie $\texttt{Booleans}$. Nu beschouwen we de algebra
$\mathfrak{P}(F)$ over de verzameling $F=\{1,2,3,4,5,6\}$ met als drager
$\mathcal{P}(F)$.

\paragraph{Wel junk}

Iedere gesloten term uit $\texttt{Booleans}$ wordt in $\mathfrak{P}(F)$
ge\"interpreteerd als \`ofwel heel F, \`ofwel de lege verzameling $\{\}$. Het
bewijs hiervan verloopt via inductie naar de structuur van de term en laten we
hier achterwege.

Laten we nu het element $\{2,4,5\}$ uit $\mathfrak{P}(F)$ bekijken. Dit is
niet $F$, ook niet $\{\}$ en dus niet de interpretatie van een gesloten
term. Hieruit volgt dat $\mathfrak{P}(F)$ junk bevat.

\paragraph{Geen confusion}

Voor iedere gesloten term $t$ uit $\texttt{Booleans}$ geldt dat
\begin{align*}
\vdash t = \texttt{true} \quad \text{of} \quad \vdash t = \texttt{false}. &&\text{(volgens voorbeeld 6.3)}
\end{align*}
Laat nu $s$ en $t$ gesloten termen zijn met $\not \vdash t = s$. Dan moet
\`ofwel $\vdash s = \texttt{true}$ en $\vdash t = \texttt{false}$, \`ofwel
$\vdash s = \texttt{false}$ en $\vdash t = \texttt{true}$. In beide gevallen
worden $s$ en $t$ als verschillende elementen van $\mathfrak{P}(F)$
ge\"interpreteerd en dus bevat $\mathfrak{P}(F)$ geen confusion.\\[2em]

\item % model voor Booleans zonder junk met confusion

Laten we nu analoog aan onderdeel a de verzamelingsalgebra
$\mathfrak{P}(\{\})$ over de lege verzameling bekijken met in de drager als
enige element $\{\}$.

\paragraph{Geen junk}

Volgens de definitie van de verzamelingsalgebra interpreteren we nu
$\texttt{true}$ als $\{\}$. Hieruit volgt direct dat $\mathfrak{P}(\{\})$ geen
junk bevat, want het enige element uit de drager is de interpretatie van een
gesloten term.

\paragraph{Wel confusion}

In de algebra $\mathfrak{P}(\{\})$ worden de interpretaties van de termen
$\texttt{true}$ en $\texttt{false}$ ge\"identificeerd. Toch is de vergelijking
$\texttt{true} = \texttt{false}$ niet afleidbaar uit de specificatie
$\texttt{Booleans}$ (bovenstaande algebra $\mathfrak{P}(F)$ is bijvoorbeeld
een tegenmodel) en dus bevat $\mathfrak{P}(\{\})$ confusion.\\[2em]

\item % model voor Booleans met junk met confusion

We bekijken de algebra $\mathfrak{B}_{3}$ voor de specificatie
$\texttt{Booleans}$ met als drager $\{A,B,C\}$ en interpretaties
\begin{align*}
\texttt{true}_{\mathfrak{B}_{3}}     &= B, \\
\texttt{false}_{\mathfrak{B}_{3}}    &= B, \\
\texttt{and}_{\mathfrak{B}_{3}}(x,y) &= B, \\
\texttt{or}_{\mathfrak{B}_{3}}(x,y)  &= B, \\
\texttt{not}_{\mathfrak{B}_{3}}(x)   &= B.
\end{align*}
Dat $\mathfrak{B}_{3}$ een model is voor $\texttt{Booleans}$ mag duidelijk
zijn (iedere term evalueert naar $B$ en dus is iedere vergelijking waar).

\paragraph{Wel junk}

Het is niet moeilijk in te zien dat iedere gesloten term uit
$\texttt{Booleans}$ wordt in $\mathfrak{B}_{3}$ ge\"interpreteerd als $B$. Dit
betekent dat de elementen $A$ en $C$ niet de interpretatie van een gesloten
term zijn en dus dat $\mathfrak{B}_{3}$ junk bevat.

\paragraph{Wel confusion}

In de algebra $\mathfrak{B}_{3}$ worden de interpretaties van de termen
$\texttt{true}$ en $\texttt{false}$ ge\"identificeerd. Toch is de vergelijking
$\texttt{true} = \texttt{false}$ niet afleidbaar uit de specificatie
$\texttt{Booleans}$ (bovenstaande algebra $\mathfrak{P}(F)$ is bijvoorbeeld
een tegenmodel) en dus bevat $\mathfrak{B}_{3}$ confusion.\\[2em]

\end{enumerate}


{\bf Opgave 6.2}

\begin{enumerate}

\item % construeer het termmodel voor Booleans

We construeren het termmodel $\mathfrak{Ter}_{\Sigma}/\sim$ voor de
specificatie $\texttt{Booleans}$ met behulp van de equivalentierelatie $\sim$
op termen uit $Ter_{\Sigma}$:
\begin{displaymath}
s \sim t \, \Longleftrightarrow \, E \vdash s = t
\end{displaymath}
waarbij E de verzameling vergelijkingen in $\texttt{Booleans}$ is.

Als drager van het termmodel nemen we nu de equivalentieklassen van de
gesloten termen onder $\sim$. We hebben eerder gezien dat iedere gesloten term
in $\texttt{Booleans}$ afleidbaar gelijk is aan \`ofwel $\texttt{true}$
\`ofwel $\texttt{false}$, dus hebben we als drager genoeg aan de twee
equivalentieklassen $[\texttt{true}]$ en $[\texttt{false}]$ van de termen
$\texttt{true}$ en $\texttt{false}$.

De interpretaties van de constanten en functiesymbolen kiezen we in het
termmodel als volgt:
\begin{align*}
\texttt{true}_{\mathfrak{Ter}_{\Sigma}/\sim}         &= [\texttt{true}] \\
\texttt{false}_{\mathfrak{Ter}_{\Sigma}/\sim}        &= [\texttt{false}] \\
\texttt{and}_{\mathfrak{Ter}_{\Sigma}/\sim}([t],[s]) &= [\texttt{and(t,s)}] \\
\texttt{or}_{\mathfrak{Ter}_{\Sigma}/\sim}([t],[s])  &= [\texttt{or(t,s)}] \\
\texttt{not}_{\mathfrak{Ter}_{\Sigma}/\sim}([t])     &= [\texttt{not(t)}] \\
\end{align*}

\item % laat zien dat het termmodel isomorf is met B2

We bekijken nu de functie $\phi : Ter_{\Sigma}/\negmedspace\sim \,\,
\rightarrow \, A$ die equivalentieklassen afbeelt op de drager $A$ van
$\mathfrak{B}_{2}$ en gedefini\"eerd is als
\begin{align*}
\phi([\texttt{true}])  &= T, \\
\phi([\texttt{false}]) &= F.
\end{align*}
Nu laten we zien dat $\phi$ een isomorfisme is van
$\mathfrak{Ter}_{\Sigma}/\sim$ naar $\mathfrak{B}_{2}$.

\paragraph{Homomorfisme}

Eerst bekijken we of $\phi$ een homomorfisme is. Voor de constante
$\texttt{true}$ hebben we:
\begin{align*}
\phi(\texttt{true}_{\mathfrak{Ter}_{\Sigma}/\sim}) &= \phi([\texttt{true}]) \\
                                                   &= T \\
                                                   &= \texttt{true}_{\mathfrak{B}_{2}}
\end{align*}
Voor de constante $\texttt{false}$ gaat dit op gelijke wijze. We bekijken nu
het functiesymbool $\texttt{not}$ toegepast op het element $x$. Voor de waarde
van $x$ zijn er twee mogelijkheden, we beperken ons hier tot
$[\texttt{true}]$:
\begin{align*}
\phi(\texttt{not}_{\mathfrak{Ter}_{\Sigma}/\sim}(x)) &= \phi(\texttt{not}_{\mathfrak{Ter}_{\Sigma}/\sim}([\texttt{true}])) \\
                                                     &= \phi([\texttt{not(true)}]) \\
                                                     &= \phi([\texttt{false}]) \\
                                                     &= F
\end{align*}
\begin{align*}
\texttt{not}_{\mathfrak{B}_{2}}(\phi(x)) &= \texttt{not}_{\mathfrak{B}_{2}}(\phi([\texttt{true}])) \\
                                         &= \texttt{not}_{\mathfrak{B}_{2}}(T) \\
                                         &= F
\end{align*}
Vervolgens bekijken we het functiesymbool $\texttt{and}$ toegepast op de
elementen $x$ en $y$. Ook hier zijn er voor $x$ en $y$ beiden twee mogelijke
waarden, we behandelen alleen het geval dat $x = [\texttt{true}]$ en $y =
[\texttt{false}]$:
\begin{align*}
\phi(\texttt{and}_{\mathfrak{Ter}_{\Sigma}/\sim}(x,y)) &= \phi(\texttt{and}_{\mathfrak{Ter}_{\Sigma}/\sim}([\texttt{true}],[\texttt{false}])) \\
                                                       &= \phi([\texttt{and}(\texttt{true},\texttt{false})]) \\
                                                       &= \phi([\texttt{false}]) \\
                                                       &= F
\end{align*}
\begin{align*}
\texttt{and}_{\mathfrak{B}_{2}}(\phi(x),\phi(y)) &= \texttt{and}_{\mathfrak{B}_{2}}(\phi([\texttt{true}]),\phi([\texttt{false}])) \\
                                                 &= \texttt{and}_{\mathfrak{B}_{2}}(T,F) \\
                                                 &= F
\end{align*}
Hetzelfde verhaal voor het functiesymbool $\texttt{or}$ zullen we achterwege
laten. Hiermee hebben we laten zien dat $\phi$ een homomorfisme is van
$\mathfrak{Ter}_{\Sigma}/\sim$ naar $\mathfrak{B}_{2}$.

\paragraph{Surjectief}

De drager van $\mathfrak{B}_{2}$ bestaat uit de elementen $T$ en $F$. In onze
definitie van $\phi$ zien we duidelijk dat beiden het beeld zijn van een
element uit $\mathfrak{Ter}_{\Sigma}/\sim$ en dus is $\phi$ surjectief.

\paragraph{Injectief}

Evenzo is gemakkelijk te zien dat de twee verschillende elementen
$[\texttt{true}]$ en $[\texttt{false}]$ met $\phi$ ook twee verschillende
beelden hebben, namelijk $T$ respectievelijk $F$. En dus is $\phi$ injectief.

Nu we hebben laten zien dat er een isomorfisme bestaat van
$\mathfrak{Ter}_{\Sigma}/\sim$ naar $\mathfrak{B}_{2}$ ($\phi$ is een
injectief en surjectief homomorfisme) weten we dus ook dat deze twee algebra's
isomorf zijn.\\[2em]

\end{enumerate}


{\bf Opgave 6.3}

\begin{enumerate}

\item % construeer het termmodel voor Naturals

Todo.

\item % laat zien dat het termmodel isomorf is met N

Todo.\\[2em]

\end{enumerate}


{\bf Opgave 6.4}

Zie uitwerking van Roel de Vrijer.\\[2em]


\end{document}
