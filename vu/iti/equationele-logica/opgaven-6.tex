\documentclass[a4paper,11pt]{article}
\usepackage[dutch]{babel}
\usepackage{amsfonts}
\usepackage{a4}
\usepackage{latexsym}
\usepackage{fitch} % http://folk.uio.no/johanw/FitchSty.html

% niet te gewichtig willen doen met veel ruimte
\setlength{\textwidth}{16cm}
\setlength{\textheight}{23.0cm}
\setlength{\topmargin}{0cm}
\setlength{\oddsidemargin}{0.2mm}
\setlength{\evensidemargin}{0.2mm}
\setlength{\parindent}{0cm}

% standaard enumerate met abc
\renewcommand\theenumi{\alph{enumi}}


\begin{document}


{\bf Uitwerkingen bij Inleiding Theoretische Informatica\\
Deel 1: Equationele Logica -- Initi\"ele modellen}\\[2em]


{\bf Opgave 6.1}

\begin{enumerate}

\item % model voor Booleans met junk zonder confusion

We hebben bij opgave 5.1 gezien dat iedere verzamelingsalgebra
$\mathfrak{P}(V)$ over een verzameling $V$ (als in voorbeeld 4.3) een model is
voor de specificatie $\texttt{Booleans}$. Nu beschouwen we de algebra
$\mathfrak{P}(F)$ over de verzameling $F=\{1,2,3,4,5,6\}$ met als drager
$\mathcal{P}(F)$.

\paragraph{Wel junk}

Iedere gesloten term uit $\texttt{Booleans}$ wordt in $\mathfrak{P}(F)$
ge\"interpreteerd als \`ofwel heel F, \`ofwel de lege verzameling $\{\}$. Het
bewijs hiervan verloopt via inductie naar de structuur van de term en laten we
hier achterwege.

Laten we nu het element $\{2,4,5\}$ uit $\mathfrak{P}(F)$ bekijken. Dit is
niet $F$, ook niet $\{\}$ en dus niet de interpretatie van een gesloten
term. Hieruit volgt dat $\mathfrak{P}(F)$ junk bevat.

\paragraph{Geen confusion}

Voor iedere gesloten term $t$ uit $\texttt{Booleans}$ geldt dat
\begin{align*}
\vdash t = \texttt{true} \quad \text{of} \quad \vdash t = \texttt{false}. &&\text{(volgens voorbeeld 6.3)}
\end{align*}
Laat nu $s$ en $t$ gesloten termen zijn met $\not \vdash t = s$. Dan moet
\`ofwel $\vdash s = \texttt{true}$ en $\vdash t = \texttt{false}$, \`ofwel
$\vdash s = \texttt{false}$ en $\vdash t = \texttt{true}$. In beide gevallen
worden $s$ en $t$ als verschillende elementen van $\mathfrak{P}(F)$
ge\"interpreteerd en dus bevat $\mathfrak{P}(F)$ geen confusion.\\[2em]

\item % model voor Booleans zonder junk met confusion

Todo.\\[2em]

\item % model voor Booleans met junk met confusion

Todo.\\[2em]

\end{enumerate}


\end{document}
