\documentclass[a4paper,11pt]{article}
\usepackage[dutch]{babel}
\usepackage{amsfonts}
\usepackage{a4}
\usepackage{latexsym}
\usepackage{fitch} % http://folk.uio.no/johanw/FitchSty.html

% niet te gewichtig willen doen met veel ruimte
\setlength{\textwidth}{16cm}
\setlength{\textheight}{23.0cm}
\setlength{\topmargin}{0cm}
\setlength{\oddsidemargin}{0.2mm}
\setlength{\evensidemargin}{0.2mm}
\setlength{\parindent}{0cm}

% standaard enumerate met abc
\renewcommand\theenumi{\alph{enumi}}


\begin{document}


{\bf Uitwerkingen bij Inleiding Theoretische Informatica\\
Deel 1: Equationele Logica -- Syntax en semantiek}\\[2em]


{\bf Opgave 5.1}

De algebra $\mathfrak{P}(V)$ (met $V$ een verzameling) heeft als drager de
machtsverzameling van $V$. De functies worden ge\"interpreteerd als

\begin{eqnarray*}
\texttt{true}_{\mathfrak{P}(V)}     & = & V , \\
\texttt{false}_{\mathfrak{P}(V)}    & = & \{\} , \\
\texttt{not}_{\mathfrak{P}(V)}(X)   & = & V - X , \\
\texttt{and}_{\mathfrak{P}(V)}(X,Y) & = & X \cap Y , \\
\texttt{or}_{\mathfrak{P}(V)}(X,Y)  & = & X \cup Y .
\end{eqnarray*}

We constateren eerst dat $\mathfrak{P}(V)$ een algebra voor
$\texttt{Booleans}$ is omdat de functietypes kloppen. Vervolgens laten we zien
dat $\mathfrak{P}(V)$ zelfs een model is voor $\texttt{Booleans}$ door te
laten zien dat alle vergelijkingen ($\texttt{[B1]} \ldots \texttt{[B5]}$) waar
zijn in $\mathfrak{P}(V)$.

\paragraph{}

We doen dit (nog) niet helemaal volgens de formele methode uit het dictaat
(met een assignment, een homomorphisme van de termalgebra naar
$\mathfrak{P}(V)$), omdat dit nog niet op het college behandeld is. We leiden
de vergelijkingen af voor willekeurige $x$ en $y$ uit $\mathcal{P}(V)$:

\begin{itemize}

\item{Vergelijking $\texttt{[B1]}$:}
\begin{eqnarray*}
\texttt{and}_{\mathfrak{P}(V)}(\texttt{true}_{\mathfrak{P}(V)},x) & = & \texttt{and}_{\mathfrak{P}(V)}(V,x) \\
                                                                  & = & V \cap x \\
                                                                  & = & x
\end{eqnarray*}

\item{Vergelijking $\texttt{[B2]}$:}
\begin{eqnarray*}
\texttt{and}_{\mathfrak{P}(V)}(\texttt{false}_{\mathfrak{P}(V)},x) & = & \texttt{and}_{\mathfrak{P}(V)}(\{\},x) \\
                                                                   & = & \{\} \cap x \\
                                                                   & = & \{\} \\
                                                                   & = & \texttt{false}_{\mathfrak{P}(V)}
\end{eqnarray*}

\item{Vergelijking $\texttt{[B3]}$:}
\begin{eqnarray*}
\texttt{not}_{\mathfrak{P}(V)}(\texttt{true}_{\mathfrak{P}(V)}) & = & \texttt{not}_{\mathfrak{P}(V)}(V) \\
                                                                & = & V - V \\
                                                                & = & \{\} \\
                                                                & = & \texttt{false}_{\mathfrak{P}(V)}
\end{eqnarray*}

\item{Vergelijking $\texttt{[B4]}$:}
\begin{eqnarray*}
\texttt{not}_{\mathfrak{P}(V)}(\texttt{false}_{\mathfrak{P}(V)}) & = & \texttt{not}_{\mathfrak{P}(V)}(\{\}) \\
                                                                 & = & V - \{\} \\
                                                                 & = & V \\
                                                                 & = & \texttt{true}_{\mathfrak{P}(V)}
\end{eqnarray*}

\item{Vergelijking $\texttt{[B5]}$:}
\begin{eqnarray*}
\texttt{or}_{\mathfrak{P}(V)}(x,y) & = & x \cup y \\
                                   & = & V - ((V - x) \cap (V - y)) \\
                                   & = & \texttt{not}_{\mathfrak{P}(V)}((V - x) \cap (V - y)) \\
                                   & = & \texttt{not}_{\mathfrak{P}(V)}(\texttt{and}_{\mathfrak{P}(V)}(V - x, V - y)) \\
                                   & = & \texttt{not}_{\mathfrak{P}(V)}(\texttt{and}_{\mathfrak{P}(V)}(\texttt{not}_{\mathfrak{P}(V)}(x),\texttt{not}_{\mathfrak{P}(V)}(y)))
\end{eqnarray*}

\end{itemize}

Hierbij is het wat intu\"itiever om $\texttt[B5]$ van achter naar voor te lezen.\\[2em]


{\bf Opgave 5.2}

\begin{enumerate}

\item % laat zien dat B een model is voor Naturals
We doen dit weer op dezelfde manier als bij opgave 5.1, dus niet met behulp
van een assignment functie.



\item % Is B een homomorf beeld van N?
Todo.

\end{enumerate}


\end{document}
