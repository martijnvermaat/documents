\documentclass[a4paper,11pt]{article}
\usepackage[dutch]{babel}
\usepackage{amsfonts}
\usepackage{a4}
\usepackage{latexsym}

% niet te gewichtig willen doen met veel ruimte
\setlength{\textwidth}{16cm}
\setlength{\textheight}{23.0cm}
\setlength{\topmargin}{0cm}
\setlength{\oddsidemargin}{0.2mm}
\setlength{\evensidemargin}{0.2mm}
\setlength{\parindent}{0cm}

% standaard enumerate met abc
\renewcommand\theenumi{\alph{enumi}}


\begin{document}


{\bf Uitwerkingen bij Inleiding Theoretische Informatica\\
Deel 1: Equationele Logica -- Syntax en semantiek}\\[2em]


{\bf Opgave 5.1}

De algebra $\mathfrak{P}(V)$ (met $V$ een verzameling) heeft als drager de
machtsverzameling van $V$. De functies worden ge\"interpreteerd als

\begin{eqnarray*}
\texttt{true}_{\mathfrak{P}(V)}     & = & V , \\
\texttt{false}_{\mathfrak{P}(V)}    & = & \{\} , \\
\texttt{not}_{\mathfrak{P}(V)}(X)   & = & V - X , \\
\texttt{and}_{\mathfrak{P}(V)}(X,Y) & = & X \cap Y , \\
\texttt{or}_{\mathfrak{P}(V)}(X,Y)  & = & X \cup Y .
\end{eqnarray*}

We constateren eerst dat $\mathfrak{P}(V)$ een algebra voor
$\texttt{Booleans}$ is omdat de functietypes kloppen. Vervolgens laten we zien
dat $\mathfrak{P}(V)$ zelfs een model is voor $\texttt{Booleans}$ door te
laten zien dat alle vergelijkingen ($\texttt{[B1]} \ldots \texttt{[B5]}$) waar
zijn in $\mathfrak{P}(V)$.

\paragraph{}

Laat $\theta$ een assignment zijn van elementen uit $\mathcal{P}(V)$ aan
de variabelen $x$ en $y$, volgens

\begin{eqnarray*}
\theta & : & \{x,y\} \rightarrow \mathcal{P}(V) \\
\theta(x) & = & X \\
\theta(y) & = & Y.
\end{eqnarray*}

We laten nu zien dat voor iedere vergelijking uit $\texttt{Booleans}$ de
rechter kant identiek is aan de linker kant onder $\bar \theta$.

\begin{itemize}

\item{$\texttt{[B1] and(true,x) = x}$}
\begin{eqnarray*}
\bar \theta(\texttt{and}(\texttt{true}, x)) & = & \texttt{and}_{\mathfrak{P}(V)}(\bar \theta(\texttt{true}), \bar \theta(x)) \\
                                            & = & \texttt{and}_{\mathfrak{P}(V)}(\texttt{true}_{\mathfrak{P}(V)}, \theta(x)) \\
                                            & = & \texttt{and}_{\mathfrak{P}(V)}(V,X) \\
                                            & = & V \cap X \\
                                            & = & X
\end{eqnarray*}

\begin{eqnarray*}
\bar \theta(x) & = & \theta(x) \\
               & = & X
\end{eqnarray*}

\item{$\texttt{[B2] and(false,x) = false}$}
\begin{eqnarray*}
\bar \theta(\texttt{and}(\texttt{false}, x)) & = & \texttt{and}_{\mathfrak{P}(V)}(\bar \theta(\texttt{false}), \bar \theta(x)) \\
                                             & = & \texttt{and}_{\mathfrak{P}(V)}(\texttt{false}_{\mathfrak{P}(V)}, \theta(x)) \\
                                             & = & \texttt{and}_{\mathfrak{P}(V)}(\{\},X) \\
                                             & = & \{\} \cap X \\
                                             & = & \{\}
\end{eqnarray*}

\begin{eqnarray*}
\bar \theta(\texttt{false}) & = & \texttt{false}_{\mathfrak{P}(V)} \\
                            & = & \{\}
\end{eqnarray*}

\item{$\texttt{[B3] not(true) = false}$}
\begin{eqnarray*}
\bar \theta(\texttt{not(true)}) & = & \texttt{not}_{\mathfrak{P}(V)}(\bar \theta(\texttt{true})) \\
                                & = & \texttt{not}_{\mathfrak{P}(V)}(\texttt{true}_{\mathfrak{P}(V)}) \\
                                & = & \texttt{not}_{\mathfrak{P}(V)}(V) \\
                                & = & V - V \\
                                & = & \{\}
\end{eqnarray*}

\begin{eqnarray*}
\bar \theta(\texttt{false}) & = & \texttt{false}_{\mathfrak{P}(V)} \\
                            & = & \{\}
\end{eqnarray*}

\item{$\texttt{[B4] not(false) = true}$}
\begin{eqnarray*}
\bar \theta(\texttt{not(false)}) & = & \texttt{not}_{\mathfrak{P}(V)}(\bar \theta(\texttt{false})) \\
                                 & = & \texttt{not}_{\mathfrak{P}(V)}(\texttt{false}_{\mathfrak{P}(V)}) \\
                                 & = & \texttt{not}_{\mathfrak{P}(V)}(\{\}) \\
                                 & = & V - \{\} \\
                                 & = & V
\end{eqnarray*}

\begin{eqnarray*}
\bar \theta(\texttt{true}) & = & \texttt{true}_{\mathfrak{P}(V)} \\
                            & = & V
\end{eqnarray*}

\item{$\texttt{[B5] or(x,y) = not(and(not(x),not(y)))}$}
\begin{eqnarray*}
\bar \theta(\texttt{or}(x,y)) & = & \texttt{or}_{\mathfrak{P}(V)}(\bar \theta(x), \bar \theta(y)) \\
                              & = & \texttt{or}_{\mathfrak{P}(V)}(\theta(x), \theta(y)) \\
                              & = & \texttt{or}_{\mathfrak{P}(V)}(X,Y) \\
                              & = & X \cup Y
\end{eqnarray*}

\begin{eqnarray*}
\bar \theta(\texttt{not}(\texttt{and}(\texttt{not}(x), \texttt{not}(y))))
      & = & \texttt{not}_{\mathfrak{P}(V)}(\bar \theta(\texttt{and}(\texttt{not}(x), \texttt{not}(y)))) \\
      & = & \texttt{not}_{\mathfrak{P}(V)}(\texttt{and}_{\mathfrak{P}(V)}(\bar \theta(\texttt{not}(x)), \bar \theta(\texttt{not}(y)))) \\
      & = & \texttt{not}_{\mathfrak{P}(V)}(\texttt{and}_{\mathfrak{P}(V)}(\texttt{not}_{\mathfrak{P}(V)}(\bar \theta(x)), \texttt{not}_{\mathfrak{P}(V)}(\bar \theta(y)))) \\
      & = & \texttt{not}_{\mathfrak{P}(V)}(\texttt{and}_{\mathfrak{P}(V)}(\texttt{not}_{\mathfrak{P}(V)}(\theta(x)), \texttt{not}_{\mathfrak{P}(V)}(\theta(y)))) \\
      & = & \texttt{not}_{\mathfrak{P}(V)}(\texttt{and}_{\mathfrak{P}(V)}(\texttt{not}_{\mathfrak{P}(V)}(X), \texttt{not}_{\mathfrak{P}(V)}(Y))) \\
      & = & \texttt{not}_{\mathfrak{P}(V)}(\texttt{and}_{\mathfrak{P}(V)}(V - X, V - Y)) \\
      & = & \texttt{not}_{\mathfrak{P}(V)}((V - X) \cap (V - Y)) \\
      & = & V - ((V - X) \cap (V - Y)) \\
      & = & X \cup Y
\end{eqnarray*}

\end{itemize}

Hiermee hebben we laten zien dat iedere vergelijking in $\texttt{Booleans}$
waar is in $\mathfrak{P}(V)$ en dus dat $\mathfrak{P}(V)$ een model is voor de
specificatie $\texttt{Booleans}$.\\[2em]


{\bf Opgave 5.2}

\begin{enumerate}

\item % laat zien dat B een model is voor Naturals
We doen dit weer op dezelfde manier als bij opgave 5.1, dus niet met behulp
van een assignment functie. De vergelijkingen maken we waar met voor $x$ en
$y$ willekeurige elementen uit $\{T,F\}$.

\begin{itemize}

\item{De vergelijking $\texttt{[A1]}$:}
\begin{eqnarray*}
\texttt{add}_{\mathfrak{B}}(x, \texttt{0}_{\mathfrak{B}}) & = & \texttt{add}_{\mathfrak{B}}(x, T) \\
                                                          & = & x
\end{eqnarray*}

\item{De vergelijking $\texttt{[A2]}$:}
\begin{eqnarray*}
\texttt{add}_{\mathfrak{B}}(x, \texttt{succ}_{\mathfrak{B}}(y)) & = & \texttt{add}_{\mathfrak{B}}(x, \neg y) \\
                                                                & = & xor(x, \neg y) \\
                                                                & = & not(xor(x,y)) \\
                                                                & = & \texttt{succ}_{\mathfrak{B}}(xor(x,y)) \\
                                                                & = & \texttt{succ}_{\mathfrak{B}}(\texttt{add}_{\mathfrak{B}}(x, y))
\end{eqnarray*}

\item{De vergelijking $\texttt{[M1]}$:}
\begin{eqnarray*}
\texttt{mul}_{\mathfrak{B}}(x, \texttt{0}_{\mathfrak{B}}) & = & \texttt{mul}_{\mathfrak{B}}(x, T) \\
                                                          & = & T \\
                                                          & = & \texttt{0}_{\mathfrak{B}}
\end{eqnarray*}

\item{De vergelijking $\texttt{[M2]}$:}
\begin{eqnarray*}
\texttt{mul}_{\mathfrak{B}}(x, \texttt{succ}_{\mathfrak{B}}(y)) & = & \texttt{mul}_{\mathfrak{B}}(x, \neg y) \\
                                                                & = & x \vee \neg y \\
                                                                & = & xor(x \vee y, x) \\
                                                                & = & \texttt{add}_{\mathfrak{B}}(x \vee y),x) \\
                                                                & = & \texttt{add}_{\mathfrak{B}}(\texttt{mul}_{\mathfrak{B}}(x, y),x)
\end{eqnarray*}

(Geloof maar gewoon dat inderdaad $x \vee \neg y = xor(x \vee y, x)$, of maak
de waarheidstafels als je het zeker wilt weten.)

\end{itemize}

\item % Is B een homomorf beeld van N?
Todo.\\[2em]

\end{enumerate}


{\bf Opgave 5.3}

Todo.


\end{document}
