\documentclass[a4paper,11pt]{article}
\usepackage[dutch]{babel}
\usepackage{a4,fullpage}
\usepackage{amsmath,amsfonts,amssymb}
\usepackage{amsthm}
\usepackage[T1]{fontenc}
\usepackage{lmodern} % Latin modern font family
\usepackage{bussproofs} % http://math.ucsd.edu/%7Esbuss/ResearchWeb/bussproofs/

% Sans-serif fonts
%\usepackage[T1experimental,lm]{sfmath} % http://dtrx.de/od/tex/sfmath.html
%\renewcommand{\familydefault}{\sfdefault}

% Some configuration for listings
\renewcommand{\labelenumi}{\arabic{enumi}.}
\renewcommand{\labelenumii}{(\alph{enumii})}

\newcounter{firstcounter}
\newcommand{\labelfirst}{(\roman{firstcounter})}
%\newcommand{\spacingfirst}{\usecounter{firstcounter}\setlength{\rightmargin}{\leftmargin}}
\newcommand{\spacingfirst}{\usecounter{firstcounter}}

\newcounter{secondcounter}
\newcommand{\labelsecond}{(\arabic{secondcounter})}
%\newcommand{\spacingsecond}{\usecounter{secondcounter}\setlength{\rightmargin}{\leftmargin}}
\newcommand{\spacingsecond}{\usecounter{secondcounter}}

\title{Inleiding Theoretische Informatica -- Lambda Calculus\\
\normalsize{Uitwerkingen van geselecteerde opgaven}}
\author{Martijn Vermaat (mvermaat@cs.vu.nl)}
\date{10 maart 2006}


\begin{document}

\maketitle


\section{Termen en reductie}


\begin{enumerate}


\item[2.]
$\lambda x. \lambda y. \texttt{0}$ of $\lambda xy. \texttt{0}$


\item[3.]

\begin{enumerate}

\item
$( \, (\lambda x. (\lambda y. (x \, (y \, z)))) \quad (\lambda x. ((y \, x) \, x)) \, )$

\end{enumerate}


\item[4.]

\begin{enumerate}

\item[(b)]
$(\lambda x. x) \quad (\lambda x. \lambda y. x \, y \, y) \quad \lambda x. \lambda y. x \, (x \, y)$

\end{enumerate}


\item[7.]

\begin{enumerate}

\item[(a)] Nee.
\item[(c)] Nee.
\item[(e)] Nee.
\item[(h)] Ja.

\end{enumerate}


\item[8.]

\begin{enumerate}

\item[(b)] $(\lambda x. \textsf{mul} \, x \, y)$

\item[(c)] $(\lambda x. \textsf{mul} \, x \, 5)$

\item[(h)] $(\lambda z. (\lambda y. \textsf{plus} \, x \, y) \, z)$

\end{enumerate}


\item[10.]

\begin{enumerate}

\item[(a)]
$\underline{(\lambda y. \textsf{mul} \, 3 \, y) \; 7}
  \quad \rightarrow_{\beta} \quad
  \textsf{mul} \, 3 \, 7
  \quad \rightarrow_{\delta} \quad
  21$

\item[(d)]
\begin{align*}
  \underline{(\lambda f \, x. f \, x) \quad (\lambda y. \textsf{plus} \, x \, y)} \quad 3
  & \quad \rightarrow_{\beta} \quad
  \underline{(\lambda z. \underline{(\lambda y. \textsf{plus} \, x \, y) \, z}) \quad 3} \\
  & \quad \rightarrow_{\beta} \quad
  \underline{(\lambda z. \textsf{plus} \, x \, z) \quad 3} \\
  & \quad \rightarrow_{\beta} \quad
  \textsf{plus} \, x \, 3
\end{align*}

\item[(f)]
\begin{align*}
  \underline{(\lambda f \, x. f \, (f \, (x))) \; (\lambda x. \textsf{mul} \, x \, 7)} \; 3
  & \quad \rightarrow_{\beta} \quad
  \underline{(\lambda x. \underline{(\lambda x. \textsf{mul} \, x \, 7) \, (\underline{(\lambda x. \textsf{mul} \, x \, 7) \, (x)})}) \; 3} \\
  & \quad \rightarrow_{\beta} \quad
  \underline{(\lambda x. \underline{(\lambda x. \textsf{mul} \, x \, 7) \, (\textsf{mul} \, x \, 7)}) \; 3} \\
  & \quad \rightarrow_{\beta} \quad
  \underline{(\lambda x. \textsf{mul} \, (\textsf{mul} \, x \, 7) \, 7) \; 3} \\
  & \quad \rightarrow_{\beta} \quad
  \textsf{mul} \, (\textsf{mul} \, 3 \, 7) \, 7 \\
  & \quad \rightarrow_{\delta} \quad
  \textsf{mul} \, 21 \, 7 \\
  & \quad \rightarrow_{\delta} \quad
  147
\end{align*}

\end{enumerate}


\end{enumerate}


\section{Reductiestrategie\"en}


\begin{enumerate}


\item[2.]

\begin{enumerate}

\item[(a)]
\begin{align*}
  (\lambda x. x \, x \, x) \quad \lambda x. x \, x \, x
  & \quad \rightarrow_{\beta} \quad
  (\lambda x. x \, x \, x) \quad (\lambda x. x \, x \, x) \quad \lambda x. x \, x \, x\\
  & \quad \rightarrow_{\beta} \quad
  (\lambda x. x \, x \, x) \quad (\lambda x. x \, x \, x) \quad (\lambda x. x \, x \, x) \quad \lambda x. x \, x \, x\\
  & \quad \rightarrow_{\beta} \quad
  (\lambda x. x \, x \, x) \quad (\lambda x. x \, x \, x) \quad (\lambda x. x \, x \, x) \quad (\lambda x. x \, x \, x) \quad \lambda x. x \, x \, x\\
  & \quad \rightarrow_{\beta} \quad
  \ldots \text{etc.}
\end{align*}

\item[(c)]
De leftmost-innermost strategie vindt altijd een normaalvorm als deze bestaat.
We vinden in dit geval geen normaalvorm, dus heeft deze term geen normaalvorm. 

\end{enumerate}


\end{enumerate}


\section{Datatypes}


\begin{enumerate}


\item[9.]

\begin{enumerate}

\item
\begin{align*}
  \textsf{xor} \, \textsf{true} \, \textsf{true} \quad & =_{\beta} \quad \textsf{false}\\
  \textsf{xor} \, \textsf{true} \, \textsf{false} \quad & =_{\beta} \quad \textsf{true}\\
  \textsf{xor} \, \textsf{false} \, \textsf{true} \quad & =_{\beta} \quad \textsf{true}\\
  \textsf{xor} \, \textsf{false} \, \textsf{false} \quad & =_{\beta} \quad \textsf{false}
\end{align*}

\item
$\textsf{xor} \, := \, \lambda a b. b \, (a \, \textsf{false} \, \textsf{true}) \, (a \, \textsf{true} \, \textsf{false})$

\item
\begin{align*}
  \textsf{xor} \, \textsf{true} \, \textsf{false}
  & \quad = \quad
  (\lambda a b. b \, (a \, \textsf{false} \, \textsf{true}) \, (a \, \textsf{true} \, \textsf{false})) \, (\lambda x y. x) \, \lambda x y. y\\
  & \quad \rightarrow_{\beta} \quad
  (\lambda b. b \, ((\lambda x y. x) \, \textsf{false} \, \textsf{true}) \, ((\lambda x y. x) \, \textsf{true} \, \textsf{false})) \, \lambda x y. y\\
  & \quad \rightarrow_{\beta} \quad
  (\lambda x y. y) \, ((\lambda x y. x) \, \textsf{false} \, \textsf{true}) \, ((\lambda x y. x) \, \textsf{true} \, \textsf{false})\\
  & \quad \rightarrow_{\beta} \quad
  (\lambda y. y) \, ((\lambda x y. x) \, \textsf{true} \, \textsf{false})\\
  & \quad \rightarrow_{\beta} \quad
  (\lambda x y. x) \, \textsf{true} \, \textsf{false}\\
  & \quad \rightarrow_{\beta} \quad
  (\lambda y. \textsf{true}) \, \textsf{false}\\
  & \quad \rightarrow_{\beta} \quad
  \textsf{true}
\end{align*}

\end{enumerate}


\end{enumerate}


\section{Recursie}


\begin{enumerate}


\item[7.]
Een eerste poging is deze recursieve definitie van $\textsf{map}$\\
$\textsf{map} \, = \, \lambda f l. (\textsf{empty} \, l) \, \textsf{nil} \,
(\textsf{cons} \, (f \, (\textsf{head} \, l)) \, (\textsf{map} \, f \, (\textsf{tail} \, l)))$

Dit kunnen we ook schrijven als\\
$\textsf{map} \, = \, (\lambda m. \lambda f l. (\textsf{empty} \, l) \, \textsf{nil} \,
(\textsf{cons} \, (f \, (\textsf{head} \, l)) \, (m \, f \, (\textsf{tail} \, l)))) \, \textsf{map}$

Nu kunnen we de fixed-point combinator $\textsf{Y}$ gebruiken om hiervan de
uiteindelijke definitie van $\textsf{map}$ te maken:\\
$\textsf{map} \, = \, \textsf{Y} \, \lambda m. \lambda f l. (\textsf{empty} \, l) \, \textsf{nil} \,
(\textsf{cons} \, (f \, (\textsf{head} \, l)) \, (\textsf{map} \, f \, (\textsf{tail} \, l)))$


\end{enumerate}


\section{Getypeerde $\lambda$-calculus}


\begin{enumerate}


\item[1.]

\begin{enumerate}

\item[(c)]
De term $\textsf{K} \, \textsf{I}$ is gelijk aan $(\lambda x y. x) \, \lambda x. x$.
\begin{prooftree}
\AxiomC{$x \: : \: C \vdash x \: : \: C$}
\UnaryInfC{$\vdash \lambda x. x \: : \: C \rightarrow C$}
\AxiomC{$x \: : \: C \rightarrow C , y \: : \: B \vdash x \: : \: C \rightarrow C$}
\UnaryInfC{$x \: : \: C \rightarrow C \vdash \lambda y. x \: : \: B \rightarrow C \rightarrow C$}
\UnaryInfC{$\vdash \lambda x y. x \: : \: (C \rightarrow C) \rightarrow B \rightarrow C \rightarrow C$}
\BinaryInfC{$\vdash (\lambda x y. x) \, \lambda x. x \: : \: B \rightarrow C \rightarrow C$}
\end{prooftree}

\end{enumerate}


\end{enumerate}


\end{document}
