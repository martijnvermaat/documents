\documentclass[a4paper,11pt]{article}
\usepackage[dutch]{babel}
\usepackage{amsfonts}
\usepackage{a4}
\usepackage{latexsym}
\usepackage{fitch} % http://folk.uio.no/johanw/FitchSty.html

% niet te gewichtig willen doen met veel ruimte
\setlength{\textwidth}{16cm}
\setlength{\textheight}{23.0cm}
\setlength{\topmargin}{0cm}
\setlength{\oddsidemargin}{0.2mm}
\setlength{\evensidemargin}{0.2mm}
\setlength{\parindent}{0cm}

% standaard enumerate met abc
\renewcommand\theenumi{\alph{enumi}}


\begin{document}


{\bf Uitwerkingen bij Inleiding Theoretische Informatica\\
Deel 1: Equationele Logica -- Initi\"ele modellen}\\[2em]


{\bf Opgave 6.1}

\begin{enumerate}

\item % model voor Booleans met junk zonder confusion

We hebben bij opgave 5.1 gezien dat iedere verzamelingsalgebra
$\mathfrak{P}(V)$ over een verzameling $V$ (als in voorbeeld 4.3) een model is
voor de specificatie $\texttt{Booleans}$. Nu beschouwen we de algebra
$\mathfrak{P}(F)$ over de verzameling $F=\{1,2,3,4,5,6\}$ met als drager
$\mathcal{P}(F)$.

\paragraph{Wel junk}

Iedere gesloten term uit $\texttt{Booleans}$ wordt in $\mathfrak{P}(F)$
ge\"interpreteerd als \`ofwel heel F, \`ofwel de lege verzameling $\{\}$. Het
bewijs hiervan verloopt via inductie naar de structuur van de term en laten we
hier achterwege.

Laten we nu het element $\{2,4,5\}$ uit $\mathfrak{P}(F)$ bekijken. Dit is
niet $F$, ook niet $\{\}$ en dus niet de interpretatie van een gesloten
term. Hieruit volgt dat $\mathfrak{P}(F)$ junk bevat.

\paragraph{Geen confusion}

Voor iedere gesloten term $t$ uit $\texttt{Booleans}$ geldt dat
\begin{align*}
\vdash t = \texttt{true} \quad \text{of} \quad \vdash t = \texttt{false}. &&\text{(volgens voorbeeld 6.3)}
\end{align*}
Laat nu $s$ en $t$ gesloten termen zijn met $\not \vdash t = s$. Dan moet
\`ofwel $\vdash s = \texttt{true}$ en $\vdash t = \texttt{false}$, \`ofwel
$\vdash s = \texttt{false}$ en $\vdash t = \texttt{true}$. In beide gevallen
worden $s$ en $t$ als verschillende elementen van $\mathfrak{P}(F)$
ge\"interpreteerd en dus bevat $\mathfrak{P}(F)$ geen confusion.\\[2em]

\item % model voor Booleans zonder junk met confusion

Laten we nu analoog aan onderdeel a de verzamelingsalgebra
$\mathfrak{P}(\{\})$ over de lege verzameling bekijken met in de drager als
enige element $\{\}$.

\paragraph{Geen junk}

Volgens de definitie van de verzamelingsalgebra interpreteren we nu
$\texttt{true}$ als $\{\}$. Hieruit volgt direct dat $\mathfrak{P}(\{\})$ geen
junk bevat, want het enige element uit de drager is de interpretatie van een
gesloten term.

\paragraph{Wel confusion}

In de algebra $\mathfrak{P}(\{\})$ worden de interpretaties van de termen
$\texttt{true}$ en $\texttt{false}$ ge\"identificeerd. Toch is de vergelijking
$\texttt{true} = \texttt{false}$ niet afleidbaar uit de specificatie
$\texttt{Booleans}$ (bovenstaande algebra $\mathfrak{P}(F)$ is bijvoorbeeld
een tegenmodel) en dus bevat $\mathfrak{P}(\{\})$ confusion.\\[2em]

\item % model voor Booleans met junk met confusion

We bekijken de algebra $\mathfrak{B}_{3}$ voor de specificatie
$\texttt{Booleans}$ met als drager $\{A,B,C\}$ en interpretaties
\begin{align*}
\texttt{true}_{\mathfrak{B}_{3}}     &= B, \\
\texttt{false}_{\mathfrak{B}_{3}}    &= B, \\
\texttt{and}_{\mathfrak{B}_{3}}(x,y) &= B, \\
\texttt{or}_{\mathfrak{B}_{3}}(x,y)  &= B, \\
\texttt{not}_{\mathfrak{B}_{3}}(x)   &= B.
\end{align*}
Dat $\mathfrak{B}_{3}$ een model is voor $\texttt{Booleans}$ mag duidelijk
zijn (iedere term evalueert naar $B$ en dus is iedere vergelijking waar).

\paragraph{Wel junk}

Het is niet moeilijk in te zien dat iedere gesloten term uit
$\texttt{Booleans}$ wordt in $\mathfrak{B}_{3}$ ge\"interpreteerd als $B$. Dit
betekent dat de elementen $A$ en $C$ niet de interpretatie van een gesloten
term zijn en dus dat $\mathfrak{B}_{3}$ junk bevat.

\paragraph{Wel confusion}

In de algebra $\mathfrak{B}_{3}$ worden de interpretaties van de termen
$\texttt{true}$ en $\texttt{false}$ ge\"identificeerd. Toch is de vergelijking
$\texttt{true} = \texttt{false}$ niet afleidbaar uit de specificatie
$\texttt{Booleans}$ (bovenstaande algebra $\mathfrak{P}(F)$ is bijvoorbeeld
een tegenmodel) en dus bevat $\mathfrak{B}_{3}$ confusion.\\[2em]

\end{enumerate}


{\bf Opgave 6.2}

\begin{enumerate}

\item % construeer het termmodel voor Booleans

We construeren het termmodel $\mathfrak{Ter}_{\Sigma}/\sim$ voor de
specificatie $\texttt{Booleans}$ met behulp van de equivalentierelatie $\sim$
op termen uit $Ter_{\Sigma}$:
\begin{displaymath}
s \sim t \, \Longleftrightarrow \, E \vdash s = t
\end{displaymath}
waarbij E de verzameling vergelijkingen in $\texttt{Booleans}$ is.

Als drager van het termmodel nemen we nu de equivalentieklassen van de
gesloten termen onder $\sim$. We hebben eerder gezien dat iedere gesloten term
in $\texttt{Booleans}$ afleidbaar gelijk is aan \`ofwel $\texttt{true}$
\`ofwel $\texttt{false}$, dus hebben we als drager genoeg aan de twee
equivalentieklassen $[\texttt{true}]$ en $[\texttt{false}]$ van de termen
$\texttt{true}$ en $\texttt{false}$.

De interpretaties van de constanten en functiesymbolen kiezen we in het
termmodel als volgt:
\begin{align*}
\texttt{true}_{\mathfrak{Ter}_{\Sigma}/\sim}         &= [\texttt{true}] \\
\texttt{false}_{\mathfrak{Ter}_{\Sigma}/\sim}        &= [\texttt{false}] \\
\texttt{and}_{\mathfrak{Ter}_{\Sigma}/\sim}([t],[s]) &= [\texttt{and}(t,s)] \\
\texttt{or}_{\mathfrak{Ter}_{\Sigma}/\sim}([t],[s])  &= [\texttt{or}(t,s)] \\
\texttt{not}_{\mathfrak{Ter}_{\Sigma}/\sim}([t])     &= [\texttt{not}(t)] \\
\end{align*}

\item % laat zien dat het termmodel isomorf is met B2

We bekijken nu de functie $\phi : Ter_{\Sigma}/\negmedspace\sim \,\,
\rightarrow \, A$ die equivalentieklassen afbeelt op de drager $A$ van
$\mathfrak{B}_{2}$ en gedefini\"eerd is als
\begin{align*}
\phi([\texttt{true}])  &= T, \\
\phi([\texttt{false}]) &= F.
\end{align*}
Nu laten we zien dat $\phi$ een isomorfisme is van
$\mathfrak{Ter}_{\Sigma}/\sim$ naar $\mathfrak{B}_{2}$.

\paragraph{Homomorfisme}

Eerst bekijken we of $\phi$ een homomorfisme is. Voor de constante
$\texttt{true}$ hebben we:
\begin{align*}
\phi(\texttt{true}_{\mathfrak{Ter}_{\Sigma}/\sim}) &= \phi([\texttt{true}]) \\
                                                   &= T \\
                                                   &= \texttt{true}_{\mathfrak{B}_{2}}
\end{align*}
Voor de constante $\texttt{false}$ gaat dit op gelijke wijze. We bekijken nu
het functiesymbool $\texttt{not}$ toegepast op het element $x$. Voor de waarde
van $x$ zijn er twee mogelijkheden, we beperken ons hier tot
$[\texttt{true}]$:
\begin{align*}
\phi(\texttt{not}_{\mathfrak{Ter}_{\Sigma}/\sim}(x)) &= \phi(\texttt{not}_{\mathfrak{Ter}_{\Sigma}/\sim}([\texttt{true}])) \\
                                                     &= \phi([\texttt{not(true)}]) \\
                                                     &= \phi([\texttt{false}]) \\
                                                     &= F
\end{align*}
\begin{align*}
\texttt{not}_{\mathfrak{B}_{2}}(\phi(x)) &= \texttt{not}_{\mathfrak{B}_{2}}(\phi([\texttt{true}])) \\
                                         &= \texttt{not}_{\mathfrak{B}_{2}}(T) \\
                                         &= F
\end{align*}
Vervolgens bekijken we het functiesymbool $\texttt{and}$ toegepast op de
elementen $x$ en $y$. Ook hier zijn er voor $x$ en $y$ beiden twee mogelijke
waarden, we behandelen alleen het geval dat $x = [\texttt{true}]$ en $y =
[\texttt{false}]$:
\begin{align*}
\phi(\texttt{and}_{\mathfrak{Ter}_{\Sigma}/\sim}(x,y)) &= \phi(\texttt{and}_{\mathfrak{Ter}_{\Sigma}/\sim}([\texttt{true}],[\texttt{false}])) \\
                                                       &= \phi([\texttt{and}(\texttt{true},\texttt{false})]) \\
                                                       &= \phi([\texttt{false}]) \\
                                                       &= F
\end{align*}
\begin{align*}
\texttt{and}_{\mathfrak{B}_{2}}(\phi(x),\phi(y)) &= \texttt{and}_{\mathfrak{B}_{2}}(\phi([\texttt{true}]),\phi([\texttt{false}])) \\
                                                 &= \texttt{and}_{\mathfrak{B}_{2}}(T,F) \\
                                                 &= F
\end{align*}
Hetzelfde verhaal voor het functiesymbool $\texttt{or}$ zullen we achterwege
laten. Hiermee hebben we laten zien dat $\phi$ een homomorfisme is van
$\mathfrak{Ter}_{\Sigma}/\sim$ naar $\mathfrak{B}_{2}$.

\paragraph{Surjectief}

De drager van $\mathfrak{B}_{2}$ bestaat uit de elementen $T$ en $F$. In onze
definitie van $\phi$ zien we duidelijk dat beiden het beeld zijn van een
element uit $\mathfrak{Ter}_{\Sigma}/\sim$ en dus is $\phi$ surjectief.

\paragraph{Injectief}

Evenzo is gemakkelijk te zien dat de twee verschillende elementen
$[\texttt{true}]$ en $[\texttt{false}]$ met $\phi$ ook twee verschillende
beelden hebben, namelijk $T$ respectievelijk $F$. En dus is $\phi$ injectief.

Nu we hebben laten zien dat er een isomorfisme bestaat van
$\mathfrak{Ter}_{\Sigma}/\sim$ naar $\mathfrak{B}_{2}$ ($\phi$ is een
injectief en surjectief homomorfisme) weten we dus ook dat deze twee algebra's
isomorf zijn.\\[2em]

\end{enumerate}


{\bf Opgave 6.3}

\begin{enumerate}

\item % construeer het termmodel voor Naturals

Met behulp van de equivalentierelatie $\sim$ ($s \sim t \Leftrightarrow$ $s =
t$ is afleidbaar uit $\texttt{Naturals}$) construeren we het termmodel
$\mathfrak{Ter}_{\Sigma}/\sim$ met als drager de equivalentieklassen van de
gesloten termen onder $\sim$.

In opgave 3.3 hebben we bewezen dat iedere gesloten term in
$\texttt{Naturals}$ afleidbaar gelijk is aan een term van de vorm
$\texttt{succ}^{n}(\texttt{0})$, dus behoort ook iedere gesloten term tot een
equivalentieklasse van de vorm $[\texttt{succ}^{n}(\texttt{0})]$. Onze drager
bestaat dan ook precies uit de klassen van deze vorm.

We bekijken nu de interpretaties van de constanten en functiesymbolen:

\begin{align*}
\texttt{0}_{\mathfrak{Ter}_{\Sigma}/\sim}            &= [\texttt{0}] \\
\texttt{succ}_{\mathfrak{Ter}_{\Sigma}/\sim}([t])    &= [\texttt{succ}(t)] \\
\texttt{add}_{\mathfrak{Ter}_{\Sigma}/\sim}([t],[s]) &= [\texttt{add}(t,s)] \\
\texttt{mul}_{\mathfrak{Ter}_{\Sigma}/\sim}([t],[s]) &= [\texttt{mul}(t,s)] \\
\end{align*}

Juist omdat iedere term afleidbaar gelijk is aan een term van de vorm
$\texttt{succ}^{n}(\texttt{0})$ zou je dezelfde interpretaties ook zo kunnen
schrijven, het resultaat is exact hetzelfde:

\begin{align*}
\texttt{0}_{\mathfrak{Ter}_{\Sigma}/\sim}
  &= [\texttt{0}] \\
\texttt{succ}_{\mathfrak{Ter}_{\Sigma}/\sim}([\texttt{succ}^{n}(\texttt{0})])
  &= [\texttt{succ}^{n+1}(\texttt{0})] \\
\texttt{add}_{\mathfrak{Ter}_{\Sigma}/\sim}([\texttt{succ}^{m}(\texttt{0})],[\texttt{succ}^{n}(\texttt{0})])
  &= [\texttt{succ}^{m+n}(\texttt{0})] \\
\texttt{mul}_{\mathfrak{Ter}_{\Sigma}/\sim}([\texttt{succ}^{m}(\texttt{0})],[\texttt{succ}^{n}(\texttt{0})])
  &= [\texttt{succ}^{m*n}(\texttt{0})]
\end{align*}

\item % laat zien dat het termmodel isomorf is met N

We laten nu zien dat dit termmodel isomorf is met het model
$\mathfrak{N}$. Hiertoe defini\"eren we een isomorphisme $\phi :
Ter_{\Sigma}/\negmedspace\sim \, \, \rightarrow \, \mathbb{N}$ dat
equivalentieklassen afbeelt op natuurlijke getallen als

\begin{align*}
\phi([\texttt{succ}^{n}(\texttt{0})]) &= n.
\end{align*}

We laten nu zien dat $\phi$ inderdaad een injectief en surjectief homomorfisme
is van $\mathfrak{Ter}_{\Sigma}/\sim$ naar $\mathfrak{N}$.

\paragraph{Homomorphisme}

We bekijken op de gebruikelijke manier hoe $\phi$ zich gedraagt met betrekking
tot de voorwaarden die gesteld worden aan een homomorfisme. Voor de constante
$\texttt{0}$ hebben we:

\begin{align*}
\phi(\texttt{0}_{\mathfrak{Ter}_{\Sigma}/\sim}) &= \phi([\texttt{0}]) \\
                                                &= 0 \\
                                                &= \texttt{0}_{\mathfrak{N}}
\end{align*}

Voor iedere $n \ge 0$ hebben we voor de het functiesymbool $\texttt{succ}$:

\begin{align*}
\phi(\texttt{succ}_{\mathfrak{Ter}_{\Sigma}/\sim}([\texttt{succ}^{n}(\texttt{0})])
  &= \phi([\texttt{succ}^{n+1}(\texttt{0})]) \\
  &= n+1 \\
  &= \texttt{succ}_{\mathfrak{N}}(n) \\
  &= \texttt{succ}_{\mathfrak{N}}(\phi([\texttt{succ}^{n}(\texttt{0})]))
\end{align*}

Op dezelfde manier kunnen we de functiesymbolen $\texttt{add}$ en
$\texttt{mul}$ beschouwen, iets dat we nu niet zullen doen. Hiermee hebben we
laten zien dat $\phi$ inderdaad een homomorfisme is van
$\mathfrak{Ter}_{\Sigma}/\sim$ naar $\mathfrak{N}$.

\paragraph{Surjectief}

Beschouw een willekeurig getal $n$ uit de drager van $\mathfrak{N}$. Volgens
$\phi$ is dit getal het beeld van de equivalentieklasse
$[\texttt{succ}^{n}(\texttt{0})]$ en deze equivalentieklasse is een element
uit de drager van $\mathfrak{Ter}_{\Sigma}/\sim$. Dit betekent dat $\phi$
surjectief is.

\paragraph{Injectief}

Beschouw nu twee equivalentieklassen $[\texttt{succ}^{m}(\texttt{0})]$ en
$[\texttt{succ}^{n}(\texttt{0})]$. Wanneer dit verschillende elementen in de
drager van het termmodel zijn, moet $m \neq n$. Omdat $m$ en $n$ precies de
respectievelijke beelden zijn van deze twee klassen onder $\phi$, zijn deze
ook verschillend. En dus is $\phi$ injectief.

Omdat $\phi$ een injectief en surjectief homomorfisme is, is het een
isomorfisme. Dit betekent dat $\mathfrak{Ter}_{\Sigma}/\sim$ en $\mathfrak{N}$
isomorf zijn.\\[2em]

\end{enumerate}


{\bf Opgave 6.5}

\begin{enumerate}

\item % afleiding voor s(s(h(a))) = s(h(s(a)))

Een afleiding voor $\texttt{s(s(h(a))) = s(h(s(a)))}$ in $\texttt{Spec}$:
\begin{equation*}
\begin{fitch}
\texttt{s(h(x)) = s(x)}                 & \texttt{[E2]}  \\ % 1
\texttt{s(s(h(x))) = s(s(x))}           & congr, 1       \\ % 2
\texttt{s(s(h(a))) = s(s(a))}           & subst, 2       \\ % 3
\texttt{s(h(s(a))) = s(s(a))}           & subst, 1       \\ % 4
\texttt{s(s(a)) = s(h(s(a)))}           & symm, 4        \\ % 5
\texttt{s(s(h(a))) = s(h(s(a)))}        & trans, 3, 5       % 6
\end{fitch}
\end{equation*}

\item % welke algebra K,L,M,N is geen model voor Spec

De algebra $\mathfrak{K}$ is geen model voor de specificatie $\texttt{Spec}$
omdat de vergelijking $\texttt{[E1]}$ hierin niet waar is. Kies bijvoorbeeld
de assignment $\theta$ met $\theta(x) = 3$. We zien dan dat
\begin{align*}
\bar \theta(\texttt{h(h(x))}) &= 3+2 \\
                              &= 5
\end{align*}
terwijl
\begin{align*}
\bar \theta(\texttt{x}) &= 3.
\end{align*}

\item % bekijk voor de overige of ze initieel zijn

We bekijken voor de overige algebra's of ze initiele modellen zijn voor $\texttt{Spec}$.

\paragraph{De algebra $\mathfrak{L}$}

Deze algebra is geen initieel model voor $\texttt{Spec}$, omdat er junk
is. Voor alle getallen ongelijk aan 0 geldt namelijk dat ze niet de
interpretatie zijn van een gesloten term.

Bovendien bevat deze algebra confusion, omdat iedere gesloten term
ge\"interpreteerd wordt als het getal 0. Dit betekent dat iedere vergelijking
van gesloten termen waar is in $\mathfrak{L}$, terwijl bijvoorbeeld
$\texttt{s(x) = x}$ niet afleidbaar is in $\texttt{Spec}$.

\paragraph{De algebra $\mathfrak{M}$}

%Deze algebra is een initieel model voor $\texttt{Spec}$.
Deze algebra is geen initieel model voor $\texttt{Spec}$, omdat er confusion
aanwezig is. De termen $\texttt{a}$ en $\texttt{h(a)}$ worden namelijk gelijk
ge\"interpreteerd, terwijl ze niet als gelijk kunnen worden afgeleid in de
specificatie $\texttt{Spec}$.

\paragraph{De algebra $\mathfrak{N}$}

Deze algebra is geen initieel model voor $\texttt{Spec}$, omdat er
confusion aanwezig is. Bijvoorbeeld de termen $\texttt{x}$ en $\texttt{h(x)}$
worden ge\"identificeerd, terwijl deze gelijkheid niet afleidbaar is in
$\texttt{Spec}$.

\end{enumerate}


\end{document}
