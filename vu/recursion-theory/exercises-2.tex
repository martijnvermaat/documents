\documentclass[a4paper,11pt]{article}
\usepackage[english]{babel}
\usepackage{a4,fullpage}
\usepackage{amsmath,amsfonts,amssymb}
\usepackage{amsthm}
\usepackage[T1]{fontenc}
\usepackage{lmodern} % Latin modern font family

\newtheorem*{lemma}{Lemma}
\newtheorem*{theorem}{Stelling}

% Sans-serif fonts
%\usepackage[T1experimental,lm]{sfmath} % http://dtrx.de/od/tex/sfmath.html
%\renewcommand{\familydefault}{\sfdefault}

% Some configuration for listings
\renewcommand{\labelenumi}{\arabic{enumi}.}
\renewcommand{\labelenumii}{(\alph{enumii})}

\newcounter{firstcounter}
\newcommand{\labelfirst}{(\roman{firstcounter})}
%\newcommand{\spacingfirst}{\usecounter{firstcounter}\setlength{\rightmargin}{\leftmargin}}
\newcommand{\spacingfirst}{\usecounter{firstcounter}}

\newcounter{secondcounter}
\newcommand{\labelsecond}{(\arabic{secondcounter})}
%\newcommand{\spacingsecond}{\usecounter{secondcounter}\setlength{\rightmargin}{\leftmargin}}
\newcommand{\spacingsecond}{\usecounter{secondcounter}}


\title{Recursion Theory (UvA 2008-2009)\\
\normalsize{Exercises Part 2 -- Martijn Vermaat (mvermaat@cs.vu.nl)}}

%\author{Martijn Vermaat (mvermaat@cs.vu.nl)}
%\date{Updated 5th December 2005}
\date{}


\begin{document}

\maketitle


\begin{enumerate}


\item % 1
\begin{quotation}
  $x \cdot y$ is primitive recursive.
\end{quotation}
\begin{equation*}
  \lambda (x, y). x \cdot y = P(\lambda x.0,
                                (\lambda (x, y). x + y) (e^3_1, e^3_2))
\end{equation*}


\item % 2
\begin{quotation}
  $x^y$ is primitive recursive.
\end{quotation}
\begin{equation*}
  \lambda (x, y). x^y = P(S \circ \lambda x.0,
                          (\lambda (x, y). x \cdot y) (e^3_1, e^3_2)) (e^2_2, e^2_1)
\end{equation*}


\item % 3
\begin{quotation}
  $x!$ is primitive recursive.
\end{quotation}
\begin{equation*}
  \lambda x.x! = P(S \circ \lambda x.0,
                   (\lambda (x, y). x \cdot y) (e^3_1, e^3_2)) (e^1_1, e^1_1)
\end{equation*}
From now on, we might write down our primitive recursive definitions in a more
informal way whenever it is obvious how it fits in the schemes (I)-(V), e.g.
\begin{align*}
  0!     &= S \circ \lambda x.0\\
  (x+1)! &=


\end{enumerate}


\end{document}
