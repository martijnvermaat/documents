\documentclass[a4paper,11pt]{article}
\usepackage[english]{babel}
\usepackage{a4,fullpage}
\usepackage{amsmath,amsfonts,amssymb}
\usepackage{amsthm}
\usepackage[T1]{fontenc}
\usepackage{lmodern} % Latin modern font family

\newtheorem*{lemma}{Lemma}
\newtheorem*{theorem}{Stelling}

% Sans-serif fonts
%\usepackage[T1experimental,lm]{sfmath} % http://dtrx.de/od/tex/sfmath.html
%\renewcommand{\familydefault}{\sfdefault}

% Some configuration for listings
\renewcommand{\labelenumi}{\arabic{enumi}.}
\renewcommand{\labelenumii}{(\alph{enumii})}

\newcounter{firstcounter}
\newcommand{\labelfirst}{(\roman{firstcounter})}
%\newcommand{\spacingfirst}{\usecounter{firstcounter}\setlength{\rightmargin}{\leftmargin}}
\newcommand{\spacingfirst}{\usecounter{firstcounter}}

\newcounter{secondcounter}
\newcommand{\labelsecond}{(\arabic{secondcounter})}
%\newcommand{\spacingsecond}{\usecounter{secondcounter}\setlength{\rightmargin}{\leftmargin}}
\newcommand{\spacingsecond}{\usecounter{secondcounter}}


\title{Recursion Theory (UvA 2008-2009)\\
\normalsize{Exercises Part 1}}

\author{Martijn Vermaat (mvermaat@cs.vu.nl)}
%\date{Updated 5th December 2005}


\begin{document}

\maketitle


\begin{enumerate}


\item % 1
\begin{quotation}
  $Sum$ defines a function: $Sum(z, x, y)$ and $Sum(z', x, y) \Rightarrow z = z'$.
\end{quotation}

\begin{proof}
Informal, by induction on $y$:
\begin{description}
\item{$y = 0$:} Assumptions are $Sum(z, x, 0)$ and $Sum(z', x, 0)$.
  $Sum(n, n', 0)$ can only be if $n = n'$, hence $z = x = z'$.
\item{$y = Sn$:} Assumptions are $Sum(z, x, Sn)$ and $Sum(z', x, Sn)$.
  IH gives that $Sum(p, m, n)$ and $Sum(p', m, n)$ imply $p = p'$.
  From the assumptions it follows that $Sum(z, x, n)$ and $Sum(z', x, n)$,
  so $z = z'$.\qedhere
\end{description}
\end{proof}


\item % 2
\begin{quotation}
  $1 + 2 + \ldots + k = \frac{1}{2} k (k + 1)$.
\end{quotation}

\begin{proof}
By induction on $k$:
\begin{description}
\item{$k = 0$:}

  $0 = \frac{1}{2} 0 (0 + 1)$.
\item{$k = i + 1$:}

  IH is $1 + 2 + \ldots + i = \frac{1}{2} i (i + 1)$.
  $1 + 2 + \ldots + i + (i + 1) = \frac{1}{2} i (i + 1) + (i + 1)$.
\end{description}
\end{proof}


\end{enumerate}


\end{document}
