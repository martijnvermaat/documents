\chapter{Evaluatie voorstel bewegwijzering}

We evalueren de door ons voorgestelde nieuwe bewegwijzering voor de RAI. Dit gebeurt net als bij de evaluaties van de bestaande systemen middels een cognitive walkthrough en een analyse op basis van cognitive dimensions. Tevens presenteren we voor deze nieuwe bewegwijzering een scenario.


\section{Cognitive walkthrough}

We gebruiken de methode van cognitive walkthrough, zoals deze uitgelegd is in \ref{sectie:cw_methode}. Wederom passen we de methode toe op de stakeholder en taak zoals uiteengezet in \ref{sectie:cw_toepassing}.

We bespreken de volgende acties die nodig zijn om het doel te bereiken in de RAI met de door ons voorgestelde nieuwe bewegwijzering:


\subsubsection{Huidige locatie bepalen}

Het bepalen van de huidige locatie is erg gemakkelijk binnen de hallen, omdat daar de vanen met halnummers hangen. Er is daaruit echter niet af te leiden waar je je ten opzichte van de rest van het complex bevindt.

Bevind je je niet in een hal, dan zul je je toevlucht moeten zoeken tot een plattegrond zoals die op enkele strategische punten hangt. Je kunt aan de pijl op de plattegrond zien waar je je bevindt.


\subsubsection{Locatie van doel bepalen}

Is het doel dichtbij, dan is het direct herkenbaar aan een icoon, halnummer, of naam van evenement. Moet je verder weg zijn, dan is uit de bewegwijzering alleen op te maken welke richting je moet lopen, maar niet waar je doel zich precies bevindt. Behalve wanneer je een plattegrond raadpleegt uiteraard, daar kun je direct de locatie van je doel op bepalen.


\subsubsection{Route naar doel bepalen}

In het door ons voorgestelde systeem is het eigenlijk niet mogelijk direct de volledige route naar het doel te bepalen, maar het eerste deel van de route is altijd duidelijk.

Wanneer het doel dichtbij is, kun je het direct herkennen en is het bepalen van de route triviaal. Wanneer het doel verder weg is, krijg je steeds genoeg informatie om door te kunnen lopen. Wanneer de route verder niet duidelijk is, kom je weer een onderdeel van de bewegwijzering tegen. Je hoeft zelf dus absoluut niet na te denken en je loopt automatisch altijd de kortste route.

Uitzondering is weer het gebruiken van een plattegrond. In dat geval kun je je route uiteraard wel direct zelf bepalen.


\subsubsection{Bepalen of het doel bereikt is}

Iedere ingang van de hallen en faciliteiten is te herkennen; de hallen aan het halnummer en evenement naam, de faciliteiten aan een icoon. Hierover kan dus nooit verwarring bestaan.


\section{Cognitive dimensions}

Verschillende cognitieve dimensies projecteren we op de door ons voorgestelde nieuwe bewegwijzering van de RAI Amsterdam. We noemen de positieve en negatieve punten van de systemen. Een korte uitleg van de gebruikte demensies is te lezen in \ref{sectie:cd_demensies}.


\subsubsection{Viscosity}

Het nieuwe systeem in de RAI heeft borden op een duidelijke plek hangen de bezoeker kan snel de bewegwijzeringsborden zien. Door de digitale borden kan je ook zien welke activiteit waar is, zo heb je meteen door waar je moet zijn. De digitale borden zijn voor de beheerder ook makkelijk bij te houden en de kosten van het bijhouden bestaat alleen uit tijd en energie.

Verder zijn er kaarten aan de muren waar mensen een directe route kunnen uitstippelen naar de verschillende hallen. Er hoeft in principe verder niemand geraadpleegd te worden over de route. Als een evenement erg groot is worden er over het algemeen ook plattegronden uitgedeeld door de organisatie van het evenement bij binnenkomst, zodat de bezoekers meteen een overzicht krijgen van alle hallen.


\subsubsection{Hidden dependencies}

Hier heb je weer het geval met iconen van de toiletten. Er zijn een aantal bordjes die wijzen naar de toiletten. Maar als je zo een bordje weghaalt van de toilet heeft de bewegwijzering erheen ook geen nut. Hetzelfde is dan ook met EHBO en al die andere voorziening gerichte borden met iconen.


\subsubsection{Premature commitment}

Dit is met de wegwijzering met de pijlen naar hallen, je komt door het volgen van de borden steeds dichter bij de hal. Je hebt alleen niet zo heel erg het gevoel van vordering want de borden worden niet specifieker in aanduiding. Wel weet je dat je op het goede spoor zit, je komt namelijk de borden met de juiste aanduiding tegen als je goed zit. Als je de vooruitgang echt wil weten kan je ook op een kaart kijken die aan de muur hangt en zien hoever je van je doel bent en de hele route uitstippelen.


\subsubsection{Abstractions}

Hier heb je hetzelfde als bij de RAI (zie \ref{sectie:cd_rai_abstractions}), alleen heb je nu ook borden die naar het doel wijzen en niet alleen de plattegrond.


\subsubsection{Secondary notation}

Door alles in dezelfde kleur en stijl te doen is het duidelijk dat het gaat om een wegwijzeringsbord. De kaarten zijn verlicht en hebben een opvallende kleur zodat ze opvallen.


\subsubsection{Visibility \& juxtaposibility}

Door alle bewegwijzeringsborden op een duidelijke plaatst te hangen zijn deze makkelijker te vinden. Door ook dezelfde kleur aan te houden weet je meteen dat het een bewegwijzeringsbord is. De digitale borden geven extra informatie weer over de hal en dat valt hoe dan ook wel op door het licht.


\section{Emoties, creativiteit en betekenis}

We bekijken de door ons voorgestelde bewegwijzering nog aan de hand van deze drie kenmerken:


\subsubsection{Emoties}

Het bewegwijzeringssysteem komt goed over, je ziet iedere keer of je de goede richting op gaat ook heb je een extra hulpmiddel (de plattegronden) waarmee je een hele route kan uitstippelen. Het komt verder professioneel en zakelijk over door de digitale borden en het feit dat de alle borden dezelfde stijl hebben.


\subsubsection{Creativiteit}

Het systeem zelf is zakelijk en niet heel creatief. Wel is het systeem er duidelijk en goed doordacht en is het ontwerp wel mooi.


\subsubsection{Betekenis}

De betekenis van de borden is erg duidelijk, Door de extra informatie van de digitaleborden weet je meteen waar je moet zijn. Verder weet je meteen wel wat er bedoelt wordt met de borden. Als je zelf de borden niet meer overzichtelijk vind ondanks de duidelijkheid kan je altijd nog op de kaart kijken die je overal wel aan de muur kan vinden. Zo kan iedereen zijn weg vinden.


\section{Scenario}

Om een idee te krijgen hoe het systeem in de praktijk gebruikt zou kunnen worden schrijven we een scenario. Een willekeurige persoon voert enkele acties uit en we letten daarbij op hoe hij de door ons voorgestelde bewegwijzering gebruikt.


\paragraph{Scenario} Meneer Kievit is naar de \emph{autoRAI} en is opzoek naar de \emph{BMW} afdeling. Hij ziet bij de ingang een overzicht van alle activiteiten en vindt daar BMW, deze is in hal 9. Verder ziet hij ook gelijk de richting waar hal 9 is. Hij gaat dus naar links toe en ziet na 100 meter een bord met hal 9 erop. Bij de ingang van een hal daaronder ziet hij dan digitaal weergegeven ``BMW''. Hij gaat de hal binnen en kijkt rond.

Meneer Kievit is nu al drie uur op de beurs en bevindt zich nog steeds in hal 9 (hij weet dit door de vaan die midden in de hal hangt). Hij heeft nog niks gedronken en heeft dus erge dorst en besluit om wat te gaan drinken. Hij gaat opzoek naar een drinkgelegenheid. Hij zoekt naar een aanwijzing voor een drinkgelegenheid, hij loopt naar de uitgang van de hal en ziet daar een bordje met een kopje (koffie) erop en ziet dat als een duidelijke aanwijzing voor een plek om iets te drinken. Hij volgt het bordje en gaat dan de gang door.

Hij komt nu hal 10 binnen en ziet daar een bordje dat hij links moet gaan om bij een drinkgelegenheid te komen. Vervolgens ziet hij weer net zo'n bordje als hij bij de hoek komt van de hal en ziet dat hij naar rechts moet. Na drie bordjes ziet hij dan een drinkgelegenheid. Hij bestelt een kop koffie en wil een slok nemen, maar het kopje blijft in zijn snor haken en de koffie valt over hem heen. De koffie is nog gloeiend heet en hij heeft zich verbrand. Hij gaat snel opzoek naar een EHBO post.

Hij ziet naast zich aan de muur een kaart met een overzicht van de beurshallen, kijkt daar snel op en ziet een teken met een kruisje: een algemeen teken voor een EHBO post. Hij ziet dat hij naar een andere hal moet en gaat naar de juiste uitgang en ziet daar al een bordje met dezelfde kruis. Hij volgt dit bordje en moet dus naar de gang en dan links en ziet dan de EHBO post ook angegeven door dezelfde kruis. Hij wordt daar geholpen. Nu heeft hij genoeg van de beurs en wil weg, hij ziet al meteen een bordje met het bekende uitgangsteken en gaat zo de beurs uit naar huis.


\section{Conclusie}

Bij het ontwerpen van een bewegwijzeringssysteem moet een keuze gemaakt worden. Geef je de gebruikers het volledige overzicht en laat je ze zelf de route bepalen? Of houd je de informatie beperkt tot dat wat nodig is en kauw je de route voor voor de gebuiker? Hier ligt het belangrijkste verschil tussen de huidige bewegwijzering in de RAI en de door ons voorgestelde bewegwijzering.

Het huidige systeem geeft nauwelijks richtingen aan, maar biedt erg veel plattegronden waarmee de gebruiker zelf de richting kan bepalen. Dit gaat soms mis, bijvoorbeeld in het geval een doorgang tijdelijk niet beschikbaar is, of wanneer de verdieping niet is weergegeven op de plattegrond. Maar in de meeste gevallen werkt deze oplossing prima. Het grote nadeel is echter dat het bepalen van je eigen locatie, het bepalen van de locatie van je doel en het uitstippelen van een route daartussen tijd kost. Dit wordt extra vervelend op een drukbezochte beurs, waar de mensen zich verdringen rond de plattegronden.

Ons leek het daarom een goed idee af te stappen van de afhankelijkheid van plattegronden. Ze zijn weliswaar zeer waardevol als aanvullende bewegwijzering, voor de gebruiker die meer wil weten over het complex, maar als primaire bewegwijzering zijn ze naar onze mening minder geschikt.

Daarom hebben wij ervoor gekozen veel gebruik te maken van borden waarop richtingen aangegeven worden. De gebruiker ziet de borden, kiest de juiste, leest de richting af en loopt door. Een totaaloverzicht ontbreekt hierdoor, maar het is snel, makkelijk en effectief.

Een ander punt waarop we het de gebruiker gemakkelijker maken is de naamgeving van de hallen. We leren hier van de filosifie achter de bewegwijzering in AHOY Rotterdam: De gebruiker komt voor een evenement en heeft dus meer aan de naam van het evenement dan aan een halnummer.