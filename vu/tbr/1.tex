\chapter{Domein van onderzoek}

Ons onderzoek zal zich richten op de RAI, het grote beurscomplex in Amsterdam Oud-Zuid. We onderzoeken hier de bewegwijzering binnen en rond de verschillende hallen die als doel heeft de weg te kunnen vinden binnen het complex.

We laten de bewegwijzering die als doel heeft de weg te kunnen vinden in het stadsdeel of zelfs binnen heel Amsterdam buiten beschouwing. Dit zijn bijvoorbeeld bordjes binnen de hallen die het station of de tramhalte wijzen, of bordjes in het station die de richting aangeven van de beursgebouwen.


\section{Situaties en activiteiten}

Het complex wordt van dag tot dag voor verschillende doeleinden gebruikt. Zo zijn er regelmatig grote evenementen in de hallen, met als bekendste voorbeelden de \emph{AutoRAI}, de \emph{FietsRAI} en de \emph{Hiswa}. Andere wisselende activiteiten zijn symposia en congressen.

Daarnaast is er nog een veelheid aan permanente activiteiten te vinden in het complex, zoals restaurants, toiletten en parkeerplaatsen.

Bij iedere activiteit hoort een scala van \emph{meta-activiteiten}, zoals kaartverkoop, schoonmaken, opbouwen en afbreken van stands en informatieverstrekking. Dit alles wordt in veel gevallen ondersteund door de aan- en afvoer van goederen.


\section{De rol van representaties}

De bewegwijzering in de RAI is op de eerste plaats van groot belang bij grote evenementen en andere wisselende activiteiten. Een aanzienlijk deel van deze bewegwijzering zal per activiteit van inhoud verschillen. Er is op bijvoorbeeld een computerbeurs bewegwijzering nodig om de verschillende stands te kunnen vinden. Op het moment dat er in deze hal(len) een nieuw evenement gehouden wordt is deze bewegwijzering niet nuttig meer.
Maar er is ook een deel dat permanent aanwezig moet zijn in het complex. Denk hierbij vooraal aan bewegwijzering naar de verschillende hallen, in- en uitgangen en andere delen van de RAI.

Voor de permanente activiteiten in de RAI is ook bewegwijzering nodig. Iedereen moet de restaurants, toiletten en parkeerplaatsen kunnen vinden en het liefst ook vanaf iedere locatie binnen de RAI. Hierbij moeten we bedenken dat ieder restaurant en iedere parkeerplaats altijd gevonden moet kunnen worden, maar dat in het algemeen bewegwijzering naar een toiletruimte (de dichtstbijzijnde) voldoende zal zijn.

Ook voor de meta-activiteiten is bewegwijzering nodig. Mensen moeten weten waar de kaartjes gekocht dienen te worden, schoonmakers moeten de weg kunnen vinden, aan- en afvoerders van goederen moeten weten waar ze moeten zijn, etc.


\section{Doel van onderzoek}

Het doel van ons onderzoek is het ontwerpen van een nieuwe representatie voor bewegwijzering binnen multifunctionele gebouwen. Door te kijken naar bestaande representaties stellen we verbeteringen voor. De door ons ontworpen representatie staat voor een electronische systeem.
