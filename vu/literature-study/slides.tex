\documentclass[notheorems]{beamer}

\usepackage[english]{babel}

\usepackage{amsmath}
\usepackage{amssymb}

\usepackage[T1]{fontenc}
\usepackage{ae,aecompl}

\usepackage{beamerthemesplit}

\setbeamertemplate{background canvas}[vertical shading][bottom=red!10,top=blue!10]
\setbeamertemplate{navigation symbols}{}
\setbeamertemplate{headline}{}
\usetheme{Warsaw}
\useinnertheme{rectangles}

\colorlet{darkred}{red!80!black}
\colorlet{darkblue}{blue!80!black}
\colorlet{darkgreen}{green!80!black}


\title{Names and Numbers in Binding}

\author{Martijn Vermaat}
\institute{mvermaat@cs.vu.nl\\
http://www.cs.vu.nl/\~{}mvermaat/}
\date{Literature Study\\
March 13, 2009}


\begin{document}


\frame{\titlepage}


\frame{

  \frametitle{Names and Numbers in Binding}

  \tableofcontents

}


\section{Mechanical Reasoning about Languages}


\frame{

  \frametitle{Mechanical Reasoning}

  Shift from on-paper reasoning to mechanical reasoning:

  \begin{itemize}
    \item History of on-paper proofs and ideas
    \item Informal mechanical implementations of ideas
    \item Add a scale increase and we have a gap
  \end{itemize}

  \uncover<2->{
    \begin{block}{Goal}
      Let's make rigorous mechanical reasoning possible.
    \end{block}
  }

}


\frame{

  \frametitle{Reasoning about Languages}

  \begin{block}{Reasoning about languages}
    Often not intrinsically hard, but cumbersome in a mechanical setting.
  \end{block}

  Why?
  \uncover<2->{
    \begin{itemize}
    \item Most languages have a notion of binding
    \item Bindings and bound variables are easy on paper, hard on a computer
    \end{itemize}
  }

}


\frame{

  \frametitle{Mechanical Reasoning about Languages}

  So we want to
  \begin{quote}
    Reason about terms with bindings in tools like Coq, in a way that is close
    to the on-paper way.
  \end{quote}

  \uncover<2->{
    We need a representation for binders and variables.
  }

}


\section{Representing Bindings}


\frame{

  \frametitle{Names and Numbers in Binding}

  \tableofcontents[currentsection]

}


\frame{

  \frametitle{Representing Bindings}

  Classical problems related to binders and variables:

  \begin{itemize}
    \item $\alpha$-conversion
    \item substitution
  \end{itemize}

  \uncover<2->{
    Let's look at some representations.
  }

}


\frame{

  \frametitle{Traditional Representation}

  \begin{block}{Running example}
    Substitution in untyped $\lambda$-calculus
  \end{block}

  \uncover<2->{
    Traditional representation with named variables:
    \begin{align*}
      M ::=             &\; x
      && \text{variable} \\
      \llap{\textbar\:} &\; \lambda x.M
      && \text{abstraction} \\
      \llap{\textbar\:} &\; M \; M
      && \text{application} \\
    \end{align*}

  }

}


\frame{

  \frametitle{Traditional Representation}

  \begin{itemize}

    \item $\alpha$-equivalent terms are routinely identified

    \item Substitution $M[N/x]$:
      \begin{align*}
        x[N/x]              &= N\\
        y[N/x]              &= y                  && \text{$y \neq x$} \\
        (\lambda y.M')[N/x] &= \lambda y.M'[N/x]  && \text{$y \neq x$ and $y$ not free in $N$} \\ % TODO: fix layout
        (M_1 \; M_2)[N/x]   &= M_1[N/x] \; M_2[N/x]
      \end{align*}

    \item Now implement this
      % alpha-conversion is very hard
      % side-conditions are declarative
      % (\x.M)[N/x] not even defined
      % use variable convention, or a more general substitution (next slide)

  \end{itemize}

}


\frame{

  \frametitle{Simple Substitution}

  Use $\alpha$-conversion to rename bound variables and define
  substituting $N$ for $x$ in $M$ inductively$^*$ on M:

  \begin{align*}
    x[N/x]              &= \text{$N$ if $y = x$, $y$ otherwise}\\
    (\lambda y.M')[N/x] &= \lambda z.M'[z/y][N/x]  && \text{$z$ not free in $N, M'$} \\ % TODO: fix layout
    (M_1 \; M_2)[N/x]   &= M_1[N/x] \; M_2[N/x]
  \end{align*}

  % Note that M'[z/y] is not a subterm of M

  \uncover<2->{
    Already difficult enough to read, but just what we would intuitively do.
    So on paper, we can get by with some handwaving.
  }

}


\frame{

  \frametitle{Simple Substitution}

  In Coq:

}


\section{Conclusions}


\frame{

  \frametitle{Names and Numbers in Binding}

  \tableofcontents[currentsection]

}


\frame{

  \frametitle{Conclusion}

  \begin{itemize}
    \item One conclusion
    \item Second conclusion
    \item Conclusion of conclusions
  \end{itemize}

}


\begin{frame}

  \frametitle{Questions and Further Reading}

  Questions?\\[4em]

  \begin{block}{Further Reading}
    \begin{itemize}
      \item Plotkin, 1975: {\em Call-by-name, Call-by-value and the lambda-calculus}
      \item Danvy and Filinski, 1992: {\em Representing control: a study of the CPS transformation}
      \item Appel, 1992: {\em Compiling with Continuations}
      \item Compcert: \texttt{http://pauillac.inria.fr/\~{}xleroy/compcert/}
      \item Lambda Tamer: \texttt{http://ltamer.sourceforge.net/}
    \end{itemize}
  \end{block}

\end{frame}


\end{document}
