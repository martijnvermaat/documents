\documentclass[a4paper,11pt]{article}
\usepackage[english]{babel}


% TODO:
% citations
% variable convention
% make it clear that there are more ways to program things in coq
% check if naming is consistent (term, variable, index, etc names)
% check other locally nameless approaches (do they use freshening?)
% natbib
% coqdoc
% 'dual' vs 'analoguous'


\usepackage{amsmath}
\usepackage{amsfonts}
\usepackage{amssymb}

\usepackage{tgtermes}
\usepackage{qtxmath}

\renewcommand{\ttdefault}{txtt}

\usepackage[scaled=0.95]{helvet}

\usepackage{url}
\makeatletter
\def\url@leostyle{%
  \@ifundefined{selectfont}{\def\UrlFont{\sf}}{\def\UrlFont{\small\ttfamily}}}
\makeatother
\urlstyle{leo}

\usepackage[T1]{fontenc}

\usepackage{listings}

\lstdefinelanguage{Coq}{
  mathescape=true,
  texcl=false,
  keywords={Require, Declare},
  morekeywords={forall, exists, with, match, let, in, if, then, else},
  morekeywords={
    Variable, Inductive, CoInductive, Fixpoint, CoFixpoint, Definition,
    Lemma, Theorem, Proof, Axiom, Local, Save, Grammar, Syntax, Intro, Trivial,
    Qed, Intros, Symmetry, Simpl, Rewrite, Apply, Elim, Assumption, Left,
    Cut, Case, Auto, Unfold, Exact, Right, Defined, Function, end,
    struct, measure},
  morekeywords={Section, Module, End},
  %emph={[1]Type, Set, nat, bool}, emphstyle={[1]\textit},
  comment=[s]{(*}{*)},
  showstringspaces=false,
}%[keywords,comments,strings]%

\lstset{
  numbers=none,
  basicstyle=\footnotesize\ttfamily,
  frame=,
  language=Coq,
  xleftmargin=0.8em,
  xrightmargin=0em,
  aboveskip=1em,
  belowskip=1em
}

% Add subsection to listing numbering
%\renewcommand\thelstlisting{\thesubsection .\arabic{lstlisting}}

\usepackage{xspace}
\newcommand{\name}[1]{\textsc{#1}\xspace}
\def\Coq{\name{Coq}}
\def\POPLmark{\name{POPLmark}}
\def\POPL{POPL}

% Must be last in preamble
\usepackage[
  pdftex,
  colorlinks,
  citecolor=black,
  filecolor=black,
  linkcolor=black,
  urlcolor=black,
  pdfauthor={Martijn Vermaat},
  pdftitle={Mechanized Reasoning about Languages with Binding},
  pdfsubject={TODO},
  pdfkeywords={lambda calculus, binding, types, program verification, poplmark}
]{hyperref}
\usepackage[figure]{hypcap}


\title{Mechanized Reasoning about Languages~with~Binding}

\author{Martijn Vermaat}
\date{}


\begin{document}

\maketitle


\begin{abstract}
  TODO.
\end{abstract}


\section{Introduction}\label{sec:introduction}

Theory and practice in programming language research are in large part
two separate worlds.
Language designers often take a pragmatic approach, where the focus is
on the observable behaviour of implementations and not much attention
is paid to prove properties of the language in a formal way.
Research on the theory of programming languages has a long tradition and
provides us with a large base of techniques and results, but is mostly
isolated in academic settings.
Applying formal reasoning to current programming language development
has the promise of rigorously defined practical languages with provable
properties, but is unfortunately not commonplace.

Many tools have become available to aid formal reasoning in a mechanical
way, such as proof assistants and theorem provers.
Researchers can use these tools to obtain machine-checked and reusable
proofs while language designers can use them to formally define languages,
clearing the way for formal reasoning.
There are several issues holding back large-scale application of such
tools, both technical and non-technical. In this literature study, we
focus on one specific technical aspect.

A major difficulty in formal reasoning about languages are {\em bindings}
and {\em bound variables}.
Traditionally, named variables are used in the representation of terms
and reasoning is done modulo $\alpha$-equivalence.
While this approach yields easy to read and sufficiently precise
reasoning on paper,
mechanical reasoning with named variables is hard:
identification of $\alpha$-equivalent terms has to be made precise and
substitution requires renaming of bound variables in order to prevent
accidental binding of free variables.

Several approaches exist to remedy these problems.
Some of them try to make it easier to work with named variables, while
others propose alternate representations for bindings and variables.
We study some of these approaches in the context of mechanical
reasoning, focussing on concrete representations for terms using
{\em names} and {\em numbers} for variables.
The approaches are illustrated with applications using the \Coq proof
assistant \cite{coq-refman-09,bertot-casteran-04}.


\section{Representing Bindings}\label{sec:representing}

The most common problems associated with bindings are {\em $\alpha$-equivalence}
and {\em $\alpha$-conversion}.
These are actually notions related to the traditional representation where
names are used for variables, and may or may not be relevant issues with
other representations.
Some representations however, have analogue notions, for example {\em lifting}
of so-called {\em de Bruijn indices} is dual to $\alpha$-conversion.

We use untyped pure $\lambda$-calculus \cite{Barendregt-84} to introduce
three concrete representations for terms using names and numbers for
variables. To illustrate, the representations are implemented in the
\Coq proof assistant, where we make use of a type for names on which
equality is decidable.
\begin{lstlisting}
Parameter name : Set.
Parameter eq_name : forall (x y : name), {x = y} + {x <> y}.
\end{lstlisting}
We also assume a way to generate fresh names with \lstinline{fresh_name},
which returns a name that is not in a given list of names.
\begin{lstlisting}
Parameter fresh_name : (list name) -> name.
\end{lstlisting}


\subsection{Named Variables}

Untyped $\lambda$-calculus has function application and abstraction over
variables. Using named variables, its terms can be represented with the
grammar
\begin{align*}
  M ::=             &\; x
  && \text{variable} \\
  \llap{\textbar\:} &\; \lambda x. \; M
  && \text{abstraction} \\
  \llap{\textbar\:} &\; M \; M
  && \text{application}
\end{align*}
where $x$ ranges over a countably infinite set of names.
An abstraction introduces a name, occurences of which are said to be
bound by it in its {\em scope}. The scope of an abstraction $\lambda
x.M$ is the body $M$ minus any subterm of $M$ that is an abstraction
over the same name $x$.

\subsubsection*{$\alpha$-equivalence and $\alpha$-conversion}

$\alpha$-conversion is the process of renaming bound variables, such
that abstractions bind exactly the variables they did originally.
Two terms are $\alpha$-equivalent if they are the same modulo
$\alpha$-conversion.
For example, the following are $\alpha$-equivalent pairs of terms:
\begin{align*}
  \lambda x&. \; x & (\lambda z&. \; z \; x) \; z & \lambda x&. \; \lambda y. \; x \; (y \; z)\\
  \lambda z&. \; z & (\lambda y&. \; y \; x) \; z & \lambda y&. \; \lambda x. \; y \; (x \; z)
\end{align*}
When reasoning about terms, it is often needed to identify $\alpha$-equivalent
terms.
This can result in a lot of extra work if the reasoning has to be rigorous,
as is the case in a tool like a proof assistant.

Manipulation of terms can result in accidental binding of variables that
where previously not bound or bound by another abstraction.
Consider substituting $x$ for $y$ in $\lambda x. \; y \; x$.
The free variable $x$ becomes bound by the abstraction over $x$ and looses
its possible contextual meaning.
This is a phenomenon that should be avoided.
A possible solution is applying $\alpha$-conversion to rename the bound
variable $x$ to a fresh name, say $z$, before performing the substitution.
Substituting $x$ for $y$ in $\lambda z. \; y \; z$ does not bind $x$.

Barendregt's variable convention is a way to avoid accidental binding of
variables by maintaining the invariant that in any context, all bound
variables in a term are chosen to be different from the free variables.
It is however not straightforward to implement this invariant directly in
mechanical developments and its effects are therefore often mimicked
by other technicalities (e.g., explicit renaming or using disjoint
sets for bound and free variables).

\subsubsection*{Substitution}

Let us consider substitution some more.
Substituting $N$ for $x$ in $M$ is usually defined as
\begin{align*}
  x[N/x]                 &= N\\
  y[N/x]                 &= y                      && \text{$x \neq y$} \\
  (\lambda y.\; M')[N/x] &= \lambda y. \; M'[N/x]  && \text{$x \neq y$ and $y$ not free in $N$} \\
  (M_1 \; M_2)[N/x]      &= M_1[N/x] \; M_2[N/x]
\end{align*}
which is cannot be directly implemented as an operation without using the
variable convention, because of the declarative side condition in the
case of abstraction.
Of course, $\alpha$-conversion can be used to eliminate the undefined
case, yielding
\begin{align*}
  x[N/x]              &= N\\
  y[N/x]              &= y                       && \text{$x \neq y$} \\
  (\lambda y.M')[N/x] &= \lambda z.M'[z/y][N/x]  && \text{$z$ not free in $N\!, M'$} \\
  (M_1 \; M_2)[N/x]   &= M_1[N/x] \; M_2[N/x]
\end{align*}
Note that this is not a {\em structurally} recursive definition, since
$M'[z/y]$ is not generally a subterm of $\lambda y.M'$.
Structural recursion is prefered over recursion on some other measure
(e.g. term size), especially in a tool like \Coq where structural
induction principles are automatically generated for inductive
definitions and structurally recursive functions don't need explicit
termination proofs.

\subsubsection*{Named Variables in \Coq}

Using named variables, $\lambda$-calculus terms can be represented in the
\Coq proof assistant with the inductive type \lstinline{term}. We also
define simple renaming of variables, to be used in other definitions.
\begin{lstlisting}
Inductive term : Set :=
  | Var : name -> term
  | Abs : name -> term -> term
  | App : term -> term -> term.

Fixpoint rename (n m : name) (t : term) {struct t} : term :=
  match t with
  | Var x =>   if eq_name x n then Var m else t
  | Abs x b => Abs (if eq_name x n then m else x) (rename n m b)
  | App f a => App (rename n m f) (rename n m a)
  end.
\end{lstlisting}

As a first try in translating the substitution operation to \Coq, the
following definition will fail because, as noted above, it is not
structurally recursive on \lstinline{t}.
\begin{lstlisting}
Fixpoint subst_ill (s : term) (n : name) (t : term)
    {measure size t} : term :=
  match t with
  | Var x   => if eq_name x n then s else t
  | Abs x b => let z := fresh_name (n :: (free_vars s)
                                         ++ (free_vars b))
               in
               Abs z (subst_ill s n (rename x z b))
  | App f a => App (subst_ill s n f) (subst_ill s n a)
  end.
\end{lstlisting}
Note that for simplicity we rename every abstraction. In practice,
this could of course be improved.

A solution to the failing definition \lstinline{subst_ill} is to use
the term size as a well-founded recursion
measure. This definition of \lstinline{subst} requires us to
additionally provide three short proofs for \Coq to be assured of its
termination.
\begin{lstlisting}
Fixpoint size (t : term) : nat :=
  match t with
  | Var _   => 0
  | Abs _ b => S (size b)
  | App f a => S (size f + size a)
  end.

Lemma size_rename : forall n m t, size (rename n m t) = size t.
Proof.
induction t as [x | |]; simpl;
[destruct (eq_name x n); reflexivity | congruence | congruence].
Qed.

Function subst (s : term) (n : name) (t : term)
    {measure size t} : term :=
  match t with
  | Var x   => if eq_name x n then s else t
  | Abs x b => let z := fresh_name (n :: (free_vars s)
                                         ++ (free_vars b))
               in
               Abs z (subst s n (rename x z b))
  | App f a => App (subst s n f) (subst s n a)
  end.
intros. rewrite size_rename. auto.
intros. simpl. destruct f; omega.
intros. simpl. destruct f; omega.
Defined.
\end{lstlisting}

\subsubsection*{Simultaneous Substitution}

In search of a structural recursive substitution operation, Stoughton
\cite{Stoughton-88} suggests the {\em simultaneous substitution} $M \sigma$
\begin{align*}
  x \sigma              &= \sigma x\\
  (\lambda x.M') \sigma &= \lambda y.(M' \; \sigma[y/x])  && \text{$y$ not free in $M', \sigma$}\\
  (M_1 \; M_2) \sigma   &= M_1 \sigma \; M_2 \sigma
\end{align*}
where
\begin{equation*}
  \sigma[N/y] \; x =
  \begin{cases}
    N        & \text{if $x = y$,}\\
    \sigma x & \text{otherwise}
  \end{cases}
\end{equation*}

Substituting $N$ for $x$ in $M$ is now $M \; \iota [N/x]$ with $\iota$
the identity substitution.
This structurally recursive operation is easily implemented in \Coq, as
shown by the following definitions.
\begin{lstlisting}
Fixpoint apply_subst (l : list (term*name)) (n : name)
    {struct l} : term :=
  match l with
  | nil        => Var n
  | (t, x)::l' => if eq_name x n then t else apply_subst l' n
  end.

Fixpoint sim_subst (l : list (term*name)) (t : term)
    {struct t} : term :=
  match t with
  | Var x =>   apply_subst l x
  | Abs x b => let z := fresh_name ((free_vars_sub l)
                                    ++ (free_vars b))
               in
               Abs z (sim_subst ((Var z, x) :: l) b)
  | App f a => App (sim_subst l f) (sim_subst l a)
  end.
\end{lstlisting}
Simple substitution can now be written as a special case of
simultaneous substitution.
\begin{lstlisting}
Definition subst' (s : term) (n : name) (t : term) : term :=
  sim_subst ((s, n) :: nil) t.
\end{lstlisting}

\subsubsection*{Summary}

The above examples illustrate that $\alpha$-conversion to prevent
accidental binding of variables is not a trivial issue in an automated
setting.
The assumption of being able to generate fresh names might be troublesome
and standard operations such as simple substitution are not structurally
recursive.

Perhaps an even greater problem is that of $\alpha$-equivalence, depending
on the task at hand.
Terms can be treated as their $\alpha$-equivalence class, properties of
terms with bound variables can sometimes be parameterized by their names,
or terms can be translated to some canonically named variant, to name just
a few techniques to overcome this problem.
Most of these require a lot of tedious work that distracts from the actual
task at hand.

An obvious advantage of using a representation with named variables is
that it is the same as the usual on-paper notation.
It can therefore be read by anyone familiar with traditional work and there
is no need to justify the relation between the original object language and
the way it is represented.


\subsection{de Bruijn Indices}

As an alternative representation for $\lambda$-calculus terms, de Bruijn
\cite{deBruijn-72} proposes a representation using natural numbers instead
of names.
A variable is represented by the number of binders between itself and its
abstraction (called a de Bruijn index), yielding the grammar
\begin{align*}
  M ::=             &\; n
  && \text{variable} \\
  \llap{\textbar\:} &\; \lambda . \; M
  && \text{abstraction} \\
  \llap{\textbar\:} &\; M \; M
  && \text{application}
\end{align*}
where $n$ ranges over the natural numbers.
The following are some example terms in the traditional representation
(top) and in the representation with de Bruijn indices (bottom):
\begin{align*}
  \lambda x&. \; x & (\lambda z&. \; z \; x) \; z & \lambda x&. \; \lambda y. \; x \; (y \; z)\\
  \lambda &. \; 0  & (\lambda &. \; 0 \; 1) \; 2  & \lambda &. \; \lambda . \; 1 \; (0 \; 2)
\end{align*}
A major advantage of using de Bruijn indices in machine-assisted
reasoning is that $\alpha$-equivalence is just term equality.

\subsubsection*{Lifting}

Abstractions no longer introduce names, eliminating the need for
$\alpha$-conversion of bound variables.
However, manipulation of terms may still accidentally capture
variables when a de Bruijn index is moved over an abstraction.
In that case the index should be updated in order to still refer to the
original binding site.
This operation is called lifting. We denote the simple lifting operation
of incrementing all free variables in $M$ by $\uparrow \! M$. For
example, $\lambda . \uparrow \! \lambda . \; 1 \; (0 \; 2)$ is
$\lambda . \lambda . \; 2 \; (0 \; 3)$.

\subsubsection*{Substitution}

An example of an operation where lifting is used, is substitution.
Substituting $N$ for free de Bruijn index $n$ can be defined
\begin{align*}
  n[N/n]             &= N\\
  m[N/n]             &= m                    && \text{$m \neq n$} \\
  (\lambda .M')[N/n] &= \lambda .M'[\uparrow \! N / n\!+\!1]\\
  (M_1 \; M_2)[N/n]  &= M_1[N/n] \; M_2[N/n]
\end{align*}
which is structurally recursive.
Note that the lifting of $N$ in the case of abstraction is needed because
$N$ is moved into the abstraction $\lambda.M'$.
For the same reason, recursively substituting $\uparrow \! N$ in the body
$M'$ has to be done for indices $n+1$ instead of $n$.

\subsubsection*{de Bruijn Indices in \Coq}

The following is a datatype definition in \Coq for $\lambda$-calculus
terms using de Bruijn indices:
\begin{lstlisting}
Inductive term : Set :=
  | Var : nat -> term
  | Abs : term -> term
  | App : term -> term -> term.
\end{lstlisting}
Substitution is easily implemented using lifting.
\begin{lstlisting}
Fixpoint lift (l : nat) (t : term) {struct t} : term :=
  match t with
  | Var n   => Var (if le_lt_dec l n then (S n) else n)
  | Abs u   => Abs (lift (S l) u)
  | App u v => App (lift l u) (lift l v)
end.

Fixpoint subst (s : term) (n : nat) (t : term)
    {struct t} : term :=
  match t with
  | Var m   => if eq_nat_dec n m then s else t
  | Abs u   => Abs (subst (lift 0 s) (S n) u)
  | App u v => App (subst s n u) (subst s n v)
end.
\end{lstlisting}

More complex operations on terms may involve quite some
manipulation of indices.
Although these manipulations are easier to carry out than
$\alpha$-conversion in the named variables representation, they
are still cumbersome and obfuscate proof scripts.

\subsubsection*{Summary}

The most obvious advantage of a representation with de Bruijn
indices is that $\alpha$-equivalence is just term equality.
This does away with the problems usually associated with
$\alpha$-equivalence, a major win in automated reasoning.
Additionally, lifting is easier than $\alpha$-conversion (but
still tedious) and there is no need to generate fresh names.

The representation diverges quite far from the traditional
representation using named variables.
Therefore, it is arguably harder to read and the relation
between the object language and formal developments using de
Bruijn indices might need some argumentation.


\subsection{Locally Nameless Representation}

A way to avoid accidental capturing of variables and thus the need for
$\alpha$-conversion is to make sure free and bound variables are drawn
from disjoint sets.
This strategy is implemented by the {\em locally nameless} representation,
where free variables are represented by names and bound variables are
represented by de Bruijn indices \cite{McBride-McKinna-04}, effectively
obeying Barendregt's variable convention.
The resulting grammar is
\begin{align*}
  M ::=             &\; x
  && \text{free variable} \\
  \llap{\textbar\:} &\; n
  && \text{bound variable} \\
  \llap{\textbar\:} &\; \lambda . \; M
    && \text{abstraction} \\
  \llap{\textbar\:} &\; M \; M
  && \text{application}
\end{align*}
where $x$ ranges over a countably infinite set of names, and $n$ over
the natural numbers.
The following are some example terms in the traditional representation
(top), using de Bruijn indices (middle), and in the locally nameless
representation (bottom):
\begin{align*}
  \lambda x&. \; x & (\lambda z&. \; z \; x) \; z & \lambda x&. \; \lambda y. \; x \; (y \; z)\\
  \lambda &. \; 0  & (\lambda &. \; 0 \; 1) \; 2  & \lambda &. \; \lambda . \; 1 \; (0 \; 2)\\
  \lambda &. \; 0  & (\lambda &. \; 0 \; x) \; z  & \lambda &. \; \lambda . \; 1 \; (0 \; z)
\end{align*}

Just like pure de Bruijn indices, the locally nameless representation
has $\alpha$-equivalence as term equality.
What sets it apart from both representations discussed above, is that
there is no need to worry about accidental capturing of variables,
since the invariant is maintained that all bound variables are
different from free variables.
There is, unfortunately, a cost to this.

\subsubsection*{Freshening}

Consider the term $\lambda . \; \lambda . \; 1 \; (0 \; z)$.
How should we proceed in order to reason about the body of the outer
abstraction?
Surely we cannot say much about the verbatim subterm
$\lambda . \; 1 \; (0 \; z)$, since it is not locally nameless.
What happend is that a previously bound variable was taken from the
scope of its binder, making it a free variable.
To maintain the locally nameless invariant, this variable should now be
represented by a name.
The substitution of a name for index $0$ is called {\em freshening}.
In this example, $\mathit{freshen}(\lambda . \; 1 \; (0 \; z), v)$ would yield
$\lambda . \; v \; (0 \; z)$.

\subsubsection*{Substitution}

Substitution of terms for free variables in a locally nameless term
means substutition of terms for names.
\begin{align*}
  x[N/x]             &= N\\
  y[N/x]             &= y                 && \text{$x \neq y$} \\
  n[N/x]             &= n\\
  (\lambda .M')[N/x] &= \lambda .M'[N/x]\\
  (M_1 \; M_2)[N/x]  &= M_1[N/x] \; M_2[N/x]
\end{align*}
There is no need for renaming, because abstractions do not bind names.
There is no need for lifting, because there are no free de Bruijn
indices.

To implement the freshening operation, it must also be possible to
replace de Bruijn indices by a term.
Substitution for free variables thus has a dual notion for bound
variables.
\begin{align*}
  x[N/n]             &= x\\
  n[N/n]             &= N\\
  m[N/n]             &= m                    && \text{$m \neq n$} \\
  (\lambda .M')[N/n] &= \lambda .M'[N / n\!+\!1]\\
  (M_1 \; M_2)[N/n]  &= M_1[N/n] \; M_2[N/n]
\end{align*}
Freshening $M$ with name $x$ can now be written
$\mathit{freshen}(M, x) = M[x/0]$.

\subsubsection*{Locally Nameless Representation in Coq}

$\lambda$-calculus terms can be represented locally nameless in \Coq with
the following datatype:
\begin{lstlisting}
Inductive term : Set :=
  | FreeVar  : name -> term
  | BoundVar : nat  -> term
  | Abs      : term -> term
  | App      : term -> term -> term.
\end{lstlisting}
Substitution of terms for free variables is defined by
\lstinline{subst_free}. The \lstinline{subst_bound} function is meant
to be used only in the definition of \lstinline{freshen}.
\begin{lstlisting}
Fixpoint subst_free (s : term) (x : name) (t : term)
    {struct t} : term :=
  match t with
  | FreeVar y  => if eq_name x y then s else t
  | BoundVar n => t
  | Abs b      => Abs (subst_free s x b)
  | App f a    => App (subst_free s x f) (subst_free s x a)
end.

Fixpoint subst_bound (s : term) (n : nat) (t : term)
    {struct t} : term :=
  match t with
  | FreeVar x  => t
  | BoundVar m => if eq_nat_dec m n then s else t
  | Abs b      => Abs (subst_bound s (S n) b)
  | App f a    => App (subst_bound s n f) (subst_bound s n a)
end.

Definition freshen (t : term) (x : name) : term :=
  subst_bound (FreeVar x) 0 t.
\end{lstlisting}

\subsubsection*{Summary}

In a way, the locally nameless representation tries to be a
\emph{best of both worlds} combination of named variables and de Bruijn
indices.
Inherited from pure de Bruijn indices are the pleasant properties that
$\alpha$-equivalence is term equality and bound variables do not need
renaming.
Using names, however, feels like a natural way to represent free
variables that presumably take some meaning from a context, and as such
makes locally nameless terms arguably easier to read than their pure de
Bruijn equivalents.

The main technical difficulty we get is that of maintaining the invariant
that all terms are de Bruijn-closed, sometimes requiring freshening.
A consequence is that we might lose structural recursion for some
operations (an abstraction body might not be a subterm after freshening).
This does not apply to our current definition of substitution, where
freshening is omitted since it can easily be seen that the invariant is
only needed for the substituted term and not for the term that is recursed
on.
For more complex operations, however, this may not be the case.


\section{Applications}

In this section, we discuss three real-world cases of representing
languages with binding. Again, we focus on applications using the \Coq
proof assistant.


\subsection{The \POPLmark Challenge}

In the introduction, we debated that mechanized formalisation of
programming language metatheory has the potential of bringing the
academic tradition of rigoruous formal reasoning to the implementation
of everyday programming languages.
The \POPLmark Challenge \cite{Poplmark-Challenge-05} proposes a set of
benchmarks to measure progress in this area, under the slogan {\em
  mechanized metatheory for the masses}.

The benchmarks are based on the metatheory of System
F$_\texttt{<\!:}$, a polymorphic $\lambda$-calculus with subtyping,
and revolve in large part around binding.

Several solutions to the chalange have since been submitted.
We discuss some of those implemented in the \Coq proof assistant.

Named variables (1a) in \Coq by Stump \cite{Stump-05}.

de Bruijn indices in \Coq by Vouillon \cite{Vouillon-05}.

Locally nameless in \Coq by Leroy \cite{Leroy-07}.

Chargu\'eraud uses proofs by induction on the derivation of the
well-formation relation (locally nameless) instead of induction on the
size of types. See \POPLmark site (and his paper?).


\subsection{Engineering Formal Metatheory}

LNgen and \cite{Aydemir-et-al-08}.


\section{Discussion}\label{sec:discussion}

\subsection{Others Representations}

Nominal, higher order abstract syntax \cite{Capretta-Felty-06}, nested datatypes (Hirschowitz and Maggesi).


\subsection{Conclusion and Related Work}


\nocite{*} % TODO: remove
\bibliographystyle{amsplain}
\bibliography{binding}


\end{document}
