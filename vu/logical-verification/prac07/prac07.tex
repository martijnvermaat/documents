\documentclass[a4paper,11pt]{article}
\usepackage[english]{babel}
\usepackage{a4,fullpage}
\usepackage{amsmath,amsfonts,amssymb}
\usepackage{bussproofs} % http://math.ucsd.edu/%7Esbuss/ResearchWeb/bussproofs/

\renewcommand{\familydefault}{\sfdefault}

\def\extraVskip{4pt} % Space above and below lines in proof trees


\title{Practical Work Week 7\\
\normalsize{Logical Verification}}
\author{
    Martijn Vermaat\\
    mvermaat@cs.vu.nl
}
\date{October 25, 2005}


\begin{document}

\maketitle


\begin{description}

\item{\bf 1.}
The following derivation of $((A \rightarrow B \rightarrow A) \rightarrow B) \rightarrow B$
using natural deduction shows that this formula is a tautology of propositional logic.

\begin{prooftree}
\AxiomC{$[((A \rightarrow B \rightarrow A) \rightarrow B)^{x}]$}
  \AxiomC{$[A^{y}]$}
  \RightLabel{$I[\_]\rightarrow$}
  \UnaryInfC{$B \rightarrow A$}
  \RightLabel{$I[y]\rightarrow$}
\UnaryInfC{$A \rightarrow B \rightarrow A$}
\RightLabel{$E\rightarrow$}
\BinaryInfC{$B$}
\RightLabel{$I[x]\rightarrow$}
\UnaryInfC{$((A \rightarrow B \rightarrow A) \rightarrow B) \rightarrow B$}
\end{prooftree}

\item{\bf 2.}
In the type derivation below, we use $\Gamma$ for the singleton $\{x:(A \rightarrow B \rightarrow A) \rightarrow B\}$.

\begin{prooftree}
\AxiomC{$\Gamma \vdash x \: : \: (A \rightarrow B \rightarrow A) \rightarrow B$}
  \AxiomC{$\Gamma, y:A, z:B \vdash y \: : \: A$}
  \UnaryInfC{$\Gamma, y:A \vdash \lambda (z:B) \, . \, y \: : \: B \rightarrow A$}
\UnaryInfC{$\Gamma \vdash \lambda (y:A) (z:B) \, . \, y \: : \: A \rightarrow B \rightarrow A$}
\BinaryInfC{$\Gamma \vdash x \, \lambda (y:A) (z:B) \, . \, y \: : \: B$}
\UnaryInfC{$\vdash \lambda (x:(A \rightarrow B \rightarrow A) \rightarrow B) \, . \, x \, \lambda (y:A) (z:B) \, . \, y
  \: : \: ((A \rightarrow B \rightarrow A) \rightarrow B) \rightarrow B$}
\end{prooftree}

\item{\bf 4.}
A natural deduction derivation of $(\forall x . P(x)) \rightarrow \neg \exists y . \neg P(y)$.

\begin{prooftree}
\
\BinaryInfC{$P(y) \rightarrow \bottom$}
  %
\UnaryInfC{$P(y)$}
\RightLabel{$E\rightarrow$}
\BinaryInfC{$\bottom$}
\RightLabel{$I[y]\rightarrow$}             % \exists y . \neg P(y)
\UnaryInfC{$\neg \exists y . \neg P(y)$}
\RightLabel{$I[x]\rightarrow$}             % \forall x . P(x)
\UnaryInfC{$(\forall x . P(x)) \rightarrow \neg \exists y . \neg P(y)$}
\end{prooftree}

\end{description}


\end{document}
