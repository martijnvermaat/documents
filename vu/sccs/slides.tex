\documentclass{beamer}

\usepackage[english]{babel}
\usepackage{listings}
\usepackage{beamerthemesplit}
\setbeamertemplate{background canvas}[vertical shading][bottom=red!10,top=blue!10]
\setbeamertemplate{navigation symbols}{}
\setbeamertemplate{headline}{}
\usetheme{Warsaw}
\useinnertheme{rectangles}

\colorlet{darkred}{red!80!black}
\colorlet{darkblue}{blue!80!black}
\colorlet{darkgreen}{green!80!black}

% By Arthur van Dam
\lstdefinelanguage{SDF}%
{otherkeywords={->,>,\{,\},;},%
  morekeywords=[1]{bla,sorts,definition,module,context,free,syntax,lexical,prefer,reject,avoid,%
    priorities,exports,imports,cons,right,left,assoc,bracket,variables,restrictions,bla}%
    sensitive,%
    morecomment=[l]{\%\%},%
    morestring=[b]"%
}[keywords,comments,strings]%

\lstset{
  numbers=none,
  basicstyle=\footnotesize\ttfamily,
  frame=tb,
  language=Java,
  captionpos=b,
  xleftmargin=1em,
  xrightmargin=1em,
  aboveskip=1em,
  belowskip=1em
}


\title{Concrete Syntax for Meta-Programming}

\subtitle{(Important notice on next page)}

\author{Martijn Vermaat}
\institute{mvermaat@cs.vu.nl\\
http://www.cs.vu.nl/\~{}mvermaat/}
\date{Scientific Communication in Computer Science\\
May 25, 2005}


\begin{document}


\frame{\titlepage}


\frame {

  \frametitle{Notice}

  \begin{block}{Important notice}
    The research presented in this talk was \emph{not} done
    by Martijn Vermaat, but by the Software Technology Group
    at Utrecht University, most notably Eelco Visser.

    The purpose of this talk was only to serve as an exercise
    in speaking in public.
  \end{block}

}


\frame{

  \frametitle{Concrete Syntax for Meta-Programming}

  \tableofcontents

}


\section{Introduction}

\frame{

  \frametitle{Meta-Programming}

  Meta-program:
  \begin{quote}
    ``Program that manipulates other programs''
  \end{quote}

  \uncover<2->{

    \begin{block}{Meta-programming is about}
      \begin{itemize}
        \item Source code analysis.
        \item Code generation.
        \item Program transformation.
      \end{itemize}
    \end{block}

  }

}


\begin{frame}

  \frametitle{Concrete syntax for manipulated language}

  \begin{block}{Manipulating code fragments}
    \begin{itemize}
      \item String literals are bad.
      \item Abstract syntax trees are good.
      \item ASTs are too verbose.
    \end{itemize}
    \uncover<2->{
      Conclusion: {\bf concrete syntax}
    }
  \end{block}

\end{frame}


\section{Motivation}

\frame{

  \frametitle{Concrete Syntax for Meta-Programming}

  \tableofcontents[currentsection]

}


\begin{frame}

  \frametitle{Meta-Language and Object Language}

  \begin{block}{Meta-language}
    \uncover<2->{
      \begin{itemize}
        \item Language of the meta-program.
        \item<4-> Example: {\bf Javascript}
      \end{itemize}
    }
  \end{block}

  \begin{block}{Object language}
    \uncover<3->{
      \begin{itemize}
        \item Language of the manipulated code.
        \item<4-> Example: {\bf XML}
      \end{itemize}
    }
  \end{block}

\end{frame}


\begin{frame}[fragile]

  \frametitle{Example: Generating XML in Javascript}

  Generate this XML fragment in Javascript:

  \begin{lstlisting}[language=XML]
<ul>
  <li>een</li>
  <li>twee</li>
</ul>
  \end{lstlisting}

\end{frame}


\subsection{Existing Methods}

\begin{frame}[fragile]

  \frametitle{Generating String Literals}

  \begin{block}{Using string literals}
    \begin{lstlisting}[language=Java]
var menu =
   "<ul>"
 + "<li>een</li>"
 + "<li>twee</li>"
 + "</ul>";
    \end{lstlisting}
  \end{block}

\end{frame}


\begin{frame}[fragile]

  \frametitle{Manipulating Abstract Syntax Trees}

  \begin{block}{Using abstract syntax}
    \begin{lstlisting}[language=Java]
var menu, item;
menu = document.createElement("ul");

item = document.createElement("li");
item.appendChild(document.createTextNode("een"));
menu.appendChild(item);

item = document.createElement("li");
item.appendChild(document.createTextNode("twee"));
menu.appendChild(item);
    \end{lstlisting}
  \end{block}

\end{frame}


\begin{frame}

  \frametitle{Comparing String Literals and Astract Syntax}

  \begin{block}{Generating String Literals}
    \begin{description}
      \item{\color{darkgreen}$\Rightarrow$} Easy to read/write
      \item{\color{darkgreen}$\Rightarrow$} Easy to implement
      \item{\color{darkred}$\Rightarrow$} No syntactic checks
      \item{\color{darkred}$\Rightarrow$} Further manipulation will be hard
    \end{description}
    \uncover<2->{
      Conclusion: {\bf bad}
    }
  \end{block}

  \begin{block}{Manipulating Abstract Syntax Trees}
    \begin{description}
      \item{\color{darkgreen}$\Rightarrow$} No syntactically incorrect code
      \item{\color{darkgreen}$\Rightarrow$} Further manipulation possible
      \item{\color{darkgreen}$\Rightarrow$} Matching on structure possible
      \item{\color{darkred}$\Rightarrow$} Knowledge of abstract syntax necessary
      \item{\color{darkred}$\Rightarrow$} Does not scale
    \end{description}
    \uncover<3->{
      Conclusion: {\bf `less bad'}
    }
  \end{block}

\end{frame}


\subsection{Concrete Object Syntax}

\begin{frame}

  \frametitle{Concrete Syntax for Meta-Programming}

  \begin{block}{Solution}
    \begin{itemize}
      \item Use concrete syntax of the object language...
      \item ... but transform to abstract syntax manipulation.
    \end{itemize}
  \end{block}

  \begin{block}{Concrete object syntax}
    \begin{description}
      \item{\color{darkgreen}$\Rightarrow$} Easy to read/write
      \item{\color{darkgreen}$\Rightarrow$} No syntactically incorrect code
      \item{\color{darkgreen}$\Rightarrow$} Further manipulation possible
      \item{\color{darkgreen}$\Rightarrow$} Matching on structure possible
    \end{description}
    \uncover<2->{
      Conclusion: {\bf best of both worlds}
    }
  \end{block}

\end{frame}


\begin{frame}[fragile]

  \frametitle{Example: Generating XML in Javascript}

  \begin{block}{Using concrete syntax}
    \begin{lstlisting}[language=Java]
var menu = <ul>
             <li>een</li>
             <li>twee</li>
           </ul>;
    \end{lstlisting}
  \end{block}

\end{frame}


\section{Implementation}

\frame{

  \frametitle{Concrete Syntax for Meta-Programming}

  \tableofcontents[currentsection]

}


\begin{frame}

  \frametitle{Implementation}

  \begin{block}{Syntax Definition}
    \uncover<2->{
      \begin{itemize}
        \item Embed object language in meta-language.
        \item Define one combined syntax.
      \end{itemize}
    }
  \end{block}

  \begin{block}{Assimilation}
    \uncover<3->{
      \begin{itemize}
        \item Rewrite concrete object syntax to meta-language constructs.
        \item By program transformation.
      \end{itemize}
    }
  \end{block}

\end{frame}


\subsection{Syntax Definition}

\begin{frame}

  \frametitle{Syntax Definition Formalism}

  \begin{block}{SDF2...}
    \begin{itemize}
      \item is declarative.
      \item is modular.
      \item combines lexical and context-free syntax.
    \end{itemize}
  \end{block}

  \uncover<2->{
    \begin{block}{Parsing}
      SGLR is a parser generator for SDF2 syntax definitions.
      \begin{itemize}
        \item Scannerless.
        \item Accepts full class of context-free grammars.
      \end{itemize}
    \end{block}
  }

\end{frame}


\begin{frame}[fragile]

  \frametitle{Example: Embedding XML in Javascript}

  \begin{itemize}
    \item Syntax definition of Javascript (\texttt{Javascript})
    \item Syntax definition of XML (\texttt{Xml})
  \end{itemize}

    \begin{block}{Combined Syntax Definition}
      \begin{lstlisting}[language=SDF]
module JavascriptXml
imports Xml
imports Javascript

exports
  context-free syntax
    Element  ->  Expr {cons("ToExpr")}
      \end{lstlisting}
    \end{block}

\end{frame}


\subsection{Assimilation}

\begin{frame}

  \frametitle{Stratego for Assimilation}

  \begin{block}{Stratego language}
    \begin{itemize}
      \item Transformation language
      \item Rewrite object fragments to meta-language constructs
      \item Supports concrete object syntax itself
    \end{itemize}
  \end{block}

\end{frame}


\begin{frame}[fragile]

  \frametitle{Example: Assimilation of XML into Javascript}

  \begin{itemize}
    \item Look for fragments of concrete XML code
    \item Transform into AST manipulating Javascript code
  \end{itemize}

  \begin{lstlisting}
var t = <p>test</p>;
  \end{lstlisting}

\uncover<1,3,5,7,9,11,13,15>{
  $\Rightarrow$
}

  \begin{lstlisting}
var t;
{
  var _t = document.createElement('p');
  var _c = document.createTextNode('test');
  _t.appendChild(_c);
  t = _t
}
  \end{lstlisting}

\end{frame}


\section{Outro}

\frame{

  \frametitle{Concrete Syntax for Meta-Programming}

  \tableofcontents[currentsection]

}


\begin{frame}

  \frametitle{Conclusions}

  \begin{block}{Concrete object syntax}
    \uncover<2->{
      \begin{description}
        \item{\color{darkgreen}$\Rightarrow$} Much easier to write and read
        \item{\color{darkgreen}$\Rightarrow$} Usual strengths of AST manipulation
        \item{\color{darkred}$\Rightarrow$} Complicates programming environment
      \end{description}
    }
  \end{block}

  \begin{block}{Future work}
    \uncover<3->{
      \begin{itemize}
        \item Integrate assimilation in meta-language compiler
        \item Apply method to library interfacing
        \item ...
      \end{itemize}
    }
  \end{block}

\end{frame}


\begin{frame}

  \frametitle{Questions and Further Reading}

  Any questions?\\[4em]

  \begin{block}{Further Reading}
    \texttt{http://metaborg.org}
  \end{block}

\end{frame}


\end{document}
