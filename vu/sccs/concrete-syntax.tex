\documentclass[a4paper,11pt]{article}
\usepackage[english]{babel}
\usepackage{a4}
%\usepackage{a4,fullpage}
%\usepackage{amsmath,amsfonts,amssymb}
%\usepackage{fancyhdr}
\usepackage{listings}


% Use sans serif font for body text
%\renewcommand{\familydefault}{\sfdefault}

\pagestyle{headings}


\title{Concrete Syntax for Meta-Programming}

\author{Martijn Vermaat\\
\texttt{mvermaat@cs.vu.nl}}
\date{8th May 2005}

\makeindex


\begin{document}

\maketitle


\lstset{
  numbers=none,
  basicstyle=\footnotesize\ttfamily,
  frame=tb,
  language=Pascal,
  captionpos=b,
  xleftmargin=1em,
  xrightmargin=1em,
  aboveskip=1em,
  belowskip=1em
}


\begin{abstract}
Discussed is a general method for meta-programming with concrete object
syntax. Todo.
\end{abstract}


\section{Introduction}

Meta-programming is concerned with analysis, generation, and transformation
of object programs. In this setting, the meta-language provides constructs
to manipulate code fragments of the object language. Meta-languages employ
standard data structures for the representation of object programs. Typically,
these data structures are used to represent abstract syntax trees rather
than textual (concrete) syntax.

The use of abstract syntax allows high-level data manipulation techniques
to be employed in the meta-language in order to manipulate the object program.
For example, object-oriented languages such as Java provide methods to store
abstract syntax trees as object hierarchies and techniques to compose and
decompose these hierarchies, while functional programming languages support
algebraic data types in combination with pattern matching (e.g. Haskell, and
the ML family of languages).

Compared to using strings of object code fragments and string concatenation
for the composition of these fragments, abstract syntax representations allow
guaranties about the well-formedness of generated object programs in addition
to the much more powerfull techniques for further manipulation of these
programs.

\paragraph{}

However, there are many domains of meta-programming where the conceptual gap
between the concrete syntax representation of object programs and the constructs
to manipulate the abstract syntax representation of these programs is greater
than one should desire. As an example, consider the following arithmetic
expression in an imaginary object language:

\begin{lstlisting}[title=Example expression in concrete syntax]
(1 + 7) * i - 1
\end{lstlisting}

A typical abstract syntax representation of this fragment in a meta-language
with algebraic data types can look like the following:

\begin{lstlisting}[title=Example expression in abstract syntax]
Sub(Mul(Add(IConst(1), IConst(7)), Id('i')), IConst(1))
\end{lstlisting}

It is clear that in this example the concrete syntax representation of the
object code is much easier to read than the abstract syntax representation in
the meta-language. Although it should be noted that in some cases the abstract
syntax representation can be more concise, in general it tends to become more
painful to manage abstract syntax as the size of object code fragments grows.


\section{Abstract Syntax vs Concrete Syntax}

Problem description, existing solutions.
Introduction of examples.
Todo.


\section{Realizing Concrete Syntax}

Combining syntax definitions, assimilation to the host language.
Todo.


\section{Discussion}


\subsection{Background}

Modularity, Scannerless parsing.
SDF, ASF+SDF, StrategoXT.
Todo.


\subsection{Related Work}

Todo.


\subsection{Future Work}

Todo.


\subsection{Conclusions}

Todo.


\begin{thebibliography}{99}

\bibitem{Visser97}E. Visser. Scannerless generalized-LR parsing.
Technical Report P9707, Programming Research Group, University of Amsterdam, July 1997.

\end{thebibliography}


\end{document}


% Meta-Programming with Concrete Object Syntax
%
% 1 Introduction
% 2 Abstract Syntax vs Concrete Syntax
%   2.1 Syntax Definition
%   2.2 Example: Instrumenting Programs
%   2.3 Concrete vs Abstract
% 3 Implementation
%   3.1 Extending the Meta-Language
%   3.2 Meta-Variables
%   3.3 Meta-Explode
% 4 Generalization
% 5 Discussion
%   5.1 Syntax Definition and Parsing
%   5.2 Desugaring Patterns
%   5.3 User-definable Syntax
%   5.4 Syntax Macros
% 6 Conclusions

% Concrete Syntax for Objects
%
% 1 Introduction
% 2 Concrete Syntax for Objects
%   2.1 Code Generation
%   2.2 XML Document Generation
%   2.3 Graphical User-Interface Construction
%   2.4 Other Applications
% 3 Realizing Concrete Syntax
%   3.1 Embedding and Assimilation
%   3.2 Java with Swul
%   3.3 Java with XML
%   3.4 Java with Java
% 4 Syntax Definition
%   4.1 SDF Overview
%   4.2 The Importance of Modularity
%   4.3 The Importance of Scannerless Parsing
% 5 Discussion
%   5.1 Previous Work
%   5.2 Related Work
%   5.3 Future Work
% 6 Conclusions
