\documentclass[a4paper,11pt]{article}
\usepackage[english]{babel}
\usepackage{a4}
%\usepackage{a4,fullpage}
%\usepackage{amsmath,amsfonts,amssymb}

% Use sans serif font for body text
\renewcommand{\familydefault}{\sfdefault}


\title{Concrete Syntax for Meta-Programming}

\author{Martijn Vermaat\\
$\texttt{mvermaat@cs.vu.nl}$}
\date{8th May 2005}

\makeindex


\begin{document}

\maketitle


\begin{abstract}
Discussed is a general method for meta-programming with concrete object
syntax. Todo.
\end{abstract}


\section{Introduction}

Todo.


\section{Discussion}


\subsection{Background}

Todo.


\subsection{Related Work}

Todo.


\subsection{Future Work}

Todo.


\end{document}


% Meta-Programming with Concrete Object Syntax
%
% 1 Introduction
% 2 Abstract Syntax vs Concrete Syntax
%   2.1 Syntax Definition
%   2.2 Example: Instrumenting Programs
%   2.3 Concrete vs Abstract
% 3 Implementation
%   3.1 Extending the Meta-Language
%   3.2 Meta-Variables
%   3.3 Meta-Explode
% 4 Generalization
% 5 Discussion
%   5.1 Syntax Definition and Parsing
%   5.2 Desugaring Patterns
%   5.3 User-definable Syntax
%   5.4 Syntax Macros
% 6 Conclusions

% Concrete Syntax for Objects
%
% 1 Introduction
% 2 Concrete Syntax for Objects
%   2.1 Code Generation
%   2.2 XML Document Generation
%   2.3 Graphical User-Interface Construction
%   2.4 Other Applications
% 3 Realizing Concrete Syntax
%   3.1 Embedding and Assimilation
%   3.2 Java with Swul
%   3.3 Java with XML
%   3.4 Java with Java
% 4 Syntax Definition
%   4.1 SDF Overview
%   4.2 The Importance of Modularity
%   4.3 The Importance of Scannerless Parsing
% 5 Discussion
%   5.1 Previous Work
%   5.2 Related Work
%   5.3 Future Work
% 6 Conclusions
