\documentclass[a4paper,11pt]{article}
\usepackage[english]{babel}
\usepackage{a4}
%\usepackage{a4,fullpage}
%\usepackage{amsmath,amsfonts,amssymb}
%\usepackage{fancyhdr}
\usepackage{listings}


% Use sans serif font for body text
%\renewcommand{\familydefault}{\sfdefault}

\pagestyle{headings}


%\title{Concrete Syntax for Object Languages}
\title{Concrete Syntax for Meta-Programming}

\author{Martijn Vermaat\\
\texttt{mvermaat@cs.vu.nl}}
\date{\today}

\makeindex


\begin{document}

\maketitle


\lstset{
  numbers=none,
  basicstyle=\footnotesize\ttfamily,
  frame=tb,
  language=Pascal,
  captionpos=b,
  xleftmargin=1em,
  xrightmargin=1em,
  aboveskip=1em,
  belowskip=1em
}


\begin{abstract}
Discussed is a general method for embedding concrete object syntax in
meta-programming (and library interfacing).

todo: abstract
\end{abstract}


\section{Introduction}\label{sec:introduction}

Meta-programming is concerned with analysis, generation, and transformation
of object programs. In this setting, the meta-language provides constructs
to manipulate code fragments of the object language. Meta-languages employ
standard data structures for the representation of object programs. Typically,
these data structures are used to represent abstract syntax trees rather
than textual (concrete) syntax.

The use of abstract syntax enables meta-programming environments to make
guaranties about well-formedness and type-correctness of generated programs.
Furthermore, abstract syntax representations allow for high-level techniques
to be employed in analysis and composition of code fragments.

However, there are many domains of meta-programming where the conceptual gap
between the concrete syntax representation of object programs and the constructs
to manipulate the abstract syntax representation of these programs is greater
than one should desire.

The techniques we present allow meta-programming systems to use abstract
syntax representations of object code, while giving the meta-programmer the
possibility of specifying the same object code using the concrete syntax of
the object language. This entails a more natural way of meta-programming with
higher readability of meta-programs, while still having all the bennefits of
abstract syntax representations of object programs.

\paragraph{}

We proceed as follows. In section \ref{sec:motivation} we introduce and explain
the problem domain. In section \ref{sec:related} we give a brief overview of
previous and related work. We conclude in section x.

todo: name all chapters


\section{Motivation}\label{sec:motivation}

Verhaaltje over meta-programming zelf. analyse, generating and transformation.

todo: verhaaltje over meta-programming

\paragraph{}

Systems for meta-programming use data structures for the representation of
object code fragments. One possible approach is to use the string type of the
meta-language for storing fragments of the object program in concrete syntax.
Composition of fragments is simply done by string concatenation, whereas
analysis and deconstruction of fragments is much harder.

A short characterization of the approach is the following. Specifying object
fragments is done in a natural way, using concrete syntax. Reading
meta-programs is easy because of this concrete syntax. Manipulating object
code can be much harder, because no structure is preserved. No guarantees
whatsoever can be made about the correctness of the generated code (i.e.
syntactical correctness, type-correctness).

\paragraph{}

Another approach is to use a datatype to store an abstract syntax representation
of the object program. The use of abstract syntax allows high-level data
manipulation techniques to be employed in the meta-language in order to
manipulate the object program. For example, object-oriented languages such
as Java provide methods to store abstract syntax trees as object hierarchies
and techniques to compose and decompose these hierarchies, while functional
programming languages support algebraic data types in combination with pattern
matching (e.g. Haskell, and the ML family of languages).

Compared to using strings of object code fragments, abstract syntax
representations allow the meta-programming system to make guaranties about the
well-formedness of generated object programs in addition to providing much more
powerfull techniques for further manipulation of these programs.

A negative aspect of this approach is that the meta-programmer has no natural
way of specifying object code. This has a negative effect on both the
construction and the readability of meta-programs. As an example, consider the
following arithmetic expression in an imaginary object language.
\begin{lstlisting}[title=Example expression in concrete syntax]
(1 + 7) * i - 1
\end{lstlisting}
A typical abstract syntax representation of this fragment in a meta-language
with algebraic data types can look like the following.
\begin{lstlisting}[title=Example expression in abstract syntax]
Sub(Mul(Add(IConst(1), IConst(7)), Id('i')), IConst(1))
\end{lstlisting}
It is clear that in this example the concrete syntax representation of the
object code is much easier to read than the abstract syntax representation in
the meta-language. Although it should be noted that in some cases the abstract
syntax representation can be more concise, in general it tends to become more
painful to manage abstract syntax as the size of object code fragments grows.

\paragraph{}

In this paper we present a method for the embedding of concrete syntax of
object languages in the syntax of meta-languages (thus thereby extending the
meta-language) and the assimilation of this embedded concrete syntax into the
meta-language. The embedding of an object language in a meta-language is done
by combining their syntax definitions. Assimilation of the embedded concrete
object code fragments is done by program transformation. Concrete object
syntax is transformed to the standard meta-language constructs for
constructing and manipulating abstract object syntax resulting in a pure
meta-program that can be compiled in the normal way.


\section{Related Work}\label{sec:related}

\subsection{User-definable Syntax}

User-definable syntax provides a way for the programmer to define certain
syntactical constructs in the language. A simple example of this are
user-definable infix operators as present in languages like Prolog and
Haskell. However, although sometimes very useful, this extending of syntax
is limited in its applications. More powerfull is the definition of syntax
in the algebraic specification formalism ASF+SDF in which the programmer
defines all syntactical constructs itself.

Experiments have been done with dynamically extending the syntax of a
language at parse-time by including declarations of syntax extensions in
the program itself. It is our opinion that this complicates the parsing
process too much while the bennefits of dynamically extending syntax within
a program file are not clear in the application of meta-programming.


\subsection{Syntax Macros}

Syntax macros are a technique to define syntactic abstractions over code
fragments. A typical syntax macro consists of a syntax definition that
accepts structured code fragments as arguments on invocation. The result
after invocation is a new code fragment constructed by the syntax macro.

Although syntax macros can provide useful abstractions, the syntax for
invocations is always fixed and usually starts with a macro delimiter or
identifier. Also, syntax macros do not abstract over the lexical syntax
of the language which our method does.

The rewriting (assimilation) of syntax macro invocations to the host
language in different syntax macro systems differs in expressivity, but
even the most expressive systems require non-terminals in the user-defined
syntax to map on a fixed non-terminal in the host language. The method
we present does not have this restriction as it completely seperates syntax
definition and assimilation.


\subsection{Metafront}

(misschien)

todo: metafront?


\section{Realizing Concrete Syntax}

In this section we will present our method for the embedding of concrete
syntax and assimilation to the meta-language. To illustrate the techniques,
x will be used as a running example.

todo: example toevoegen of zinnetje weghalen


\subsection{Syntax Definition}

Traditional systems for meta-programming need at least a syntax definition
of the object language combined with a parser and a pretty printer. In this
setting, object programs can be parsed to an abstract syntax representation
in the meta-language and this representation can be pretty-printed to
concrete object language code, usually after modifications by the meta-program.

Our proposed method consists of the following. The syntax definitions of the
meta-language and the object language are combined to form a syntax definition
of the meta-language extended with the embedding of the object language.
Meta-programs can be written in this syntax (and thus using concrete syntax
for object code) and are transformed to programs in the traditional syntax of
the meta-language. The transformation rewrites fragments of concrete object
code to constructs in the meta-language working on an abstract syntax representation
of the object code.

\subsubsection*{SDF and SGLR}

For syntax definition, we use the Syntax Definition Formalism SDF which combines
lexical and context-free syntax in a declarative and modular system. SDF definitions
are declarative in that it that they are not operational implementations of a
specific parser or parsing algorithm. Furthermore, SDF provides constructs to
combine syntax definitions and this is essential for our application. Because
there is no proper subclass of context-free grammars that is closed under union
SDF allows the full class of context-free grammars to be defined. Describing
both lexical and context-free syntax in a single definition is important as it
allows combining languages with completely different lexical syntax.

We employ the scannerless generalized LR parsing algorithm as implemented in
SGLR. The algorithm is scannerless because it does not include a separate lexical
scanner (which is important for reasons stated above). Generalized LR parsing is
an adaptation of LR parsing that results in a parse forrest rather than a single
parse tree. Multiple LR parsing states will be maintaned in parallel when multiple
productions can be choosen. Generalized LR and Early parsing are the only existing
algorithms that allow the full class of context-free grammars as is dictated by
the use of SDF.


\subsection{Assimilation}

We use the Stratego language for the assimilation of embedded object code fragments
in concrete syntax into the meta-language by transformation. Stratego is a
transformation language that itself supports meta-programming with concrete object
syntax implemented using the method described here. Abstract syntax of object code
in Stratego is stored in the ATerm format, a generic format for describing
structured data comparable to but more concise than the XML format.

Assimilation of embedded fragments can be done by a generic transformation to the
abstract syntax representation of the meta-language. This transormation can be
automatically generated from the combined syntax definition of the meta-language
and the object language. Indeed, the Stratego language uses this technique to
assimilate arbitrary object languages into the core Stratego language by rewriting
embedded concrete code fragments to the corresponding ATerm constructs.

Another approach is defining a custom transformation for assimilation, rather than
generating one automatically from the syntax definition. This allows one to exploit
the domain-specific characteristics of the embedding and thus to rewrite the
concrete syntax to more direct or efficient representations in the meta-language.

todo: nog meer kletsen over assimilation?


\subsection{Meta-Variables}

Concrete object code fragments embedded in the meta-program may need to refer to
meta-language variables (typically having as value another object code fragment).
This is called escaping and can be done by using a fixed syntactical construct
for meta-language variables inside object code.
However, it would be preferable if we could just write the name of a meta-variable
in concrete syntax fragments. This can be achieved by using SDF variable
declarations where we define (patterns for) identifiers that should be interpreted
as meta-variables by the parser.

todo: voorbeeldje van meta-variables ('as you see, this makes it even more concise')


\section{Discussion}

todo: discussion


\subsection{Future Work}

todo: future work


\subsection{Conclusions}

todo: conclusions


\begin{thebibliography}{99}

\bibitem{Visser97}E. Visser. Scannerless generalized-LR parsing.
Technical Report P9707, Programming Research Group, University of Amsterdam, July 1997.

todo: bibliography

\end{thebibliography}


\end{document}





% Meta-Programming with Concrete Object Syntax
%
% 1 Introduction
% 2 Abstract Syntax vs Concrete Syntax
%   2.1 Syntax Definition
%   2.2 Example: Instrumenting Programs
%   2.3 Concrete vs Abstract
% 3 Implementation
%   3.1 Extending the Meta-Language
%   3.2 Meta-Variables
%   3.3 Meta-Explode
% 4 Generalization
% 5 Discussion
%   5.1 Syntax Definition and Parsing
%   5.2 Desugaring Patterns
%   5.3 User-definable Syntax
%   5.4 Syntax Macros
% 6 Conclusions

% Concrete Syntax for Objects
%
% 1 Introduction
% 2 Concrete Syntax for Objects
%   2.1 Code Generation
%   2.2 XML Document Generation
%   2.3 Graphical User-Interface Construction
%   2.4 Other Applications
% 3 Realizing Concrete Syntax
%   3.1 Embedding and Assimilation
%   3.2 Java with Swul
%   3.3 Java with XML
%   3.4 Java with Java
% 4 Syntax Definition
%   4.1 SDF Overview
%   4.2 The Importance of Modularity
%   4.3 The Importance of Scannerless Parsing
% 5 Discussion
%   5.1 Previous Work
%   5.2 Related Work
%   5.3 Future Work
% 6 Conclusions
