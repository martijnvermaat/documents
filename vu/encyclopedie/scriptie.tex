\documentclass[11pt]{report}
\usepackage[dutch]{babel}
\usepackage{a4}
\usepackage{latexsym}
\usepackage[
        colorlinks=true,
        linkcolor=zwart,
        citecolor=zwart,
        pdftitle={Veerman gevraagd (m/v)},
        pdfsubject={De kloof tussen de twee culturen},
        pdfauthor={Martijn Vermaat}
]{hyperref}
\usepackage{graphicx}
\usepackage{color}
\definecolor{zwart}{rgb}{0,0,0}

\title{Veerman gevraagd (m/v)}
\author{
	Martijn Vermaat \\ mvermaat@cs.vu.nl
}
\date{Amsterdam, Vrije Universiteit, 16 juni 2003}

\begin{document}

\maketitle


\begin{abstract}
In de Westerse cultuur kennen we al sinds de Verlichting twee kampen van wetenschappers: de alfa's en de b\`eta's. Er is veel discussie over de kloof die de twee van elkaar zou scheiden. Wat heeft nou gezorgd voor deze tweedeling en waarom lijkt het alsof de kloof alleen maar groter is geworden? We bekijken wat de verschillen tussen de alfa en de b\`eta cultuur zijn (en wat precies met beiden bedoeld wordt), wat de gevolgen zijn van de schijnbaar grote afstand tussen de twee en welke oplossingen we hebben voor eventuele problemen.
\end{abstract}

\tableofcontents


\chapter{De twee culturen}


\section{Een verkenning}


\subsection{Culturen?}

Het zal niemand onbekend zijn, het bestaan van twee distincte culturen binnen de wetenschap. Aan de ene kant zien we de mens- en geesteswetenschappen, ook wel alfa cultuur genoemd en aan de andere kant zien we de exacte en natuurwetenschappen, welke de b\`eta cultuur vormen. Het precies aangeven van welke wetenschappen onder welke cultuur vallen is niet zo eenvoudig. Over het algemeen wordt ook een derde cultuur van de sociale en economische wetenschappen, de gamma cultuur, genoemd en in enkele gevallen spreekt men zelfs over een omega cultuur met welke dan filosofie en religie bedoeld wordt.

Een van de weinige dingen die in vrijwel alle verdelingen blijft gelden is dat met de b\`eta cultuur de exacte en natuurwetenschappen bedoeld worden. Deze blijken dus altijd apart van `de rest' gezien te worden. Maar over de verdeling van die rest is men het lang niet altijd eens. Regelmatig wordt dit voor het gemak zelfs op een hoop gegooid om aangeduid te kunnen worden onder een noemer, `alfa' of `alfa-gamma'. Overigens wordt in veel gevallen niet alleen de studie van de kunst, maar ook de kunst zelf hier geplaatst.


\subsection{Verdeling?}

Wat moeten we met een dergelijke verdeling? Waarop is deze eigenlijk gebaseerd? Kijken we naar de alfa cultuur dan kunnen we zeggen dat deze voornamelijk gericht is op gevoel en subjectief van aard is. De b\`eta cultuur daarentegen richt zich voornamelijk op de werkelijkheid en is objectief van aard. Tegenstanders van deze tweedeling op methodologie betogen dat kennis nooit objectief is en er dus zelfs geen wezenlijk verschil bestaat tussen kunst en wetenschap. Kennis wordt immers door de mens zelf gecre\"eerd. Natuurwetenschappers zullen niet akkoord gaan met deze visie, maar onder filosofen is hij een stuk populairder.

Maar toch kunnen we hierin een tweedeling vinden. Waar de alfa cultuur uit gaat van de mens gaat de b\`eta cultuur uit van de werkelijkheid en of deze werkelijkheid dan wel of niet door de mens gecre\"eerd is, het doel van de twee culturen is duidelijk verschillend. Vanaf een hoger niveau lijkt het wellicht een pot nat, maar als natuurkundige zijn de feiten van betekenis en niets anders, terwijl iemand bij het lezen van een roman nooit zijn beklag zal doen bij de schrijver omdat een bepaalde gebeurtenis niet waar gebeurd lijkt te kunnen zijn.


\section{Historie}


\subsection{De Oudheid}

In de Oudheid kunnen we eigenlijk geen duidelijke alfa en b\`eta cultuur herkennen. Toch zijn er al wel enkele aanwijzingen voor een tweedeling te ontdekken. Zo grijpt filosoof \emph{Hans Achterhuis} terug op \emph{Plato}'s mythe van \emph{Prometheus}, uit de dialoog \emph{Protagoras}. De halfgod \emph{Prometheus} rooft van de goden technische kennis en vuur. De oppergod \emph{Zeus} vond dat niet toereikend en zond daarom de bode \emph{Hermes} om de mensen `eergevoel en recht' te brengen. Volgens \emph{Achterhuis} zien we daar reeds de twee culturen. Maar de huidige wetenschappen zijn ingedeeld op na-Middeleeuws model en daarom is in dit verband de Renaissance interessanter om te beschouwen.


\subsection{De \emph{Homo Universalis}}

Toen in Europa na de Middeleeuwen de Renaissance begon (in Itali\"e vanaf de 14$^{de}$ en in heel Europa vanaf de 16$^{de}$ eeuw) bestond bij veel mensen de behoefte in alle, of in ieder geval zo veel mogelijk, gebieden uit te blinken. Wonderbaarlijk genoeg zijn hier verschillende mensen zeer goed in geslaagd en dat heeft ervoor gezorgd dat we nu bij de term \emph{Homo Universalis} denken aan \emph{Da Vinci}, \emph{Michelangelo}, \emph{Copernicus}, \emph{Machiavelli} en anderen.

Maar deze opvatting verdween nog voor het einde van de Renaissance -- vanaf de Verlichting (ruwweg vanaf 1650) is de wetenschap steeds verder opgedeeld in segmenten die onderling steeds minder contact hadden. De \emph{Homo Universalis} kennen we wel nog tot op de dag van vandaag, maar wordt maar zeer zelden als toonaangevend gezien binnen een bepaalde wetenschap, misschien alleen binnen `meta-wetenschappen' als wetenschapsgeschiedenis.


\subsection{Alfa en b\`eta}\label{sec:alfaenbeta}

We kunnen stellen dat de klassieke indeling in alfa, b\`eta en gamma wetenschappen zich gevormd heeft tijdens de Verlichting. Opmerkelijk is de toewijzing van de Griekse letter alfa aan de taal- en geschiedeniswetenschappen. Tegenwoordig wordt door veel mensen de b\`eta cultuur als de hogere van de twee gezien, mede door de hogere significantie in de ontwikkeling van de maatschappij (wellicht kunnen we de huidige cultuur als geheel een `b\`eta cultuur' noemen). Getuige ook de betekenis van het Engelse `science' -- letterlijk `kennis', `weten', maar in de praktijk worden er slechts de exacte en natuurwetenschappen mee bedoeld. Het opstapje hiervoor werd al genomen in de Verlichting, waar feiten en waarheid troef speelden in de wetenschap. Een dergelijke instelling rechtvaardigt de toewijzing van de eerste letter uit het Griekse alfabet aan de b\`eta wetenschappen, zo zouden we denken -- het is niet gebeurd. 

De keuze voor de taal- en geschiedeniswetenschappen is beter te verklaren met behulp van het imperialisme tijdens de Verlichting. De expansie van Europa maakte dat deze wetenschappen belangrijker werden en wellicht werden zij daarvoor beloond met de letter alfa. De exacte en natuurwetenschappen waren `in' , maar minder significant in het dagelijks leven en kregen de letter b\`eta, de letter gamma bleef over voor sociologie en psychologie die zich eindelijk ook los hadden weten te maken van filosofie en theologie.

Zouden de Griekse letters nu opnieuw toegewezen moeten worden aan de verschillende wetenschapsculturen dan zou dit zeker anders gebeuren. De exacte en natuurwetenschappen zouden met de letter alfa heen gaan in deze technologie cultuur.


\chapter{De kloof tussen de twee culturen}


\section{De `two cultures' van \emph{Snow}}


\subsection{\emph{C.P. Snow}}

Wanneer we het hebben over de kloof tussen de alfa en de b\`eta cultuur kunnen we onmogelijk voorbij gaan aan \emph{C.P. Snow} en zijn \emph{The Two Cultures and the Scientific Revolution}\cite{Snow}. In 1956 gebruikte \emph{Snow} `The Two Cultures' voor het eerst als titel van een artikel in \emph{The New Statesman}, later heeft hij de inhoud van dit artikel uitgewerkt tot zijn lezing voor de universiteit van Cambridge in 1959: \emph{The Two Cultures and the Scientific Revolution}. In de eerste plaats was \emph{Snow}, na zijn korte bestaan als natuurwetenschapper, een schrijver van romans, maar hij werd onverwacht wereldwijd bekend na de publicatie van zijn lezing in 1959.


\subsection{\emph{The Two Cultures and the Scientific Revolution}}

In deze \emph{Rede Lecture} bespreekt \emph{Snow} de problemen van de wetenschap in het na-oorlogse Engeland. Er zou onvoldoende begrepen zijn wat de betekenis van de industri\"ele revolutie was en het onderwijs zou daardoor te weinig aangepast zijn. De ervaringswereld van natuurwetenschappers verschilde te veel met die van de literaire intelligentia. Tijdens de wetenschappelijke revolutie die volgde uit de industri\"ele revolutie faalden natuurwetenschappers te denken in termen van toepasbare kennis. \emph{Snow} bekijkt verschillende onderwijssystemen wereldwijd en komt tot de conclusie dat het Russische onderwijssysteem nog het best uitgerust lijkt voor de wetenschappelijke revolutie.

De twee genoemde culturen communiceren volgens hem te weinig met elkaar, maar later in zijn lezing stapt hij over op meer sociaal maatschappelijke beschouwingen en heeft hij het vooral over omvormingen van het onderwijs opdat Engeland beter kan concurreren met de Sovjet Unie. Hoewel het debat dat uit zijn lezing voort kwam (en tot op heden voort duurt) zich vooral richt op de verschillen tussen de alfa en b\`eta wetenschappen is dat niet alleen waar het in de lezing om draait.


\subsection{Reacties op \emph{The Two Cultures}}

Ondanks het feit dat het \emph{Snow} in zijn lezing vooral gaat om het bredere, maatschappelijke beeld van de situatie en mogelijke verbeteringen daarvan, komt er een flink aantal stevige uitspraken over de alfa en b\`eta culturen in voor. In eerste instantie is te lezen dat de schuld in beide kampen gevonden kan worden, wetenschappers uit beide culturen staan niet genoeg open voor de andere cultuur. Maar even verderop lijkt hij terug te komen op deze mening. Zo betoogt hij, niet gehinderd door enige mens- of geesteswetenschappelijke kennis, dat vooral de alfa wetenschappers zich schuldig maken aan het behouden van de kloof tussen de twee culturen. Zo zouden ze de centrale functie van de b\`eta wetenschappen in de maatschappij niet onderkennen, behalve als toekomstbeeld, maar in dat geval wensen ze er gewoon op dat de toekomst niet bestaat.

Hij gaat zelfs verder. De b\`eta cultuur kent meer discussie, doorgaans veel significanter en altijd op een hoger conceptueel niveau, dan de alfa cultuur. Dat schiet velen in het verkeerde keelgat. Tot meer dan 50 jaar na de publicatie van \emph{The Two Cultures} krijgt het stuk regelmatig zware kritiek te verduren.\cite{Sparreboom} Nog regelmatig worden passages aangehaald, waarbij vooral alfa en gamma wetenschappers de significantie ervan proberen te minimaliseren. Vaak wordt daarbij ook nog de onrechtvaardige betekenis van het Engelse woord `science' aangehaald, waar zoals genoemd alleen de b\`eta wetenschappen mee bedoeld worden.

\paragraph{}

De bekendste en meeste directe kritiek op \emph{The Two Cultures} kwam echter al na een jaar, in de vorm van een lezing van \emph{F.R. Leavis}\cite{Leavis} welke gepubliceerd is in \emph{The Spectator}. Bedenk dat in die tijd de Engelse pers anders bekend stond dan tegenwoordig en doorgaans nog zeer beleefd te werk ging. Dat is niet af te zien aan de reactie van \emph{Leavis}, die werkelijk ieder argument, iedere zin en ieder woord van \emph{Snow} vertrapt. Onder de titel \emph{Two Cultures? The Significance of C. P. Snow} lezen we dat \emph{Leavis} ervan walgde dat \emph{Snow} compleet voorbij ging aan geschiedenis, literatuur, de geschiedenis de beschaving, en de menselijke significantie van de industri\"ele revolutie. Om een uitspraak van \emph{Leavis} te noemen: ``It is ridiculous to credit him with any capacity for serious thinking about the problems on which he offers to advise the world.''

\paragraph{}

Of \emph{Snow} nu wel of niet de schuld bij de juiste kant legde, het idee van de tweedeling binnen de wetenschap sprak hoe dan ook veel mensen aan. Hoewel velen tegenwoordig het bestaan van deze tweedeling compleet ontkennen (de reden is doorgaans \`of de gamma cultuur heeft de kloof altijd al gedempt, \`of er bestaat in wezen geen verschil tussen natuurwetenschappen en geesteswetenschappen) kan het geen toeval zijn dat zovelen op zijn lezing reageerden. Onder andere \emph{Snow} zelf concludeert enkele jaren later dat, afgezien van de geldigheid van zijn redeneringen, het grote aantal reacties wel moet betekenen dat er in ieder geval een kern van waarheid in zijn verhaal zit.\cite{Berkel} En daar is wat voor te zeggen. Of de tweedeling theoretisch niet bestaat of niet kan bestaan doet feitelijk niet terzake, in de praktijk hebben reeds zo veel mensen de kloof gezien en is er reeds zo veel over geschreven dat er tenminste \emph{iets} aan de hand moet zijn.


\section{Is er eigenlijk wel een kloof?}


Een vraag die we ons zelf misschien eerder hadden moeten stellen. We hebben min of meer vast gesteld dat er verschillende uitersten bestaan binnen de wetenschap, met aan de ene kant de alfa wetenschappen en aan de andere kant de b\`eta wetenschappen. Er is veel geschreven over de steeds breder wordende kloof tussen deze twee. Maar bestaat deze kloof \"uberhaupt wel? Als hij in de praktijk al wordt gezien (al dan niet als een probleem), kijken we dan niet door de verkeerde bril? Beelden we ons niet die hele kloof in?


\subsection{Er is maar een wetenschap}\label{sec:eenwetenschap}

Zoals al enkele keren genoemd betogen velen tegenwoordig, meer dan 40 jaar na \emph{Snow}, dat het onderscheid tussen de alfa wetenschappen en de b\`eta wetenschappen slechts een schijnbaar onderscheid is. Uitgaande van de \emph{Frankfurter Schule}\footnote{De in de jaren dertig ontstane Frankfurter Schule is een groep van ge\"engageerde wetenschappers, die zich afzetten tegen de zogenaamd waardevrije objectieve wetenschap van het logisch positivisme. Er bestaat niet zoiets als objectieve kennis, kennis is altijd gekleurd door de achtergrond en idee\"en van de kenner.} kunnen we tot de conclusie komen dat het bestuderen van de werkelijkheid, de natuur, niet wezenlijk verschilt van het beoefenen van kunst -- het andere uiterste tegenover de b\`eta cultuur. In de kunst heeft men niets te maken met de echte werkelijkheid, men cre\"eert een eigen werkelijkheid. En hoewel de b\`eta wetenschapper het menselijk aspect irrelevant vindt en slechts de werkelijkheid, de waarheid van de natuur, wil bestuderen, gaat het hier volgens de \emph{Frankfurter Schule} niet om een objectieve werkelijkheid, maar ook om een door de mens gecre\"erde werkelijkheid. En daarmee is het belangrijkste verschil tussen alfa en b\`eta verdwenen.

\paragraph{}

Er is nog een manier om de exacte wetenschap met de kunsten te verenigen. In het boek \emph{G\"odel, Escher, Bach -- Een eeuwige gouden band}\cite{Hofstadter} doet \emph{Douglas R. Hofstadter} grote moeite de geniale muziek van \emph{Bach} en de kunstzinnige tekeningen van \emph{Escher} te rijmen met de wiskunde. Een centrale rol spelen de zogenaamde `merkwaardige lussen'. Waar deze zich voor doen gebeuren merkwaardige dingen. \emph{Hofstadter} ziet deze lussen overal terug komen: als basis van de moeilijk te bevatten stelling van \emph{G\"odel} (wiskunde), in de onwerkelijkste delen van \emph{Escher}'s tekeningen en in de mooiste composities van \emph{Bach}. Hij ziet hierin zelfs de sleutel om van wiskunde naar kunst te komen en andersom (en in een ander perspectief: de basis van ware kunstmatige intelligentie).

Een andere aanwijzing in deze richting vinden we in de persoon van \emph{Nietzsche}. Hij beschouwt de filosoof \emph{Socrates} waarvan bekend is dat hij regelmatig in zijn dromen bezocht werd door een figuur die hem steeds weer zei: ``Socrates, musiceer!''. \emph{Socrates} dacht altijd dat de figuur met musiceren doelde op de hoogste muzenkunst filosoferen en kon dus gerust zijn. Maar later in zijn leven, in afwachting van zijn doodvonnis, komt hij met schrik tot het inzicht dat ook het be\"oefenen van muziek bedoeld kon zijn. We kunnen hier ook de tweedeling in ontdekken van de hogere wetenschap tegenover de kunsten. Maar zoals \emph{Nietzsche} opmerkt gaat het om een en hetzelfde, want wanneer de logica tot in haar uitersten bedreven wordt ``slingert zij zich om zichzelf heen om zichzelf tenslotte in de staart te bijten'' en daar moet zij ook ``uiteindelijk omslaan in kunst. En dat is het eigenlijke doel van dit mechanisme''. \emph{Nietzche} heeft het hier duidelijk over dezelfde `merkwaardige lussen' als \emph{Hofstadter} in zijn \emph{G\"odel, Escher, Bach}.

\paragraph{}

Maar laten we ons niet te veel meeslepen door filosofische beschouwingen. Zoals gezegd wordt de tweedeling in de praktijk wel degelijk ervaren en om de mogelijke kloof tussen de twee culturen verder te kunnen onderzoeken zullen we moeten aannemen dat ze in ieder geval bestaan.


\subsection{Er is geen kloof}\label{sec:geenkloof}

Met de aanname dat er een alfa en een b\`eta cultuur bestaat in ons achterhoofd kunnen we kijken of er dan eigenlijk wel een kloof tussen beiden bestaat. Er zijn in wezen vier mogelijkheden. De eerste is simpelweg het bestaan van een kloof, zonder dat we daar iets aan kunnen doen. De tweede mogelijkheid is het bestaan van een kloof met een of meer bruggen die beide kanten met elkaar verbindt. Deze bruggen zijn in feite oplossingen voor een aanwezig probleem en zullen verderop aan bod komen. Het kan ook zo zijn dat er nooit sprake geweest is van een kloof en dat de twee kanten, als we goed kijken, samen slechts een eiland vormen. Dit is de derde mogelijkheid en is in \ref{sec:eenwetenschap} besproken. Onze laatste en vierder mogelijkheid is dat de kloof er in feite wel is, maar reeds lang geleden gedempt is.

\paragraph{}

Volgens velen is de kloof inderdaad gedempt. Uiteraard doelt men dan op de opvulling door de gamma cultuur, de sociale en economische wetenschappen. Waar de alfa cultuur slechts mens en woord en de b\`eta cultuur slechts feit en woord bevat, bevat de gamma cultuur alle drie: mens, feit en woord. Bevinden we ons ergens in de kloof, het gebied tussen de alfa cultuur en de b\`eta cultuur, dan vallen we niet, maar staan we in de gamma cultuur. Deze verbindt beide culturen niet slechts op bepaalde punten zoals een brug dat doet, maar dempt de hele kloof.

Indien dit juist is kunnen we dus vanuit de alfa cultuur via de gamma cultuur naar de b\`eta cultuur lopen en terug. Maar wat merken we dan nog van de kloof? Merken we \"uberhaupt nog dat de kloof er is, al is deze gedempt door de gamma cultuur?


\section{De kloof in de praktijk}


\subsection{Eigenschappen van de culturen}

\emph{Gerrit Krol} geeft in verschillende lezingen typerende beschrijvingen van alfa wetenschappers en b\`eta wetenschappers. De in \ref{sec:geenkloof} genoemde ingredie\"enten -- mens, feit en woord -- komen bijvoorbeeld van \emph{Krol}. In het openingscollege van de Vermeerlezing 2001 aan de TU Delft stelt \emph{Krol}\cite{Krol} dat het in de alfa cultuur om de illusie en in de b\`eta cultuur om het getal draait. In een gedicht bestaat een werkelijkheid, maar deze is niet dezelfde als de `echte werkelijkheid'. Het is een illusie die door iedere lezer anders ge\"\i{}nterpreteerd kan worden, want een alfageest interpreteert een woord niet naar de letter, maar naar de geest. In de b\`eta cultuur daarentegen komen twee verschillende mensen tot dezelfde voorspelling. Wat niet in getallen kan worden uitgedrukt bestaat niet en de enige werkelijkheid is de `echte werkelijkheid' -- illusies bestaan niet.

\emph{Krol} wijst ook op de trots van beide culturen\footnote{Zie ook \emph{De trots van alfa en b\'eta}\cite{Krolea} met verhandelingen van onder andere \emph{Gerrit Krol} en \emph{Hans Achterhuis}, uitgegeven door De Bezige Bij}. Terwijl men in de b\`eta wereld trots is op de aanwezigheid van dingen en de afwezigheid van de mens, is men in de alfa wereld trots op de afwezigheid van de ware werkelijkheid en de aanwezigheid van illusies. Volgens \emph{Krol} kunnen we alfa en b\`eta zien als twee bastions, twee elkaar vijandige families die trots zijn op zichzelf. En alfa is blij dat hij geen b\`eta is en andersom.


\subsection{Asymmetrie}

De tweedeling tussen alfa en b\`eta is overigens niet symmetrisch. Er zijn vele voorbeelden te noemen van b\`eta wetenschappers die uiteindelijk uitstekend uit de voeten blijken te kunnen in de alfa cultuur. Zo ontvingen \emph{Gerrit Krol}, \emph{Rutger Kopland} en \emph{Eduard Jan Dijksterhuis}\cite{Hooykaas} de P.C. Hooftprijs en weet Paul Verhoeven (afgestudeerd wis- en natuurkundige) zich goed staande te houden achter de camera. Andersom zijn voorbeelden moeilijk te vinden. Hebben we ooit een alfa de Nobelprijs voor natuurkunde zien winnen?

Mensen uit de alfa cultuur bekennen het over het algemeen direct als ze niet weten hoe de wetten van \emph{De Morgan} luiden en lijken in veel gevallen zelfs trots op dit gebrek aan kennis. Maar geven natuurkundigen het snel toe als ze niets van muziek weten, of als ze nooit films kijken? Nee, in de meeste gevallen zijn b\`eta wetenschappers zelfs goed geinformeerd als het gaat om de alfa cultuur. Binnen de b\`eta cultuur heerst geen taboe op de alfa cultuur, maar binnen de alfa cultuur wel op de b\`eta cultuur, zo lijkt het.


\chapter{De problemen van de kloof}


\section{Wat zijn de gevolgen van de kloof?}


\subsection{Negatieve gevolgen}

Waarom houdt de koof tussen de twee culturen zoveel mensen bezig? Waarom volgen er vaak verhitte discussies uit? Er moeten problemen vastzitten aan de afstand tussen alfa en b\`eta. \emph{Snow} noemt als belangrijkste problemen de miscommunicatie en het onbegrip aan beide kanten (maar dan vooral aan de alfa kant). De b\`eta wetenschappers zijn volgens \emph{Snow} niet goed op de hoogte van wat de maatschappij nodig heeft en dus niet praktisch genoeg bezig. Maar de meeste schuld ligt bij de alfa wetenschappers, zo zegt \emph{Snow}. Zij ontkennen hardnekkig het belang van de b\`eta wetenschap, leven met een bord voor hun kop en zien niet in hoe ze van de b\`eta wetenschappen gebruik kunnen maken. \emph{Snow} gaat zelfs zo ver om te stellen dat deze problemen de wetenschappelijke revolutie tegenhouden en dat het westen op deze manier de race verliest van de Sovjet Unie. Dit is overigens ook een makkelijk punt van kritiek op \emph{The Two Cultures}, want wat is er tegenwoordig, wanneer de kloof nog altijd bestaat, nou helemaal over van de zogenaamde koppositie van Rusland op wetenschappelijk en economisch gebied?

\paragraph{}

Een ander probleem van de, volgens velen overigens steeds groter wordende, kloof is de gebrekkige uitwisseling van kennis tussen de twee culturen. De wetenschappers zijn over het algemeen niet direct vrijgevig als het op de `overkant' aankomt en andersom wordt er bij het verzamelen van kennis voornamelijk uit eigen bron geput, de bron van de eigen cultuur.

Gebrekkige samenwerking komt het duidelijkst naar boven bij gedwongen samenwerking. Denk hierbij aan onderzoeksgebieden als kunstmatige intelligentie, computerspraak, maar ook aan de alfa wetenschapper die samen moet werken met de statisticus. Over het algemeen blijft het contact beperkt, of neigt het hele project aan een kant van de kloof te blijven hangen. Als gevolg van de grote afstand komt het vaak voor dat wetenschappers zich eerst lange tijd moeten inlezen in elkaars vakgebied alvorens samen onderzoek te kunnen verrichten. Zou nu de afstand kleiner zijn, de andere cultuur bekender zijn, dan was samenwerking een stuk gemakkelijker.

\paragraph{}

Naast praktische problemen hebben vooral alfa wetenschappers nogal eens wat te klagen over de tweedeling. Want bij een tweedeling is er over het algemeen een kant die het moet ontgelden en dat is volgens velen in dit geval de alfa cultuur. We hebben eerder in \ref{sec:alfaenbeta} het woord `science' al genoemd. Aan de betekenis daarvan is al goed te zien hoe men in de Westerse cultuur tegen de wetenschap aan kijkt: daar valt in eerste instantie alleen de b\`eta cultuur onder. Voor de alfa wetenschappen wordt doorgaans het woord `humanities' gebruikt. De mens- en geesteswetenschappen willen van dit imago af en betogen daarom tegenwoordig regelmatig dat de hele tweedeling slechts een verzinsel is.


\subsection{Positieve gevolgen}

Zijn er dan misschien ook redenen om blij te zijn met de kloof? We kunnen ons voorstellen dat specialisatie noodzakelijk is voor verdere ontwikkeling en in dat geval is aan enige afstand tussen de wetenschappen niet te ontkomen. Het lijkt haast onmogelijk tegenwoordig nog diepgaande kennis op meerdere, ver uit elkaar liggende, wetenschappelijke gebieden te hebben. Rond het begin van de Renaissance zien we weliswaar de \emph{Homo Universalis} met zelfs verschillende wetenschappelijke successen, maar sindsdien is iets dergelijks toch nauwelijks meer voor gekomen.

Kunnen we zelfs stellen dat het verwijderen van de kloof tussen alfa en b\`eta negatieve gevolgen zal hebben op de wetenschappelijke vooruitgang? Dat gaat misschien wat ver. Een aardig experiment zou zijn om te kijken wat er gebeurt wanneer scholieren niet reeds op de middelbare school hoeven te kiezen tussen bepaalde specialisaties en slechts een algemene vervolgopleiding volgen. Waarschijnlijk zal het nooit zo ver komen, maar de afgelopen jaren zijn wel enkele stappen in die richting gedaan. Op de middelbare school zijn tot in het laatste jaar steeds meer vakken (al dan niet als bijvak) verplicht en op de universiteiten zien we steeds meer brede studies opkomen, studies die als het ware tussen enkele bestaande studies in hangen en zelfs over meerdere faculteiten lopen. Of dit een positief effect zal hebben op de wetenschap als geheel? De tijd zal het ons leren.


\section{Oplossingen voor de problemen}


\subsection{De veerman in de persoon van \emph{Dijksterhuis}}

Ook \emph{Eduard Jan Dijksterhuis}, wetenschapshistoricus, ziet eind jaren 60 de groter wordende kloof tussen de twee culturen. Gedurende zijn hele leven heeft hij getracht in Nederland deze volgens hem heilloze tegenstelling op te heffen. \emph{Dijksterhuis} is een van de voorbeelden waar een b\`eta wetenschapper ook de alfa cultuur verovert. Als wiskundige werd hij opgenomen in de afdeling Letteren van de Koninklijke Akademie, voor zijn wetenschapshistorische werk ontving hij de P.C. Hooftprijs en op latere leeftijd kreeg hij in Utrecht een dubbelaanstelling in zowel de faculteit van wis- en natuurkunde als in die van letteren en wijsbegeerte. Zoals eerder opgemerkt komt omgekeerd -- succes in de b\`eta cultuur voor een alfa wetenschapper -- maar zelden voor.

\paragraph{}

Als wetenschapshistoricus zag \emph{Dijksterhuis} in zichzelf \emph{de} mogelijkheid de kloof te overbruggen. In zijn bekende veermanmetafoor ziet hij de kloof gevuld met een sterke stroom water:

\begin{quote}
``Stroomopwaarts gaande zult gij echter een veer aantreffen dat u naar de overzijde kan brengen. Het veer heet geschiedenis der exacte wetenschappen en ik zal mij gelukkig prijzen, wanneer ik uw veerman mag zijn.''
\end{quote}

Met de overzijde bedoelt \emph{Dijksterhuis} de b\`eta wetenschappen. Deze zijde ziet \emph{Dijksterhuis} niet per s\'e als `de alfa cultuur', maar eerder als `de niet-b\`eta cultuur'. Eigenlijk doet dit detail van tweedeling niet zo zeer ter zake, in vrijwel alle tweedelingen wordt in de eerste plaats uitgegaan van een scherp begrensde b\`eta cultuur en logischerwijs wordt de alfa cultuur daar tegenover gezet.

In ieder geval is het duidelijk dat hij als veerman de `leek op alfa gebied' ge\"\i{}nteresseerd denkt te kunnen maken voor de b\`eta wetenschappen. Hier zien we dus dat ook \emph{Dijksterhuis} een onderscheid maakt tussen alfa en b\`eta -- vooral de alfa cultuur zou een veerman nodig hebben om zich te kunnen verdiepen in de b\`eta cultuur. \emph{Dijksterhuis} heeft zich met dit doel voor ogen altijd ingezet de alfa cultuur te interesseren voor de b\`eta cultuur. Hij was ervan overtuigd dat de geschiedenis van de b\`eta wetenschappen hiervoor bij uitstek geschikt is.


\subsection{Andere veermannen aanwezig?}

Maar \emph{Dijksterhuis} is inmiddels overleden. Was hij de enige veerman, of vinden we verder stroomopwaarts andere veren? Natuurlijk zijn er meer wetenschapshistorici in Nederland en wellicht zien sommigen van hen zichzelf ook als veerman. Misschien hoeven we ons niet te beperken tot de wetenschapshistorici, maar komen ook anderen in aanmerking voor de functie als veerman.

\paragraph{}

Zo is er waarschijnlijk geen andere Nederlandse schrijver te vinden die zo veel verschillende b\`eta wetenschappen in zijn literaire werk weet te stoppen als \emph{Gerrit Krol}. Ook \emph{Krol} komt uit de b\`eta cultuur, hij studeerde wiskunde en werkte een half leven als informaticus en systeemanalist voor Shell. Hij weet de alfa cultuur niet alleen te inspireren door middel van allerlei passages over de exacte en natuurwetenschappen, hij analyseert ook de tweedeling in verschillende lezingen en publicaties. De conclusie van zijn openingscollege van de Vermeerlezing aan de TU Delft in 2001 is dat de verhouding tussen alfa en b\`eta een veelvormige is, maar dat het doorgaans niet meer dan een flirt wordt. Hij vindt het vreemd dat de twee aparte culturen bestaan en tegelijkertijd voor een bepaald soort evenwicht lijken te zorgen. Er is geen strijd of ruzie, hooguit een flinke eigendunk, zo karakteriseert \emph{Krol} de verhouding tussen de twee culturen.

\paragraph{}

Een ander sprekend voorbeeld van een veerman is \emph{Douglas Hofstadter}. Misschien moeten we hier niet van een veerman spreken, maar zelfs van een bruggenbouwer. Hij weet in zijn beroemde boek \emph{G\"odel, Escher, Bach} de kunsten permanent te verbinden met de exacte wetenschappen. Niet alleen de veelvormigheid van de tekst -- hij wisselt beschouwingen af met metaforische dialogen die opgebouwd zijn volgens steeds een ander stuk van \emph{Bach} -- zorgt voor deze verbinding, maar ook inhoudelijk weet hij de kunst van \emph{Escher} en de muziek van \emph{Bach} op wonderlijke wijze te verbinden met de wiskunde van \emph{G\"odel}. Door veel computergekken wordt het boek als `bijbel' gezien, maar het be\"\i{}vloedde ook een generatie van filosofen, wiskundigen en taalkundigen. Daarbij brengt \emph{Hofstadter} de b\`eta wetenschappen op zo'n romantische manier, dat het de alfa persoon haast wel moet interesseren. Als dat geen typisch voorbeeld is van het overbruggen van de kloof...

\paragraph{}

Om nog een aardig voorbeeld te noemen: in 2000 verscheen de dichtbundel \emph{Wis- en natuurlyriek}\cite{DrsP} door \emph{Drs. P} (pseudoniem van Heinz Polzer) en \emph{Marjolein Kool}. In deze dichtbundel doen de schrijvers een poging zowel alfa's als b\`eta's te lokken met ``onderwerpen uit de exacte vakken gehuld in vormvast verzen''. Honderdtien van deze gedichten (plus nog acht in het `chemisch supplement') proberen de b\`eta cultuur in een alfa sfeer te gieten. Een voorbeeld van een kort gedicht door \emph{Drs.P}:

\begin{quote}
Niels Bohr \\
Kwam nucleaire wetenswaardigheden op het spoor \\
en als enthousiast en openhartig type \\
Publiceerde hij zijn correspondentieprincipe.
\end{quote}


\subsection{De techniek als brug}

Vanuit een ander gezichtspunt kunnen we de techniek als brug zien. Eigenlijk doet \emph{Krol} dit ook al wanneer hij de boekdrukkunst als voorbeeld van een brug noemt. In dit voorbeeld brengt de techniek de natuurkundige en de literaire wetenschappen bij elkaar.

We kunnen dit idee toepassen op andere technieken, zoals de tv en video, maar ook op het gebruik van nieuwe materialen en technieken binnen de beeldende en de schilderskunst. Het zijn meestal de exacte en natuurwetenschappen die het middel brengen, terwijl de alfa cultuur het doel brengt. Ze maken niet alleen handig gebruik van elkaar, maar zijn zelfs afhankelijk van elkaar. Video kunstenaars bestuderen te technische achtergrond van film en video en hebben steeds meer kennis van de computer. De alfa wereld maakt via de techniek kennis met de b\`eta wereld.

\paragraph{}

We kunnen de techniek ook op een andere manier als brug zien fungeren, in de vorm van het internet. Is het namelijk niet zo dat het internet voor het eerst iedereen van iedere cultuur in staat stel gemakkelijk en snel andere culturen te verkennen? En maakt tegenwoordig niet iedereen gebruik van internet? Sommigen zeggen dat de digitale revolutie de kloof tussen kenners en niet-kenners alleen maar groter maakt. Maar we zien juist het omgekeerde gebeuren. De niet-kenners verdwijnen en iedereen gaat mee in de digitale revolutie. Het zijn alleen de ouderen die niet opgegroeid zijn met de computer die er niet in mee kunnen. De jeugd weet daarentegen niet beter. Met een druk op de knop staat een uitleg van de kooi van \emph{Faraday} op je scherm. Het is niet moeilijk in te zien dat dit veel mogelijkheden biedt de kloof te overbruggen.


\chapter{Een conclusie}


\section{De kloof is er, of hij nou bestaat of niet}

Wat kunnen we met al deze kennis? Een analyserende conclusie is snel getrokken: hoezeer de kloof tussen de alfa en de b\`eta cultuur ook weg te praten valt, al dan niet filofisch, in de praktijk hebben we er dagelijks mee te maken. Of deze theoretisch gezien nou wel of niet bestaat is dan eigenlijk niet zo belangrijk.

Niet voor niets maakte de lezing van \emph{Snow} zo veel los. Ook al kwam er veel kritiek, veel mensen herkenden de problemen die \emph{Snow} aankaartte in \emph{The Two Cultures}. Maar ook zonder deze lezing zou de discussie op gang gekomen zijn. Het is niet zo dat \emph{Snow} een verborgen probleem `ontdekte', hij was slechts de eerste die zijn waarnemingen in een groter perspectief plaatste, waardoor hij veel aandacht kreeg.


\section{Wat moet er gebeuren?}

Eigenlijk moeten we ons eerste afvragen \`of er wel iets moet gebeuren. Waarschijnlijk zal het antwoord geen volmondig `ja' zijn. Er zijn weliswaar problemen, maar het lijkt erop dat enige afstand tussen de wetenschappen noodzakelijk is voor ontwikkeling. We zouden de lespakketen op de middelbare scholen nog breder kunnen maken, we zouden studies als wiskunde, natuurkunde en Nederlands kunnen opofferen voor nieuwe bredere studies. Maar het lijkt er niet op dat dat verstandig is. Diepgaande kennis en daarmee specialisatie blijft van belang.

\paragraph{}

Een andere weg is er een die niet het onderscheid probeert weg te nemen, maar de betrokkenheid van de alfa cultuur met de b\`eta cultuur en andersom te vergroten. De veerman van \emph{Dijksterhuis} kan hier een grote rol in spelen, maar ook schrijvers als \emph{Hofstadter} en \emph{Krol} leveren hier een aardige bijdrage aan. Het verenigen van de twee culturen is onbegonnen werk, want, zoals \emph{Krol} zegt: ``Beide genres lijken elkaar uit te sluiten: een alfa is alles behalve een be\`eta en omgekeerd, een b\`eta kan geen alfa zijn''. We kunnen slechts proberen ze kennis met elkaar te laten maken.


\begin{thebibliography}{10}

\bibitem{Snow}C. P. Snow: \emph{The Two Cultures and the Scientific Revolution}, Cambridge University Press, Cambridge, 1960.
\bibitem{Leavis}R. R. Leavis: \emph{Two Cultures? The Significance of C. P. Snow}, The Spectator, Cambridge, 1961.
\bibitem{Hooykaas}R. Hooykaas: \emph{Biografisch Woordenboek van Nederland 1}, Nijhoff, Den Haag, 1979.
\bibitem{Hofstadter}D. R. Hofstadter: \emph{G\"odel, Escher, Bach. Een eeuwige gouden band}, Olympus, 1985.
\bibitem{Berkel}K. van Berkel: \emph{De wetenschapsgeschiedenis als brug tussen de twee culturen}, Universiteit van Utrecht, Utrecht, 1997.
\bibitem{Krolea}G. Krol ea: \emph{De trots van alfa en b\`ta}, De Bezige Bij, Amsterdam, 1997.
\bibitem{DrsP}Drs. P, Marjolein Kool: \emph{Wis- en natuurkundelyriek. Met chemisch supplement}, Nijgh \& van Ditmar, Amsterdam, 2000.
\bibitem{Krol}G. Krol: \emph{De onhandige mens. Vermeerlezing 2001}, B\`eta Imaginations, Delft, 2001.
\bibitem{Sparreboom}M. Sparreboom: \emph{Geesteswetenschappen: beleid en organisatie}, Erasmus Universiteit Rotterdam, Rotterdam, 2002.

\end{thebibliography}


\end{document}
