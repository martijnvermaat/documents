\chapter{Een conclusie}


\section{De kloof is er, of hij nou bestaat of niet}

Wat kunnen we met al deze kennis? Een analyserende conclusie is snel getrokken: hoezeer de kloof tussen de alfa en de b\`eta cultuur ook weg te praten valt, al dan niet filofisch, in de praktijk hebben we er dagelijks mee te maken. Of deze theoretisch gezien nou wel of niet bestaat is dan eigenlijk niet zo belangrijk.

Niet voor niets maakte de lezing van \emph{Snow} zo veel los. Ook al kwam er veel kritiek, veel mensen herkenden de problemen die \emph{Snow} aankaartte in \emph{The Two Cultures}. Maar ook zonder deze lezing zou de discussie op gang gekomen zijn. Het is niet zo dat \emph{Snow} een verborgen probleem `ontdekte', hij was slechts de eerste die zijn waarnemingen in een groter perspectief plaatste, waardoor hij veel aandacht kreeg.


\section{Wat moet er gebeuren?}

Eigenlijk moeten we ons eerste afvragen \`of er wel iets moet gebeuren. Waarschijnlijk zal het antwoord geen volmondig `ja' zijn. Er zijn weliswaar problemen, maar het lijkt erop dat enige afstand tussen de wetenschappen noodzakelijk is voor ontwikkeling. We zouden de lespakketen op de middelbare scholen nog breder kunnen maken, we zouden studies als wiskunde, natuurkunde en Nederlands kunnen opofferen voor nieuwe bredere studies. Maar het lijkt er niet op dat dat verstandig is. Diepgaande kennis en daarmee specialisatie blijft van belang.

\paragraph{}

Een andere weg is er een die niet het onderscheid probeert weg te nemen, maar de betrokkenheid van de alfa cultuur met de b\`eta cultuur en andersom te vergroten. De veerman van \emph{Dijksterhuis} kan hier een grote rol in spelen, maar ook schrijvers als \emph{Hofstadter} en \emph{Krol} leveren hier een aardige bijdrage aan. Het verenigen van de twee culturen is onbegonnen werk, want, zoals \emph{Krol} zegt: ``Beide genres lijken elkaar uit te sluiten: een alfa is alles behalve een be\`eta en omgekeerd, een b\`eta kan geen alfa zijn''. We kunnen slechts proberen ze kennis met elkaar te laten maken.
