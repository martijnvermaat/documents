\begin{thebibliography}{10}

\bibitem{Snow}C. P. Snow: \emph{The Two Cultures and the Scientific Revolution}, Cambridge University Press, Cambridge, 1960.
\bibitem{Leavis}R. R. Leavis: \emph{Two Cultures? The Significance of C. P. Snow}, The Spectator, Cambridge, 1961.
\bibitem{Hooykaas}R. Hooykaas: \emph{Biografisch Woordenboek van Nederland 1}, Nijhoff, Den Haag, 1979.
\bibitem{Hofstadter}D. R. Hofstadter: \emph{G\"odel, Escher, Bach. Een eeuwige gouden band}, Olympus, 1985.
\bibitem{Berkel}K. van Berkel: \emph{De wetenschapsgeschiedenis als brug tussen de twee culturen}, Universiteit van Utrecht, Utrecht, 1997.
\bibitem{Krolea}G. Krol ea: \emph{De trots van alfa en b\`ta}, De Bezige Bij, Amsterdam, 1997.
\bibitem{DrsP}Drs. P, Marjolein Kool: \emph{Wis- en natuurkundelyriek. Met chemisch supplement}, Nijgh \& van Ditmar, Amsterdam, 2000.
\bibitem{Krol}G. Krol: \emph{De onhandige mens. Vermeerlezing 2001}, B\`eta Imaginations, Delft, 2001.
\bibitem{Sparreboom}M. Sparreboom: \emph{Geesteswetenschappen: beleid en organisatie}, Erasmus Universiteit Rotterdam, Rotterdam, 2002.

\end{thebibliography}