\chapter{De problemen van de kloof}


\section{Wat zijn de gevolgen van de kloof?}


\subsection{Negatieve gevolgen}

Waarom houdt de koof tussen de twee culturen zoveel mensen bezig? Waarom volgen er vaak verhitte discussies uit? Er moeten problemen vastzitten aan de afstand tussen alfa en b\`eta. \emph{Snow} noemt als belangrijkste problemen de miscommunicatie en het onbegrip aan beide kanten (maar dan vooral aan de alfa kant). De b\`eta wetenschappers zijn volgens \emph{Snow} niet goed op de hoogte van wat de maatschappij nodig heeft en dus niet praktisch genoeg bezig. Maar de meeste schuld ligt bij de alfa wetenschappers, zo zegt \emph{Snow}. Zij ontkennen hardnekkig het belang van de b\`eta wetenschap, leven met een bord voor hun kop en zien niet in hoe ze van de b\`eta wetenschappen gebruik kunnen maken. \emph{Snow} gaat zelfs zo ver om te stellen dat deze problemen de wetenschappelijke revolutie tegenhouden en dat het westen op deze manier de race verliest van de Sovjet Unie. Dit is overigens ook een makkelijk punt van kritiek op \emph{The Two Cultures}, want wat is er tegenwoordig, wanneer de kloof nog altijd bestaat, nou helemaal over van de zogenaamde koppositie van Rusland op wetenschappelijk en economisch gebied?

\paragraph{}

Een ander probleem van de, volgens velen overigens steeds groter wordende, kloof is de gebrekkige uitwisseling van kennis tussen de twee culturen. De wetenschappers zijn over het algemeen niet direct vrijgevig als het op de `overkant' aankomt en andersom wordt er bij het verzamelen van kennis voornamelijk uit eigen bron geput, de bron van de eigen cultuur.

Gebrekkige samenwerking komt het duidelijkst naar boven bij gedwongen samenwerking. Denk hierbij aan onderzoeksgebieden als kunstmatige intelligentie, computerspraak, maar ook aan de alfa wetenschapper die samen moet werken met de statisticus. Over het algemeen blijft het contact beperkt, of neigt het hele project aan een kant van de kloof te blijven hangen. Als gevolg van de grote afstand komt het vaak voor dat wetenschappers zich eerst lange tijd moeten inlezen in elkaars vakgebied alvorens samen onderzoek te kunnen verrichten. Zou nu de afstand kleiner zijn, de andere cultuur bekender zijn, dan was samenwerking een stuk gemakkelijker.

\paragraph{}

Naast praktische problemen hebben vooral alfa wetenschappers nogal eens wat te klagen over de tweedeling. Want bij een tweedeling is er over het algemeen een kant die het moet ontgelden en dat is volgens velen in dit geval de alfa cultuur. We hebben eerder in \ref{sec:alfaenbeta} het woord `science' al genoemd. Aan de betekenis daarvan is al goed te zien hoe men in de Westerse cultuur tegen de wetenschap aan kijkt: daar valt in eerste instantie alleen de b\`eta cultuur onder. Voor de alfa wetenschappen wordt doorgaans het woord `humanities' gebruikt. De mens- en geesteswetenschappen willen van dit imago af en betogen daarom tegenwoordig regelmatig dat de hele tweedeling slechts een verzinsel is.


\subsection{Positieve gevolgen}

Zijn er dan misschien ook redenen om blij te zijn met de kloof? We kunnen ons voorstellen dat specialisatie noodzakelijk is voor verdere ontwikkeling en in dat geval is aan enige afstand tussen de wetenschappen niet te ontkomen. Het lijkt haast onmogelijk tegenwoordig nog diepgaande kennis op meerdere, ver uit elkaar liggende, wetenschappelijke gebieden te hebben. Rond het begin van de Renaissance zien we weliswaar de \emph{Homo Universalis} met zelfs verschillende wetenschappelijke successen, maar sindsdien is iets dergelijks toch nauwelijks meer voor gekomen.

Kunnen we zelfs stellen dat het verwijderen van de kloof tussen alfa en b\`eta negatieve gevolgen zal hebben op de wetenschappelijke vooruitgang? Dat gaat misschien wat ver. Een aardig experiment zou zijn om te kijken wat er gebeurt wanneer scholieren niet reeds op de middelbare school hoeven te kiezen tussen bepaalde specialisaties en slechts een algemene vervolgopleiding volgen. Waarschijnlijk zal het nooit zo ver komen, maar de afgelopen jaren zijn wel enkele stappen in die richting gedaan. Op de middelbare school zijn tot in het laatste jaar steeds meer vakken (al dan niet als bijvak) verplicht en op de universiteiten zien we steeds meer brede studies opkomen, studies die als het ware tussen enkele bestaande studies in hangen en zelfs over meerdere faculteiten lopen. Of dit een positief effect zal hebben op de wetenschap als geheel? De tijd zal het ons leren.


\section{Oplossingen voor de problemen}


\subsection{De veerman in de persoon van \emph{Dijksterhuis}}

Ook \emph{Eduard Jan Dijksterhuis}, wetenschapshistoricus, ziet eind jaren 60 de groter wordende kloof tussen de twee culturen. Gedurende zijn hele leven heeft hij getracht in Nederland deze volgens hem heilloze tegenstelling op te heffen. \emph{Dijksterhuis} is een van de voorbeelden waar een b\`eta wetenschapper ook de alfa cultuur verovert. Als wiskundige werd hij opgenomen in de afdeling Letteren van de Koninklijke Akademie, voor zijn wetenschapshistorische werk ontving hij de P.C. Hooftprijs en op latere leeftijd kreeg hij in Utrecht een dubbelaanstelling in zowel de faculteit van wis- en natuurkunde als in die van letteren en wijsbegeerte. Zoals eerder opgemerkt komt omgekeerd -- succes in de b\`eta cultuur voor een alfa wetenschapper -- maar zelden voor.

\paragraph{}

Als wetenschapshistoricus zag \emph{Dijksterhuis} in zichzelf \emph{de} mogelijkheid de kloof te overbruggen. In zijn bekende veermanmetafoor ziet hij de kloof gevuld met een sterke stroom water:

\begin{quote}
``Stroomopwaarts gaande zult gij echter een veer aantreffen dat u naar de overzijde kan brengen. Het veer heet geschiedenis der exacte wetenschappen en ik zal mij gelukkig prijzen, wanneer ik uw veerman mag zijn.''
\end{quote}

Met de overzijde bedoelt \emph{Dijksterhuis} de b\`eta wetenschappen. Deze zijde ziet \emph{Dijksterhuis} niet per s\'e als `de alfa cultuur', maar eerder als `de niet-b\`eta cultuur'. Eigenlijk doet dit detail van tweedeling niet zo zeer ter zake, in vrijwel alle tweedelingen wordt in de eerste plaats uitgegaan van een scherp begrensde b\`eta cultuur en logischerwijs wordt de alfa cultuur daar tegenover gezet.

In ieder geval is het duidelijk dat hij als veerman de `leek op alfa gebied' ge\"\i{}nteresseerd denkt te kunnen maken voor de b\`eta wetenschappen. Hier zien we dus dat ook \emph{Dijksterhuis} een onderscheid maakt tussen alfa en b\`eta -- vooral de alfa cultuur zou een veerman nodig hebben om zich te kunnen verdiepen in de b\`eta cultuur. \emph{Dijksterhuis} heeft zich met dit doel voor ogen altijd ingezet de alfa cultuur te interesseren voor de b\`eta cultuur. Hij was ervan overtuigd dat de geschiedenis van de b\`eta wetenschappen hiervoor bij uitstek geschikt is.


\subsection{Andere veermannen aanwezig?}

Maar \emph{Dijksterhuis} is inmiddels overleden. Was hij de enige veerman, of vinden we verder stroomopwaarts andere veren? Natuurlijk zijn er meer wetenschapshistorici in Nederland en wellicht zien sommigen van hen zichzelf ook als veerman. Misschien hoeven we ons niet te beperken tot de wetenschapshistorici, maar komen ook anderen in aanmerking voor de functie als veerman.

\paragraph{}

Zo is er waarschijnlijk geen andere Nederlandse schrijver te vinden die zo veel verschillende b\`eta wetenschappen in zijn literaire werk weet te stoppen als \emph{Gerrit Krol}. Ook \emph{Krol} komt uit de b\`eta cultuur, hij studeerde wiskunde en werkte een half leven als informaticus en systeemanalist voor Shell. Hij weet de alfa cultuur niet alleen te inspireren door middel van allerlei passages over de exacte en natuurwetenschappen, hij analyseert ook de tweedeling in verschillende lezingen en publicaties. De conclusie van zijn openingscollege van de Vermeerlezing aan de TU Delft in 2001 is dat de verhouding tussen alfa en b\`eta een veelvormige is, maar dat het doorgaans niet meer dan een flirt wordt. Hij vindt het vreemd dat de twee aparte culturen bestaan en tegelijkertijd voor een bepaald soort evenwicht lijken te zorgen. Er is geen strijd of ruzie, hooguit een flinke eigendunk, zo karakteriseert \emph{Krol} de verhouding tussen de twee culturen.

\paragraph{}

Een ander sprekend voorbeeld van een veerman is \emph{Douglas Hofstadter}. Misschien moeten we hier niet van een veerman spreken, maar zelfs van een bruggenbouwer. Hij weet in zijn beroemde boek \emph{G\"odel, Escher, Bach} de kunsten permanent te verbinden met de exacte wetenschappen. Niet alleen de veelvormigheid van de tekst -- hij wisselt beschouwingen af met metaforische dialogen die opgebouwd zijn volgens steeds een ander stuk van \emph{Bach} -- zorgt voor deze verbinding, maar ook inhoudelijk weet hij de kunst van \emph{Escher} en de muziek van \emph{Bach} op wonderlijke wijze te verbinden met de wiskunde van \emph{G\"odel}. Door veel computergekken wordt het boek als `bijbel' gezien, maar het be\"\i{}vloedde ook een generatie van filosofen, wiskundigen en taalkundigen. Daarbij brengt \emph{Hofstadter} de b\`eta wetenschappen op zo'n romantische manier, dat het de alfa persoon haast wel moet interesseren. Als dat geen typisch voorbeeld is van het overbruggen van de kloof...

\paragraph{}

Om nog een aardig voorbeeld te noemen: in 2000 verscheen de dichtbundel \emph{Wis- en natuurlyriek}\cite{DrsP} door \emph{Drs. P} (pseudoniem van Heinz Polzer) en \emph{Marjolein Kool}. In deze dichtbundel doen de schrijvers een poging zowel alfa's als b\`eta's te lokken met ``onderwerpen uit de exacte vakken gehuld in vormvast verzen''. Honderdtien van deze gedichten (plus nog acht in het `chemisch supplement') proberen de b\`eta cultuur in een alfa sfeer te gieten. Een voorbeeld van een kort gedicht door \emph{Drs.P}:

\begin{quote}
Niels Bohr \\
Kwam nucleaire wetenswaardigheden op het spoor \\
en als enthousiast en openhartig type \\
Publiceerde hij zijn correspondentieprincipe.
\end{quote}


\subsection{De techniek als brug}

Vanuit een ander gezichtspunt kunnen we de techniek als brug zien. Eigenlijk doet \emph{Krol} dit ook al wanneer hij de boekdrukkunst als voorbeeld van een brug noemt. In dit voorbeeld brengt de techniek de natuurkundige en de literaire wetenschappen bij elkaar.

We kunnen dit idee toepassen op andere technieken, zoals de tv en video, maar ook op het gebruik van nieuwe materialen en technieken binnen de beeldende en de schilderskunst. Het zijn meestal de exacte en natuurwetenschappen die het middel brengen, terwijl de alfa cultuur het doel brengt. Ze maken niet alleen handig gebruik van elkaar, maar zijn zelfs afhankelijk van elkaar. Video kunstenaars bestuderen te technische achtergrond van film en video en hebben steeds meer kennis van de computer. De alfa wereld maakt via de techniek kennis met de b\`eta wereld.

\paragraph{}

We kunnen de techniek ook op een andere manier als brug zien fungeren, in de vorm van het internet. Is het namelijk niet zo dat het internet voor het eerst iedereen van iedere cultuur in staat stel gemakkelijk en snel andere culturen te verkennen? En maakt tegenwoordig niet iedereen gebruik van internet? Sommigen zeggen dat de digitale revolutie de kloof tussen kenners en niet-kenners alleen maar groter maakt. Maar we zien juist het omgekeerde gebeuren. De niet-kenners verdwijnen en iedereen gaat mee in de digitale revolutie. Het zijn alleen de ouderen die niet opgegroeid zijn met de computer die er niet in mee kunnen. De jeugd weet daarentegen niet beter. Met een druk op de knop staat een uitleg van de kooi van \emph{Faraday} op je scherm. Het is niet moeilijk in te zien dat dit veel mogelijkheden biedt de kloof te overbruggen.
