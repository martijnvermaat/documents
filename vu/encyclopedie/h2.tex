\chapter{De kloof tussen de twee culturen}


\section{De `two cultures' van \emph{Snow}}


\subsection{\emph{C.P. Snow}}

Wanneer we het hebben over de kloof tussen de alfa en de b\`eta cultuur kunnen we onmogelijk voorbij gaan aan \emph{C.P. Snow} en zijn \emph{The Two Cultures and the Scientific Revolution}\cite{Snow}. In 1956 gebruikte \emph{Snow} `The Two Cultures' voor het eerst als titel van een artikel in \emph{The New Statesman}, later heeft hij de inhoud van dit artikel uitgewerkt tot zijn lezing voor de universiteit van Cambridge in 1959: \emph{The Two Cultures and the Scientific Revolution}. In de eerste plaats was \emph{Snow}, na zijn korte bestaan als natuurwetenschapper, een schrijver van romans, maar hij werd onverwacht wereldwijd bekend na de publicatie van zijn lezing in 1959.


\subsection{\emph{The Two Cultures and the Scientific Revolution}}

In deze \emph{Rede Lecture} bespreekt \emph{Snow} de problemen van de wetenschap in het na-oorlogse Engeland. Er zou onvoldoende begrepen zijn wat de betekenis van de industri\"ele revolutie was en het onderwijs zou daardoor te weinig aangepast zijn. De ervaringswereld van natuurwetenschappers verschilde te veel met die van de literaire intelligentia. Tijdens de wetenschappelijke revolutie die volgde uit de industri\"ele revolutie faalden natuurwetenschappers te denken in termen van toepasbare kennis. \emph{Snow} bekijkt verschillende onderwijssystemen wereldwijd en komt tot de conclusie dat het Russische onderwijssysteem nog het best uitgerust lijkt voor de wetenschappelijke revolutie.

De twee genoemde culturen communiceren volgens hem te weinig met elkaar, maar later in zijn lezing stapt hij over op meer sociaal maatschappelijke beschouwingen en heeft hij het vooral over omvormingen van het onderwijs opdat Engeland beter kan concurreren met de Sovjet Unie. Hoewel het debat dat uit zijn lezing voort kwam (en tot op heden voort duurt) zich vooral richt op de verschillen tussen de alfa en b\`eta wetenschappen is dat niet alleen waar het in de lezing om draait.


\subsection{Reacties op \emph{The Two Cultures}}

Ondanks het feit dat het \emph{Snow} in zijn lezing vooral gaat om het bredere, maatschappelijke beeld van de situatie en mogelijke verbeteringen daarvan, komt er een flink aantal stevige uitspraken over de alfa en b\`eta culturen in voor. In eerste instantie is te lezen dat de schuld in beide kampen gevonden kan worden, wetenschappers uit beide culturen staan niet genoeg open voor de andere cultuur. Maar even verderop lijkt hij terug te komen op deze mening. Zo betoogt hij, niet gehinderd door enige mens- of geesteswetenschappelijke kennis, dat vooral de alfa wetenschappers zich schuldig maken aan het behouden van de kloof tussen de twee culturen. Zo zouden ze de centrale functie van de b\`eta wetenschappen in de maatschappij niet onderkennen, behalve als toekomstbeeld, maar in dat geval wensen ze er gewoon op dat de toekomst niet bestaat.

Hij gaat zelfs verder. De b\`eta cultuur kent meer discussie, doorgaans veel significanter en altijd op een hoger conceptueel niveau, dan de alfa cultuur. Dat schiet velen in het verkeerde keelgat. Tot meer dan 50 jaar na de publicatie van \emph{The Two Cultures} krijgt het stuk regelmatig zware kritiek te verduren.\cite{Sparreboom} Nog regelmatig worden passages aangehaald, waarbij vooral alfa en gamma wetenschappers de significantie ervan proberen te minimaliseren. Vaak wordt daarbij ook nog de onrechtvaardige betekenis van het Engelse woord `science' aangehaald, waar zoals genoemd alleen de b\`eta wetenschappen mee bedoeld worden.

\paragraph{}

De bekendste en meeste directe kritiek op \emph{The Two Cultures} kwam echter al na een jaar, in de vorm van een lezing van \emph{F.R. Leavis}\cite{Leavis} welke gepubliceerd is in \emph{The Spectator}. Bedenk dat in die tijd de Engelse pers anders bekend stond dan tegenwoordig en doorgaans nog zeer beleefd te werk ging. Dat is niet af te zien aan de reactie van \emph{Leavis}, die werkelijk ieder argument, iedere zin en ieder woord van \emph{Snow} vertrapt. Onder de titel \emph{Two Cultures? The Significance of C. P. Snow} lezen we dat \emph{Leavis} ervan walgde dat \emph{Snow} compleet voorbij ging aan geschiedenis, literatuur, de geschiedenis de beschaving, en de menselijke significantie van de industri\"ele revolutie. Om een uitspraak van \emph{Leavis} te noemen: ``It is ridiculous to credit him with any capacity for serious thinking about the problems on which he offers to advise the world.''

\paragraph{}

Of \emph{Snow} nu wel of niet de schuld bij de juiste kant legde, het idee van de tweedeling binnen de wetenschap sprak hoe dan ook veel mensen aan. Hoewel velen tegenwoordig het bestaan van deze tweedeling compleet ontkennen (de reden is doorgaans \`of de gamma cultuur heeft de kloof altijd al gedempt, \`of er bestaat in wezen geen verschil tussen natuurwetenschappen en geesteswetenschappen) kan het geen toeval zijn dat zovelen op zijn lezing reageerden. Onder andere \emph{Snow} zelf concludeert enkele jaren later dat, afgezien van de geldigheid van zijn redeneringen, het grote aantal reacties wel moet betekenen dat er in ieder geval een kern van waarheid in zijn verhaal zit.\cite{Berkel} En daar is wat voor te zeggen. Of de tweedeling theoretisch niet bestaat of niet kan bestaan doet feitelijk niet terzake, in de praktijk hebben reeds zo veel mensen de kloof gezien en is er reeds zo veel over geschreven dat er tenminste \emph{iets} aan de hand moet zijn.


\section{Is er eigenlijk wel een kloof?}


Een vraag die we ons zelf misschien eerder hadden moeten stellen. We hebben min of meer vast gesteld dat er verschillende uitersten bestaan binnen de wetenschap, met aan de ene kant de alfa wetenschappen en aan de andere kant de b\`eta wetenschappen. Er is veel geschreven over de steeds breder wordende kloof tussen deze twee. Maar bestaat deze kloof \"uberhaupt wel? Als hij in de praktijk al wordt gezien (al dan niet als een probleem), kijken we dan niet door de verkeerde bril? Beelden we ons niet die hele kloof in?


\subsection{Er is maar een wetenschap}\label{sec:eenwetenschap}

Zoals al enkele keren genoemd betogen velen tegenwoordig, meer dan 40 jaar na \emph{Snow}, dat het onderscheid tussen de alfa wetenschappen en de b\`eta wetenschappen slechts een schijnbaar onderscheid is. Uitgaande van de \emph{Frankfurter Schule}\footnote{De in de jaren dertig ontstane Frankfurter Schule is een groep van ge\"engageerde wetenschappers, die zich afzetten tegen de zogenaamd waardevrije objectieve wetenschap van het logisch positivisme. Er bestaat niet zoiets als objectieve kennis, kennis is altijd gekleurd door de achtergrond en idee\"en van de kenner.} kunnen we tot de conclusie komen dat het bestuderen van de werkelijkheid, de natuur, niet wezenlijk verschilt van het beoefenen van kunst -- het andere uiterste tegenover de b\`eta cultuur. In de kunst heeft men niets te maken met de echte werkelijkheid, men cre\"eert een eigen werkelijkheid. En hoewel de b\`eta wetenschapper het menselijk aspect irrelevant vindt en slechts de werkelijkheid, de waarheid van de natuur, wil bestuderen, gaat het hier volgens de \emph{Frankfurter Schule} niet om een objectieve werkelijkheid, maar ook om een door de mens gecre\"erde werkelijkheid. En daarmee is het belangrijkste verschil tussen alfa en b\`eta verdwenen.

\paragraph{}

Er is nog een manier om de exacte wetenschap met de kunsten te verenigen. In het boek \emph{G\"odel, Escher, Bach -- Een eeuwige gouden band}\cite{Hofstadter} doet \emph{Douglas R. Hofstadter} grote moeite de geniale muziek van \emph{Bach} en de kunstzinnige tekeningen van \emph{Escher} te rijmen met de wiskunde. Een centrale rol spelen de zogenaamde `merkwaardige lussen'. Waar deze zich voor doen gebeuren merkwaardige dingen. \emph{Hofstadter} ziet deze lussen overal terug komen: als basis van de moeilijk te bevatten stelling van \emph{G\"odel} (wiskunde), in de onwerkelijkste delen van \emph{Escher}'s tekeningen en in de mooiste composities van \emph{Bach}. Hij ziet hierin zelfs de sleutel om van wiskunde naar kunst te komen en andersom (en in een ander perspectief: de basis van ware kunstmatige intelligentie).

Een andere aanwijzing in deze richting vinden we in de persoon van \emph{Nietzsche}. Hij beschouwt de filosoof \emph{Socrates} waarvan bekend is dat hij regelmatig in zijn dromen bezocht werd door een figuur die hem steeds weer zei: ``Socrates, musiceer!''. \emph{Socrates} dacht altijd dat de figuur met musiceren doelde op de hoogste muzenkunst filosoferen en kon dus gerust zijn. Maar later in zijn leven, in afwachting van zijn doodvonnis, komt hij met schrik tot het inzicht dat ook het be\"oefenen van muziek bedoeld kon zijn. We kunnen hier ook de tweedeling in ontdekken van de hogere wetenschap tegenover de kunsten. Maar zoals \emph{Nietzsche} opmerkt gaat het om een en hetzelfde, want wanneer de logica tot in haar uitersten bedreven wordt ``slingert zij zich om zichzelf heen om zichzelf tenslotte in de staart te bijten'' en daar moet zij ook ``uiteindelijk omslaan in kunst. En dat is het eigenlijke doel van dit mechanisme''. \emph{Nietzche} heeft het hier duidelijk over dezelfde `merkwaardige lussen' als \emph{Hofstadter} in zijn \emph{G\"odel, Escher, Bach}.

\paragraph{}

Maar laten we ons niet te veel meeslepen door filosofische beschouwingen. Zoals gezegd wordt de tweedeling in de praktijk wel degelijk ervaren en om de mogelijke kloof tussen de twee culturen verder te kunnen onderzoeken zullen we moeten aannemen dat ze in ieder geval bestaan.


\subsection{Er is geen kloof}\label{sec:geenkloof}

Met de aanname dat er een alfa en een b\`eta cultuur bestaat in ons achterhoofd kunnen we kijken of er dan eigenlijk wel een kloof tussen beiden bestaat. Er zijn in wezen vier mogelijkheden. De eerste is simpelweg het bestaan van een kloof, zonder dat we daar iets aan kunnen doen. De tweede mogelijkheid is het bestaan van een kloof met een of meer bruggen die beide kanten met elkaar verbindt. Deze bruggen zijn in feite oplossingen voor een aanwezig probleem en zullen verderop aan bod komen. Het kan ook zo zijn dat er nooit sprake geweest is van een kloof en dat de twee kanten, als we goed kijken, samen slechts een eiland vormen. Dit is de derde mogelijkheid en is in \ref{sec:eenwetenschap} besproken. Onze laatste en vierder mogelijkheid is dat de kloof er in feite wel is, maar reeds lang geleden gedempt is.

\paragraph{}

Volgens velen is de kloof inderdaad gedempt. Uiteraard doelt men dan op de opvulling door de gamma cultuur, de sociale en economische wetenschappen. Waar de alfa cultuur slechts mens en woord en de b\`eta cultuur slechts feit en woord bevat, bevat de gamma cultuur alle drie: mens, feit en woord. Bevinden we ons ergens in de kloof, het gebied tussen de alfa cultuur en de b\`eta cultuur, dan vallen we niet, maar staan we in de gamma cultuur. Deze verbindt beide culturen niet slechts op bepaalde punten zoals een brug dat doet, maar dempt de hele kloof.

Indien dit juist is kunnen we dus vanuit de alfa cultuur via de gamma cultuur naar de b\`eta cultuur lopen en terug. Maar wat merken we dan nog van de kloof? Merken we \"uberhaupt nog dat de kloof er is, al is deze gedempt door de gamma cultuur?


\section{De kloof in de praktijk}


\subsection{Eigenschappen van de culturen}

\emph{Gerrit Krol} geeft in verschillende lezingen typerende beschrijvingen van alfa wetenschappers en b\`eta wetenschappers. De in \ref{sec:geenkloof} genoemde ingredie\"enten -- mens, feit en woord -- komen bijvoorbeeld van \emph{Krol}. In het openingscollege van de Vermeerlezing 2001 aan de TU Delft stelt \emph{Krol}\cite{Krol} dat het in de alfa cultuur om de illusie en in de b\`eta cultuur om het getal draait. In een gedicht bestaat een werkelijkheid, maar deze is niet dezelfde als de `echte werkelijkheid'. Het is een illusie die door iedere lezer anders ge\"\i{}nterpreteerd kan worden, want een alfageest interpreteert een woord niet naar de letter, maar naar de geest. In de b\`eta cultuur daarentegen komen twee verschillende mensen tot dezelfde voorspelling. Wat niet in getallen kan worden uitgedrukt bestaat niet en de enige werkelijkheid is de `echte werkelijkheid' -- illusies bestaan niet.

\emph{Krol} wijst ook op de trots van beide culturen\footnote{Zie ook \emph{De trots van alfa en b\'eta}\cite{Krolea} met verhandelingen van onder andere \emph{Gerrit Krol} en \emph{Hans Achterhuis}, uitgegeven door De Bezige Bij}. Terwijl men in de b\`eta wereld trots is op de aanwezigheid van dingen en de afwezigheid van de mens, is men in de alfa wereld trots op de afwezigheid van de ware werkelijkheid en de aanwezigheid van illusies. Volgens \emph{Krol} kunnen we alfa en b\`eta zien als twee bastions, twee elkaar vijandige families die trots zijn op zichzelf. En alfa is blij dat hij geen b\`eta is en andersom.


\subsection{Asymmetrie}

De tweedeling tussen alfa en b\`eta is overigens niet symmetrisch. Er zijn vele voorbeelden te noemen van b\`eta wetenschappers die uiteindelijk uitstekend uit de voeten blijken te kunnen in de alfa cultuur. Zo ontvingen \emph{Gerrit Krol}, \emph{Rutger Kopland} en \emph{Eduard Jan Dijksterhuis}\cite{Hooykaas} de P.C. Hooftprijs en weet Paul Verhoeven (afgestudeerd wis- en natuurkundige) zich goed staande te houden achter de camera. Andersom zijn voorbeelden moeilijk te vinden. Hebben we ooit een alfa de Nobelprijs voor natuurkunde zien winnen?

Mensen uit de alfa cultuur bekennen het over het algemeen direct als ze niet weten hoe de wetten van \emph{De Morgan} luiden en lijken in veel gevallen zelfs trots op dit gebrek aan kennis. Maar geven natuurkundigen het snel toe als ze niets van muziek weten, of als ze nooit films kijken? Nee, in de meeste gevallen zijn b\`eta wetenschappers zelfs goed geinformeerd als het gaat om de alfa cultuur. Binnen de b\`eta cultuur heerst geen taboe op de alfa cultuur, maar binnen de alfa cultuur wel op de b\`eta cultuur, zo lijkt het.
