\chapter{De twee culturen}


\section{Een verkenning}


\subsection{Culturen?}

Het zal niemand onbekend zijn, het bestaan van twee distincte culturen binnen de wetenschap. Aan de ene kant zien we de mens- en geesteswetenschappen, ook wel alfa cultuur genoemd en aan de andere kant zien we de exacte en natuurwetenschappen, welke de b\`eta cultuur vormen. Het precies aangeven van welke wetenschappen onder welke cultuur vallen is niet zo eenvoudig. Over het algemeen wordt ook een derde cultuur van de sociale en economische wetenschappen, de gamma cultuur, genoemd en in enkele gevallen spreekt men zelfs over een omega cultuur met welke dan filosofie en religie bedoeld wordt.

Een van de weinige dingen die in vrijwel alle verdelingen blijft gelden is dat met de b\`eta cultuur de exacte en natuurwetenschappen bedoeld worden. Deze blijken dus altijd apart van `de rest' gezien te worden. Maar over de verdeling van die rest is men het lang niet altijd eens. Regelmatig wordt dit voor het gemak zelfs op een hoop gegooid om aangeduid te kunnen worden onder een noemer, `alfa' of `alfa-gamma'. Overigens wordt in veel gevallen niet alleen de studie van de kunst, maar ook de kunst zelf hier geplaatst.


\subsection{Verdeling?}

Wat moeten we met een dergelijke verdeling? Waarop is deze eigenlijk gebaseerd? Kijken we naar de alfa cultuur dan kunnen we zeggen dat deze voornamelijk gericht is op gevoel en subjectief van aard is. De b\`eta cultuur daarentegen richt zich voornamelijk op de werkelijkheid en is objectief van aard. Tegenstanders van deze tweedeling op methodologie betogen dat kennis nooit objectief is en er dus zelfs geen wezenlijk verschil bestaat tussen kunst en wetenschap. Kennis wordt immers door de mens zelf gecre\"eerd. Natuurwetenschappers zullen niet akkoord gaan met deze visie, maar onder filosofen is hij een stuk populairder.

Maar toch kunnen we hierin een tweedeling vinden. Waar de alfa cultuur uit gaat van de mens gaat de b\`eta cultuur uit van de werkelijkheid en of deze werkelijkheid dan wel of niet door de mens gecre\"eerd is, het doel van de twee culturen is duidelijk verschillend. Vanaf een hoger niveau lijkt het wellicht een pot nat, maar als natuurkundige zijn de feiten van betekenis en niets anders, terwijl iemand bij het lezen van een roman nooit zijn beklag zal doen bij de schrijver omdat een bepaalde gebeurtenis niet waar gebeurd lijkt te kunnen zijn.


\section{Historie}


\subsection{De Oudheid}

In de Oudheid kunnen we eigenlijk geen duidelijke alfa en b\`eta cultuur herkennen. Toch zijn er al wel enkele aanwijzingen voor een tweedeling te ontdekken. Zo grijpt filosoof \emph{Hans Achterhuis} terug op \emph{Plato}'s mythe van \emph{Prometheus}, uit de dialoog \emph{Protagoras}. De halfgod \emph{Prometheus} rooft van de goden technische kennis en vuur. De oppergod \emph{Zeus} vond dat niet toereikend en zond daarom de bode \emph{Hermes} om de mensen `eergevoel en recht' te brengen. Volgens \emph{Achterhuis} zien we daar reeds de twee culturen. Maar de huidige wetenschappen zijn ingedeeld op na-Middeleeuws model en daarom is in dit verband de Renaissance interessanter om te beschouwen.


\subsection{De \emph{Homo Universalis}}

Toen in Europa na de Middeleeuwen de Renaissance begon (in Itali\"e vanaf de 14$^{de}$ en in heel Europa vanaf de 16$^{de}$ eeuw) bestond bij veel mensen de behoefte in alle, of in ieder geval zo veel mogelijk, gebieden uit te blinken. Wonderbaarlijk genoeg zijn hier verschillende mensen zeer goed in geslaagd en dat heeft ervoor gezorgd dat we nu bij de term \emph{Homo Universalis} denken aan \emph{Da Vinci}, \emph{Michelangelo}, \emph{Copernicus}, \emph{Machiavelli} en anderen.

Maar deze opvatting verdween nog voor het einde van de Renaissance -- vanaf de Verlichting (ruwweg vanaf 1650) is de wetenschap steeds verder opgedeeld in segmenten die onderling steeds minder contact hadden. De \emph{Homo Universalis} kennen we wel nog tot op de dag van vandaag, maar wordt maar zeer zelden als toonaangevend gezien binnen een bepaalde wetenschap, misschien alleen binnen `meta-wetenschappen' als wetenschapsgeschiedenis.


\subsection{Alfa en b\`eta}\label{sec:alfaenbeta}

We kunnen stellen dat de klassieke indeling in alfa, b\`eta en gamma wetenschappen zich gevormd heeft tijdens de Verlichting. Opmerkelijk is de toewijzing van de Griekse letter alfa aan de taal- en geschiedeniswetenschappen. Tegenwoordig wordt door veel mensen de b\`eta cultuur als de hogere van de twee gezien, mede door de hogere significantie in de ontwikkeling van de maatschappij (wellicht kunnen we de huidige cultuur als geheel een `b\`eta cultuur' noemen). Getuige ook de betekenis van het Engelse `science' -- letterlijk `kennis', `weten', maar in de praktijk worden er slechts de exacte en natuurwetenschappen mee bedoeld. Het opstapje hiervoor werd al genomen in de Verlichting, waar feiten en waarheid troef speelden in de wetenschap. Een dergelijke instelling rechtvaardigt de toewijzing van de eerste letter uit het Griekse alfabet aan de b\`eta wetenschappen, zo zouden we denken -- het is niet gebeurd. 

De keuze voor de taal- en geschiedeniswetenschappen is beter te verklaren met behulp van het imperialisme tijdens de Verlichting. De expansie van Europa maakte dat deze wetenschappen belangrijker werden en wellicht werden zij daarvoor beloond met de letter alfa. De exacte en natuurwetenschappen waren `in' , maar minder significant in het dagelijks leven en kregen de letter b\`eta, de letter gamma bleef over voor sociologie en psychologie die zich eindelijk ook los hadden weten te maken van filosofie en theologie.

Zouden de Griekse letters nu opnieuw toegewezen moeten worden aan de verschillende wetenschapsculturen dan zou dit zeker anders gebeuren. De exacte en natuurwetenschappen zouden met de letter alfa heen gaan in deze technologie cultuur.
