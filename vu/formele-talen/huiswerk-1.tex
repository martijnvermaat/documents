\documentclass[a4paper,11pt]{article}
\usepackage[dutch]{babel}
\usepackage{a4}
\usepackage{color, rotating}
\usepackage{latexsym}
\usepackage[
    colorlinks,
    pdftitle={Formele Talen - Inleveropgaven I},
    pdfsubject={Formele Talen},
    pdfauthor={Martijn Vermaat}
]{hyperref}

\title{Formele Talen - Inleveropgaven I}
\author{
    Martijn Vermaat\\
    mvermaat@cs.vu.nl
}
\date{16 november 2004}

\begin{document}
\maketitle


\section*{Opgave 1}


\begin{description}


\item[a)]

Een grammatica die $L$ genereert:

\begin{eqnarray*}
S & \rightarrow & aAb \, | \, aAc \, | \, bAa \, | \, bAc \, | \, cAa \, | \, cAb \nonumber \\
A & \rightarrow & \lambda \, | \, aA \, | \, bA \, | \, cA \nonumber
\end{eqnarray*}

\item[b)]

Een dfa die $L$ accepteerd:\\[1em]

\input{huiswerk-1-opg1b.pdftex_t}


\end{description}


\section*{Opgave 2}


\begin{description}


\item[a)]

Een dfa die $L$ genereert:\\[1em]

\input{huiswerk-1-opg2a.pdftex_t}

\item[b)]

We maken er eerst een nfa van waarbij we de trapstate $q_{3}$ weg laten en
de drie accepterende states vervangen door \'e\'en accepterende state:\\[1em]

\input{huiswerk-1-opg2b-1.pdftex_t}

Vervolgens halen we $q_{1}$ weg volgens het algoritme op pagina's 82 en 83
van het boek:\\[1em]

\input{huiswerk-1-opg2b-2.pdftex_t}

En zo ook $q_{2}$:\\[1em]

\input{huiswerk-1-opg2b-3.pdftex_t}

Volgens het recept uit datzelfde algoritme kunnen we uit deze nfa de
volgende reguliere expressie construeren:

\begin{eqnarray*}
r = (11^{*}00 + 0)^{*} + (11^{*}\lambda + \lambda + 11^{*}0\lambda)
\end{eqnarray*}


\end{description}


\section*{Opgave 3}


We bewijzen dat de taal

\begin{displaymath}
\{a^{n} \, | \, n \mbox{ is \emph{geen} priemgetal}\}
\end{displaymath}

niet regulier is door met behulp van de pompstelling te bewijzen dat het
complement van deze taal niet regulier is.

\paragraph{Bewijs}

We nemen aan dat de taal

\begin{displaymath}
L = \{a^{n} \, | \, n \mbox{ is een priemgetal}\}
\end{displaymath}

regulier is. Omdat $L$ een oneindige taal is (er zijn oneindig veel
priemgetallen) kunnen we de pompstelling gebruiken. Volgens deze stelling
bestaat er een geheel getal $m$, z\'o, dat voor iedere $w \in L$ met
$|w| \geq m$ er een opdeling

\begin{displaymath}
w = xyz
\end{displaymath}

bestaat met $|xy| \leq m$ en $|y| \geq 1$ waarbij

\begin{displaymath}
w_{i} = xy^{i}z
\end{displaymath}

ook in $L$ zit (voor alle $i \geq 0$).

Gegeven het gehele getal $m$, bekijken we de string

\begin{displaymath}
w = a^{n} \, \mbox{ ($n$ is het $m^{de}$ priemgetal).}
\end{displaymath}

Nu bekijken we met behulp van de genoemde opdeling $xyz$ van $w$ de string

\begin{displaymath}
w_{n} = xy^{n}z \, \mbox{ ($n = |w|+1$).}
\end{displaymath}

Nu hebben we $|w_{n}| = |w|+(|w|*|n|)$, waarbij we zeker weten dat deze
lengte deelbaar is door $|w|$ (met als resultaat $|n|+1$). En dus zit $w_{n}$
niet in $L$. Maar de pompstelling zegt dat $w_{n}$ wel in $L$ zit. Dit is
een tegenspraak.

Uit deze tegenspraak concluderen we dat onze aanname niet juist was en dus
dat $L$ regulier is.

Doordat de eigenschap `regulier' gesloten is onder de operatie
complement\footnote{Eigenlijk gebruiken we hier dat het `\emph{niet}-regulier'
zijn gesloten is onder de complement operatie. Dit is ook algemeen geldig,
omdat een dubbel complement de originele taal geeft.}, is ook het complement

\begin{displaymath}
\{a^{n} \, | \, n \mbox{ is \emph{geen} priemgetal}\}
\end{displaymath}

van $L$ niet regulier. Dit wilden we bewijzen.

\hfill\rule{2.1mm}{2.mm}


\section*{Opgave 4}



\end{document}
