\documentclass[a4paper,11pt]{article}
\usepackage[dutch]{babel}
\usepackage{a4}
\usepackage{color, rotating}
\usepackage{latexsym}
\usepackage[
    colorlinks,
    pdftitle={Formele Talen - Inleveropgaven I},
    pdfsubject={Formele Talen},
    pdfauthor={Martijn Vermaat}
]{hyperref}

\title{Formele Talen - Inleveropgaven I}
\author{
    Martijn Vermaat\\
    mvermaat@cs.vu.nl
}
\date{16 november 2004}

\begin{document}
\maketitle


\section*{Opgave 1}


\begin{description}


\item[a)]

Een grammatica die $L$ genereert:

\begin{eqnarray}
S & \rightarrow & aAb \, | \, aAc \, | \, bAa \, | \, bAc \, | \, cAa \, | \, cAb \nonumber \\
A & \rightarrow & \lambda \, | \, aA \, | \, bA \, | \, cA \nonumber
\end{eqnarray}

\item[b)]

Een dfa die $L$ accepteerd:\\[1em]

\input{huiswerk-1-opg1b.pdftex_t}


\end{description}


\section*{Opgave 2}


\begin{description}


\item[a)]

Een dfa die $L$ genereert:\\[1em]

\input{huiswerk-1-opg2a.pdftex_t}


\end{description}


\end{document}
