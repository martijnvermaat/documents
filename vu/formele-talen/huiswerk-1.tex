\documentclass[a4paper,11pt]{article}
\usepackage[dutch]{babel}
\usepackage{a4}
\usepackage{color, rotating}
\usepackage{latexsym}
\usepackage[
    colorlinks,
    pdftitle={Formele Talen - Inleveropgaven I},
    pdfsubject={Formele Talen},
    pdfauthor={Martijn Vermaat}
]{hyperref}

\title{Formele Talen - Inleveropgaven I}
\author{
    Martijn Vermaat\\
    mvermaat@cs.vu.nl
}
\date{16 november 2004}

\begin{document}
\maketitle


\section*{Opgave 1}


\begin{description}


\item[a)]

Een grammatica die $L$ genereert:

\begin{eqnarray*}
S & \rightarrow & aAb \, | \, aAc \, | \, bAa \, | \, bAc \, | \, cAa \, | \, cAb \nonumber \\
A & \rightarrow & \lambda \, | \, aA \, | \, bA \, | \, cA \nonumber
\end{eqnarray*}

\item[b)]

Een dfa die $L$ accepteerd:\\[1em]

\input{huiswerk-1-opg1b.pdftex_t}


\end{description}


\section*{Opgave 2}


\begin{description}


\item[a)]

Een dfa die $L$ genereert:\\[1em]

\input{huiswerk-1-opg2a.pdftex_t}

\item[b)]

We maken er eerst een nfa van waarbij we de trapstate $q_{3}$ weg laten en
de drie accepterende states vervangen door \'e\'en accepterende state:\\[1em]

\input{huiswerk-1-opg2b-1.pdftex_t}

Vervolgens halen we $q_{1}$ weg volgens het algoritme op pagina's 82 en 83
van het boek:\\[1em]

\input{huiswerk-1-opg2b-2.pdftex_t}

En zo ook $q_{2}$:\\[1em]

\input{huiswerk-1-opg2b-3.pdftex_t}

Volgens het recept uit datzelfde algoritme kunnen we uit deze nfa de
volgende reguliere expressie construeren:

\begin{eqnarray*}
r = (11^{*}00 + 0)^{*} + (11^{*}\lambda + \lambda + 11^{*}0\lambda)
\end{eqnarray*}


\end{description}


\section*{Opgave 3}


We bewijzen dat de taal

\begin{displaymath}
\{a^{n} \, | \, n \mbox{ is \emph{geen} priemgetal}\}
\end{displaymath}

niet regulier is door met behulp van de pompstelling te bewijzen dat het
complement van deze taal niet regulier is.

\paragraph{Bewijs.}

We nemen aan dat de taal

\begin{displaymath}
L = \{a^{n} \, | \, n \mbox{ is een priemgetal}\}
\end{displaymath}

regulier is. Omdat $L$ een oneindige taal is (er zijn oneindig veel
priemgetallen) kunnen we de pompstelling gebruiken. Volgens deze stelling
bestaat er een geheel getal $m$, z\'o, dat voor iedere $w \in L$ met
$|w| \geq m$ er een opdeling

\begin{displaymath}
w = xyz
\end{displaymath}

bestaat met $|xy| \leq m$ en $|y| \geq 1$ waarbij

\begin{displaymath}
w_{i} = xy^{i}z
\end{displaymath}

ook in $L$ zit (voor alle $i \geq 0$).

Gegeven het gehele getal $m$, bekijken we de string

\begin{displaymath}
w = a^{n} \, \mbox{ ($n$ is het $m^{de}$ priemgetal).}
\end{displaymath}

Nu bekijken we met behulp van de genoemde opdeling $xyz$ van $w$ de string

\begin{displaymath}
w_{n} = xy^{n}z \, \mbox{ ($n = |w|+1$).}
\end{displaymath}

Nu hebben we $|w_{n}| = |w|+(|w|*|n|)$, waarbij we zeker weten dat deze
lengte deelbaar is door $|w|$ (met als resultaat $|n|+1$). En dus zit $w_{n}$
niet in $L$. Maar de pompstelling zegt dat $w_{n}$ wel in $L$ zit. Dit is
een tegenspraak.

Uit deze tegenspraak concluderen we dat onze aanname niet juist was en dus
dat $L$ regulier is.

Doordat de eigenschap `regulier' gesloten is onder de operatie
complement\footnote{Eigenlijk gebruiken we hier dat het `\emph{niet}-regulier'
zijn gesloten is onder de complement operatie. Dit is ook algemeen geldig,
omdat een dubbel complement de originele taal geeft.}, is ook het complement

\begin{displaymath}
\{a^{n} \, | \, n \mbox{ is \emph{geen} priemgetal}\}
\end{displaymath}

van $L$ niet regulier. Dit wilden we bewijzen.

\hfill\rule{2.1mm}{2.mm}


\section*{Opgave 4}


Omdat $L_{1}$ een reguliere taal over $\{a,b\}$ is, bestaat er een reguliere
expressie $r_{1}$ die precies $L_{1}$ beschrijft.

De reguliere expressie $r_{2}$ defini\"eren we door herschrijfregels op
$r_{1}$ toe te passen:

\begin{eqnarray}
r_{2} & \mapsto & r_{1}' \label{regel:regex} \\
(\psi \omega)' & \mapsto & \psi' \omega + \psi \omega' \label{regel:con} \\
(\psi + \omega)' & \mapsto & \psi' + \omega' \label{regel:keuze} \\
(\psi^{*})' & \mapsto & \psi^{*} \psi' \psi^{*} \label{regel:herh} \\
\alpha' & \mapsto & \lambda \label{regel:atom} \\
\lambda' & \mapsto & \emptyset \label{regel:lambda} \\
\emptyset' & \mapsto & \emptyset \label{regel:empty}
\end{eqnarray}

Waarbij $\psi$ en $\omega$ reguliere expressies zijn en $\alpha$ een
element uit $\{a,b\}$. De taal die door $r_{2}$ beschreven wordt
is regulier.

We zullen nu bewijzen dat de reguliere expressie $r_{2}$ precies $L_{2}$
beschrijft.

\paragraph{Bewijs.}

We bewijzen met inductie naar de structuur van $r_{1}$. Als basisstap
nemen we $r_{1}$ gelijk aan een element uit $\{a,b,\lambda,\emptyset\}$.
Volgens de definitie van $L_{2}$ (uit de opgave) is $L_{2} = \{\lambda\}$
indien $r_{1} \in \{a,b\}$ en $L_{2} = \emptyset$ als $r_{2} \in
\{\lambda,\emptyset\}$.

Regel \ref{regel:atom} uit onze herschrijfregels zegt dat $r_{2}$ geschreven
kan worden als $\lambda$ wanneer $r_{1} \in \{a,b\}$. Regels
\ref{regel:lambda} en \ref{regel:empty} zeggen dat $r_{2}$ geschreven kan
worden als $\emptyset$ wanneer $r_{1} \in \{\lambda,\emptyset\}$. Deze twee
beschrijven precies de taal $\{\lambda\}$ respectievelijk $\emptyset$ en
dus hebben we hiermee ons basisgeval bewezen.

\paragraph{}

Onze inductiestap bestaat uit het bewijzen dat we met onze herschrijfregels
voor iedere reguliere expressie $r_{1}$ een reguliere expressie $r_{2}$
kunnen maken die precies $L_{2}$ beschrijft waarbij we aan mogen nemen dat
dit al bewezen is voor iedere subexpressie van $r_{1}$. We behandelen alle
gevallen van de structuur van $r_{1}$ apart (met uitzondering van de gevallen
waarin deze precies $a$, $b$, $\lambda$, of $\emptyset$ is, want dat hebben
we in de basisstap gedaan).

\paragraph{}

\begin{description}

\item[Concatenatie]

Wanneer $r_{1}$ van de vorm $\psi \omega$ is (met $\psi$ en $\omega$ reguliere
expressies), dan zien we aan de definitie van $L_{2}$ dat we $L_{2}$ krijgen
door in de strings van $L_{1}$ een letter weg te laten uit \emph{ofwel} het
deel dat beschreven wordt door $\psi$ \emph{ofwel} het deel dat beschreven
wordt door $\omega$. Het is niet moeilijk in te zien dat dat precies is wat
regel \ref{regel:con} beschrijft.

\item[Keuze]

Evenzo, als $r_{1}$ van de vorm $\psi + \omega$ is, zegt de definitie van
$L_{2}$ dat we deze die strings bevat die beschreven worden door \emph{ofwel}
$\psi$ met daaruit een letter weggelaten, \emph{ofwel} $\omega$ met daaruit
een letter weggelaten. Regel \ref{regel:keuze} beschrijft precies dit.

\item[Ster-afsluiting]

Als laatste geval bekijken we $r_{1}$ wanneer deze van de vorm $\psi^{*}$
is. Het is al laat, dus ik laat het erbij dat het niet moeilijk is om in
te zien dat $L_{2}$ precies die strings bevat die bestaan uit het nul of
meer keer herhalen van $\psi$, gevolgd door $\psi$ met daaruit een letter
weggelaten, gevolgd door wederom het nul of meer keer herhalen van $\psi$.
Deze strings worden precies beschreven door \ref{regel:herh}.

\end{description}

Nu we hebben bewezen dat de reguliere expressie $r_{2}$ precies de taal
$L_{2}$ beschrijft, volgt onmiddelijk dat $L_{2}$ een reguliere taal is.

\hfill\rule{2.1mm}{2.mm}


\end{document}
