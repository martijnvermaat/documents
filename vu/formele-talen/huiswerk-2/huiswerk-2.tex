\documentclass[a4paper,11pt]{article}
\usepackage[dutch]{babel}
\usepackage{amsfonts}
\usepackage{a4}
\usepackage{color, rotating}
\usepackage{latexsym}
\usepackage[
    colorlinks,
    pdftitle={Formele Talen - Inleveropgaven II},
    pdfsubject={Formele Talen},
    pdfauthor={Martijn Vermaat}
]{hyperref}

\title{Formele Talen - Inleveropgaven II}
\author{
    Martijn Vermaat\\
    mvermaat@cs.vu.nl
}
\date{14 december 2004}

\begin{document}
\maketitle


\section*{Opgave 1}


\begin{displaymath}
L = \{a^{n}b^{m}a^{n} \, | \, n,m \geq 0\}
\end{displaymath}


\begin{description}


\item[a)]

We gebruiken de pompstelling voor reguliere talen om te bewijzen dat $L$ niet
regulier is.

Stel dat $L$ wel regulier is. Volgens de pompstelling bestaat er nu een
positief getal $m$ zo dat iedere $w \in L$ met $|w| \geq m$ als volgt
opgedeeld kan worden in $x$, $y$ en $z$:

\begin{eqnarray*}
& w = xyz \mbox{,} & \\
\mbox{met} & & \\
& |xy| \leq m \quad \mbox{en} \quad |y| \geq 1 \mbox{,} & \\
\mbox{zo dat} & & \\
& xy^{i}z \in L \quad \mbox{voor alle $i \in \mathbb{N}$.} &
\end{eqnarray*}

Laten we bij deze $m$ de string $w = a^{m}ba^{m}$ bekijken. Dan kan $w$
geschreven worden als $xyz$ met:

\begin{eqnarray*}
x & = & a^{|x|} \mbox{,} \\
y & = & a^{|y|} \mbox{,} \\
z & = & a^{m-|x|-|y|}ba^{m} \mbox{.}
\end{eqnarray*}

Nu moet ook de volgende string in $L$ zitten:

\begin{eqnarray*}
xy^{2}z & = & a^{|x|}a^{2|y|}a^{m-|x|-|y|}ba^{m} \\
 & = & a^{m+|y|}ba^{m} \mbox{.}
\end{eqnarray*}

Maar omdat $|y| \geq 1$ volgt $m+|y| \not = m$ en dus zit $xy^{2}z$ niet
in $L$. Dit is in tegenspraak met onze eerdere veronderstelling, dus moet
onze aanname onjuist geweest zijn en is $L$ niet regulier.

\hfill\rule{2.1mm}{2.mm}


\item[b)]

Een contextvrije grammatica die $L$ genereert:

\begin{eqnarray*}
S & \rightarrow & a S a \quad | \quad B \\
B & \rightarrow & bB \quad | \quad \lambda
\end{eqnarray*}


\item[c)]

Een nondeterministische pushdown automaat die $L$ accepteert:\\[1em]

\input{huiswerk-2-opg1c.pdftex_t}


\end{description}


\section*{Opgave 2}


\begin{description}


\item[a)]

\begin{enumerate}

\item

Na eliminatie van $\lambda$-producties:

\begin{eqnarray*}
S & \rightarrow & SAa \, | \, Sa \, | \, BBb \, | \, Bb \, | \, b \\
A & \rightarrow & CC \, | \, C \, | \, a \\
B & \rightarrow & C \, | \, Sb \\
C & \rightarrow & SDE \, | \, SE \\
D & \rightarrow & A \, | \, ab
\end{eqnarray*}

\item

Na eliminatie van unit-producties:

\begin{eqnarray*}
S & \rightarrow & SAa \, | \, Sa \, | \, BBb \, | \, Bb \, | \, b \\
A & \rightarrow & CC \, | \, a \, | \, SDE \, | \, SE \\
B & \rightarrow & Sb \, | \, SDE \, | \, SE \\
C & \rightarrow & SDE \, | \, SE \\
D & \rightarrow & ab \, | \, CC \, | \, a \, | \, SDE \, | \, SE
\end{eqnarray*}

\item

Na verwijderen van nutteloze producties:

\begin{eqnarray*}
S & \rightarrow & SAa \, | \, Sa \, | \, BBb \, | \, Bb \, | \, b \\
A & \rightarrow & a \\
B & \rightarrow & Sb
\end{eqnarray*}

\end{enumerate}


\item[b)]

Dezelfde grammatica in Chomsky normaalvorm:

\begin{eqnarray*}
S & \rightarrow & SY_{1} \, | \, SX_{a} \, | \, BY_{2} \, | \, BX_{b} \, | \, b \\
A & \rightarrow & a \\
B & \rightarrow & SX_{b} \\
X_{a} & \rightarrow & a \\
X_{b} & \rightarrow & b \\
Y_{1} & \rightarrow & AX_{a} \\
Y_{2} & \rightarrow & BX_{b}
\end{eqnarray*}


\item[c)]

We schrijven nu dezelfde grammatica om naar Greibach normaalvorm. We gebruiken
hierbij de ordening $<$ op de non-terminals gedefini\"eerd als

\begin{displaymath}
S < A < B \mbox{.}
\end{displaymath}

... hier moet ik even over nadenken...


\end{description}


\section*{Opgave 3}

Hieronder volgt een visuele voortelling van het CYK parseer algoritme
toegepast op $baaba$ en de gegeven grammatica. Voor $V_{nm}=\{\varphi\}$
schrijven we $\{\varphi\}_{nm}$.\\[1em]

\input{huiswerk-2-opg3.pdftex_t}\\[1em]

We zien dat $S \in V_{15}$ en dus zit $baaba$ in de taal die door de gegeven
grammatica wordt gegenereerd.


\section*{Opgave 4}





\end{document}
