\documentclass[a4paper,11pt]{article}
\usepackage[english]{babel}
\usepackage{a4,fullpage}
\usepackage{amsmath,amsfonts,amssymb}
\usepackage{amsthm}
\usepackage[T1]{fontenc}
\usepackage{lmodern} % Latin modern font family
\usepackage[specials]{synttree} % http://www.matijs.net/software/synttree/

% Sans-serif fonts
%\usepackage[T1experimental,lm]{sfmath} % http://dtrx.de/od/tex/sfmath.html
%\renewcommand{\familydefault}{\sfdefault}

% Some configuration for listings
\renewcommand{\labelenumi}{\arabic{enumi}.}
\renewcommand{\labelenumii}{(\alph{enumii})}

\newcounter{firstcounter}
\newcommand{\labelfirst}{(\roman{firstcounter})}
%\newcommand{\spacingfirst}{\usecounter{firstcounter}\setlength{\rightmargin}{\leftmargin}}
\newcommand{\spacingfirst}{\usecounter{firstcounter}}

\newcounter{secondcounter}
\newcommand{\labelsecond}{(\arabic{secondcounter})}
%\newcommand{\spacingsecond}{\usecounter{secondcounter}\setlength{\rightmargin}{\leftmargin}}
\newcommand{\spacingsecond}{\usecounter{secondcounter}}


\title{Logic and Structure\\
\normalsize{Solutions to Selected Exercises (from Fourth Edition)}}

\author{Martijn Vermaat (mvermaat@cs.vu.nl)}
\date{Updated 5th December 2005}


\begin{document}

\maketitle


\section{Propositional Logic}


\subsection{Propositions and Connectives}


\begin{enumerate}

\item % 1
$p_{1}, p_{2}, p_{3}, \neg p_{2}, \neg p_{3}, p_{1} \leftrightarrow p_{2}, p_{3} \vee (p_{1} \leftrightarrow p_{2}),
\neg p_{2} \rightarrow (p_{3} \vee (p_{1} \leftrightarrow p_{2})),\\
(\neg p_{2} \rightarrow (p_{3} \vee (p_{1} \leftrightarrow p_{2}))) \wedge \neg p_{3}$.

$p_{1}, p_{2}, p_{4}, p_{7}, \bot, \neg \bot, p_{7} \rightarrow \neg \bot, \neg p_{2}, p_{4} \leftrightarrow \neg p_{2},
(p_{4} \leftrightarrow \neg p_{2}) \rightarrow p_{1},\\
(p_{7} \rightarrow \neg \bot) \leftrightarrow ((p_{4} \leftrightarrow \neg p_{2}) \rightarrow p_{1})$.

$p_{1}, p_{2}, p_{1} \rightarrow p_{2}, (p_{1} \rightarrow p_{2}) \rightarrow p_{1},
((p_{1} \rightarrow p_{2}) \rightarrow p_{1}) \rightarrow p_{2},
(((p_{1} \rightarrow p_{2}) \rightarrow p_{1}) \rightarrow p_{2}) \rightarrow p_{1}$.

\item % 2
Suppose $((\neg \in X$ and $X$ satisfies (i), (ii), and (iii) of Definition
1.1.2. Now $Y = X - \{((\neg\}$ is smaller than $X$ while still satisfying
those rules making $X$ not the smallest set satisfying them, hence $((\neg$
cannot be in $PROP$. (By the same reasoning as in the example on page 8.)

\item % 3
Let $\varphi, \psi$ be formulas. We use induction on $\chi$ to show
$\varphi \in Sub(\psi) \And \psi \in Sub(\chi) \Rightarrow \varphi \in Sub(\chi)$
for all $\chi$.
($Sub$ as in Definition 1.1.7.)
\begin{list}{\labelfirst}{\spacingfirst}
\item $\chi$ is atomic, therefore $Sub(\chi) = \{\chi\}$. Suppose
$\varphi \in Sub(\psi)$ and $\psi \in \{\chi\}$. Now $\psi = \chi$ and
$Sub(\psi) = \{\chi\}$. So we also have $\varphi = \chi$ and thus
$\varphi \in Sub(\chi)$.

\item $\chi = \chi_{1} \square \chi_{2}$. By definition,
$Sub(\chi) = Sub(\chi_{1}) \cup Sub(\chi_{2}) \cup \{\chi_{1} \square \chi_{2}\}$.
Suppose $\varphi \in Sub(\psi)$ and $\psi \in Sub(\chi)$. Now we have three
cases:
  \begin{list}{\labelsecond}{\spacingsecond}
    \item $\psi \in Sub(\chi_{1})$. By induction we may assume
      $\varphi \in Sub(\chi_{1})$. Hence $\varphi \in Sub(\chi)$.
    \item $\psi \in Sub(\chi_{2})$. Like (1).
    \item $\psi = \chi_{1} \square \chi_{2}$. Now $\psi = \chi$
      and trivially $\varphi \in Sub(\chi)$.
  \end{list}

\item $\chi = \neg \chi_{1}$. Again by definition,
$Sub(\chi) = Sub(\chi_{1}) \cup \{\neg \chi_{1}\}$. Suppose $\varphi \in Sub(\psi)$
and $\psi \in Sub(\chi)$. We have two cases:
  \begin{list}{\labelsecond}{\spacingsecond}
    \item $\psi \in Sub(\chi_{1})$. Same as (1) in (ii).
    \item $\psi = \neg \chi_{1}$. Because $\psi = \chi$, also
      $\varphi \in Sub(\chi)$.
  \end{list}

\end{list}

By the Induction Principle we now have
$\varphi \in Sub(\psi) \And \psi \in Sub(\chi) \Rightarrow \varphi \in Sub(\chi)$
for all formulas $\chi$.

\item % 4
Let $\varphi \in Sub(\psi)$ and suppose $\psi_{0}, \ldots, \psi_{n}$ is a
formation sequence with $\psi_{n} = \psi$. We show
$\varphi \in \{\psi_{0}, \ldots, \psi_{n}\}$ by induction on $n$:
\begin{description}
\item{$n = 0$:} $\psi$ must be atomic, so $Sub(\psi) = \{\psi\}$. This
means $\varphi = \psi$ and thus $\varphi \in \{\psi_{0}, \ldots, \psi_{n}\}$.

\item{$n > 0$:} We consider possible structures for $\psi_{n}$:
  \begin{list}{\labelfirst}{\spacingfirst}
    \item $\psi_{n}$ is atomic. Same as $n = 0$ case above.
    \item $\psi = \psi_{a} \square \psi_{b}$. Note that $\psi_{a}$ and
      $\psi_{b}$ are in $\{\psi_{0}, \ldots, \psi_{n-1}\}$ (by Definition 1.1.4).
      Let $n_{a}, n_{b}$ be such that $\psi_{n_{a}} = \psi_{a}$ and
      $\psi_{n_{b}} = \psi_{b}$. Also,
      $Sub(\psi) = Sub(\psi_{a}) \cup Sub(\psi_{b}) \cup \{\psi_{a} \square \psi_{b}\}$
      which leaves three cases for $\varphi$:
        \begin{list}{\labelsecond}{\spacingsecond}
          \item $\varphi \in Sub(\psi_{a})$. By induction,
            $\varphi \in \{\psi_{0}, \ldots, \psi_{n_{a}}\}$ and thus
            $\varphi \in \{\psi_{0}, \ldots, \psi_{n}\}$.
          \item $\varphi \in Sub(\psi_{a})$. Same as (1).
          \item $\varphi = \psi_{a} \square \psi_{b}$. Now $\varphi = \psi$,
            so $\varphi \in \{\psi_{0}, \ldots, \psi_{n}\}$.
        \end{list}
    \item $\psi = \neg \psi_{a}$. Analogous to (ii).
  \end{list}
\end{description}

\item % 5
By induction on $\psi$ we show that $\varphi \in Sub(\psi)$ if $\varphi$ is in a shortest
formation sequence of $\varphi$.
\begin{list}{\labelfirst}{\spacingfirst}
\item Atomic $\psi$ has a shortest formation sequence $\psi$. If $\varphi$ is
  in this sequence, $\varphi = \psi$ and $\varphi \in Sub(\psi)$ (for
  $Sub(\psi) = \{\psi\}$).
\item Suppose $\varphi$ is in a shortest formation sequence of $\psi_{0} \square \psi_{1}$, then
  clearly $\varphi$ is in a shortest formation sequence of either $\psi_{0}$ or $\psi_{1}$, or
  $\varphi = \psi_{0} \square \psi_{1}$. We consider all three cases:
  \begin{list}{\labelsecond}{\spacingsecond}
    \item By induction $\varphi \in Sub(\psi_{0})$. Now,
      $Sub(\psi_{0} \square \psi_{1} = Sub(\psi_{0}) \cup Sub(\psi_{1}) \cup \{\psi_{0} \square \psi_{1}\}$
      and thus $\varphi \in Sub(\psi_{0} \square \psi_{1})$.
    \item Same as case (1).
    \item A formula is always a subformula of itself.
  \end{list}
\item Suppose $\varphi$ is in a shortest formation sequence of $\neg \psi$, then $\varphi$ is in
  a shortest formation sequence of $\psi$ or $\varphi = \neg \psi$. The two cases are handled like we did
  in (ii).
\end{list}

\item % 6
\begin{enumerate}
\item % 6a
We show that $r(\varphi) \leq$ number of occurrences of connectives of $\varphi$ ($c(\varphi)$) by
induction on $\varphi$:
  \begin{list}{\labelfirst}{\spacingfirst}
    \item For atomic $\varphi$, $r(\varphi) = 0$, $c(\varphi) \in \{0,1\}$.
    \item Let $r(\varphi_{0}) \leq c(\varphi_{0})$ and $r(\varphi_{1}) \leq c(\varphi_{1})$. Now
      $r(\varphi_{0} \square \varphi_{1}) = max(r(\varphi_{0}), r(\varphi_{1}) + 1)$ and
      $c(\varphi_{0} \square \varphi_{1}) = c(\varphi_{0}) + c(\varphi_{1}) + 1)$, so clearly
      $r(\varphi_{0} \square \varphi_{1}) \leq c(\varphi_{0} \square \varphi_{1})$.
    \item Like (ii).
  \end{list}
\item % 6b
  $r(\neg p_{6}) = c(\neg p_{6}) = 1$, $r(\neg p_{3} \wedge \neg p_{1}) < c(\neg p_{3} \wedge \neg p_{1})$.
\item % 6c
  $r((\neg p_{2} \rightarrow (p_{3} \vee (p_{1} \leftrightarrow p_{2}))) \wedge \neg p_{3}) = 4$,\\
  $r((p_{7} \rightarrow \neg \bot) \leftrightarrow ((p_{4} \leftrightarrow \neg p_{2}) \rightarrow p_{1})) = 4$,\\
  $r((((p_{1} \rightarrow p_{2}) \rightarrow p_{1}) \rightarrow p_{2}) \rightarrow p_{1}) = 4$.
\item % 6d
We use induction on $\psi$ to show $r(\varphi) < r(\psi)$ if $\varphi$ is a proper subformula of $\psi$:
\begin{list}{\labelfirst}{\spacingfirst}
\item For atomic $\psi$ there is no proper subformula $\varphi$.
\item Let $r(\varphi) < r(\psi_{0})$ if $\varphi$ is a proper subformula of $\psi_{0}$,
  $r(\varphi) < r(\psi_{1})$ if $\varphi$ is a proper subformula of $\psi_{1}$. Now suppose
  $\varphi$ is a proper subformula of $\psi_{0} \square \psi_{1}$. Then $\varphi$ is a
  subformula of $\psi_{0}$ or a subformula of $\psi_{1}$. We consider both cases:
  \begin{list}{\labelsecond}{\spacingsecond}
    \item $\varphi = \psi_{0}$ or $\varphi$ is a proper subformula of $\psi_{0}$. If the
      former, $r(\varphi) < r(\psi)$. If the latter, $r(\varphi) < r(\psi_{0})$ by our
      induction hypothesis, hence again $r(\varphi) < r(\psi)$.
    \item Analogous to (1).
  \end{list}
\item Like (ii).
\end{list}
\end{enumerate}

\item % 7
\begin{enumerate}
\item The trees for the three propositions in exercise 1:\\ % 7a
\parbox[t]{1.8in}{
\synttree[$\wedge$
  [$\rightarrow$
    [$\neg$[$p_{2}$]]
    [$\vee$
      [$p_{3}$]
      [$\leftrightarrow$[$p_{1}$][$p_{2}$]]
    ]
  ]
  [$\neg$[$p_{3}$]]
]}
\parbox[t]{1.8in}{
\synttree[$\leftrightarrow$
  [$\rightarrow$
    [$p_{7}$]
    [$\neg$[$\bot$]]
  ]
  [$\rightarrow$
    [$\wedge$
      [$p_{4}$]
      [$\neg$[$p_{2}$]]
    ]
    [$p_{1}$]
  ]
]}
\parbox[t]{1.5in}{
\synttree[$\rightarrow$
  [$\rightarrow$
    [$\rightarrow$
      [$\rightarrow$
        [$p_{1}$]
        [$p_{2}$]
      ]
      [$p_{1}$]
    ]
    [$p_{2}$]
  ]
  [$p_{1}$]
]}

\item % 7b
$\neg \neg \neg \bot$,\\
$(p_{0} \rightarrow \bot) \rightarrow ((p_{0} \leftrightarrow p_{1}) \wedge p_{5})$,\\
$\neg (\neg p_{1} \rightarrow \neg p_{1})$.
\end{enumerate}

\item % 8
\begin{enumerate}
\item % 8a
We use induction on $\varphi$ to show that $c(\varphi) + a(\varphi) \leq \#(T(\varphi))$
($c$ as in exercise 6a, $a(\varphi)$ gives the number of atoms of $\varphi$) given that
$\varphi$ does not contain $\bot$:
\begin{list}{\labelfirst}{\spacingfirst}
\item If $\varphi$ is atomic, $c(\varphi) + a(\varphi) = \#(T(\varphi)) = 1$.
\item Let $c(\varphi_{0}) + a(\varphi_{0}) \leq \#(T(\varphi_{0}))$ and
  $c(\varphi_{1}) + a(\varphi_{1}) \leq \#(T(\varphi_{1}))$. Now,
  $c(\varphi_{0} \square \varphi_{1}) + a(\varphi_{0} \square \varphi_{1}) =
  c(\varphi_{0}) + c(\varphi_{1}) + 1 + a(\varphi_{0}) + a(\varphi_{1}) - n$ and
  $\#(T(\varphi_{0} \square \varphi_{1})) = \#(T(\varphi_{0})) + \#(T(\varphi_{1})) + 1$
  where $n \leq 0$ is the number of atoms occuring in both $\varphi_{0}$ and $\varphi{1}$.
  This shows
  $c(\varphi_{0} \square \varphi_{1}) + a(\varphi_{0} \square \varphi_{1}) \leq \#(T(\varphi_{0} \square \varphi_{1}))$.
\item Like (ii).
\end{list}

\item % 8b
\item % 8c
\item % 8d
\end{enumerate}

\item % 9

\item % 10

\item % 11

\item % 12
\begin{enumerate}
\item % 12a
\item % 12b
\item % 12c
\item % 12d
\end{enumerate}

\end{enumerate}


\subsection{Semantics}


\end{document}
