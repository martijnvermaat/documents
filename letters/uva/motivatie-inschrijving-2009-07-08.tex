\documentclass[a4paper,11pt]{brief}
\usepackage[dutch]{babel}
\usepackage{a4,fullpage}
\usepackage[latin1]{inputenc}
\usepackage[T1]{fontenc}
\usepackage{times} % pxfonts or txfonts or lmodern or palatino or times

\date{8 juli 2009}

\address{\textsf{Martijn Vermaat}\\
Platanenweg 18 hs\\
1091 KS Amsterdam}

\signature{Martijn Vermaat,\\
master student informatica\\
Vrije Universiteit Amsterdam}

\betreft{Inschrijving bijvakstudent 2008/2009}

\begin{document}

\begin{letter}{Service \& Informatiecentrum\\
Hoofd studentenzaken\\
Binnengasthuisstraat 9\\
1012 ZA Hoofddorp}

\opening{Geachte heer, mevrouw,}

Afgelopen collegejaar 2008/2009 heb ik het vak \emph{Recursion Theory},
\begin{quote}
vakcode MOLRT6\\
admin. code OWII\\
studielast 6 ects\\
docent dr. P.H. Rodenburg,
\end{quote}
gevolgd als bijvakstudent en op 22 december 2008 de bijbehorende
examinatie succesvol afgerond (cijfer 10).

Ik ben master student aan de Vrije Universiteit bij de opleiding
\emph{Computer Science} (studentnummer 1362917).

Tijdens de loop van het vak is mijn inschrijving slechts gedeeltelijk
verwerkt (UvA studentnummer 5991641), wachtend op de noodzakelijke
schriftelijke toestemming door de examencommissie van de Vrije
Universiteit.

Op 27 januari 2009 heb ik deze schriftelijke toestemming ontvangen,
maar daarop wegens ziekte enkele maanden niet verder gehandeld.

Bij deze verzoek ik u vriendelijk mij met terugwerkende kracht in te
schrijven als bijvakstudent aan de Universiteit van Amsterdam met als
doel het behaalde resultaat bij bovengenoemd vak in de studentenadministratie
op te nemen.

Hiertoe voeg ik bij een ingevuld inschrijvingsformulier 2008/2009 met
alle benodigde bijlagen.

\closing{Met vriendelijke groet,}

\end{letter}

\end{document}
