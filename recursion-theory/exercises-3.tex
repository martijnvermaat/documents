\documentclass[a4paper,11pt]{article}
\usepackage[english]{babel}
\usepackage{a4}
\usepackage[cm]{fullpage}
\usepackage{amsmath,amsfonts,amssymb}
\usepackage{amsthm}
\usepackage[T1]{fontenc}
\usepackage{lmodern} % Latin modern font family

\newtheorem*{lemma}{Lemma}
\newtheorem*{theorem}{Stelling}

\newcommand{\dotmin}{\buildrel\textstyle.\over{\hbox{\vrule height3pt depth0pt width0pt}{\smash-}}}

% Sans-serif fonts
%\usepackage[T1experimental,lm]{sfmath} % http://dtrx.de/od/tex/sfmath.html
%\renewcommand{\familydefault}{\sfdefault}

% Some configuration for listings
\renewcommand{\labelenumi}{\arabic{enumi}.}
\renewcommand{\labelenumii}{(\alph{enumii})}

\newcounter{firstcounter}
\newcommand{\labelfirst}{(\roman{firstcounter})}
%\newcommand{\spacingfirst}{\usecounter{firstcounter}\setlength{\rightmargin}{\leftmargin}}
\newcommand{\spacingfirst}{\usecounter{firstcounter}}

\newcounter{secondcounter}
\newcommand{\labelsecond}{(\arabic{secondcounter})}
%\newcommand{\spacingsecond}{\usecounter{secondcounter}\setlength{\rightmargin}{\leftmargin}}
\newcommand{\spacingsecond}{\usecounter{secondcounter}}


\title{Recursion Theory (UvA autumn 2008)\\
\normalsize{Exercises Part 3 -- Martijn Vermaat (mvermaat@cs.vu.nl)}}

%\author{Martijn Vermaat (mvermaat@cs.vu.nl)}
%\date{Updated 5th December 2005}
\date{}


\begin{document}

\maketitle


\begin{enumerate}


\item % 1
\begin{enumerate}
\item Turing program computing $\lambda x.0$:

\begin{quote}
\begin{tabular}{ccccc}
  $q_1$ & $1$ & $q_1$ & $B$ & $R$\\
  $q_1$ & $B$ & $q_0$ & $B$ & $R$
\end{tabular}
\end{quote}

Just erase the input number.

\item Turing program computing $\lambda x.k$:

\begin{quote}
\begin{tabular}{ccccc}
  $q_1$ & $1$ & $q_1$ & $B$ & $R$\\
  $q_1$ & $B$ & $q_2$ & $B$ & $R$\\
  \vdots & \vdots & \vdots & \vdots & \vdots\\
  $q_i$ & $B$ & $q_{i+1}$ & $1$ & $R$\\
  \vdots & \vdots & \vdots & \vdots & \vdots\\
  $q_{k+1}$ & $B$ & $q_{k+2}$ & $1$ & $R$\\
  $q_{k+2}$ & $B$ & $q_0$ & $B$ & $R$
\end{tabular}
\end{quote}

Again, erase the input number and write $k$ times
a $1$.

\item Turing program computing $\lambda x.2x$:

\begin{quote}
\begin{tabular}{ccccc}
  $q_1$ & $1$ & $q_2$ & $B$ & $R$\\ % begin van input, verander in B
  $q_2$ & $B$ & $q_0$ & $B$ & $R$\\ % is dit einde input dan stoppen
  $q_2$ & $1$ & $q_3$ & $1$ & $R$\\
  $q_3$ & $B$ & $q_4$ & $B$ & $R$\\ % -> zoek einde input
  $q_3$ & $1$ & $q_3$ & $1$ & $R$\\
  $q_4$ & $B$ & $q_5$ & $1$ & $R$\\ % -> zoek einde output en zet er een 1 achter
  $q_4$ & $1$ & $q_4$ & $1$ & $R$\\
  $q_5$ & $B$ & $q_6$ & $1$ & $L$\\ % zet er nog een 1 achter
  $q_6$ & $B$ & $q_7$ & $B$ & $L$\\ % <- zoek einde output
  $q_6$ & $1$ & $q_6$ & $1$ & $L$\\
  $q_7$ & $B$ & $q_1$ & $B$ & $R$\\ % <- zoek einde input
  $q_7$ & $1$ & $q_7$ & $1$ & $L$
\end{tabular}
\end{quote}

For every $1$ in the input, erase it and append $11$ to
the output (which we place to the right of the input).

\item Turing program computing $\lambda x y.x+y$:

\begin{quote}
\begin{tabular}{ccccc}
  $q_1$ & $B$ & $q_2$ & $1$ & $R$\\ % zoek einde x en zet er een 1 achter
  $q_1$ & $1$ & $q_1$ & $1$ & $R$\\
  $q_2$ & $B$ & $q_3$ & $B$ & $L$\\ % zoek einde y
  $q_2$ & $1$ & $q_2$ & $1$ & $R$\\
  $q_3$ & $1$ & $q_4$ & $B$ & $L$\\ % <- haal een 1 weg
  $q_4$ & $1$ & $q_5$ & $B$ & $L$\\ % <- haal een 1 weg
  $q_5$ & $1$ & $q_0$ & $B$ & $L$   % <- haal een 1 weg
\end{tabular}
\end{quote}

Write a $1$ over the $B$ in between inputs $x$ and $y$. Then, starting
from the right of the second input, erase three $1$'s.
\end{enumerate}


\end{enumerate}


\end{document}
