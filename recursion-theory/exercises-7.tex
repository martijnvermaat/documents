\documentclass[a4paper,11pt]{article}
\usepackage[english]{babel}
\usepackage{a4}
\usepackage[cm]{fullpage}
\usepackage{amsmath,amsfonts,amssymb}
\usepackage{amsthm}
\usepackage[T1]{fontenc}
\usepackage{lmodern} % Latin modern font family
\usepackage{enumitem}
\usepackage{fitch} % http://folk.uio.no/johanw/FitchSty.html

\newtheorem*{lemma}{Lemma}
\newtheorem*{theorem}{Stelling}

\newcommand{\dotmin}{\buildrel\textstyle.\over{\hbox{\vrule height3pt depth0pt width0pt}{\smash-}}}

% Corner quotes (from http://www.phil.cam.ac.uk/teaching_staff/Smith/LaTeX/resources/godelquotes.txt)
\newbox\gnBoxA
\newdimen\gnCornerHgt
\setbox\gnBoxA=\hbox{$\ulcorner$}
\global\gnCornerHgt=\ht\gnBoxA
\newdimen\gnArgHgt
\def\Godelnum #1{%
	\setbox\gnBoxA=\hbox{$#1$}%
	\gnArgHgt=\ht\gnBoxA%
	\ifnum \gnArgHgt<\gnCornerHgt
		\gnArgHgt=0pt%
	\else
		\advance \gnArgHgt by -\gnCornerHgt%
	\fi
	\raise\gnArgHgt\hbox{$\ulcorner$} \box\gnBoxA %
		\raise\gnArgHgt\hbox{$\urcorner$}}

% Sans-serif fonts
%\usepackage[T1experimental,lm]{sfmath} % http://dtrx.de/od/tex/sfmath.html
%\renewcommand{\familydefault}{\sfdefault}

% Some configuration for listings
\renewcommand{\labelenumi}{\arabic{enumi}.}
\renewcommand{\labelenumii}{(\alph{enumii})}

\newcounter{firstcounter}
\newcommand{\labelfirst}{(\roman{firstcounter})}
%\newcommand{\spacingfirst}{\usecounter{firstcounter}\setlength{\rightmargin}{\leftmargin}}
\newcommand{\spacingfirst}{\usecounter{firstcounter}}

\newcounter{secondcounter}
\newcommand{\labelsecond}{(\arabic{secondcounter})}
%\newcommand{\spacingsecond}{\usecounter{secondcounter}\setlength{\rightmargin}{\leftmargin}}
\newcommand{\spacingsecond}{\usecounter{secondcounter}}


\title{Recursion Theory (UvA autumn 2008)\\
\normalsize{Exercises Part 7 -- Martijn Vermaat (mvermaat@cs.vu.nl)}}

%\author{Martijn Vermaat (mvermaat@cs.vu.nl)}
%\date{Updated 5th December 2005}
\date{}


\begin{document}

\maketitle


\paragraph{Exercise 32:1}

Suppose $\psi(x_1, x_2)$ represents `$x_2$ codes a proof of the formula
with G\"odel number $Sub(x_1, x_1)$'. Let $n$ be the G\"odel number of
$\forall x_2 \neg \psi(x_1, x_2)$.

\begin{enumerate}[label=(\roman*)]

\item
Show that if $T$ is consistent, $T \not\vdash \forall x_2 \neg \psi(\underline{n}, x_2)$.

\begin{proof}
Suppose $T \vdash \forall x_2 \neg \psi(\underline{n}, x_2)$,
i.e. there is a proof of $\forall x_2 \neg \psi(\underline{n}, x_2)$ in $T$.
Let's say this proof has G\"odel number $p$.

The G\"odel number of $\forall x_2 \neg \psi(\underline{n}, x_2)$ is $Sub(n, n)$,
because $Sub(\Godelnum{\forall x_2 \neg \psi(x_2, x_2)}, n) = \Godelnum{\forall x_2 \neg \psi(\underline{n}, x_2)}$.

Reformulating, $p$ codes a proof of the formula with G\"odel number $Sub(n, n)$.
This is represented by $\psi(\underline{n}, \underline{p})$,
so $T \vdash \psi(\underline{n}, \underline{p})$.
This is in contradiction with $T$ being consistent.
\end{proof}

\item
Prove that if $T$ is also $\omega$-consistent, $\neg \forall x_2 \neg \psi(\underline{n}, x_2)$
is not a theorem of $T$ either.

\begin{proof}
Suppose $T \vdash \neg \forall x_2 \neg \psi(\underline{n}, x_2)$.

We have $T \vdash \neg \psi(\underline{n}, \underline{p})$ for any $p$,
because $\psi(\underline{n}, \underline{p})$ represents that $p$ codes
a proof of the formula with G\"odel number $Sub(n, n)$ and there is no
such proof by (i).

By $T$ $\omega$-consistent, $T \not\vdash \neg \forall x_2 \neg \psi(\underline{n}, x_2)$,
and we have a contradiction.
\end{proof}

\end{enumerate}


\paragraph{Exercise 32:2}

Note that a real number is computable if and only if its absolute value is
computable. Therefore, we only have to consider positive real numbers.

\begin{enumerate}[label=(\alph*)]

\item
Show that every rational number is computable.

\begin{proof}
Consider the rational number $\frac{m}{n}$ with $m, n \in \omega$ and $n \neq 0$.
Take $f(x) = m$ and $g(x) = n$.
Both are obviously computable and $|\frac{m}{n}| = \frac{f(x)}{g(x)}$ for all $x > 0$.
\end{proof}

\item
Show that $e$ and $\pi$ are computable.

\begin{proof}
\begin{enumerate}[label=(\roman*)]
\item Take $f(x) = \Sigma_{n=0}^x \frac{1}{n!}$ and $g(x) = 1$.
Now we have $|\Sigma_{n=0}^\infty \frac{1}{n!}| - \frac{f(x)}{g(x)} = \Sigma_{n=x+1}^\infty$
which is smaller than $\frac{1}{x}$ for $x > 0$.
\item \ldots \qedhere
\end{enumerate}
\end{proof}

\item
Sorry, no time left.

\end{enumerate}


\paragraph{Exercise 32:3}

\begin{enumerate}[label=(\alph*)]

\item
Show:
\begin{equation*}
  T \text{ is axiomatizable } \quad \Longleftrightarrow \quad Thm_T \text{ is computably enumerable}
\end{equation*}

\begin{proof}
\begin{description}
\item[($\Rightarrow$)]
There is an axiomatized theory $T'$ with $Thm_T = Thm_{T'}$. Since the set of G\"odel
numbers of theorems of an axiomatized theory is computably enumerable, $Thm_T$ is
computably enumerable.

\item[($\Leftarrow$)]
By $Thm_T$ computably enumerable, there is a computable function $f(x)$ with $Thm_T$ as
its range. Define
\begin{equation*}
  A = \{ \bigwedge_{i=0}^n f(i) \text{ : } n \in \omega \} \text{ ,}
\end{equation*}
where we take $\bigwedge_{i=0}^n \varphi_i$ to be the G\"odel number of the conjunction
of the formulas coded by $\varphi_0 \ldots \varphi_n$. $A$ is computable, since its
characteristic function can decide for any input if it is the G\"odel number of a
conjunction of formulas $\varphi_0 \ldots \varphi_k$ such that $f(i) = \Godelnum{\varphi_i}$.

Consider the first-order theory $T'$ with axioms $A$. It is axiomatized and $Thm_T = Thm_{T'}$.\qedhere

\end{description}
\end{proof}

\item
The theory with as axioms the theorems of $N$ is axiomatizable (it is equivalent to
$N$ and $N$ is axiomatized), but not axiomatized ($Thm_N$ is not computable).

\end{enumerate}


\paragraph{Exercise 34}

Prove that $K$ is not an index set.

\begin{proof}
By Proposition 33.2, there exists an $n \in K$ such that $W_n = \{n\}$.
By the Padding Lemma, there is an $m > n$ with $\varphi_m = \varphi_n$.
If $K$ were an index set, $m$ would be in $K$.
But this implies $m \in W_m$ while $W_m = W_n = \{n\}$.
Contradiction.
\end{proof}


\end{document}
