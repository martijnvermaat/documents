\documentclass[a4paper,11pt]{article}
\usepackage[english]{babel}
\usepackage{a4}
\usepackage[cm]{fullpage}
\usepackage{amsmath,amsfonts,amssymb}
\usepackage{amsthm}
\usepackage[T1]{fontenc}
\usepackage{lmodern} % Latin modern font family
\usepackage{enumitem}
\usepackage{fitch} % http://folk.uio.no/johanw/FitchSty.html

\newtheorem*{lemma}{Lemma}
\newtheorem*{theorem}{Stelling}

\newcommand{\dotmin}{\buildrel\textstyle.\over{\hbox{\vrule height3pt depth0pt width0pt}{\smash-}}}

% Sans-serif fonts
%\usepackage[T1experimental,lm]{sfmath} % http://dtrx.de/od/tex/sfmath.html
%\renewcommand{\familydefault}{\sfdefault}

% Some configuration for listings
\renewcommand{\labelenumi}{\arabic{enumi}.}
\renewcommand{\labelenumii}{(\alph{enumii})}

\newcounter{firstcounter}
\newcommand{\labelfirst}{(\roman{firstcounter})}
%\newcommand{\spacingfirst}{\usecounter{firstcounter}\setlength{\rightmargin}{\leftmargin}}
\newcommand{\spacingfirst}{\usecounter{firstcounter}}

\newcounter{secondcounter}
\newcommand{\labelsecond}{(\arabic{secondcounter})}
%\newcommand{\spacingsecond}{\usecounter{secondcounter}\setlength{\rightmargin}{\leftmargin}}
\newcommand{\spacingsecond}{\usecounter{secondcounter}}


\title{Recursion Theory (UvA autumn 2008)\\
\normalsize{Exercises Part 6 -- Martijn Vermaat (mvermaat@cs.vu.nl)}}

%\author{Martijn Vermaat (mvermaat@cs.vu.nl)}
%\date{Updated 5th December 2005}
\date{}


\begin{document}

\maketitle


\begin{enumerate}[leftmargin=*,label=\textbf{Exercise 26}]


\item
\begin{description}

\item[(iv)]
Show that the term $x_1 \cdot x_2$ represents multiplication, thus
\begin{align*}
  n_1 \text{ times } n_2 \text{ equals } m \quad \Longrightarrow \quad N \vdash \underline{n_1} \cdot \underline{n_2} = \underline{m} \text{ .}
\end{align*}

By induction on $n_2$:
\begin{description}[leftmargin=1cm,style=nextline]
\item[Case $n_2 = 0$]
  We must show $N \vdash \underline{n_1} \cdot \underline{0} = \underline{0}$.
  Substitute $\underline{n_1}$ for $x$ in $\mathbf{N5}$.
\item[Case $n_2 + 1$]
  We must show $N \vdash \underline{n_1} \cdot S\underline{n_2} = \underline{m}$.
  \begin{equation*}
    \begin{fitch}
      \underline{n_1} \cdot S\underline{n_2} = (\underline{n_1} \cdot \underline{n_2}) + \underline{n_1} & subs $\mathbf{N6}$ \\ % 1
      \underline{n_1} \cdot \underline{n_2} = \underline{m - n_1}                                        & IH \\ % 2
      \underline{n_1} \cdot S\underline{n_2} = \underline{m - n_1} + \underline{n_1}                     & subs 1, 2 \\ % 1
      \underline{n_1} \cdot S\underline{n_2} = \underline{m}                                             & subs 3, \textbf{26(iii)} \\ % 1
    \end{fitch}
  \end{equation*}
\end{description}

\item[(vi)]
Show that if $n_1 < n_2$, then $N \vdash \underline{n_1} \neq x_0 + \underline{n_2}$.
We ignore representability of $<$ and note that $n_2 = n_1 + n_3 + 1$ for some $n_3$.
  \begin{equation*}
    \begin{fitch}
      \fh \underline{n_1} = x_0 + \underline{n_2}          & ass \\ % 1
      \fa \underline{n_1} = x_0 + S\underline{n_1 + n_3}   & subs 1, \textbf{26(iii)} \\ % 2
      \fa \underline{n_1} = S(x_0 + \underline{n_1 + n_3}) & $\mathbf{N4}$ \\ % 3 % TODO: right names for =e and subs, write =e out

    \end{fitch}
  \end{equation*}

\end{description}


\end{enumerate}


\end{document}
