\documentclass[a4paper,11pt]{article}
\usepackage[english]{babel}
\usepackage{a4}
\usepackage[cm]{fullpage}
\usepackage{amsmath,amsfonts,amssymb}
\usepackage{amsthm}
\usepackage[T1]{fontenc}
\usepackage{lmodern} % Latin modern font family
\usepackage{enumitem}
\usepackage{fitch} % http://folk.uio.no/johanw/FitchSty.html

\newtheorem*{lemma}{Lemma}
\newtheorem*{theorem}{Stelling}

\newcommand{\dotmin}{\buildrel\textstyle.\over{\hbox{\vrule height3pt depth0pt width0pt}{\smash-}}}

% Sans-serif fonts
%\usepackage[T1experimental,lm]{sfmath} % http://dtrx.de/od/tex/sfmath.html
%\renewcommand{\familydefault}{\sfdefault}

% Some configuration for listings
\renewcommand{\labelenumi}{\arabic{enumi}.}
\renewcommand{\labelenumii}{(\alph{enumii})}

\newcounter{firstcounter}
\newcommand{\labelfirst}{(\roman{firstcounter})}
%\newcommand{\spacingfirst}{\usecounter{firstcounter}\setlength{\rightmargin}{\leftmargin}}
\newcommand{\spacingfirst}{\usecounter{firstcounter}}

\newcounter{secondcounter}
\newcommand{\labelsecond}{(\arabic{secondcounter})}
%\newcommand{\spacingsecond}{\usecounter{secondcounter}\setlength{\rightmargin}{\leftmargin}}
\newcommand{\spacingsecond}{\usecounter{secondcounter}}


\title{Recursion Theory (UvA autumn 2008)\\
\normalsize{Exercises Part 6 -- Martijn Vermaat (mvermaat@cs.vu.nl)}}

%\author{Martijn Vermaat (mvermaat@cs.vu.nl)}
%\date{Updated 5th December 2005}
\date{}


\begin{document}

\maketitle


\paragraph{Exercise 26}

\begin{description}

\item[(iv)]
Show that the term $x_1 \cdot x_2$ represents multiplication, thus
\begin{align*}
  n_1 \text{ times } n_2 \text{ equals } m \quad \Longrightarrow \quad N \vdash \underline{n_1} \cdot \underline{n_2} = \underline{m} \text{ .}
\end{align*}

By induction on $n_2$:
\begin{description}
\item[($n_2 = 0$)]
  We must show $N \vdash \underline{n_1} \cdot \underline{0} = \underline{0}$.
  Substitute $\underline{n_1}$ for $x$ in $\mathbf{N5}$.
\item[($n_2 + 1$)]
  We must show $N \vdash \underline{n_1} \cdot S\underline{n_2} = \underline{m}$.
  \begin{equation*}
    \begin{fitch}
      \underline{n_1} \cdot S\underline{n_2} = (\underline{n_1} \cdot \underline{n_2}) + \underline{n_1} & inst $\mathbf{N6}$ \\ % 1
      \underline{n_1} \cdot \underline{n_2} = \underline{m - n_1}                                        & IH \\ % 2
      \underline{n_1} \cdot S\underline{n_2} = \underline{m - n_1} + \underline{n_1}                     & subs 1, 2 \\ % 3
      \underline{n_1} \cdot S\underline{n_2} = \underline{m}                                             & subs 3, \textbf{26(iii)} \\ % 4
    \end{fitch}
  \end{equation*}
\end{description}

\item[(vi)]
Show that if $n_1 < n_2$, then $N \vdash \underline{n_1} \neq x_0 + \underline{n_2}$.
We ignore representability of $<$ and note that $n_2 = n_1 + n_3 + 1$ for some $n_3$.
\begin{equation*}
  \begin{fitch}
    \fh \underline{n_1} = x_0 + \underline{n_2}                      & \\ % 1
    \fa \underline{n_1} = x_0 + SS^{n_1}\underline{n_3}               & subs 1, \textbf{26(iii)} \\ % 2
    \fa \underline{n_1} = SS^{n_1}(x_0 + \underline{n_3})             & $n_1+1$ times subs with inst $\mathbf{N4}$ \\ % 3
    \fa \underline{0} = S(x_0 + \underline{n_3})                     & $n_1$ times $\mathbf{MP}$ with inst $\mathbf{N2}$ \\ % 4
    \fa S(x_0 + \underline{n_3}) = \underline{0}                     & 4, symmetry of $=$ \\ % 5
    \fa S(x_0 + \underline{n_3}) \neq \underline{0}                  & inst $\mathbf{N1}$ \\ % 6
    \fa \bot                                                        & contradiction 5, 6 \\ % 7
    \underline{n_1} \neq x_0 + \underline{n_2}                       & $\neg$i 1-7 \\ % 8
  \end{fitch}
\end{equation*}

\item[(vii)]
Show that monus is represented by $(x_0 = \underline{0} \wedge x_1 < x_2) \vee x_1 = x_0 + x_2$.

Suppose $n_1 \dotmin n_2 = m$. We consider two cases:
\begin{description}
\item[($n_1 < n_2$)]
  We know $m$ is $0$ and deduce both directions.
  \begin{equation*}
    \begin{fitch}
      \fh (x_0 = \underline{0} \wedge \underline{n_1} < \underline{n_2}) \vee \underline{n_1} = x_0 + \underline{n_2} & \\ % 1
      \fa \fh x_0 = \underline{0} \wedge \underline{n_1} < \underline{n_2} & \\ % 2
      \fa \fa x_0 = \underline{0} & $\wedge$e 2 \\ % 3
      \fa \fh \underline{n_1} = x_0 + \underline{n_2} & \\ % 4
      \fa \fa \underline{n_1} \neq x_0 + \underline{n_2} & \textbf{26(vi)} \\ % 5
      \fa \fa \bot & contradiction 4, 5 \\ % 6
      \fa \fa x_0 = \underline{0} & $\bot$e 6 \\ % 7
      \fa x_0 = \underline{0} & $\vee$e 1, 2-3, 4-7 \\ % 8
      (x_0 = \underline{0} \wedge \underline{n_1} < \underline{n_2}) \vee \underline{n_1} = x_0 + \underline{n_2} \rightarrow x_0 = \underline{0} & $\rightarrow$i 1-8 \\ % 9
    \end{fitch}
  \end{equation*}
  \begin{equation*}
    \begin{fitch}
      \fh x_0 = \underline{0} & \\ % 1
      \fa \underline{n_1} < \underline{n_2} & \textbf{26(v)} \\ % 2
      \fa x_0 = \underline{0} \wedge \underline{n_1} < \underline{n_2} & $\wedge$i 1, 2 \\ % 3
      \fa (x_0 = \underline{0} \wedge \underline{n_1} < \underline{n_2}) \vee \underline{n_1} = x_0 + \underline{n_2} & $\vee$i 3 \\ % 4
      x_0 = \underline{0} \rightarrow (x_0 = \underline{0} \wedge \underline{n_1} < \underline{n_2}) \vee \underline{n_1} = x_0 + \underline{n_2} & $\rightarrow$i 1-4 \\ % 5
    \end{fitch}
  \end{equation*}

\item[($n_1 \not< n_2$)]
  Now $n_1$ is $m + n_2$, and we give both deductions again.
  \begin{equation*}
    \begin{fitch}
      \fh (x_0 = \underline{0} \wedge \underline{n_1} < \underline{n_2}) \vee \underline{n_1} = x_0 + \underline{n_2} & \\ % 1
      \fa \fh x_0 = \underline{0} \wedge \underline{n_1} < \underline{n_2} & \\ % 2
      \fa \fa \underline{n_1} < \underline{n_2} & $\wedge$e 2 \\ % 3
      \fa \fa \underline{n_1} \not< \underline{n_2} & \textbf{26(v)} \\ % 4
      \fa \fa \bot  & contradiction 3, 4 \\ % 5
      \fa \fa x_0 = \underline{m} & $\bot$e 5 \\ % 6
      \fa \fh \underline{n_1} = x_0 + \underline{n_2} & \\ % 7
      \fa \fa \underline{n_1} = \underline{m} + \underline{n_2} & \textbf{26(iii)} \\ % 8
      \fa \fa x_0 + \underline{n_2} = \underline{m} + \underline{n_2} & subs 7, 8 \\ % 9
      \fa \fa x_0 + \underline{n_2} = S^{n_2}x_0 & $n_2$ times $\mathbf{N4}$, $\mathbf{N3}$ \\ % 10
      \fa \fa \underline{m} + \underline{n_2} = S^{n_2}\underline{m} & $n_2$ times $\mathbf{N4}$, $\mathbf{N3}$ \\ % 11
      \fa \fa S^{n_2}x_0 = S^{n_2}\underline{m} & subs, subs, using 9, 10, 11 \\ % 12
      \fa \fa x_0 = \underline{m} & $n_2$ times $\mathbf{MP}$ with $\mathbf{N2}$ \\ % 13
      \fa x_0 = \underline{m} & $\vee$e 1, 2-6, 7-13 \\ % 14
      (x_0 = \underline{0} \wedge \underline{n_1} < \underline{n_2}) \vee \underline{n_1} = x_0 + \underline{n_2} \rightarrow x_0 = \underline{m} & $\rightarrow$i 1-14 \\ % 15
    \end{fitch}
  \end{equation*}
  \begin{equation*}
    \begin{fitch}
      \fh x_0 = \underline{m} & \\ % 1
      \fa \underline{n_1} = \underline{m} + \underline{n_2} & \textbf{26(iii)} \\ % 2
      \fa \underline{n_1} = x_0 + \underline{n_2} & subs 1, 2 \\ % 3
      \fa (x_0 = \underline{0} \wedge \underline{n_1} < \underline{n_2}) \vee \underline{n_1} = x_0 + \underline{n_2} & $\vee$i 3 \\ % 4
      x_0 = \underline{m} \rightarrow (x_0 = \underline{0} \wedge \underline{n_1} < \underline{n_2}) \vee \underline{n_1} = x_0 + \underline{n_2} & $\rightarrow$i 1-4 \\ % 5
    \end{fitch}
  \end{equation*}
\end{description}

\end{description}

\paragraph{Exercise 27:5}


\end{document}
