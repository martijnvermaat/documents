\documentclass[a4paper,11pt]{article}
\usepackage[english]{babel}
\usepackage{a4}
\usepackage[cm]{fullpage}
\usepackage{amsmath,amsfonts,amssymb}
\usepackage{amsthm}
\usepackage[T1]{fontenc}
\usepackage{lmodern} % Latin modern font family
\usepackage{enumitem}
\usepackage{fitch} % http://folk.uio.no/johanw/FitchSty.html

\newtheorem*{lemma}{Lemma}
\newtheorem*{theorem}{Theorem}

\newcommand{\dotmin}{\buildrel\textstyle.\over{\hbox{\vrule height3pt depth0pt width0pt}{\smash-}}}

% Sans-serif fonts
%\usepackage[T1experimental,lm]{sfmath} % http://dtrx.de/od/tex/sfmath.html
%\renewcommand{\familydefault}{\sfdefault}

% Some configuration for listings
\renewcommand{\labelenumi}{\arabic{enumi}.}
\renewcommand{\labelenumii}{(\alph{enumii})}

\newcounter{firstcounter}
\newcommand{\labelfirst}{(\roman{firstcounter})}
%\newcommand{\spacingfirst}{\usecounter{firstcounter}\setlength{\rightmargin}{\leftmargin}}
\newcommand{\spacingfirst}{\usecounter{firstcounter}}

\newcounter{secondcounter}
\newcommand{\labelsecond}{(\arabic{secondcounter})}
%\newcommand{\spacingsecond}{\usecounter{secondcounter}\setlength{\rightmargin}{\leftmargin}}
\newcommand{\spacingsecond}{\usecounter{secondcounter}}


\title{Recursion Theory (UvA autumn 2008)\\
\normalsize{Exercises Part 13 -- Martijn Vermaat (mvermaat@cs.vu.nl)}}

%\author{Martijn Vermaat (mvermaat@cs.vu.nl)}
%\date{Updated 5th December 2005}
\date{}


\begin{document}

\maketitle


\paragraph{Exercise 2.5.6}

Let $(X_s)_{s \in \omega}$ and $(Y_s)_{s \in \omega}$ be computable enumerations of c.e. sets $X$ and $Y$.

\begin{enumerate}[label=(\alph*)]

\item Both $X \setminus Y$ and $X \searrow Y$ are c.e sets.
\begin{proof}
  Since $z \in X_s - Y_s$ is computable, $X \setminus Y$ is $\Sigma_1$ (hence c.e.).
  The c.e. sets are closed under intersection, so $X \searrow Y$ is c.e. as well.
\end{proof}

\item $X \setminus Y = (X - Y) \cup (X \searrow Y)$.
\begin{proof}
  By definition, we have
  \begin{align*}
    & \exists s(z \in X_s - Y_s) \quad\! \Rightarrow \quad\! \exists s(z \in X_s) &&
    & \exists s(z \in X_s) \quad\! \& \quad\! \neg \exists s(z \in Y_s) \quad\! \Rightarrow \quad\! \exists s(z \in X_s - Y_s)
  \end{align*}
  which give use two useful facts:
  \begin{align}
    & X \setminus Y \subseteq X \label{fact:1}\\
    & X - Y \subseteq X \setminus Y \label{fact:2}
  \end{align}
  \begin{align*}
    X \setminus Y & = (X \setminus Y) \cap (X \cup Y)         && \text{by (\ref{fact:1})}\\
    & = (X \setminus Y) \cap ((X - Y) \cup Y)\\
    & = ((X - Y) \cup (X \setminus Y)) \cap ((X - Y) \cup Y)  && \text{by (\ref{fact:2})}\\
    & = (X - Y) \cap ((X \setminus Y) \cap Y)\\
    & = (X - Y) \cup (X \searrow Y)                           && \text{by definition.}\qedhere
  \end{align*}
\end{proof}

\item If $X - Y$ is not c.e. then $X \searrow Y$ is infinite and indeed noncomputable.
\begin{proof}
  Assume $X - Y$ is not c.e. and suppose $X \searrow Y$ is computable. We have
  \begin{align*}
    X - Y = (X \setminus Y) \cap X \cap \overline{X \searrow Y}
  \end{align*}
  where all intersection operands are c.e. and thus $X - Y$ c.e. to a contradiction.
\end{proof}

\item (Reduction Principle for c.e. sets) Given any two c.e. sets $A$ and $B$,
  there exist c.e. sets $A_1 \subseteq A$ and $B_1 \subseteq B$ such that
  $A_1 \cap B_1 = \emptyset$ and $A_1 \cup B_1 = A \cup B$.
\begin{proof}
  Let $A = W_x$ and $B = W_y$, $A_1 = W_x \setminus W_y$ and $B_1 = W_y \setminus W_x$.
  By the Padding Lemma, we can assume $x \neq y$ without loss of generalization.

  Both $A_1$ and $B_1$ are c.e. by (a).
  By (\ref{fact:1}), $A_1 \subseteq A$ and $B_1 \subseteq B$.
  Furthermore, there is no $z$ such that $\exists s(z \in W_{x,s} - W_{y,s}) \; \& \; \exists s(z \in W_{y,s} - W_{x,s})$, so $A_1 \cap B_1 = \emptyset$.
  Finally, $A_1 \cup B_1 = A \cup B$ follows from
  \begin{align*}
    & \exists s(z \in W_{x,s} - W_{y,s}) \quad\! \text{or} \quad\! \exists s(z \in W_{y,s} - W_{x,s}) \quad\! \Longleftrightarrow \quad\! \exists s(z \in W_{x,s}) \quad\! \text{or} \quad\! \exists s(z \in W_{y,s}) \text{ .}
  \end{align*}
  Note that the ($\Leftarrow$) direction relies on fact (2.24), stating that in every stage, at most one element is enumerated in at most one set $W_e$.
\end{proof}

\end{enumerate}


\end{document}
