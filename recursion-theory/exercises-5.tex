\documentclass[a4paper,11pt]{article}
\usepackage[english]{babel}
\usepackage{a4}
\usepackage[cm]{fullpage}
\usepackage{amsmath,amsfonts,amssymb}
\usepackage{amsthm}
\usepackage[T1]{fontenc}
\usepackage{lmodern} % Latin modern font family
\usepackage{enumitem}

\newtheorem*{lemma}{Lemma}
\newtheorem*{theorem}{Stelling}

\newcommand{\dotmin}{\buildrel\textstyle.\over{\hbox{\vrule height3pt depth0pt width0pt}{\smash-}}}

% Sans-serif fonts
%\usepackage[T1experimental,lm]{sfmath} % http://dtrx.de/od/tex/sfmath.html
%\renewcommand{\familydefault}{\sfdefault}

% Some configuration for listings
\renewcommand{\labelenumi}{\arabic{enumi}.}
\renewcommand{\labelenumii}{(\alph{enumii})}

\newcounter{firstcounter}
\newcommand{\labelfirst}{(\roman{firstcounter})}
%\newcommand{\spacingfirst}{\usecounter{firstcounter}\setlength{\rightmargin}{\leftmargin}}
\newcommand{\spacingfirst}{\usecounter{firstcounter}}

\newcounter{secondcounter}
\newcommand{\labelsecond}{(\arabic{secondcounter})}
%\newcommand{\spacingsecond}{\usecounter{secondcounter}\setlength{\rightmargin}{\leftmargin}}
\newcommand{\spacingsecond}{\usecounter{secondcounter}}


\title{Recursion Theory (UvA autumn 2008)\\
\normalsize{Exercises Part 5 -- Martijn Vermaat (mvermaat@cs.vu.nl)}}

%\author{Martijn Vermaat (mvermaat@cs.vu.nl)}
%\date{Updated 5th December 2005}
\date{}


\begin{document}

\maketitle


\begin{enumerate}[leftmargin=*,start=8,label=\textbf{Exercise 1.6.\arabic*}]


\item
\begin{enumerate}[label=(\alph*)]

\item
For $f_1(x)$, if $x$ is the index of a partial computable function, take
$z$ such that $\varphi_z = \varphi_{f(x)}$
and return it if it is not equal to $f_1(y)$ for all $y < x$.
Repeat this if necessary.
If $x$ is not such an index, return a number that is not equal to $f_1(y)$ for
all $y < x$.

The padding lemma says we can effectively find infinitely
many such $z$'s and since there are only finitely many $y < x$ we will find
a result. This makes $f_1$ (total) computable. It is also obviously injective.

\item
?

\end{enumerate}


\item
\begin{enumerate}[label=\arabic*.]
\item Take any $\varphi_e$.
If $\varphi_e(0)\downarrow$, we have $\varphi_e = \psi_{\langle \varphi_e(0)+1, e \rangle}$.
Otherwise, $\varphi_e = \psi_{\langle 0, e \rangle}$.
This shows $\{\psi_n\}_{n \in \omega} = \mathcal{P}$.

\item Suppose $\psi$ is acceptable.
There is a computable function $g$, such that $\psi_{g(x)} = \varphi_x$.
We show $K_0 \leq_1 A$ with $A = \{x \text{ : } \pi_1(g(x)) \neq 0 \}$.
Let
\begin{align*}
  \gamma(x, y) = \begin{cases}
    1        & \text{if $x \in K_0$ ,}\\
    \uparrow & \text{otherwise}
  \end{cases}
\end{align*}
like $\psi$ in the proof of Theorem 1.5.10.

Take $f(x) = s^1_1(e, x)$ such that $\varphi_e = \gamma$.
Then
\begin{align*}
  x \in K_0 \quad \Rightarrow \quad \varphi_{f(x)} = \lambda y.1 \quad \Rightarrow \quad \pi_1(g(f(x))) \neq 0 \quad \Rightarrow \quad f(x) \in A \text{ ,}\\
  x \notin K_0 \quad \Rightarrow \quad \varphi_{f(x)} = \lambda y.\uparrow \quad \Rightarrow \quad \pi_1(g(f(x))) = 0 \quad \Rightarrow \quad f(x) \notin A \text{ .}
\end{align*}
But since $A$ is computable, $K_0$ is computable.
This is in contradiction with Corollary 1.5.6 ($K_0$ is not computable), hence $\psi$ is not acceptable.
\end{enumerate}


\end{enumerate}


\end{document}
