\documentclass[a4paper,11pt]{article}
\usepackage[english]{babel}
\usepackage{a4,fullpage}
\usepackage{amsmath,amsfonts,amssymb}
\usepackage{amsthm}
\usepackage[T1]{fontenc}
\usepackage{lmodern} % Latin modern font family

\newtheorem*{lemma}{Lemma}
\newtheorem*{theorem}{Stelling}

% Sans-serif fonts
%\usepackage[T1experimental,lm]{sfmath} % http://dtrx.de/od/tex/sfmath.html
%\renewcommand{\familydefault}{\sfdefault}

% Some configuration for listings
\renewcommand{\labelenumi}{\arabic{enumi}.}
\renewcommand{\labelenumii}{(\alph{enumii})}

\newcounter{firstcounter}
\newcommand{\labelfirst}{(\roman{firstcounter})}
%\newcommand{\spacingfirst}{\usecounter{firstcounter}\setlength{\rightmargin}{\leftmargin}}
\newcommand{\spacingfirst}{\usecounter{firstcounter}}

\newcounter{secondcounter}
\newcommand{\labelsecond}{(\arabic{secondcounter})}
%\newcommand{\spacingsecond}{\usecounter{secondcounter}\setlength{\rightmargin}{\leftmargin}}
\newcommand{\spacingsecond}{\usecounter{secondcounter}}


\title{Recursion Theory (UvA autumn 2008)\\
\normalsize{Exercises Part 1 -- Martijn Vermaat (mvermaat@cs.vu.nl)}}

%\author{Martijn Vermaat (mvermaat@cs.vu.nl)}
%\date{Updated 5th December 2005}
\date{}


\begin{document}

\maketitle


\begin{enumerate}


\item % 1
Using Peano's axioms, prove that the predicate $Sum$,
\begin{equation*}
  Sum(z, x, y) \leftrightarrow \forall R \, ( \forall x \, R(x, x, 0) \wedge \forall x y z \, (R(z, x, y) \rightarrow R(Sz, x, Sy)) ) \rightarrow R(z, x, y) ) \text{ ,}
\end{equation*}
defines a function: $Sum(z, x, y) \wedge Sum(z', x, y) \rightarrow z = z'$ .

\begin{proof}
By induction on $y$. Let
\begin{equation*}
  Y = \{ y \in \mathbb{N} \text{ : } \forall x z z' \: Sum(z, x, y) \wedge Sum(z', x, y) \rightarrow z = z' \} \text{ .}
\end{equation*}

We will prove that $Y$ contains exactly Peano's natural numbers. What is
left to prove is thus $\mathbb{N} \subseteq Y$.

Define
\begin{equation*}
  A = \{ R \subseteq \mathbb{N}^3 \text{ : } \forall x \, R(x, x, 0) \wedge \forall x y z \, (R(z, x, y) \rightarrow R(Sz, x, Sy)) \} \text{ .}
\end{equation*}

\begin{description}
\item{$0 \in Y$:} Suppose $Sum(z, x, 0) \wedge Sum(z', x, 0)$.
  Assuming $z \neq x$ (or $z' \neq x$ symmetrically) rises a contradition --
  let $R = \mathbb{N}^3 \setminus \{ (z, x, 0) \}$, then $R \in A$ and
  $Sum(z, x, 0)$ implies $R(z, x, 0)$.

  Now $z = x = z'$ concludes the base case.

\item{$y \in Y \Rightarrow Sy \in Y$:} Suppose $y \in Y$ and
  $Sum(z, x, Sy) \wedge Sum(z', x, Sy)$.
  Assuming $\neg \exists a \, z = Sa$ (likewise for $z'$) rises a
  contradiction (consider $R = \mathbb{N}^3 \setminus \{ (z, x, Sy) \}$,
  argue as in the base case above).

  So there are $a, a'$ such that $z = Sa$ and $z' = Sa'$.
  If $\neg Sum(a, x, y)$, then there exists an $R \in A$ with
  $(a, x, y) \notin R$.
  This shows a contradiction (consider $R \setminus \{ (z, x, Sy) \}$).
  Analoguous for $a'$.

  We conclude $Sum(a, x, y) \wedge Sum(a', x, y)$ and by the induction
  hypothesis $y \in Y$ we get $a = a'$.
  Hence $z = Sa = Sa' = z'$.
\end{description}

Now $\mathbb{N} \subseteq Y$ follows from Peano axiom (V).
\end{proof}


\item % 2
Show $1 + 2 + \ldots + k = \frac{1}{2} k (k + 1)$ .

\begin{proof}
By induction on $k$:
\begin{description}
\item{$k = 0$:} $0 = \frac{1}{2} 0 (0 + 1)$.
\item{$k = i + 1$:} IH is $1 + 2 + \ldots + i = \frac{1}{2} i (i + 1)$. Algebra gives:
  \begin{align*}
    1 + 2 + \ldots + i + (i + 1) &= \tfrac{1}{2} i (i + 1) + (i + 1)\\
                                 &= \tfrac{1}{2} (i (i + 1) + 2 (i + 1))\\
                                 &= \tfrac{1}{2} (i + 2) (i + 1)\\
                                 &= \tfrac{1}{2} (i + 1) ((i + 1) + 1)\qedhere
  \end{align*}
\end{description}
\end{proof}


\end{enumerate}


\end{document}
