\documentclass[a4paper,11pt]{article}
\usepackage[english]{babel}
\usepackage{a4}
\usepackage[cm]{fullpage}
\usepackage{amsmath,amsfonts,amssymb}
\usepackage{amsthm}
\usepackage[T1]{fontenc}
\usepackage{lmodern} % Latin modern font family

\newtheorem*{lemma}{Lemma}
\newtheorem*{theorem}{Stelling}

\newcommand{\dotmin}{\buildrel\textstyle.\over{\hbox{\vrule height3pt depth0pt width0pt}{\smash-}}}

% Sans-serif fonts
%\usepackage[T1experimental,lm]{sfmath} % http://dtrx.de/od/tex/sfmath.html
%\renewcommand{\familydefault}{\sfdefault}

% Some configuration for listings
\renewcommand{\labelenumi}{\arabic{enumi}.}
\renewcommand{\labelenumii}{(\alph{enumii})}

\newcounter{firstcounter}
\newcommand{\labelfirst}{(\roman{firstcounter})}
%\newcommand{\spacingfirst}{\usecounter{firstcounter}\setlength{\rightmargin}{\leftmargin}}
\newcommand{\spacingfirst}{\usecounter{firstcounter}}

\newcounter{secondcounter}
\newcommand{\labelsecond}{(\arabic{secondcounter})}
%\newcommand{\spacingsecond}{\usecounter{secondcounter}\setlength{\rightmargin}{\leftmargin}}
\newcommand{\spacingsecond}{\usecounter{secondcounter}}


\title{Recursion Theory (UvA autumn 2008)\\
\normalsize{Exercises Part 4 -- Martijn Vermaat (mvermaat@cs.vu.nl)}}

%\author{Martijn Vermaat (mvermaat@cs.vu.nl)}
%\date{Updated 5th December 2005}
\date{}


\begin{document}

\maketitle


\begin{enumerate}


\item % 1
Suppose $B = A \oplus \overline{A}$ for some set $A \subset \omega$. Prove $B \leq_1 \overline{B}$.

\begin{proof}
Note that $\overline{B} = \overline{A} \oplus A$. Then,
\begin{equation*}
  f(x) = \begin{cases}
    x + 1 & \text{if $x$ is even,}\\
    x - 1 & \text{if $x$ is odd}
  \end{cases}
\end{equation*}
reduces $B$ to $\overline{B}$ (and is 1-1).
\end{proof}


\item % 2
Prove that $deg_m(A \oplus B) = deg_m(A) \vee deg_m(B)$, thus that the $m$-degree of $A \oplus B$
is the least upper bound of the $m$-degrees of $A$ and $B$.

We must prove two things:

\begin{enumerate}
\item $A \leq_m A \oplus B$ and $B \leq_m A \oplus B$.
\begin{proof}
$f_A(x) = 2x$ and $f_B(x) = 2x + 1$ reduce $A$, respectively $B$, to $A \oplus B$.
\end{proof}

\item If $A \leq_m C$ and $B \leq_m C$ then $A \oplus B \leq_m C$.
\begin{proof}
Let $\phi_A$ reduce $A$ to $C$ and $\phi_B$ reduce $B$ to $C$. Then
\begin{equation*}
  \psi(x) = \begin{cases}
    \phi_A(\frac{x}{2})   & \text{if $x$ is even,}\\
    \phi_B(\frac{x-1}{2}) & \text{if $x$ is odd}
  \end{cases}
\end{equation*}
reduces $A \oplus B$ to $C$.
\end{proof}
\end{enumerate}
Note that in (a), the constructed $\psi$ is in general not injective, so the
proof does not carry over to 1-reducibility.


\item % 3
Prove that $K_0$, $K_1$ and $K$ are 1-equivalent.


\end{enumerate}


\end{document}
