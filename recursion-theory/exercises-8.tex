\documentclass[a4paper,11pt]{article}
\usepackage[english]{babel}
\usepackage{a4}
\usepackage[cm]{fullpage}
\usepackage{amsmath,amsfonts,amssymb}
\usepackage{amsthm}
\usepackage[T1]{fontenc}
\usepackage{lmodern} % Latin modern font family
\usepackage{enumitem}
\usepackage{fitch} % http://folk.uio.no/johanw/FitchSty.html

\newtheorem*{lemma}{Lemma}
\newtheorem*{theorem}{Stelling}

\newcommand{\dotmin}{\buildrel\textstyle.\over{\hbox{\vrule height3pt depth0pt width0pt}{\smash-}}}

% Sans-serif fonts
%\usepackage[T1experimental,lm]{sfmath} % http://dtrx.de/od/tex/sfmath.html
%\renewcommand{\familydefault}{\sfdefault}

% Some configuration for listings
\renewcommand{\labelenumi}{\arabic{enumi}.}
\renewcommand{\labelenumii}{(\alph{enumii})}

\newcounter{firstcounter}
\newcommand{\labelfirst}{(\roman{firstcounter})}
%\newcommand{\spacingfirst}{\usecounter{firstcounter}\setlength{\rightmargin}{\leftmargin}}
\newcommand{\spacingfirst}{\usecounter{firstcounter}}

\newcounter{secondcounter}
\newcommand{\labelsecond}{(\arabic{secondcounter})}
%\newcommand{\spacingsecond}{\usecounter{secondcounter}\setlength{\rightmargin}{\leftmargin}}
\newcommand{\spacingsecond}{\usecounter{secondcounter}}


\title{Recursion Theory (UvA autumn 2008)\\
\normalsize{Exercises Part 8 -- Martijn Vermaat (mvermaat@cs.vu.nl)}}

%\author{Martijn Vermaat (mvermaat@cs.vu.nl)}
%\date{Updated 5th December 2005}
\date{}


\begin{document}

\maketitle


\paragraph{Exercise 2.2.9}

Prove that $K$ is not an index set.

\begin{proof}
By Corollary 2.2.3, there exists an $n \in K$ such that $W_n = \{n\}$.
By the Padding Lemma, there is an $m > n$ with $\varphi_m = \varphi_n$.
If $K$ were an index set, $m$ would be in $K$.
But this implies $m \in W_m$ while $W_m = W_n = \{n\}$.
Contradiction.
\end{proof}


\paragraph{Exercise 2.2.10}

\begin{enumerate}[label=(\alph*)]

\item
Prove that the set $M$ of minimal indices of partial computable functions
is infinite but contains no infinite computably enumerable subset.

\begin{proof}
$M$ is infinite because it includes an index for every constant function.

Suppose there is an $M' \subseteq M$ which is infinite and computably enumerable.
Let $f$ be a computable function enumerating $M'$.
Then from an index $e$ we can compute some minimal index greater than $e$, i.e.
$g(e) = f( \; \mu i \, (f(i) > e) \; )$ is computable.

By the Recursion Theorem, there is some $n$ with $\varphi_{g(n)} = \varphi_n$.
However, since $g(n) > n$ and $g(n)$ is minimal, $\varphi_{g(n)} \neq \varphi_n$.
Contradiction.
\end{proof}

\item

\end{enumerate}


\paragraph{Exercise 2.2.11}


\end{document}
