\documentclass[a4paper,11pt]{article}
\usepackage[english]{babel}
\usepackage{a4}
\usepackage[cm]{fullpage}
\usepackage{amsmath,amsfonts,amssymb}
\usepackage{amsthm}
\usepackage[T1]{fontenc}
\usepackage{lmodern} % Latin modern font family
\usepackage{enumitem}
\usepackage{fitch} % http://folk.uio.no/johanw/FitchSty.html

\newtheorem*{lemma}{Lemma}
\newtheorem*{theorem}{Theorem}

\newcommand{\dotmin}{\buildrel\textstyle.\over{\hbox{\vrule height3pt depth0pt width0pt}{\smash-}}}

% Sans-serif fonts
%\usepackage[T1experimental,lm]{sfmath} % http://dtrx.de/od/tex/sfmath.html
%\renewcommand{\familydefault}{\sfdefault}

% Some configuration for listings
\renewcommand{\labelenumi}{\arabic{enumi}.}
\renewcommand{\labelenumii}{(\alph{enumii})}

\newcounter{firstcounter}
\newcommand{\labelfirst}{(\roman{firstcounter})}
%\newcommand{\spacingfirst}{\usecounter{firstcounter}\setlength{\rightmargin}{\leftmargin}}
\newcommand{\spacingfirst}{\usecounter{firstcounter}}

\newcounter{secondcounter}
\newcommand{\labelsecond}{(\arabic{secondcounter})}
%\newcommand{\spacingsecond}{\usecounter{secondcounter}\setlength{\rightmargin}{\leftmargin}}
\newcommand{\spacingsecond}{\usecounter{secondcounter}}


\title{Recursion Theory (UvA autumn 2008)\\
\normalsize{Exercises Part 9 -- Martijn Vermaat (mvermaat@cs.vu.nl)}}

%\author{Martijn Vermaat (mvermaat@cs.vu.nl)}
%\date{Updated 5th December 2005}
\date{}


\begin{document}

\maketitle


\paragraph{Exercise 37:2}

If $A$ is computably enumerable and $\psi$ is partial computable, then $\psi[A]$ and $\psi^{-1}[A]$ are computably enumerable.

\begin{proof}
\begin{enumerate}[label=(\roman*)]
  \item
    %$A$ is the domain of some partial computable function $\varphi$. Define
    %\begin{equation*}
    %  \chi(x) = \mu y \, ( \psi(y) = x \, \wedge \, \varphi(y) \! \downarrow ) \text{ ,}
    %\end{equation*}
    %and note that $\chi(x)$ is partial computable. Now $\psi[A]$ is the domain of $\chi$.
    % TODO: \varphi(y)\downarrow is not p.c.
  \item
    $A$ is the domain of some partial computable function $\varphi$.
    Now $\psi^{-1}[A]$ is the domain of $\varphi \circ \psi$ which is partial computable.
    \qedhere
\end{enumerate}
\end{proof}


\paragraph{Exercise 37:3}

Let $f$ be a total function. $f$ is a computable function iff it is a computable relation.

\begin{proof}
\begin{description}
  \item{(If)}
    Assume $f$ is a computable relation, i.e. $\chi_f(x, y)$ is computable.
    We can define the function $f$ as $f(x) = \mu y \, (\chi_f(x, y) = 1)$.
    This is a computable function because it is total and $\chi_f(x, y)$ is computable.
  \item{(Only if)}
    Assume $f$ is a computable function.
    The characteristic function of the relation $f$ is
    \begin{equation*}
      \chi_f(x, y) = \begin{cases}
        1 & \text{if $f(x) = y$ ,}\\
        0 & \text{otherwise}
      \end{cases}
    \end{equation*}
    which is computable.
    \qedhere
\end{description}
\end{proof}


\paragraph{Exercise 2.3.12}

%\begin{enumerate}[label=(\alph*)]

%\item
%\emph{(Double Recursion Theorem)} For any computable functions $f(x, y)$ and $g(x, y)$ there exist $a$ and $b$ such that $\varphi_a = \varphi_{f(a, b)}$ and $\varphi_b = \varphi_{g(a, b)}$.

%\begin{proof}
%\end{proof}

%\item
%\emph{(Double Recursion Theorem with Parameters)} For any computable functions $f(x, y, z)$ and $g(x, y, z)$ there exist computable functions $a(z)$ and $b(z)$ such that $\varphi_{a(z)} = \varphi_{f(a(z), b(z), z)}$ and $\varphi_{b(z)} = \varphi_{g(a(z), b(z), z)}$.

%\begin{proof}
%\end{proof}

%\end{enumerate}


\end{document}
