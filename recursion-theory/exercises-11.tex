\documentclass[a4paper,11pt]{article}
\usepackage[english]{babel}
\usepackage{a4}
\usepackage[cm]{fullpage}
\usepackage{amsmath,amsfonts,amssymb}
\usepackage{amsthm}
\usepackage[T1]{fontenc}
\usepackage{lmodern} % Latin modern font family
\usepackage{enumitem}
\usepackage{fitch} % http://folk.uio.no/johanw/FitchSty.html

\newtheorem*{lemma}{Lemma}
\newtheorem*{theorem}{Theorem}

\newcommand{\dotmin}{\buildrel\textstyle.\over{\hbox{\vrule height3pt depth0pt width0pt}{\smash-}}}

% Sans-serif fonts
%\usepackage[T1experimental,lm]{sfmath} % http://dtrx.de/od/tex/sfmath.html
%\renewcommand{\familydefault}{\sfdefault}

% Some configuration for listings
\renewcommand{\labelenumi}{\arabic{enumi}.}
\renewcommand{\labelenumii}{(\alph{enumii})}

\newcounter{firstcounter}
\newcommand{\labelfirst}{(\roman{firstcounter})}
%\newcommand{\spacingfirst}{\usecounter{firstcounter}\setlength{\rightmargin}{\leftmargin}}
\newcommand{\spacingfirst}{\usecounter{firstcounter}}

\newcounter{secondcounter}
\newcommand{\labelsecond}{(\arabic{secondcounter})}
%\newcommand{\spacingsecond}{\usecounter{secondcounter}\setlength{\rightmargin}{\leftmargin}}
\newcommand{\spacingsecond}{\usecounter{secondcounter}}


\title{Recursion Theory (UvA autumn 2008)\\
\normalsize{Exercises Part 11 -- Martijn Vermaat (mvermaat@cs.vu.nl)}}

%\author{Martijn Vermaat (mvermaat@cs.vu.nl)}
%\date{Updated 5th December 2005}
\date{}


\begin{document}

\maketitle


\paragraph{Exercise 3.1.17}

Given c.e. sets $A$ and $B$ prove that $B \leq_T A$ iff there is a computable
function $h$ such that for $D_y$ as in Definition 2.3.4 we have:
\begin{align*}
  x \in \overline{B} \quad \Longleftrightarrow \quad (\exists y) \; [ \; y \in W_{h(x)} \quad \& \quad D_y \subseteq \overline{A} \; ] \text{ .}
\end{align*}

\begin{proof}
($\Rightarrow$)
We code finite strings $\sigma$ over $\{0, 1\}$ as canonical indexes of
finite sets as follows:
\begin{align*}
  \sigma_y(x) = 1 \quad & \Longleftrightarrow \quad x \in D_y \text{ ,} \\
  \sigma_y(x) = 0 \quad & \Longleftrightarrow \quad x \notin D_y \quad \& \quad \exists v \! > \! x (v \in D_y) \text{ .}
\end{align*}
From $B \leq_T A$ we have $B \leq_T \overline{A}$.
Let $\chi_B = \Phi^{\overline{A}}_e$, then
\begin{align*}
  x \in \overline{B} \quad \Longleftrightarrow \quad (\exists y) \; [ \; \Phi^{\sigma_y}_e(x) = 0 \quad \& \quad \sigma_y \prec \chi_{\overline{A}} \; ] \text{ .}
\end{align*}
Since $f(y) = \sigma_y$ is computable, the relation $\Phi^{\sigma_y}_e(x) = 0$ is c.e. on $(y, e, x)$.
Let it be the domain of $\phi_i(e, x, y)$ and take $h(x) = s^2_1(i, e, x)$. We have
\begin{align*}
  \Phi^{\sigma_y}_e(x) = 0 \quad \Longleftrightarrow \quad \phi_i(e, x, y) \! \downarrow \quad \Longleftrightarrow \quad y \in W_{h(x)} \text{ .}
\end{align*}
Furthermore, $\sigma_y \prec \chi_{\overline{A}}$ implies $D_y \subseteq \overline{A}$.
However, not the other way around (a prefix of $\sigma_y$ may mis $1$'s), but this does not
matter here because it still uses only information from $\overline{A}$. Combined, we get
\begin{align*}
  (\exists y) \; [ \; \Phi^{\sigma_y}_e(x) = 0 \quad \& \quad \sigma_y \prec \chi_{\overline{A}} \; ] \quad \Longleftrightarrow \quad (\exists y) \; [ \; y \in W_{h(x)} \quad \& \quad D_y \subseteq \overline{A} \; ] \text{ .}
\end{align*}

($\Leftarrow$)
\ldots
\end{proof}


\paragraph{Exercise 3.1.18}

Consider $A = \overline{K}$ and $R^A = \{ (x,y) | x \in \overline{K} \}$ which is $A$-computable.
If there was a p.c. function $\psi$ with $\psi(x)\!\downarrow \; \Leftrightarrow \; \exists y R^A(x,y)$,
$R^A$ would be c.e. and hence also $\overline{K}$. But $\overline{K}$ is not c.e.


\paragraph{Exercise 3.1.19}

\ldots


\end{document}
