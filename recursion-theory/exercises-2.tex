\documentclass[a4paper,11pt]{article}
\usepackage[english]{babel}
\usepackage{a4,fullpage}
\usepackage{amsmath,amsfonts,amssymb}
\usepackage{amsthm}
\usepackage[T1]{fontenc}
\usepackage{lmodern} % Latin modern font family

\newtheorem*{lemma}{Lemma}
\newtheorem*{theorem}{Stelling}

\newcommand{\dotmin}{\buildrel\textstyle.\over{\hbox{\vrule height3pt depth0pt width0pt}{\smash-}}}

% Sans-serif fonts
%\usepackage[T1experimental,lm]{sfmath} % http://dtrx.de/od/tex/sfmath.html
%\renewcommand{\familydefault}{\sfdefault}

% Some configuration for listings
\renewcommand{\labelenumi}{\arabic{enumi}.}
\renewcommand{\labelenumii}{(\alph{enumii})}

\newcounter{firstcounter}
\newcommand{\labelfirst}{(\roman{firstcounter})}
%\newcommand{\spacingfirst}{\usecounter{firstcounter}\setlength{\rightmargin}{\leftmargin}}
\newcommand{\spacingfirst}{\usecounter{firstcounter}}

\newcounter{secondcounter}
\newcommand{\labelsecond}{(\arabic{secondcounter})}
%\newcommand{\spacingsecond}{\usecounter{secondcounter}\setlength{\rightmargin}{\leftmargin}}
\newcommand{\spacingsecond}{\usecounter{secondcounter}}


\title{Recursion Theory (UvA autumn 2008)\\
\normalsize{Exercises Part 2 -- Martijn Vermaat (mvermaat@cs.vu.nl)}}

%\author{Martijn Vermaat (mvermaat@cs.vu.nl)}
%\date{Updated 5th December 2005}
\date{}


\begin{document}

\maketitle


\begin{enumerate}


\item % 1
\begin{enumerate}
\item $x \cdot y$ is primitive recursive:
\begin{equation*}
  \lambda (x, y). x \cdot y = P(\lambda x.0,
                                (\lambda (x, y). x + y) (e^3_2, e^3_3))
\end{equation*}

\item $x^y$ is primitive recursive:
\begin{equation*}
  \lambda (x, y). x^y = P(S \circ \lambda x.0,
                          (\lambda (x, y). x \cdot y) (e^3_2, e^3_3)) (e^2_2, e^2_1)
\end{equation*}

\item $x!$ is primitive recursive:
\begin{equation*}
  \lambda x.x! = P(S \circ \lambda x.0,
                   (\lambda (x, y). x \cdot y) (e^3_2, e^3_3)) (e^1_1, e^1_1)
\end{equation*}

\item The predecessor function $x \dotmin 1$ is primitive recursive:
\begin{equation*}
  \lambda x.x \dotmin 1 = P(\lambda x.0,
                            e^3_1) (e^1_1, e^1_1)
\end{equation*}

\item The monus function
\begin{equation*}
  x \dotmin y =
  \begin{cases}
    x - y & \text{if $x > y$,}\\
    0     & \text{otherwise}
  \end{cases}
\end{equation*}
is primitive recursive:
\begin{equation*}
  \lambda (x, y).x \dotmin y = P(e^1_1, (\lambda x.x \dotmin 1) (e^3_2)) (e^2_2, e^2_1)
\end{equation*}
\end{enumerate}

From now on, we might write down our primitive recursive definitions in a more
informal way whenever it is obvious how they fit in the schemes (I)--(V), e.g.
\begin{align*}
  x \dotmin 0       &= x\\
  x \dotmin (y + 1) &= (x \dotmin y) \dotmin 1
\end{align*}

If two relations (of the same arity) $P$ and $Q$ are primitive recursive, then
their conjunction is defined by the characteristic function
\begin{equation*}
  \chi^{\wedge}_{P,Q}(\bar{x}) = \chi_{P}(\bar{x}) \cdot \chi_{Q}(\bar{x})
\end{equation*}
and hence is primitive recursive.


\item % 2
$\overline{sg}(x) = 0 \lhd (x > 0) \rhd 1$ is primitive recursive:
\begin{equation*}
  \overline{sg}(x) = 1 \dotmin x
\end{equation*}

Now, if $P$ is primitive recursive, then its negation is characterized by
\begin{equation*}
  \chi^{\neg}_{P}(\bar{x}) = (\overline{sg} \circ \chi_P) (\bar{x}) \text{ .}
\end{equation*}


\item % 3
Definition by cases is primitive recursive. Let $g_1, \ldots g_n$ be primitive
recursive functions, and $R_1, \ldots R_n$ mutually exclusive primitive
recursive relations, then
\begin{equation*}
  f(\bar{x}) = \sum^i_{1 \ldots n} \, \chi_{R_i} \, \cdot \, g_i(\bar{x})
\end{equation*}
is primitive recursive.


\item % 4
If $f$ is primitive recursive, then so are $\prod_{y < x} f(y, \bar{x})$:
\begin{align*}
  \textstyle{\prod_{y < 0}} f(y, \bar{x})     &= 1\\
  \textstyle{\prod_{y < (x + 1)}} f(y, \bar{x}) &= \textstyle{\prod_{y < x}} f(y, \bar{x}) \, \cdot \, f(x, \bar{x})
\end{align*}
and $\sum_{y < x} f(y, \bar{x})$:
\begin{align*}
  \textstyle{\sum_{y < 0}} f(y, \bar{x})     &= 0\\
  \textstyle{\sum_{y < (x + 1)}} f(y, \bar{x}) &= \textstyle{\sum_{y < x}} f(y, \bar{x}) \, + \, f(x, \bar{x})
\end{align*}


\item % 5
If $R$ is primitive recursive, then so is $\forall y \! < \! z \, R(y, \bar{x})$, characterized by:
\begin{equation*}
  \chi^{\forall}_{R,z}(\bar{x}) = \textstyle{\prod_{y < z}} \, \chi_R(y, \bar{x}) \text{ .}
\end{equation*}
Likewise for $\exists y \! < \! z \, R(y, \bar{x})$, characterized by:
\begin{equation*}
  \chi^{\exists}_{R,z}(\bar{x}) = sg( \textstyle{\sum_{y < z}} \, \chi_R(y, \bar{x}) ) \text{ , where } sg = \overline{sg} \circ \overline{sg} \text{ .}
\end{equation*}


\end{enumerate}


\end{document}
