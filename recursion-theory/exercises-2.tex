\documentclass[a4paper,11pt]{article}
\usepackage[english]{babel}
\usepackage{a4,fullpage}
\usepackage{amsmath,amsfonts,amssymb}
\usepackage{amsthm}
\usepackage[T1]{fontenc}
\usepackage{lmodern} % Latin modern font family

\newtheorem*{lemma}{Lemma}
\newtheorem*{theorem}{Stelling}

\newcommand{\dotmin}{\buildrel\textstyle.\over{\hbox{\vrule height3pt depth0pt width0pt}{\smash-}}}

% Sans-serif fonts
%\usepackage[T1experimental,lm]{sfmath} % http://dtrx.de/od/tex/sfmath.html
%\renewcommand{\familydefault}{\sfdefault}

% Some configuration for listings
\renewcommand{\labelenumi}{\arabic{enumi}.}
\renewcommand{\labelenumii}{(\alph{enumii})}

\newcounter{firstcounter}
\newcommand{\labelfirst}{(\roman{firstcounter})}
%\newcommand{\spacingfirst}{\usecounter{firstcounter}\setlength{\rightmargin}{\leftmargin}}
\newcommand{\spacingfirst}{\usecounter{firstcounter}}

\newcounter{secondcounter}
\newcommand{\labelsecond}{(\arabic{secondcounter})}
%\newcommand{\spacingsecond}{\usecounter{secondcounter}\setlength{\rightmargin}{\leftmargin}}
\newcommand{\spacingsecond}{\usecounter{secondcounter}}


\title{Recursion Theory (UvA autumn 2008)\\
\normalsize{Exercises Part 2 -- Martijn Vermaat (mvermaat@cs.vu.nl)}}

%\author{Martijn Vermaat (mvermaat@cs.vu.nl)}
%\date{Updated 5th December 2005}
\date{}


\begin{document}

\maketitle


\begin{enumerate}


\item % 1
\begin{enumerate}
\item $x \cdot y$ is primitive recursive:
\begin{equation*}
  \lambda (x, y). x \cdot y = P(\lambda x.0,
                                (\lambda (x, y). x + y) (e^3_2, e^3_3))
\end{equation*}

\item $x^y$ is primitive recursive:
\begin{equation*}
  \lambda (x, y). x^y = P(S \circ \lambda x.0,
                          (\lambda (x, y). x \cdot y) (e^3_2, e^3_3)) (e^2_2, e^2_1)
\end{equation*}

\item $x!$ is primitive recursive:
\begin{equation*}
  \lambda x.x! = P(S \circ \lambda x.0,
                   (\lambda (x, y). x \cdot y) (e^3_2, e^3_3)) (e^1_1, e^1_1)
\end{equation*}

\item The predecessor function $x \dotmin 1$ is primitive recursive:
\begin{equation*}
  \lambda x.x \dotmin 1 = P(\lambda x.0,
                            e^3_1) (e^1_1, e^1_1)
\end{equation*}

\item The monus function
\begin{equation*}
  x \dotmin y =
  \begin{cases}
    x - y & \text{if $x > y$,}\\
    0     & \text{otherwise}
  \end{cases}
\end{equation*}
is primitive recursive:
\begin{equation*}
  \lambda (x, y).x \dotmin y = P(e^1_1, (\lambda x.x \dotmin 1) (e^3_2)) (e^2_2, e^2_1)
\end{equation*}
\end{enumerate}

From now on, we might write down our primitive recursive definitions in a more
informal way whenever it is obvious how they fit in the schemes (I)--(V), e.g.
\begin{align*}
  x \dotmin 0       &= x\\
  x \dotmin (y + 1) &= (x \dotmin y) \dotmin 1
\end{align*}

If two relations (of the same arity) $P$ and $Q$ are primitive recursive, then
their conjunction is defined by the characteristic function
\begin{equation*}
  \chi^{\wedge}_{P,Q}(\bar{x}) = \chi_{P}(\bar{x}) \cdot \chi_{Q}(\bar{x})
\end{equation*}
and hence is primitive recursive.


\item % 2
$\overline{sg}(x) = 0 \lhd (x > 0) \rhd 1$ is primitive recursive:
\begin{equation*}
  \overline{sg}(x) = 1 \dotmin x
\end{equation*}

Now, if $P$ is primitive recursive, then its negation is primitive recursive
and characterized by $\overline{sg} \circ \chi_P$.


\item % 3
Definition by cases is primitive recursive. Let $g_1, \ldots g_n$ be primitive
recursive functions, and $R_1, \ldots R_n$ mutually exclusive primitive
recursive relations, then
\begin{equation*}
  f(\bar{x}) = \sum^n_{i = 1} \, \chi_{R_i} \, \cdot \, g_i(\bar{x})
\end{equation*}
is primitive recursive.


\item % 4
If $f$ is primitive recursive, then so are $\prod_{y < x} f(y, \bar{x})$:
\begin{align*}
  \textstyle{\prod_{y < 0}} f(y, \bar{x})     &= 1\\
  \textstyle{\prod_{y < (x + 1)}} f(y, \bar{x}) &= \textstyle{\prod_{y < x}} f(y, \bar{x}) \, \cdot \, f(x, \bar{x})
\end{align*}
and $\sum_{y < x} f(y, \bar{x})$:
\begin{align*}
  \textstyle{\sum_{y < 0}} f(y, \bar{x})     &= 0\\
  \textstyle{\sum_{y < (x + 1)}} f(y, \bar{x}) &= \textstyle{\sum_{y < x}} f(y, \bar{x}) \, + \, f(x, \bar{x})
\end{align*}


\item % 5
If $R$ is primitive recursive, then so is $\forall y \! < \! z \, R(y, \bar{x})$, characterized by
\begin{equation*}
  \chi^{\forall}_{R,z}(\bar{x}) = \textstyle{\prod_{y < z}} \, \chi_R(y, \bar{x}) \text{ .}
\end{equation*}
Likewise for $\exists y \! < \! z \, R(y, \bar{x})$, characterized by
\begin{equation*}
  \chi^{\exists}_{R,z}(\bar{x}) = sg( \textstyle{\sum_{y < z}} \, \chi_R(y, \bar{x}) ) \text{ , where } sg = \overline{sg} \circ \overline{sg} \text{ .}
\end{equation*}

Furthermore, if $R$ is primitive recursive, then so is $\lambda \bar{x}. \mu y \! < \! z R(y, \bar{x})$:
\begin{align*}
  \mu y \! < \! 0 R(y, \bar{x})             &= 0\\
  \mu y \! < \! (z \! + \! 1) R(y, \bar{x}) &=
  \begin{cases}
    \mu y \! < \! z R(y, \bar{x}) & \text{if $\mu y \! < \! z R(y, \bar{x}) < z$ ,}\\
                                  & \text{otherwise:}\\
    \begin{cases}
       z     & \text{if $R(z, \bar{x})$ ,}\\
       z + 1 & \text{otherwise.}
    \end{cases}
  \end{cases}
\end{align*}

An alternative approach: Let $\chi_{\overline{mon}R}$ be the characteristic function of the
primitive recursive relation $\forall y \! \leq \! z \neg R(y, \bar{x})$.
Effectively, this takes $\chi_R$, makes it monotone in its first argument,
and inverses the result.

Now we have
\begin{equation*}
  \mu y \! \leq \! z R(y, \bar{x}) = \sum^{z}_{y=0} \chi_{\overline{mon}R}(y, \bar{x}) \text{ ,}
\end{equation*}
which uses $\leq$ instead of $<$, but this can be cured with some fiddling.


\item % 6
The set of primes is primitive recursive:
\begin{equation*}
  Prime(x) \quad \Leftrightarrow \quad
  x \text{ is prime } \quad \Leftrightarrow \quad
  x \neq 0 \, \wedge \, \forall y z \! \leq \! x \: (y \! \cdot \! z = x \, \rightarrow \, y = 1 \vee z = 1) \text{ .}
\end{equation*}

Let $p_i$ be the $i^{\text{th}}$ prime number starting with $p_0 = 2$.
This function is primitive recursive:
\begin{align*}
  p_0    &= 2 \\
  p_{x+1} &= \mu p \! < \! 2 \! \cdot \! p_x \: (Prime(p) \wedge p > p_x)
\end{align*}

We make use of Bertrand's Postulate, stating that for any $n>1$, there is
a prime $p$ with $n < p < 2\! \cdot n$.

A sequence $(k_0, \ldots k_{m\!-\!1})$ is coded by the number $\prod_{i<m} p_i^{k_i+1}$.
If $n$ is the code for a sequence (of at least length $i-1$), the projection
$(n)_i$ gives us the $i^\text{th}$ number of the sequence.
The projection function is primitive recursive:
\begin{equation*}
  (n)_i \quad = \quad \mu x \! < \! n \; ( \exists y \! < \! n \; p_i^{x+1} \cdot y = n ) \wedge \neg ( \exists y \! < \! n \; p_i^{x+2} \cdot y = n )
\end{equation*}

Actually, Soare defines $(n)_i$ to return $k_i+1$ if $k_i$ is
the $i^{\text{th}}$ number in the sequence coded by $n$. Substituting $x$ and
$x+1$ for $x+1$ and $x+2$ respectively in the above definition makes it
conforming to Soare's. However, in that case we only have unique
sequence codes if we disallow zero's in sequences.


\end{enumerate}


\end{document}
